\chapter{Model-View-Controller}
Il \textbf{Model-View-controller} (\textbf{MVC}) è un pattern architetturale molto
diffuso nello sviluppo di sistemi software, in particolare nell'ambito della
programmazione orientata agli oggetti e in applicazioni web, in grado di separare
la logica di presentazione dei dati dalla logica di business. Partiamo analizzando
il pattern architetturale:
\begin{figure}[!ht]
    \centering
    \includegraphics[width=0.5\textwidth]{img/mvc/mvcpattern.png}
    \caption{Model View Controller}
    \label{fig:enter-label}
\end{figure}

Dove con le frecce piene si hanno le invocazioni di metodi e con quelle
tratteggiate gli eventi. Analizzando meglio le tre componenti si ha che:
\begin{itemize}
    \item \textbf{Model}: incapsula lo stato dell'applicazione, fornisce i metodi
          per accedere ai dati utili dell'applicazione. È implementato dalle classi che
          realizzano la logica applicativa.
    \item \textbf{View}: renderizza il modello, richiede l'aggiornamento dai
          modelli, invia le operazioni dell'utente al \textit{controller} e gli
          permette di selezionare la vista. Visualizza i dati contenuti nel model e
          si occupa dell'interazione con utenti e agenti. Presenta lo stato del model all'utente.
    \item \textbf{Controller}: definisce il comportamento dell'applicazione,
          riceve i comandi dell'utente (in genere attraverso il \textit{View}) e li
          attua modificando lo stato degli altri due componenti. Si ha un controller
          per funzionalità. È un mediatore tra view e controller.
\end{itemize}
La \textit{view} raccoglie gli input dell'utente e li inoltra al \textit{controller},
che li mappa in operazioni sul \textit{model}, che viene modificato. A questo punto
il controller seleziona la nuova view da mostrare all'utente, che a sua volta
interagisce per avere i dati con il model. Cambi del model sono notificati alla
view per eventuali cambiamenti dei dati.

Per la costruzione del controller si hanno vari design pattern tra cui:
\begin{itemize}
      \item \textbf{Page Controller}: si ha un controller per pagina che si occupa di:
            \begin{itemize}
                  \item controllare i parametri
                  delle richieste e richiamare la business logic sulla base della richiesta
                  \item selezionare la view successiva da mostrare
                  \item preparare i dati per la presentazione
            \end{itemize}
            Si ha il seguente flusso di controllo:
            \begin{enumerate}
                  \item Il controllo si sposta dal server Web al Page Controller.
                  \item Estrae i parametri dalla richiesta.
                  \item Utilizza alcuni oggetti di business.
                  \item Decide la view successiva.
                  \item Prepara i dati da mostrare nella view successiva
                  \item Porta il controllo (e i dati) alla view.
            \end{enumerate}
            Dal punto di vista implementativo si possono usare soluzioni con poca
            logica di controllo o con una forte logica di controllo. Con questo pattern
            si implementa un controller per ciascuna pagina logica, avendo, per la
            scalabilità, la crescita finita del controller proporzionale al numero
            di pagine necessarie, producendo un numero comunque finito e gestibile
            di file di codice piccoli.
      \item \textbf{Front Controller}: definisce un singolo componente per gestire
            tutte le richieste. Ogni volta che si riceve una richiesta, il front controller
            rappresentato dall'\textbf{handler} si avvale della collaborazione di una gerarchia
            di classi, le quali rappresentano i comandi che possono essere richiesti dall'utente
            attraverso l'interfaccia. Questo approccio risulta scalabile perché al crescere della
            dimensione dell'applicazione ciò che cresce non è l'handler ma cresce la gerarchia
            dei comandi. Nel dettaglio l'handler:
            \begin{itemize}
                  \item Riceve la richiesta dal server.
                  \item Esegue operazioni generali/comuni a tutti i comandi.
                  \item Decide l'operazione che deve essere eseguita e alloca l'istanza
                        del comando.
                  \item Delega l'esecuzione al comando istanziato.
            \end{itemize}
            mentre il comando:
            \begin{itemize}
                  \item Estrae i parametri dalla richiesta.
                  \item Invoca metodi implementati nella business logic.
                  \item Determina la vista successiva.
                  \item Dà il controllo al View.
            \end{itemize}
            Il front controller è più complesso del page controller. Inoltre evita
            la duplicazione del codice tra i vari controller, permette una semplice
            configurazione del server avendo una sola servlet, permette di gestire
            dinamicamente nuovi comandi, facilita l'estensione del controller e
            i comandi possono essere implementati sia come metodi, sia come classi.
            Al crescere dell'applicazione cresce la gerarchia di comandi, mantenendo
            quasi invariato il front controller.
      \item \textbf{Intercepting Filter}: è utile per gestire richieste e risposte
            prima che vengano servite. Viene spesso usato insieme al front controller per
            “decorare” richieste e risposte con funzioni aggiuntive, come autenticazione,
            logging, conversione dati etc$\dots$

            Dal punto di vista implementativo il web server fornisce Filter Manager
            e Filter chain, dovendo quindi implementare e dichiarare solo i filters.
            \begin{figure}[!ht]
                  \centering
                  \includegraphics[width=0.5\textwidth]{img/mvc/filter.png}
                  \caption{Comportamento del Intercepting Filter}
            \end{figure}

            I filtri sono composti da:
            \begin{itemize}
                  \item \textbf{preprocessing}: operazioni prima delle invocazioni
                  \item \textbf{invocazione}: si invocano le funzionalità della
                        filter chain
                  \item \textbf{postprocessing}: operazioni dopo le invocazioni.
            \end{itemize}
      \item \textbf{Application Controller}: usato in quanto all'aumentare della
            complessità del flusso, la gestione del flusso potrebbe essere concentrata in
            una classe. Solitamente viene usato dal page controller o dal front conteoller.
\end{itemize}
Possiamo definire anche i design pattern per l'implementazione della view:
\begin{itemize}
    \item \textbf{Template View}: che genera codice HTML dalle pagine template
          che includono chiamate al modello. Dove nel codice HTML si hanno vari marker
          e l'\textbf{helper} genera i dati del dominio e separa la vista dalla logica
          di implementazione.

          L'helper è creato dal controller ed è accedibile da una vista. Si usano
          anche i metodi helper del controller per recuperare i dati dal Model,
          permettendo di preparare dei dati ad hoc per la parte di presentazione,
          tramite getter specifici.
    \item \textbf{Transformation View}: che trasforma le entità di dominio in HTML.
          Dove per il model tipicamente ogni classe implementa un metodo \textit{.toXML()}
          mentre per il transformer si ha tipicamente l'implementazione tramite
          \textit{eXtensible Style Language for Transformations} (XSLT). Si ha quindi che ci
          si concentra sull'entità che deve essere trasformata, piuttosto che sulla pagina
          di output. Si ha una produzione interamente dinamica della pagina da parte del
          transformer. Il trasformer legge la conversione XML e le regole XSLT,
          crea la pagina dinamicamente, inserendo i dati del model definito come
          XML nel posto specificato dalle regole XSLT.

          Purtroppo è difficile includere la logica di implementazione nella view
          anche se il testing è facile. È facile da applicare a dati in formato
          XML anche se What You See Is What You Get (WYSIWYG) sarebbe più intuitivo e
          facile da implementare. Al contrario di template view non vediamo cosa
          stiamo facendo fino a che non visualizziamo. Template view è usato spesso
          per costruire solo parti di pagine.
    \item \textbf{Two-Steps View} dove si ha la generazione della pagina HTML in
          due passaggi:
          \begin{enumerate}
              \item Si genera un pagina logica.
              \item Si renderizza la pagina.
          \end{enumerate}
          Spesso collassano in un'unica fase.

          Dal punto di vista implementativo si può implementare secondo due metodologie:
          \begin{itemize}
              \item 2 passi XSLT: Una sequenza di due trasformazione XSLT, la prima per la
                    costruzione della pagina logica e la seconda per il suo render.
              \item template-view: Una vista template con tag custom, la pagina HTML è quindi
                    la pagina logica e i tag custom il render.
          \end{itemize}
          Si ha che il “Look \& Feel” è facile da cambiare perché il rendering
          non è ottenuto tramite interazione con la strutturazione delle informazioni
          che devono essere visualizzate nella pagina.
\end{itemize}
\section{Object Relational Mapping}
Come sappiamo non si ha una relazione diretta tra programmazione a oggetti e
modello relazionale. Per questo si usa l'\textbf{Object Relational Mapping}
(\textbf{ORM}) quando si lavora con la programmazione a oggetti e dati persistenti
in un database con le classi che vanno “mappate” nelle entità e viceversa. Molti
dei concetti si applicano ad altri mapping con modelli anche NoSQL. Si sfrutta
quindi un \textbf{API gateway} tra il client e le risorse dati che “incapsula”
l'accesso a una risorsa esterna con una classe e traduce le richieste di accesso
alla risorsa esterna in chiamate all'API. Con questo approccio si nasconde l'accesso
alle risorse e la complessità delle API.

Si hanno tipicamente 4 pattern per il database gateway:
\begin{itemize}
    \item \textbf{Table Data Gateway}: che prevede un oggetto per tabella. Prevede
          classi stateless ed è comodo nel paradigma procedurale ma meno in quello ad
          oggetti, in quanto i metodi ritornano dati row e non oggetti.
    \item \textbf{Row Data Gateway}: che prevede un oggetto per record. Si possono
          avere più copie di oggetti persistenti in memoria, avendo una classe gateway
          che funge da alter-ego del database che viene chiamata dalla classe coi metodi finder.
    \item \textbf{Active Record}: che prevede un oggetto per record.
    \item \textbf{Data Mapper}: che prevede un layer applicativo dedicato all'ORM.
\end{itemize}
\subsection{Active record}
Considerate le classi che implementano i dati che devono essere persistenti si
arricchiscono le classi con metodi per permettere l'interazione col database. Un
\textbf{active record} include quindi:
\begin{itemize}
    \item La business logic.
    \item Il mapping logico al database, avendo  metodi statici come \textit{finder}
          (invocabili avendo accesso alla classe) che comunque restituiscono risultati
          già nel dominio (oggetti o collezioni di oggetti) e non statici i metodi gateway per lavorare con le istanze.
\end{itemize}
Tra i metodi abbiamo quindi:
\begin{itemize}
    \item \textbf{Load}: che crea un'istanza a partire dai risultati di una query
          SQL. Potrebbe essere necessario creare altri oggetti con accesso a più tabelle
          nel casi di reference.
    \item \textbf{Costruttore}: per creare nuove istanze che saranno poi rese
          persistenti invocando metodi appositi.
    \item \textbf{Finder} (statici): che incapsulano la query SQL e ritornano una
          collezione di oggetti.
    \item \textbf{Write}: tra cui si hanno:
          \begin{itemize}
              \item \textbf{Update}: per aggiornare un record a seconda dei valori degli attributi.
              \item \textbf{Insert}: per aggiungere un record a seconda dei valori degli attributi.
              \item \textbf{Delete}: per cancellare il record corrispondente all'oggetto in analisi.
          \end{itemize}
          serve sincronizzazione tra oggetto in memoria ed entità nel database.
          Per farlo si ha un attributo identificatore nella classe che compare anche
          nel record e che serve per trovare il record corrispondente ad un oggetto.
    \item \textbf{Getter} e \textbf{Setter}: che a seconda del caso devono essere
          subito sincronizzati con il database
    \item \textbf{Business}
\end{itemize}
Non è il metodo migliore dal momento che nella stessa classe ho sia i metodi 
della logica di business, sia i metodi per il mapping nel db. In aggiunta, forza 
anche la corrispondenza perfetta tra OOP e ER design. Il lato positivo è la 
semplicità.
	
\subsection{Data mapper}
Con il pattern \textbf{Data Mapper} invece si ha un layer software indipendente,
dedicato alla persistenza e all'ORM, in modo che la logica di dominio non conosca
la struttura del database e nemmeno le query SQL. Si usano quindi interfacce per
consentire l'accesso ai mapper indipendentemente dalla loro implementazione.

Framework moderni implementano questo tipo di pattern (che offre la parte di
mapping senza doverli implementare, a solo configurare).
\subsection{Ulteriori pattern ORM}
In generale si hanno dei pattern di comportamento per risolvere altri problemi ORM:
\begin{itemize}
    \item \textbf{Unit of work}
    \item \textbf{Identity map}
    \item \textbf{Lazy load}
\end{itemize}
e anche pattern strutturali:
\begin{itemize}
    \item \textbf{Value holder}
    \item \textbf{Identify field}
    \item \textbf{Embedded value}
    \item \textbf{LOB}
\end{itemize}
\subsubsection{Unit of work}
In questo caso si approfondiscono le operazioni ACID cercando di capire quando
sincronizzare le informazioni in memory con il database. Questo aspetto è importante
introducendo il concetto di \textbf{business transaction} per operazioni con
semantica di tipo all or nothing dove o si accettano tutte le operazioni richieste
o nessuna. Le business transaction vanno quindi mappate nelle \textbf{system transaction} fatte dal sistema sul database.

Si hanno varie soluzioni:
\begin{itemize}
	\item \ref{fig:bt-uguale-st} aggiornare il database ogni volta che si ha una modifica in memoria. 
      Si lega la business transaction con la system transaction. Infattibile 
      perché si avrebbe che la transazione del DB sarebbe troppo lunga, 
      comportando che l'aggiornamento impedisce l'accesso agli altri
      \begin{figure}[!ht]
            \centering
            \includegraphics[width=0.5\textwidth]{img/mvc/bt-uguale-st.png}
            \caption{System transaction uguale alla Business transaction}
            \label{fig:bt-uguale-st}
      \end{figure}
	\item \ref{fig:op-uguale-st}  aggiornare il database ad ogni modifica in memoria. Per la business 
      transaction  si hanno tante system transaction una per ogni modifica. Si 
      rischia inconsistenza perché si hanno diversi aggiornamenti ma in periodi 
      diversi.
      \begin{figure}[!ht]
            \centering
            \includegraphics[width=0.5\textwidth]{img/mvc/op-uguale-st.png}
            \caption{System transaction per ogni singola operazione nella Business
            transaction}
            \label{fig:op-uguale-st}
      \end{figure}
	\item \ref{fig:bt-uguale-st} aggiornamento finale al termine della business transaction.
      \begin{figure}[!ht]
            \centering
            \includegraphics[width=0.5\textwidth]{img/mvc/st-finale.png}
            \caption{System transaction alla fine della Business transaction}
            \label{fig:bt-uguale-st}
      \end{figure}
\end{itemize}
La \textbf{unit of work} quindi traccia ogni cambiamento (coi metodi register) e
lo attuta in una singola system transaction finale (con il commit). Per fare questo incapsula ogni update/insert/delete, mantenendo l'integrità del database ed evitando deadlock.

A proposito di \textit{locking} si ha che due business transaction possono essere
attive nello stesso momento e bisogna capire come devono lavorare sul database.
In questo caso si hanno due strategie:
\begin{itemize}
    \item \textbf{Optimistic locking}: usato quando si hanno poche probabilità
          di generare conflitti, avendo diversi utenti che lavorano spesso su dati
          differenti, e si bassa su un numero che identifica la versione degli oggetti
          registrati. Due business transaction possono quindi interferire accedendo e
          modificando i record ma si sfrutta il numero di versione per tenere traccia
          delle modifiche e se una transazione vede che il numero di versione di un record è
          incrementato e in tal caso si ricarica il record tramite un rollback.
    \item \textbf{Pessimistic locking}: usato quando hanno alte probabilità di
          generare conflitti. Gli oggetti sono bloccati non appena vengono usati riducendo
          a concorrenza. Non si permette quindi alle transazioni di lavorare
          concorrentemente riducendo però le prestazioni del sistema ma aumentando la sicurezza.
\end{itemize}
Poi, per avvisare la unit of work che un'operazione di persistenza è stata
schedulata, si hanno due pattern:
\begin{itemize}
    \item \textbf{Caller registration}: dove chi modifica l'entità lo notifica
          direttamente alla unit of work. Questa soluzione è rischiosa in quanto il
          programmatore potrebbe dimenticarsi ma permette maggior flessibilità, permettendo
          di decidere dinamicamente se i cambiamenti devono “riflettersi” sullo storage
          persistente.
    \item \textbf{Object registration}: dove è l'entità modificata a notifica la
          unit of work, che però deve obbligatoriamente avere accesso globale. Ogni
          cambiamento viene comunque notificato. Solitamente si usa questa soluzione ma
          nella variante in cui le classi persistenti non includono il codice per la
          notifica ma tramite una strumentazione trasparente offerta dai framework.
\end{itemize}
\subsubsection{Identity map}
In questo caso si studia il load di un'istanza direttamente a partire dal database.
Se un certo record viene richiesto due volte si generano due oggetti uguali.
In pratica uno stesso oggetto può essere caricato più volte, o da azioni differenti
dello stesso utente rischiando inconsistenze, se una istanza viene modificata, oltre
ad avere ridondanza di dati o da utenti diversi in questo spesso è intenzionale.

\textbf{Identity map} è quindi un oggetto con la responsabilità di identificare
gli oggetti caricati in una sessione, funzionando come una sorta di cache,
controllando l'eventuale presenza dell'oggetto “in cache” prima di caricarne uno
nuovo, se già presente si ritorna un riferimento
all'oggetto. Identity map quindi mantiene i riferimenti agli oggetti tramite un
sistema chiave-valore.

Generalmente la chiave della mappa è quella primaria del database. Si hanno
diverse identity map:
\begin{itemize}
    \item Una per applicazione, se chiavi di entità differenti sono disgiunte
    \item Una per classe
    \item Una per tabella, generalmente meglio se una per classe
\end{itemize}
\subsubsection{Lazy load}
A volte è necessario avere solo una parte di un oggetto per eseguire una certa
operazione anche perché alcuni attributi possono essere particolarmente pesanti.
Si parla quindi di efficienza.

Si procede quindi con il pattern lazy load che crea oggetti inizializzando solo
alcuni attributi lasciando comunque possibilità si caricare in seguito ulteriori
attributi non ancora inizializzati se ce ne fosse necessità, in modo on-demand,
tramite i finder.
\subsubsection{Value holder}
Con questa tecnica si usa un oggetto, il value holder, come uno storage
intermediario per un attributo. Anche in questo caso si ha un'inizializzazione
“lazy” implementata da questo oggetto che caso per caso potrebbe essere istanziato
con diverse strategie di load. Si separa la locgia di dominio da quella “lazy” di loading.
\subsubsection{Identify field}
Normalmente si hanno:
\begin{itemize}
    \item L'identità in memory rappresentata dalla reference dell'oggetto.
    \item L'identità nel database rappresentata dalla chiave primaria.
\end{itemize}
e si ha bisogno di accoppiare/sincronizzare le due identità per la consistenza
dei dati.

La soluzione è usare l'identity field ovvero un “nuovo” attributo nell'oggetto
il cui valore diventa la “chiave primaria” (solitamente chiamato id, di tipo long).
Non si usano i normali attributi/identificatori della classe perché potrebbero
modificare nel tempo di sviluppo del programma.

Si hanno infatti due tipi di chiave:
\begin{itemize}
    \item Una chiave derivata da attributi dell'oggetto (che sono usati anche
          dall'applicazione). Si ha comunque rischio di duplicazione, avere assenza di
          valore e avere cambiamenti sulla logica di dominio che impattano su quella di persistenza.
    \item Chiavi dedicate, sia semplici (un solo attributi) che composte (di più
          attributi), avendo una chiave per tabella e una per database.
          Nella pratica spesso hanno tipo long che ben supporta il check di uguaglianza
          (==) e la generazione di nuove chiavi.
\end{itemize}
Si hanno varie strategie per la generazione delle chiavi:
\begin{itemize}
    \item Direttamente dal database (con sequenze o id delle colonne)
    \item Tramite GUID, ovvero Globally Unique IDentifier, tramite opportuni
          software e libreria
    \item Dall'applicazione, principalmente in due modi:
          \begin{itemize}
              \item \textbf{Table scan}, usando una query SQL per determinare il valore
                    della prossima chiave.
              \item \textbf{Key table}, usando tabelle speciali con il nome delle altre
                    tabelle e il valore successivo della chiave.
          \end{itemize}
\end{itemize}
I moderni framework ORM supportano i vari design pattern appena discussi e quelli
mancanti, relativi a chiavi esterne, cascading, ereditarietà $\dots$ In ogni caso
i pattern appena discussi sono tutti supportati dai moderni framework ORM.
\subsubsection{Embedded value}
Con questo pattern si hanno semplici oggetti, senza un chiaro concetto di identità,
che possono essere inclusi in oggetti che fungono da alter-ego delle tabelle
anziché creare tabelle dedicate.

\subsection{JPA}
Standard di implementazione di ORM in Java.
Grande set di pattern/meccanismi già implementati mediante:
\begin{itemize}
      \item sistema di annotazioni
      \item servizio
\end{itemize}

Avremo 4 macroelementi:
\begin{itemize}
      \item database: che saverà degli oggetti, istanze di entity class, classi
      con un mapping ad un record. Il mapping è garantito attraverso le annotazioni
      \item classi della logica di dominio
      \item entity manager: servizio che permette di lavorare sulle entità persistenti
\end{itemize}

\subsubsection{Entity manager}
Gli oggetti persistenti sono anche chiamati come Pojo perché oggetti normali Java
ma con annotazioni. Sono uguali fino a quando non interagiscono con l'entity manager.

Pojo possono avere 2 stati:
\begin{itemize}
      \item managed: entity manager è consapevole nella presenza in memoria
      \item unmanaged: entity manager non è consapevole nella presenza in memoria
\end{itemize}
Essenziale lo stato per implementare il pattern Unit of Work e Identity map.
In caso di creazione o cancellazione di entità dobbiamo notificarlo alla UoW mentre
nel caso di modifica queste vengono applicate automaticamente. 
EM cattura le notifiche e c'è una trasparenza nella notifica.

Un'entità diventa managed quando si crea da una classe con annotazione entity mentre
è unmanaged sono quegli oggetti che si inviano tra Sistemi distribuiti o che si ricevono.
quindi in questo caso bisogna notificarlo all'EM.

Il servizio EM è in grado di gestire solo le classi che dichiariamo (insimee di classi persistenti = persistence unit)
e vengono dichiarati nel file persistence.xml e si deve specificare una connessione al db.
A runtime la persistence unit si chiama persistence context. (si possono dichiarare più persistence unit e context)

l'annotazioen @Entity specifica un oggetto persistente e avremo una tabella 
per pemorizzare gli oggetti con le colonne con gli stessi nomi degli attributi.
Si possono utilizzare annotazioni per modificare il mapping come i nomi dei fields 
e delle tabelle. 

Definito il mapping utilizzeremo EM per accedere ai dati. Creiamo l'EM e per 
notificare la creazione si usa la funzione di persist. (UoW) La scrittura non è 
immediata perché aspetta quando si aprirà la transazione. Dall'EM posso recuperare
l'entità con i metodi finder che chiedono la classe dell'oggetto richiesto e id.
Il risultato dei metodi sono sempre oggetti.
getReference è caricamento Lazy
Un altro modo di estrazione degli oggetti è attraverso una query simile a SQL,
non si usano i nomi delle tabelle o field delle tabelle ma bensì ai nomi delle classi
e attributi (sempre secondo Obj oriented).
Per modificare le entità si modificano gli attributi e alla prossima transazione
verrà salvata in automatico. La modifica non viene riportata dal db se l'entità
è unmanaged. Per renderla effettiva allora si usa merge che:
\begin{itemize}
      \item se l'entità non esiste allora viene inserita
      \item se esiste un'altra entità si sovrascrive l'unmanaged sul db
\end{itemize}

refresh posso aggiornare l'entità 

per aprire e chiudere la transaction ci sono dei metodi dell'EM. 


SequenceGen(nome generatore in Java, nome generatore nel DB)

Cerca di usare sempre un'attributo ad hoc per le chiavi

possiamo fare il mapping di un'entità su più tabelle con SecondaryTable e l'entità
sarà partizionata tra le due tabelle che hanno i record associati con lo stesso id.
Utile per accedere a database legacy e codificare più tabelle in un'entità.

Mapping delle associazioni comportano il caricamento anche delle entità associate

1-1 monodirezionale: mappato come relazione 1-1 in dib con una fk nella tabella 
padre. Posso fare il join con chiavi primarie quindi diventà un'entita spartita 
tra 2 tabelle

1-n monodirezionale: in memoria Customer con collezione sui telefoni e lo possiamo
gestire con una tabella di join. Si deve specificare il join table e le colonne 
delle fk.

n-1 monodirezionale: esattamente uguale a 1-n ma al contrario

relazioni bidirezionali: raramente si utilizzano e per crearlo aggiungo un field
in entrabe le classi e nel mapping specifico da una parte la colonna uguale all'altra.

n-m monodirezionale: 

Si progetta direttamente le classi e non il DB.

Si mappa Ereditarietà perché permette di ridurre la duplicazione e di risolvere
le query correttamente.

single table: tutto in una tabella e si introduce un attributo discriminatore.
richiede che sia possibile avere null negli attributi. Veloce e semplice ma 
si vincola ad non avere not null.

1 tabella per classe: supporta i constraint, mapping semplice, ma complesso la
gestione em e il DBMS deve supportare la union.

1 tabella per sottoclasse: supporta i constraint, non necessita union, ma lento
e 1 istanza in più tabelle. (ottimale in termini di spazio)docdo
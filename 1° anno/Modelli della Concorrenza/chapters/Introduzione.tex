\chapter{Introduzione}
Possiamo definire diverse semantiche per la programmazione sequenziale:
\begin{itemize}
    \item \textbf{Operazionale}: ho una macchina astratta e definisco i vari
          passi della computazione:
          \begin{equation}
              Input \longrightarrow M \longrightarrow Output
          \end{equation}
    \item \textbf{Denotazionale}: dato un programma funzionale $P$ ho una
          funzione definita come:
          \begin{equation}
              f: \text{Dati}_{I} \to \text{Dati}_{O} \ ( \lambda-\text{calcolo})
          \end{equation}
    \item \textbf{Assiomatica}: l'input e l'output sono rappresentati come
          formule logiche. Questa tipologia prende il nome di \textit{Triple di
              Hoare}.
          \begin{equation}
              \{Input\} P \{Output\}
          \end{equation}
\end{itemize}
Nella programmazione sequenziale si hanno due punti chiave che devono essere
garantiti:
\begin{itemize}
    \item \textbf{Terminazione del programma}.
    \item \textbf{Composizionalità} tra più comandi per ottenere un programma,
          ovvero:
          \begin{equation}
              \begin{aligned}
                  s_1: x = 2 \ \ \{x = V\} \ x = 2 \ \{x = 2\} \\
                  s_2: x = 3 \ \ \{x = V\} \ x = 3 \ \{x = 3\} \\
                  \{x = V\} \ s_1; \ s_2 \ \{x = 3\}           \\
              \end{aligned}
          \end{equation}
          Se esiste un programma $s_1'$ tale che mi permette di ottenere lo
          stesso risultato si $s_1$, allora posso sostituirlo a $s_1$.
          \begin{equation}
              \begin{aligned}
                  s_1': \{x = V\} \ x = 1; \ x = x + 1  \ \{x = 2\} \\
                  \{x = V\} \ s_1'; \ s_2 \ \{x = 3\}               \\
              \end{aligned}
          \end{equation}
\end{itemize}


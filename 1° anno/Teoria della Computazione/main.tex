\documentclass[a4paper,15pt, oneside]{book}
\usepackage[italian]{babel}
\usepackage[utf8]{inputenc}
\usepackage[a4paper,top=2.5cm,bottom=2.5cm,left=2cm,right=2cm]{geometry}
\usepackage{amssymb}
\usepackage{amsthm}
\usepackage{graphics}
\usepackage{amsfonts}
\usepackage{amsmath}
\usepackage{amstext}
\usepackage{engrec}
\usepackage{rotating}
\usepackage[safe,extra]{tipa}
\usepackage[table,xcdraw]{xcolor}

\usepackage{multirow}
\usepackage{hyperref}
\usepackage{enumerate}
\usepackage{braket}
\usepackage{marginnote}
\usepackage{pgfplots}
\usepackage{cancel}
\usepackage{polynom}
\usepackage{booktabs}
\usepackage{enumitem}
\usepackage{algorithm}
\usepackage{algpseudocode}
\usepackage{framed}
\usepackage{pdfpages}
\usepackage{pgfplots}
\usepackage{fancyhdr}
\usepackage{caption}
\usepackage{subcaption}
\usepackage{setspace}


\pagestyle{fancy}
\fancyhead[L,RO]{\slshape \rightmark}
\fancyfoot[C]{\thepage}
\title{Teoria della Computazione}
\author{Tommaso Ferrario (\href{https://github.com/TommasoFerrario18}{@TommasoFerrario18}) \\\\
Telemaco Terzi (\href{https://github.com/Tezze2001}{@Tezze2001})}
\date{October 2023}

\pgfplotsset{compat=1.13}

\begin{document}

\maketitle
\newtheorem{teorema}{Teorema}
\newtheorem{dimostrazione}{Dimostrazione}
\newtheorem{definizione}{Definizione}
\newtheorem{esempio}{Esempio}
\newtheorem{osservazione}{Osservazione}
\newtheorem{nota}{Nota}
\newtheorem{corollario}{Corollario}
\tableofcontents
\renewcommand{\chaptermark}[1]{
  \markboth{\chaptername
    \ \thechapter.\ #1}{}}
\renewcommand{\sectionmark}[1]{\markright{\thesection.\ #1}}

\chapter{Introduzione}
In questo corso si analizzeranno modelli probabilistici di machine learning.
I modelli probabilistici permettono di quantificare l'incertezza delle informazioni
e successivamente prendere delle decisioni. Questi passi sono fondamentali perché
siamo costantemente circondati da tantissimi dati (evidenze) che per loro natura sono incerti,
quindi dobbiamo avere un modo per gestire l'incertezza con l'obiettivo di prendere 
delle decisioni sensate.

Quando devo prendere delle decisioni sui dati che analizzo, posso avere diversi
problemi:
\begin{itemize}
    \item Dati mancanti/inesatti
    \item Evidenze inconsistenti
    \item Diverse fonti di incertezza
\end{itemize}

I \textbf{modelli probabilistici} rappresentano l'incertezza tramite le dipendenze
tra le variabili (struttura del modello) e le probabilità (parametri del modello).
Rappresentando l'incertezza, si riesce ad automatizzare la parte di previsione
di conseguenza risultano:
\begin{itemize}
    \item scalabili
    \item robusti
    \item adattivi
\end{itemize}

Per ora sono stati visti solo modelli \textbf{discriminativi}, al contrario i
modelli probabilistici sono \textbf{generativi} perché sono caratterizzati da
variabili aleatorie, le quali possono essere campionate e permettono di effettuare
delle simulazioni. Essi si basano sulla teoria della probabilità e quindi ci
permetteranno di inferire quantità sconosciute e allo stesso tempo apprendere.

A livello di matematica si partirà da Teorema di Bayes per adattare i modelli
nel tempo.

La fase di inferenza si classifica in due tipi:
\begin{itemize}
    \item \textbf{Diagnostica}: dalla classe si ricava l'evidenza.
    \item \textbf{Prognostica}: dall'evidenza si ricava la classe.
\end{itemize}

\begin{esempio}
    Definizione di un sistema di Ranking per gli scacchi basato sulla statistica.
    L'idea si basa sul fatto che non è possibile la misura diretta, ogni risultato
    della partita dipende dalle persone. Quindi al termine di ogni partita bisogna 
    aggiornare la posizione in base al risultato della partita e chi ha giocato.

    Si esprime l'assunzione che la variabile che modella la mossa ha una distribuzione
    normale.

    Abbiamo due variabili gaussiane che rappresentano il valore del giocatore, Quando
    si conosce l'esito della partita, si cambia la distribuzione delle due variabili.

    $y$ è il risultato della partita
    $\pi$ sono le caratteristiche del giocatore
    $s$ è il livello del problema.

    L'inferenza sarà diagnostica (dall'esito all'inizio).
\end{esempio}

\begin{esempio}
    Supponiamo di chiedere un finanziamento di $10000$ per un auto con scadenza
    a $1$ anno. Il finanziamento può essere o al tasso fisso o al tasso variabile.
    I modelli probabilistici permettono di identificare il miglior tasso pur non
    sapendo come cambieranno.
\end{esempio}

In generale i sistemi che modellano anche l'incertezza dovrebbero funzionare meglio
rispetto a quelli che non la gestiscono. Per modellare l'incertezza, si utilizzerà
la teoria della probabilità.

\chapter{Macchina di Turing}
\section{Macchina di Turing}
La \textbf{macchina di Turing} consiste di un controllo finito che può trovarsi
in un stato, scelto in un insieme finito. Tale macchina è composta da un
\textbf{nastro}, potenzialmente infinito, diviso in \textbf{celle}, ognuna delle
quali può contenere un simbolo scelto in un insieme finito chiamato
\textbf{alfabeto}.

L'input è una stringa di lunghezza finita formata da simboli scelti
dall'\textit{alfabeto} di input ($\Sigma$), e viene inizialmente posto sul nastro.
In tutte le altre celle, che si estendono sia a destra che a sinistra senza limiti,
è presente il simbolo \textit{blank} ($\sqcup$), il quale specifica che in quella
posizione non è presente un simbolo dell'alfabeto. C'è però un eccezione, infatti,
all'inizio della sequenza di input è presente il simbolo $\triangleright$, il
quale indica la posizione dell'input sul nastro.
\begin{definizione}[\textbf{Macchina di Turing}]
    La \textbf{macchina di Turing} è definita come una quadrupla:
    \begin{equation}
        M = (Q, \Sigma, q_0, \delta)
    \end{equation}
    dove:
    \begin{itemize}
        \item $Q$ è un insieme \textit{finito} di stati.
        \item $\Sigma$ è un alfabeto \textit{finito} al quale sono aggiunti due
              caratteri di controllo:
              \begin{enumerate}
                  \item $\triangleright$: indica il punto di partenza della
                        sequenza di input.
                  \item $\sqcup$ (\textit{blank}): il quale è presente in tutte
                        le celle del nastro, escluse quelle contenenti l'input,
                        nell'istante di partenza.
              \end{enumerate}
        \item $q_0$ rappresenta lo stato iniziale.
        \item $\delta$ è la funzione di transizione definita come:
              \begin{equation}
                  \delta: Q \times \Sigma \longrightarrow Q \times \Sigma \times
                  \{\rightarrow, \leftarrow, -\}
              \end{equation}
              Tale funzione esprime il comportamento passo per passo della
              macchina di Turing. Essa prende in input uno stato e un simbolo,
              e restituisce come output una tripla composta da un nuovo stato,
              il simbolo scritto nella posizione indicata dalla testina e lo
              spostamento della testina.

              In base allo stato in cui mi trovo e al valore presente sotto la
              testina si applica la funzione di transazione.
    \end{itemize}
\end{definizione}
\begin{osservazione}
    Dato che l'alfabeto dei simboli è finito, la funzione di transizione è
    definibile, ovvero è un algoritmo.
\end{osservazione}
La macchina di Turing si arresta quando non ho più transazioni valide oppure
quando entra in uno stato accettante.

Per esprimere la computazione di una macchina di Turing usiamo una
\textbf{configurazione}, ovvero sulla base della definizione della macchina di
Turing e dello stato attuale devo definire tutti i passi.

La \textbf{configurazione} descrive in ogni istante lo stato della macchina,
viene rappresentata da una quadrupla definita come:
\begin{equation}
    (q, \ \text{simbolo sotto la testina}, \
    \text{stringa a sinistra della testina}, \
    \text{stringa a destra della testina})
\end{equation}
Descriviamo le \textbf{mosse} della macchina di Turing $M = (Q, \Sigma, q_0, \delta)$
mediante la notazione $\vdash$.
\begin{definizione}[\textbf{Computazione}]
    Una sequenza di configurazioni in cui si trova la macchina prende il nome di
    \textbf{computazione}.
\end{definizione}
\begin{teorema}[\textbf{Tesi di Church-Turing}]
    Se un problema è umanamente calcolabile, allora esisterà una macchina di
    Turing in grado di risolverlo.
\end{teorema}
È una tesi che non ha dimostrazione formale ma è stata dimostrata empiricamente
nel corso degli anni. Portando quindi a dire che il calcolo è ciò che può essere
eseguito con un Macchina di Turing. Quindi ciò che è computabile è computabile da
una macchina di Turing o da un suo equivalente.
\begin{esempio} [\textbf{Successore}]
    Si scriva la macchina di Turing che calcoli il successore di un numero binario,
    che sarà l'input (e si da per scontato che sia correttamente formattato avendo
    solo 0 o 1 come simboli). Si trascuri il riporto (nel senso che non aggiungo
    ulteriori bit).

    Definisco quindi la macchina di Turing come:
    \begin{itemize}
        \item $Q = \{ini, incr, uno, zero, H\}$
        \item $\Sigma = \{1, 0, \triangleright, \sqcup\}$
        \item La funzione di transizione come:
              \begin{equation}
                  \begin{array}{lcl}
                      (ini, 0 / 1)           & \to & (ini, 0 / 1, \to)          \\
                      (ini, \sqcup)          & \to & (incr, \sqcup, \gets)      \\
                      (incr, 0)              & \to & (H, 1, -)                  \\
                      (incr, 1)              & \to & (incr, 0, \gets)           \\
                      (incr, \triangleright) & \to & (uno, \triangleright, \to) \\
                      (uno, 0)               & \to & (zero, 1, \to)             \\
                      (zero, 0)              & \to & (zero, 0, \to)             \\
                      (zero, \sqcup)         & \to & (H, 0, -)                  \\
                  \end{array}
              \end{equation}
        \item $ini$ come lo stato iniziale.
    \end{itemize}
\end{esempio}
Ogni operazione sulla macchina di Turing ha lo stesso tempo e quindi posso usare
il numero di passi per calcolare il tempo di risoluzione. Il tempo di calcolo di
una macchina di Turing è definito come il numero di configurazioni che la macchina
$M$ attraversa quando riceve in input un valore $x$. Questo valore si indica come:
\begin{equation}
    t_M(x)
\end{equation}
Più in generale possiamo esprimere i tempi di calcolo come:
\begin{equation}
    T_M(n) = \max_{x \in \Sigma^{\ast}} \left\{t_M(x) \ | \ |x| = n \right\}
\end{equation}
ovvero sto cercando il tempo del caso peggiore tra tutti gli input della stessa
dimensione.
\subsection{Macchina di Turing multi-nastro}
Per semplificare le computazioni svolte con la macchina di Turing, è possibile
definire una macchina che utilizza più nastri. Tale macchina viene chiamata
\textbf{macchina multi-nastro} nella quale sono presenti $k$ nastri, ognuno dei
quali può essere utilizzato per operazioni di lettura e scrittura.

Questa macchina ha una testina per ogni nastro, la quale è associata  alla cella
su cui voglio eseguire le operazioni.
\begin{esempio} [\textbf{Decidere se una stringa è nella forma} $a^nb^nc^n$]
    Vogliamo decidere se la stringa in input è del tipo $a^nb^nc^n$. Per
    risolvere questo problema è possibile utilizzare una macchina di Turing
    multi-nastro.

    Ad esempio, utilizzando 3 tre nastri posso inserire la stringa di input su
    tutti e tre e posizionare la testina del primo nastro all'inizio della
    sequenza di $a$, quella sul secondo nastro all'inizio della sequenza di $b$
    e quella sul terzo all'inizio della sequenza di $c$.
\end{esempio}
Una macchina di Turing a k nastri \textbf{non} è più potente di una macchina di
Turing a singolo nastro. Esiste sempre una traduzione verso una macchina di
Turing a singolo nastro.
\begin{teorema}[\textbf{Equivalenza tra macchina mono nastro e macchina multi nastro}]
    Sia $M$ una macchina di Turing con $k$ nastri, allora esiste una macchina $M'$
    a nastro singolo equivalente, ovvero che mi permette di ottenere lo stesso
    risultato.
\end{teorema}
\begin{dimostrazione}
    Una semplice metodologia per passare da una macchina di Turing multi nastro
    a una a nastro singolo, consiste nel concatenare gli input presenti sui vari
    nastri in un unico nastro e separarli con l'ausilio di uno specifico
    carattere, solitamente $\#$. Oltre a ciò, posso introdurre una variante
    dei simboli dell'alfabeto in modo da poter indicare le posizioni delle $k$
    testine sul singolo nastro. Infine, è necessario aumentare il numero degli
    stati, passando da $|Q|$ a $|Q| \times |\Sigma|^k$ per rappresentare le
    varie configurazioni della macchina.
\end{dimostrazione}
\begin{teorema}[\textbf{Equivalenza tra macchina multi nastro e macchina mono
            nastro tempi di calcolo}]
    Se alla macchina multi-nastro $M$ è associata una funzione di complessità
    $T_M(n)$, allora alla macchina con un solo nastro $M'$ corrisponde una
    funzione di complessità $T_{M'}(n) = \mathcal{O}((T_M(n))^2)$.
\end{teorema}
\begin{dimostrazione}
    Se la lunghezza dei $k$ nastri è $\leq T_M(x)$ allora la lunghezza della
    macchina mono-nastro sarà $\leq k \cdot T_M(x)$. Per eseguire la scansione
    del nastro sarà richiesto un tempo pari a $\mathcal{O}(k \cdot T_M(x))$,
    dato che $T_M(x)$ è il tempo di esecuzione del programma, allora il tempo di
    calcolo della macchina mono-nastro sarà pari a:
    \begin{equation}
        \mathcal{O}(T_M(x)) \cdot T_M(x) = \mathcal{O}((T_M(x))^2)
    \end{equation}
\end{dimostrazione}
Data una macchina di Turing $M$ con alfabeto $\Sigma$, tale che $|\Sigma| > 4$,
esiste sempre una macchina di Turing $M'$ equivalente tale che $|\Sigma| = 4$,
ovvero che risolve lo stesso problema con l'utilizzo di un \textbf{alfabeto
    binario} a cui sono aggiunti i simboli $\sqcup$ e $\triangleright$.
\begin{equation}
    \forall x \in \Sigma ^ \ast \ M(x) = M'(f(x))
\end{equation}
Per convertire i simboli di un alfabeto $\Sigma$ in binario è possibile utilizzare
la seguente conversione:
\begin{equation}
    \sigma \in \Sigma \ \text{posso rappresentarlo con} \ \log_{2}|\Sigma | = k
\end{equation}
Il tempo di esecuzione richiesto per risolvere lo stesso problema utilizzando un
alfabeto binario è lo stesso della macchina con un alfabeto composto da più simboli.
\begin{equation}
    T_M(x) \ \to \ T_{M'}(f(x)) = \mathcal{O}(k \cdot T_M(x)) = T_{M'}(f(x)) =
    \mathcal{O}(T_M(x))
\end{equation}
\subsection{Macchina di Turing universale}
Esiste una macchina di Turing che mi permette di eseguire macchine di Turing.
Tale macchina è chiamata \textbf{macchina universale}, indicata con il simbolo
$U$. La macchina universale $U$ è tale che $\forall \alpha, x \in \Sigma^{\ast}$,
allora $U(\alpha, x) = M_{\alpha}(x)$ dove $M_{\alpha}$ è la macchina di Turing
rappresentata da $\alpha$.

Posso rappresentare $U$ come una macchina con tre nastri:
\begin{enumerate}
    \item Il primo nastro contiene la descrizione della macchina $M_{\alpha}$.
    \item Il secondo contiene l'input della macchina $M_{\alpha}$.
    \item Il terzo contiene le informazioni relative allo stato corrente di $M_{\alpha}$.
\end{enumerate}
Il tempo di calcolo di tale macchina dipende dalla dimensione del programma:
\begin{equation}
    t_U(\alpha, x) = \mathcal{O}(|\alpha| \cdot t_{M_{\alpha}}(x))
\end{equation}
\section{Linguaggi}
Le macchine di Turing calcolano funzioni del tipo:
\begin{equation}
    f: \Sigma^{\ast} \to \Sigma^{\ast}
\end{equation}
Oltre a questo, una macchina di Turing può \textit{decidere} un linguaggio
fornendo una risposta binaria ($\{0, 1\}$), ovvero data una stringa in input è
sempre in grado di dire se appartiene o meno al linguaggio.
\begin{definizione}[\textbf{Linguaggio ricorsivo}]
    Un linguaggio è \textbf{decidibile} o \textbf{ricorsivo}, se esiste almeno
    una macchina di Turing che \textbf{decide} il linguaggio, fermandosi sempre
    in $1$ o $0$.
    \begin{equation}
        M(x) = \begin{cases}
            1 & \text{se} \ x \in L \\
            0 & \text{altrimenti}
        \end{cases}
    \end{equation}
    Solitamente è un linguaggio finito, ma potrebbe anche essere infinito.
\end{definizione}
\begin{definizione}[\textbf{Linguaggio ricorsivamente enumerabile}]
    Un linguaggio è \textbf{semi-decidibile} o \textbf{ricorsivamente enumerabile},
    se esiste almeno una macchina di Turing che \textbf{accetta} il linguaggio:
    \begin{equation}
        M(x) = \begin{cases}
            1                & \text{se} \ x \in L     \\
            0 \ \lor \ \perp & \text{se} \ x \not\in L
        \end{cases}
    \end{equation}
    dove $\perp$ rappresenta il fatto che una macchina di Turing non termina.

    Quindi, un linguaggio ricorsivamente enumerabile è un linguaggio per cui data
    una stringa in input, la macchina di Turing si ferma, altrimenti la macchina
    di Turing potrebbe o fermarsi o andare avanti all'infinito nella computazione.
\end{definizione}
\begin{teorema}[\textbf{Complemento di un lingiaggio ricorsivo è ricorsivo}]
    Sia $L$ un linguaggio ricorsivo allora $\overline{L}$ è ricorsivo.
\end{teorema}
\begin{dimostrazione}
    Dato che $L$ è un linguaggio ricorsivo, allora esiste sempre una macchina che
    mi decide il linguaggio, ovvero che mi restituisce $1$ se $x \in L$ e $0$
    altrimenti. $\overline{L}$ è il linguaggio che contiene tutte le stringhe
    che non appartengono a $L$. Quindi, posso definire una macchina che mi
    decide $\overline{L}$ partendo da quella che mi decide $L$ e invertendo gli
    output.
    \begin{figure}[!ht]
        \centering
        \includegraphics[width=0.5\textwidth]{img/MacchineTuring/dimostrazione1.png}
        \caption{Rappresentazione grafica della dimostrazione che il complemento
            di un linguaggio ricorsivo è ricorsivo}
    \end{figure}
\end{dimostrazione}
\begin{teorema}[\textbf{Linguaggio ricorsivo è anche ricorsivamente enumerabile}]
    Preso un linguaggio $L$ costruito alfabeto $\Sigma$ si ha che se
    $L \subseteq \Sigma^{\ast}$ se $L$ è ricorsivo è anche ricorsivamente
    enumerabile.
\end{teorema}
\begin{dimostrazione}
    Se $L$ è ricorsivo esiste una macchina di Turing che riconosce se, data una
    stringa $x$, tale che $x \in L$, rispondendo $1$ e risponde $0$ se $x \not\in
        L$.

    Costruisco ora una macchina di Turing $M'$ che se il linguaggio non è
    ricorsivo allora vado in loop ($\perp$). Quindi, quando la macchina sta per
    andare in $0$ modifico $\delta$ per ottenere il loop, ottenendo una macchina
    che va in loop se $x \not\in L$, mentre restituisce $1$ se $x \in L$ e
    quindi $L$ è ricorsivamente enumerabile.
    \begin{figure}[!ht]
        \centering
        \includegraphics[width=0.5\textwidth]{img/MacchineTuring/dimostrazione2.png}
        \caption{Rappresentazione grafica della dimostrazione che un linguaggio
            ricorsivo è anche ricorsivamente enumerabile}
    \end{figure}
\end{dimostrazione}
\begin{teorema} \label{teo-rec-en-comp}
    $L$ è ricorsivo se e solo se $L$ è ricorsivamente enumerabile e il
    complementare di $L$, ossia $\overline{L}$, è ricorsivamente enumerabile.
\end{teorema}
\begin{dimostrazione}
    Rispetto alla dimostrazione precedente, che garantisce che se $L$ è ricorsivo
    allora è ricorsivamente enumerabile, devo definire, per il complementare,
    una macchina di Turing che va in loop se $x \in L$. Presa quindi una macchina
    di Turing $M$ che decide $L$, definisco $M'$ che accetta $L$, che quindi è
    ricorsivamente enumerabile. Definisco ora $M''$ che restituisce $1$ se
    $x \in \overline{L}$ ovvero $x \not\in L$ ed entra in loop se $x \not\in
        \overline{L}$, ovvero $x \in L$.

    Quindi $M$ decide $L$ eseguendo alternativamente $M'$ e $M''$ prestando
    attenzione alla gestione degli output.
\end{dimostrazione}
Vogliamo ora dimostrare che esistono dei problemi di decisione che non possono
essere risolti da una macchina di Turing. Per realizzare questa dimostrazione
partiamo dal fatto che $|A| < |\mathcal{P}(A)|$, ovvero che la cardinalità
dell'insieme $A$ è sempre minore della cardinalità dell'insieme delle parti di
$A$ ($\mathcal{P}(A)$).

Con le informazioni ottenute in precedenza abbiamo visto che possiamo
rappresentare una macchina di Turing partendo dall'alfabeto binario definito
come $B = \{0, 1\}$. Da questo, possiamo affermare che il numero di macchine di
Turing che possiamo realizzare sarà al più il numero di stringhe realizzabili
con l'utilizzo di $B$, ovvero $|B^{\ast}|$:
\begin{equation}
    |M| = |B^{\ast}|
\end{equation}
Inoltre, utilizzando l'alfabeto $B$, possiamo definire il linguaggio di
decisione come:
\begin{equation}
    L_D = \{x \in B^{\ast} \ | \ f_D(x) = 1\} \subseteq B^{\ast}
\end{equation}
quindi, possiamo affermare che il numero di problemi di decisione che si possono
avere è uguale a $|P(B^{\ast})|$. Partendo dal concetto matematico prima
presentato, possiamo affermare che:
\begin{equation}
    |M| = |B^{\ast}| < |P(B^{\ast})|
\end{equation}
ovvero che esistono problemi di decisione che \textit{non posso} risolvere con
una macchina di Turing.
\subsection{Halting Problem}
Un esempio di problema di decisione che non può essere risolto da una macchina
di Turing è l'\textbf{halting problem}. Questo problema consiste nel determinare
se esiste un algoritmo che dati in input una macchina di Turing $M$ e una stringa
$x$, mi dica se la macchina $M$ eseguita sull'input $x$ termina.
\begin{definizione}
    Posso definire formalmente l'\textbf{halting problem} con l'utilizzo del
    linguaggio $L_H$, definito come:
    \begin{equation}
        L_H = \{M \cdot \# \cdot x \ | \ M(x) \neq \perp \}
    \end{equation}
    ovvero come l'insieme di stringhe composte dalla descrizione di una macchina
    di Turing a cui è concatenato l'input tali per cui $M$ eseguita sull'input
    $x$ termina. Si utilizza il carattere $\#$ per separare la descrizione della
    macchina e l'input.
\end{definizione}
\begin{teorema}[\textbf{Halting problem non è decidibile}]
    Il linguaggio $L_H$ non è ricorsivo e quindi l'halting problem non è
    decidibile.
\end{teorema}
\begin{dimostrazione} [\textit{Per assurdo}]
    Assumiamo, per assurdo, che $L_H$ sia ricorsivo. Quindi esiste una macchina
    di Turing $M_H$ che prende in input $(M \cdot \# \cdot x)$ e mi fornisce in
    output:
    \begin{equation}
        M_H (M \cdot \# \cdot x) = \begin{cases}
            Y & \text{se} \ M(x) \ \neq \ \perp \\
            N & \text{se} \ M(x) \  = \ \perp
        \end{cases}
    \end{equation}
    Se esiste tale macchina, allora posso costruire una macchina di Turing $C$,
    costruita partendo da $M_H$ che prende in input $M$ e mi restituisce:
    \begin{equation}
        C(M) = \begin{cases}
            Y & \text{se} \ M_H(M, M) = N \\
            N & \text{se} \ M_H(M, M) = Y
        \end{cases}
    \end{equation}
    A questo punto se fornisco alla macchina $C$ se stessa come input ottenendo
    due possibili situazioni:
    \begin{itemize}
        \item $C(C) = Y$ allora $M_H(C, C) = N$ ma quindi mi aspetto che
              $C(C) = \perp$ il che mi porta ad un assurdo.
        \item $C(C) = \perp$ allora $M_H(C, C) =  Y$ ma quindi mi aspetto che
              $C(C) \neq \perp$ il che mi porta ad un assurdo.
    \end{itemize}
    Posso quindi affermare che non esiste una macchina $M_H$ che decide $L_H$.
    \begin{figure}[!ht]
        \centering
        \includegraphics[width=0.5\textwidth]{img/MacchineTuring/halt.png}
        \caption{Rappresentazione grafica della dimostrazione dell'Halting problem}
    \end{figure}
\end{dimostrazione}
\begin{teorema}[\textbf{Halting problem è ricorsivamente enumerabile}] \label{teo-lh-rec-en}
    Il linguaggio $L_H$ è ricorsivamente enumerabile, ovvero il problema è
    parzialmente decidibile.
\end{teorema}
\begin{dimostrazione}
    Esiste una macchina di Turing $M_A$ che accetta il linguaggio $L_H$, ovvero
    tale che $M_A(M, x) = Y$ se e solo se $M(x) \neq \ \perp$.

    In precedenza abbiamo definito la macchina di Turing universale, la quale
    $U(M, x) = M(x)$, allora:
    \begin{equation}
        M_A(M, x) = \begin{cases}
            Y     & \text{se e solo se} \ U(M, x) = Y \ \lor \ N \\
            \perp & \text{se} \ U(M, x) = \perp
        \end{cases}
    \end{equation}
    Esiste quindi una macchina che accetta il linguaggio $L_H$.
\end{dimostrazione}
\begin{teorema}
    $\overline{L_H}$ non è ricorsivamente enumerabile.
\end{teorema}
\begin{dimostrazione} [\textit{Per assurdo}]
    Assumiamo per assurdo che $\overline{L_H}$ sia ricorsivamente enumerabile.
    Dal teorema \ref{teo-lh-rec-en} sappiamo che $L_H$ è ricorsivamente
    enumerabile, allora per quanto definito nel teorema \ref{teo-rec-en-comp} si
    ha che $L_H$ è ricorsivo, il che è assurdo in quanto è stato dimostrato che
    $L_H$ non è ricorsivo.
\end{dimostrazione}
\section{Problemi decidibili}
Vogliamo ora studiare i problemi decidibili classificandoli sulla base dei tempi
di esecuzione, considerando la dimensione dell'input e il caso peggiore. Per fare
questo utilizzeremo il tempo di esecuzione di una macchina di Turing. Questo
valore viene calcolato tramite il numero di passi che la macchina compie.
\begin{definizione}[\textbf{Tempo di calcolo}]
    Definiamo $t_M(x)$ come il tempo di calcolo di una macchina di Turing $M$ su
    input $x$. Non è il caso peggiore, ma dipendente dal singolo input specifico.
    Il tempo di calcolo è il numero di passi che esegue $M$ su input $x$ per
    dare una risposta.
\end{definizione}
Non si usa comunque il numero di passi nella realtà ma si usa la notazione
$\mathcal{O}$-grande, studiando il caso peggiore, ovvero il numero massimo di
passi.
\begin{definizione}[\textbf{Complessità temporale}]
    Definiamo $T_M(n)$ come la \textbf{funzione di complessità temporale} come:
    \begin{equation}
        T_M(n) = \max \{t_M(x) \ | \ |x| = n\}
    \end{equation}
\end{definizione}
\begin{definizione}[\textbf{Classe P}]
    Definisco la classe $P$ in base alle macchine di Turing. La classe $P$ è
    definita come:
    \begin{equation}
        P = \{L \ | \ L \ \text{ è deciso da una macchina di Turing in tempo }
        \mathcal{O}(p(n))\}
    \end{equation}
\end{definizione}
\begin{definizione}[\textbf{Classe DTIME}]
    Definiamo la classe $DTIME(f(n))$ come la classe dei linguaggi decisi da una
    macchina di Turing entro un tempo $f(n)$:
    \begin{equation}
        DTIME(f(n)) = \{L \subseteq \Sigma^{\ast} \ | \ \exists M \
        \text{decide} \ L \ \text{in tempo} \ \mathcal{O}(f(n)) \}
    \end{equation}
    Quindi $DTIME(n)$ rappresenta l'insieme dei problemi di decisione che possono
    essere risolti con un algoritmo che lavora in tempo $\mathcal{O}(n)$. Quindi:
    \begin{equation}
        P = \bigcup_{c \ \in \ \mathbb{N}} DTIME(n^c)
    \end{equation}
    Infatti $P$ è l'unione di tutte le classi DTIME con funzioni polinomiali.
\end{definizione}
Possiamo dire che vale la seguente relazione tra le classi $DTIME$:
\begin{equation}
    DTIME(n) \subseteq DTIME(n^2) \subseteq DTIME(2^{n^c})
\end{equation}
Possiamo anche definire la classe \textbf{EXPTIME} come la classe di linguaggi
decidibili da una macchina di Turing in tempo esponenziale:
\begin{equation}
    EXPTIME =  \bigcup_{c \ \in \ \mathbb{N}} DTIME(2^{n^c})
\end{equation}
Inoltre, si ha che vale la seguente relazione tra le classi $P$ e $EXPTIME$:
\begin{equation}
    P \subseteq EXPTIME
\end{equation}
\begin{teorema}
    Se un problema è nella classe $P$ allora è risolvibile in un tempo efficiente.
\end{teorema}
Devo anche definire la complessità spaziale oltre a quella temporale.
\begin{definizione}[\textbf{Spazio}]
    Definisco lo \textbf{spazio} di calcolo $s_M(x)$ come il numero di celle del
    nastro usate dalla macchina di Turing $M$ con input $x$ durante la
    computazione.

    Il calcolo non è semplice come per il tempo, avendo anche decrementi. Quindi
    più che “celle usate” studiamo le “celle visitate”.
\end{definizione}
\begin{definizione}[\textbf{Complessità spaziale}]
    Definisco $S_M(n)$ come la \textbf{funzione di complessità spaziale}:
    \begin{equation}
        S_M(n) =  \max\{s_M(x) \ | \ |x| = n\}
    \end{equation}
\end{definizione}
Esiste una relazione tra la complessità spaziale e quella temporale per una
macchina di Turing. Se una computazione dura $n$ passi di tempo, allora posso
dire che al più ho usato $n$ celle di spazio, questo perché è possibile che in
alcune configurazioni la testina non si muove, ma nel caso peggiore si sposta
sempre. Si ha quindi:
\begin{equation}
    S_M(n) \leq T_M(n) + n
\end{equation}
con $+ \ n$ perché sul nastro abbiamo comunque l'input di lunghezza $n$.
\begin{teorema}
    Se il tempo è limitato allora lo spazio è limitato ma non vale l'opposto.
\end{teorema}
\begin{teorema}
    Se ho una macchina di Turing $M$ che lavora in spazio finito e tempo infinito,
    esiste una macchina di Turing $M'$ che fa la stessa cosa di $M$ in tempo
    limitato. Quindi se lo spazio è limitato allora il tempo è limitato.
\end{teorema}
\begin{dimostrazione}
    Infatti la macchina $M'$ può trovarsi in un numero finito di stati $K$ e
    avendo spazio limitato ho un numero limitato $S_M(n)$ di celle in cui si trova
    la testina. Ho anche un numero finito di simboli in alfabeto $\Sigma$ e quindi:
    \begin{equation}
        S_{M'}(n) \leq |k| \cdot |S_M(n)| \cdot |\Sigma|^{|S_M(n)|}
    \end{equation}
    avendo che prima o poi ritorno a stati già visti quindi la macchina se supera
    la quantità appena definita capisce di essere in loop. Quindi
    $|k| \cdot |S_M(n)| \cdot |\Sigma|^{|S_M(n)|}$ è anche un limite temporale
    per la seconda macchina.
\end{dimostrazione}
Quindi data una certa macchina che lavora in un certo spazio $S(n)$ posso costruire
una macchina equivalente che da la stessa risposta in tempo limitato $T(n)$. Si ha
che se ho un problema che si risolve in spazio polinomiale, per la formula appena
scritta avrò tempo esponenziale. Invece, al contrario, tempo polinomiale comporta
spazio polinomiale.
\section{Macchine di Turing non deterministiche}
Una \textbf{macchina di Turing non deterministica} si distingue da quella
deterministica nella funzione di transizione $\delta$. Nella macchina non
deterministica la funzione di transizione associa a ogni stato $q$ e a ogni
simbolo di nastro $X$ un \textit{insieme} di triple:
\begin{equation}
    \delta(q, X) = \{(q_1, Y_1, D_1), (q_2, Y_2, D_2), \dots, (q_k, Y_k, D_k)\}
\end{equation}
A ogni passo una macchina di Turing non deterministica sceglie una delle triple
come mossa, ovviamente lo stato, il simbolo di nastro e la direzione appartengono
alla stessa tripla. Nelle macchine non deterministiche la computazione non è una
sequenza di configurazioni ma un albero di computazione. Da ogni stato posso
passare a uno tra più stati, a seconda della scelta, formando così un albero. Il
singolo passo di computazione non è univocamente definito. Ogni singolo ramo
comunque è equivalente al passo di computazione della macchina di Turing
deterministica.
\begin{definizione}[\textbf{Linguaggio accettato}]
    Un linguaggio $L$ è \textbf{accettato} da una macchina di Turing non
    deterministica $N$ se per tutte le stringhe che fanno parte del linguaggio
    esiste almeno una computazione che termina nello stato $Y$, ovvero esiste
    una computazione per cui:
    \begin{equation}
        \forall x \in L \Rightarrow N(x) = Y
    \end{equation}
    Nel caso in cui nessuna delle computazioni termina nello stato $Y$, allora
    l'input non è accettato.
\end{definizione}
\begin{definizione}[\textbf{Linguaggio deciso}]
    Un linguaggio $L$ è \textbf{deciso} da una macchina di Turing non
    deterministica $N$ se, qualora la stringa $x$ appartenga il linguaggio,
    esiste almeno una computazione tale per cui:
    \begin{equation}
        x \in L \to N(x) = Y
    \end{equation}
    altrimenti, se $x$ non appartiene al linguaggio (\textbf{rifiuta}), per
    tutte le computazioni, si ha che:
    \begin{equation}
        x \not\in L \to N(x) = N
    \end{equation}
    Non devo quindi avere loop in questo caso, tutte devono dare $N$.
\end{definizione}
\begin{teorema}[\textbf{Equivalenza tra macchine di Turing deterministica e non
            deterministica}]
    Se $M_N$ è una macchina di Turing non deterministica, esiste una macchina di
    Turing deterministica $M_D$ tale che accettano gli stessi linguaggi $L(M_N)
        = L(M_D)$.
\end{teorema}
\begin{dimostrazione}
    La macchina di Turing deterministica $M_D$ può simulare la macchina di Turing
    non deterministica $M_N$ procedendo con una visita in ampiezza dell'albero
    di computazione. Viene eseguita una visita in ampiezza in modo da evitare di
    incappare in loop.
\end{dimostrazione}
La macchina di Turing non deterministica $N$ decide il linguaggio $L$ in tempo
$t_N(n)$, ovvero il tempo di calcolo è dato dall'altezza del ramo più lungo.
Come per le macchine deterministiche, definiamo $T_N(n)$ come:
\begin{equation}
    T_N(n) = \max \{t_N(x) \ | \ |x| = n\}
\end{equation}
Data una macchina di Turing non deterministica $N$ che decide $L$ in tempo
$T_N(n)$ esiste una macchina di Turing deterministica $M$ che decide $L$ in tempo
$2^{\mathcal{O}(T_N(n))}$. Questa complessità deriva dal fatto che se ogni nodo
dell'albero ha al massimo $b$ figli, allora l'albero di computazione ha al
massimo $b^{T_N(n)}$ foglie. Il numero interno di nodi è al massimo $b^{T_N(n)}
    - 1$ e quindi il numero totale di nodi è $< 2 \cdot b^{T_N(n)}$. Inoltre, un
cammino dalla radice alla foglia è $\mathcal{O}(T_N(n))$. Il tempo di esecuzione
della macchina $M$ è:
\begin{equation}
    \mathcal{O}(T(n) \cdot b^{T(n)}) = 2^{\mathcal{O}(T_N(n))}
\end{equation}
\begin{definizione}[\textbf{Classe NTIME}]
    Definiamo una funzione di tempo per una macchina di Turing non deterministica
    come:
    \begin{equation}
        NTIME(f(n)) = \{L \ | \ L \ \text{è deciso da una macchina di Turing non
            deterministica in tempo } \mathcal{O}(f(n))\}
    \end{equation}
\end{definizione}
Quest'ultima definizione ci permette di definire la classe di \textbf{problemi NP}
come l'insieme dei linguaggi $L$ che sono decisi in tempo polinomiale da una
macchina di Turing non deterministica:
\begin{equation}
    NP = \bigcup_{c \in \mathbb{N}} NTIME(n^c)
\end{equation}
\begin{osservazione}
    Sia $P$ l'insieme dei \textbf{problemi decidibili in tempo polinomiale} da
    una macchina di Turing deterministica, e $NP$ l'insieme dei \textbf{problemi
        decidibili in tempo polinomiale} da una macchina di Turing non
    deterministica. Sappiamo che è vero che $P \subseteq NP$ ma non è ancora
    dimostrato che $NP \subseteq P$.
\end{osservazione}
Possiamo però affermare che:
\begin{equation}
    P \subseteq NP \subseteq EXPTIME
\end{equation}
\subsection{Verificatori}
La classe $NP$ rappresenta una classe di linguaggi che sono verificabili in
tempo polinomiale da una macchina di Turing non deterministica. Verificare un
linguaggio significa che per ogni input $x \in L$ si ha una stringa $c$ detta
\textbf{certificato} che possiamo usare per verificare che effettivamente $x \in
    L$.
\begin{definizione}[\textbf{Verificatore}]
    Un \textbf{verificatore} di un linguaggio $L$ è una macchina di Turing
    deterministica $V$ tale per cui:
    \begin{equation}
        L = \{x | V \ \text{accetta} \ \langle x, c \rangle \
        \text{per qualunque stringa} \ c\}
    \end{equation}
    Questa macchina accetta le stringhe appartenenti al linguaggio eseguendo le
    istruzioni specificate dal certificato $c$.
\end{definizione}
Se $V$ richiede tempo polinomiale rispetto $x$ a per accettare/rifiutare, allora
$V$ è un verificatore in tempo polinomiale per $x$. Se esiste un verificatore in
tempo polinomiale per $L$, allora $L$ è verificabile in tempo polinomiale.

Dato che $V$ deve eseguire in tempo polinomiale rispetto a $x$, ne segue che $c$,
il certificato, deve essere di lunghezza polinomiale rispetto a $x$, altrimenti
non avremmo il tempo di leggere tutto $c$.

Supponiamo di avere una macchina di Turing non deterministica $M$ che lavora in
tempo polinomiale, costruiamo un verificatore $V$ che lavora in tempo polinomiale
per lo stesso linguaggio deciso da $M$. Se su input $x$ la macchina $M$ accetta,
significa che esiste una computazione accettante $C_1, \dots, C_k$ di lunghezza
polinomiale rispetto a $x$. Per ogni passaggio da $C_i$ a $C_{i + 1}$ è stata
applicata una transizione definita dalla funzione di transizione di $M$.

Usiamo come certificato $c$ la sequenza di transizioni applicate lungo tutta la
computazione, le quali sono in numero polinomiale. Queste transizioni ci
identificano una specifica computazione della macchina di Turing non
deterministica $M$. Il verificatore deve solo simulare quella computazione, senza
fare scelte non deterministiche, dato che le transizioni da fare sono tutte in
$c$ e verificare che la computazione accetti.

Abbiamo mostrato che per ogni linguaggio accettato da una macchina di Turing non
deterministica che lavora in tempo polinomiale possiamo costruire un verificatore
polinomiale per lo stesso linguaggio.

Dobbiamo mostrare ora l'inclusione in senso inverso. Sfruttiamo il fatto che
possiamo usare una macchina di Turing non deterministica per scrivere un
certificato sul nastro e, se esiste un certificato che ci permette di accettare,
questo sarà presente in una computazione.

Sia $V$ un verificatore in tempo polinomiale per $L$. Assumiamo $V$ esegua in
tempo $n^k$. Costruiamo una macchina di Turing non deterministica che su input
$x$ esegue due compiti:
\begin{itemize}
    \item Genera in modo non deterministico un certificato $c$ di lunghezza al
          più $n^k$.
    \item Simula $V$ su input $\langle x, c \rangle$ e accetta se $V$ accetta,
          mentre rifiuta se $V$ rifiuta.
\end{itemize}
Abbiamo mostrato che per ogni linguaggio per il quale esiste un verificatore
polinomiale possiamo costruire una macchina di Turing non deterministica che
decide lo stesso linguaggio in tempo polinomiale. Quindi le due definizioni di
$NP$ sono equivalenti.
\begin{esempio}
    Consideriamo il problema del cammino Hamiltonian. Tale problema consiste nel
    dato un grafo orientato $G = (V, E)$ trovare trovare un cammino che visita
    tutti i nodi una sola volta. Consideriamo una variante in cui conosciamo il
    nodo di partenza $s$ e il nodo di arrivo $t$.
    \begin{equation}
        HAMPATH = \{\langle G, s, t \rangle \ | \ G \ \text{ha un cammino
            Hamiltonian da} \ s \ \text{a} \ t\}
    \end{equation}
    Per questo problema si può ottenere una semplice soluzione che lo risolve in
    tempo esponenziale, ovvero provare tutti i possibili cammini. Questa
    soluzione è però inefficiente.

    Questo problema ha una caratteristica chiamata \textit{verificabilità
        polinomiale}. Se in qualche modo si riesce a trovare un cammino
    Hamiltonian, allora possiamo verificare l'esistenza di tale cammino in tempo
    polinomiale.

    Per il problema del cammino Hamiltonian il certificato è rappresentato da
    un percorso Hamiltonian da $s$ a $t$. Il verificatore $V$ prende in input
    $\langle G, s, t, c \rangle$ e controlla che $c$ sia un cammino Hamiltonian
    da $s$ a $t$ in $G$. Per fare questo, il verificatore controlla che il primo
    nodo di $c$ sia $s$, che l'ultimo nodo sia $t$ e che ogni nodo di $c$ sia
    connesso al successivo.
    Da questo segue che il verificatore $V$ lavora in tempo polinomiale rispetto
    alla lunghezza del certificato $c$.
\end{esempio}
\subsection{Classi di problemi}
Definiamo ora altre classi di problemi:
\begin{itemize}
    \item \textbf{coP}: ovvero la classe di linguaggi di cui posso decidere la
          \textbf{non appartenenza} in tempo polinomiale:
          \begin{equation}
              coP = \{L \ | \ \overline{L} \in P\}
          \end{equation}
    \item \textbf{$\overline{P}$}: ovvero la classe di linguaggi che \textbf{non
              sono decidibili} in tempo polinomiale:
          \begin{equation}
              \overline{P} = \{L \ | \ L \not\in P\}
          \end{equation}
    \item \textbf{coNP}: ovvero la classe di linguaggi di cui posso decidere la
          \textbf{non appartenenza} in tempo polinomiale da una macchina di
          Turing non deterministica:
          \begin{equation}
              coNP = \{L \ | \ \overline{L} \in NP\}
          \end{equation}
\end{itemize}
\begin{nota}
    È importante notare che:
    \begin{equation}
        \overline{P} \neq coP
    \end{equation}
    \begin{equation}
        P \subseteq NP \ \land \ P \subseteq coNP \ \text{quindi} \ P
        \subseteq NP \cap coNP
    \end{equation}
\end{nota}
\begin{teorema}
    \begin{equation}
        P = coP
    \end{equation}
\end{teorema}
\begin{dimostrazione}
    Se un linguaggio $L \in P$, allora esiste una macchina di Turing
    deterministica che decide $L$ in tempo polinomiale:
    \begin{equation}
        \forall x \in \Sigma^{\ast} = \begin{cases}
            x \in L \Rightarrow M(x) = Y \\
            x \not\in L \Rightarrow M(x) = N
        \end{cases}
    \end{equation}
    Posso inoltre creare una macchina $M'$ che decide $\overline{L}$ in tempo
    polinomiale:
    \begin{equation}
        \forall x \in \Sigma^{\ast} = \begin{cases}
            x \in \overline{L} \Rightarrow M'(x) = Y \\
            x \not\in \overline{L} \Rightarrow M'(x) = N
        \end{cases}
    \end{equation}
    per ottenere questa macchina è sufficiente modificare gli stati di
    accettazione e rifiuto della macchina $M$.
\end{dimostrazione}
La dimostrazione precedente non può essere utilizzata per dimostrare che $NP =
    coNP$ dato che, nel caso di macchine non deterministiche è sufficiente che
esista un singolo ramo di computazione che termina in uno stato di accettazione.
Invertendo l'output di questa macchina si ottiene una macchina che non decide
$\bar{L}$ dato che sarebbero necessaire tutte computazioni che terminano in uno
stato accettante.
\section{Riduzioni polinomiali}
Le \textbf{riduzioni polinomiali} tra problemi sono delle procedure che per ogni
istanza del problema $A$ la trasformano in un'istanza per un problema diverso $B$.
Quindi un'istanza di $A$, $I_A$, ha due risposte, $Y$ o $N$, ma posso passare
tramite una determinata funzione $f$, definita come:
\begin{equation}
    f: I_A \to I_B
\end{equation}
a un istanza $I_B$ tale che $B$ avrà risposte, uguali a quelle di $A$, $Y$ o $N$.
\begin{equation}
    \forall x \in I_A \ \begin{cases}
        A(x) = Y \Rightarrow B(f(x)) = Y \\
        A(x) = N \Rightarrow B(f(x)) = N
    \end{cases}
\end{equation}
In altre parole, si cerca una funzione $f$ che mi permette di convertire le
istanze del mio problema di partenza in istanze di un problema che so risolvere.
\begin{definizione}[\textbf{Riduzione polinomiale}]
    La \textbf{riducibilità polinomiale} tra problemi è definita come:
    \begin{equation}
        f: \Sigma^{\ast} \to \Sigma^{\ast}
    \end{equation}
    la quale deve essere calcolabile in tempo polinomiale da una macchina di
    Turing deterministica, tale che:
    \begin{equation}
        \forall x \in \Sigma^{\ast}, \ x \in L_A \ \text{se e solo se} \
        f(x) \in L_B
    \end{equation}
    Indichiamo l'operazione di riduzione polinomiale come:
    \begin{equation}
        L_A \leq_P L_B
    \end{equation}
\end{definizione}
\begin{osservazione}
    Si utilizza il simbolo minore uguale ($\leq$), per indicare una relazione
    d'ordine nella complessità dei problemi.
\end{osservazione}
\begin{teorema}
    Dato un linguaggio $L_A$ riducibile polinomialmente a $L_B$ ($L_A \leq_P L_B$),
    si ha che se $L_B \in P$ allora sicuramente anche $L_A \in P$.
\end{teorema}
\begin{dimostrazione}
    Infatti e esiste un algoritmo polinomiale per $L_B$ allora, avendo una
    trasformazione polinomiale ho che $f(x) + L_B$ è ancora polinomiale.
\end{dimostrazione}
\begin{teorema}[\textbf{Transitività}]
    La riduzione polinomiale gode della proprietà di transitività, ovvero:
    \begin{equation}
        L_A \ \leq_P \ L_B \ \land \ L_B \ \leq_P \ L_C \ \text{allora} \ L_A
        \ \leq_P \ L_C
    \end{equation}
    Posso ottenere questa riduzione applicando la composizione di funzioni.
\end{teorema}
\begin{definizione}[\textbf{NP-hard}]
    Un linguaggio $L$ è \textbf{NP}$\_$\textbf{hard} se vale che:
    \begin{equation}
        \forall L' \in NP \ \text{si ha} \ L' \leq_P L
    \end{equation}
\end{definizione}
\begin{definizione}[\textbf{NP-completo}]
    Un linguaggio $L$ si dice \textbf{NP}$\_$\textbf{completo} se valgono i
    seguenti punti:
    \begin{itemize}
        \item $L \in NP$
        \item $L \in NP\_hard$
    \end{itemize}
\end{definizione}
\begin{teorema}
    Se $L_A \leq_P L_B$ e $L_A \in NP\_hard$ allora so che anche $L_B \in NP\_hard$.
\end{teorema}
\begin{dimostrazione}
    Se $L_A \leq_P L_B$ per la definizione di $NP\_hard$ so che esistono n
    linguaggi che sono riducibili polinomialmente a $L_A$. Utilizzando la
    proprietà transitiva posso affermare che $L_B$ è $NP\_hard$.
\end{dimostrazione}
\begin{teorema}[\textbf{Teorema Cook-Levin}]
    Il problema di soddisfacibilità SAT è \textbf{NP}$\_$\textbf{completo}.
    Questo problema prende in input una formula booleana $\phi$ in forma normale
    congiunta (CNF), ovvero che ha una congiunzione ($\land$) come legame tra le
    clausole. Una clausola è un $\lor$ di letterali, ovvero di variabili booleane
    $x_i$ o $\overline{x_i}$. In output ho se la forma sia soddisfacibile o meno.
\end{teorema}
\begin{figure}[!ht]
    \centering
    \includegraphics[width=0.5\textwidth]{img/MacchineTuring/classificazioneProblemi.png}
    \caption{Classificazione dei problemi.}
\end{figure}
% https://en.wikipedia.org/wiki/Succinct_data_structure
\chapter{Strutture dati succinte}
Le \textbf{strutture dati succinte} sono una classe di strutture dati che forniscono
un compromesso tra l'efficienza dell'accesso ai dati e la quantità di spazio
utilizzato per memorizzare i dati. Queste strutture cercano di minimizzare l'uso
dello spazio di memoria, mantenendo nel contempo un accesso efficiente ai dati.

Le strutture dati succinte sono un tipo di struttura dati che utilizza una quantità
di spazio “vicina” al limite inferiore teorico dell'informazione, ma che, a
differenza di altre rappresentazioni compresse, consente ancora operazioni di
interrogazione efficienti.

Supponiamo che $\mathcal{Z}$ sia il numero ottimale teorico di bit necessari per
memorizzare alcuni dati. Una rappresentazione di questi dati viene chiamata:
\begin{itemize}
    \item \textbf{Implicita} se richiede $\mathcal{Z} + \mathcal{O}(1)$ bit di
          spazio. (es. $\mathcal{Z} + 14$ bit)
    \item \textbf{Succinta} se richiede $\mathcal{Z} + o(\mathcal{Z})$ bit di spazio.
          (es. $\mathcal{Z} + \log \mathcal{Z}$ oppure $\mathcal{Z} + \sqrt{
                  \mathcal{Z}}$ bit)
    \item \textbf{Compatta} se richiede $\mathcal{O}(\mathcal{Z})$ bit di spazio.
          (es. $5 \cdot \mathcal{Z}$ bit)
\end{itemize}
\begin{nota}
    $o(\mathcal{Z})$ si riferisce al concetto matematico di \textit{o-piccolo},
    ovvero:
    \begin{equation}
        \lim_{x \to x_0} \frac{f(x)}{g(x)} = 0 \Rightarrow f(x) = o_{x_0} (g(x))
        \ \text{con} \ x_0 = + \infty
    \end{equation}
\end{nota}
\section{Bitvector}
Un modo di rappresentare le strutture dati succinte è attraverso l'utilizzo di
\textbf{bitvector}.
\begin{definizione}[\textbf{Bitvector}]
    Si definisce un \textbf{bitvector} $B$ come un array di lunghezza $n$, popolato
    da elementi binari ($\{0, 1\}$). Formalmente si ha:
    \begin{equation}
        B[i] \in \{0, 1\}, \ \forall i \ \text{tale che} \ 1 \leq i \leq n
    \end{equation}
    In alternativa all'alfabeto binario è possibile utilizzare i valori booleani
    vero e falso ($\{\top, \bot\}$).
\end{definizione}
Su questa struttura è possibile effettuare due operazioni:
\begin{itemize}
    \item \textbf{rank}: restituisce il numero di elementi uguali a $q$ che sono
          presenti nella struttura dati fino a $x$.
          \begin{equation}
              rank_q(x) = \sum_{k = 1}^{k \leq i} B[k], \ \forall i \
              \text{tale che} \ 1 \leq i \leq n \ \land \ B[k] = q
          \end{equation}
    \item \textbf{select}: restituisce la posizione della $x$-esima occorrenza
          di $q$.
          \begin{equation}
              select_q(x) = \min{\{k \in [0, \dots, n): \ rank_q(k) = x\}}
          \end{equation}
\end{itemize}
\begin{nota}
    \begin{equation}
        rank_q(select_q(i)) = i
    \end{equation}
\end{nota}
È possibile ottenere una struttura dati succinta, usando $o(n)$ bit aggiuntivi,
che permetta di effettuare le operazioni di \textit{rank} e \textit{select} in
tempo costante $\mathcal{O}(1)$.
\subsection{Funzione rank}
Vediamo ora come è possibile rendere il tempo di esecuzione dell'operazione di
rank costante ($\mathcal{O}(1)$).

La soluzione più semplice, ovvero memorizzare tutti i valori di $rank(i)$
necessiterebbe di $\mathcal{O}(n \log n)$ bit il che non lo rende un o-piccolo
di $n$. Una soluzione alternativa consiste nel memorizzare ogni $l-$esimo valore
$rank(i)$, e a questo punto, quando si esegue una query scorriamo i restanti $l
    - 1$ bit.

Questi valori vengono salvati in un vettore \textit{first} $F[0 \dots n / l]$,
dove l'operatore $/$ indica la divisione intera, tale che:
\begin{itemize}
    \item $F[0] = 0$
    \item $F[i / l] = rank(i)$ se $i \mod l = 0$
\end{itemize}
Se $l = \left(\left\lceil \frac{\log n}{2} \right\rceil \right)^2$ si ha un ordine
di $\frac{n}{\log n}$ bit per l'array $F$. Con questa prima operazione è possibile
eseguire una rank come:
\begin{equation}
    rank(i) = F[i/l] + C(B[l \cdot (i / l) + 1 \dots i])
\end{equation}
con $C(B')$ che rappresenta una funzione che conta i simboli $\sigma = 1$ in $B'$.
Questa soluzione mi porta a una rank eseguita in tempo $\mathcal{O}(\log^2 n)$.

Aumentando la quantità di informazioni salvate in memoria, possiamo ridurre
ulteriormente il tempo di esecuzione della funzione di rank.

Nello specifico, in ogni blocco di lunghezza $l$, indotto da $F$, memorizziamo
ogni $k$ posizioni il numero di simboli $\sigma = 1$ a partire dall'inizio del
blocco, escludendo la posizione iniziale in cui il valore di rank è già in $F$.
Otteniamo un vettore \textit{second} $S[0 \dots l /k]$ dove:
\begin{itemize}
    \item $S[i / k] = 0$ se $k \mod l = 0$
    \item $S[i / k] = rank_{B[l \cdot (i / l) + 1 \dots i]} (i - l \cdot (i / l)
              + 1)$ se $i \mod k = 0$
\end{itemize}
Questa soluzione richiede $\mathcal{O}\left(\left(\frac{n}{k}\right) \log l\right)$
bit aggiuntivi. In particolare, scegliendo $k = \left\lceil \frac{\log n}{2}
    \right\rceil$ si ottiene uno spazio di $\mathcal{O}(\frac{n \log \log n}{\log n})$
bit. Introducendo questo secondo vettore è possibile eseguire una rank come:
\begin{equation}
    rank(i) = F[i / l] + S[i / k] + C(B[k \cdot (i / k) + 1 \dots i])
\end{equation}
Questa soluzione mi porta a una rank eseguita in tempo $\mathcal{O}(\log n)$.

Per ottenere una rank in tempo costante si utilizza la tecnica \textbf{Four
    Russians technique}. Questa tecnica consiste nel salvare una look-up table
third $T$, di dimensioni $2^{k - 1} \times k - 1$, i valori di $rank(i')$
per tutte le posizioni $k - 1$, in tutte le possibili configurazioni indotte dai
blocchi definiti per $S$. Così facendo la lookup-table $T$ richiede uno spazio
per essere memorizzata pari a $\mathcal{O}(\sqrt{n} \log n \log \log n)$ bit.

Si definisce:
\begin{equation}
    c_i = B \left[k \cdot \left(\frac{i}{k}\right) + 1 \dots k \cdot \left(\frac{i}{k} 
    + 1\right) - 1\right]
\end{equation}
come il bitvector di lunghezza $k - 1$ che copre il $(k + 1)-$esimo blocco. Questo
valore ci permette di identificare la riga della lookup table $T$ da cui prendere
il valore di $rank(i)$. Mentre la colonna si ottiene da $i \mod k$. In questo
modo è possibile eseguire una rank in tempo costante $\mathcal{O}(1)$.
\begin{equation}
    rank(B) = \begin{cases}
        F[i/l]                                   & \text{se} \ i \mod l = 0    \\
        F[i/l] + S[i/k]                          & \text{se} \ i \mod l \neq 0
        \land i \mod k = 0                                                     \\
        F[i/l] + S[i/k] + T[c_i][(i \mod k) - 1] & \text{se} \ i \mod l \neq 0
        \land i \mod k \neq 0                                                  \\
    \end{cases}
\end{equation}
Così facendo, memorizzando un $o(n)$ bit in aggiunta alla struttura, posso
eseguire l'operazione di rank in tempo costante.
\begin{nota}
    Le operazioni per arrivare alla costruzione delle struttura dati che mi
    permette di eseguire la rank in tempo costante sono eseguite in tempo lineare
    ($\mathcal{O}(n)$).
\end{nota}
\begin{esempio}
    Vediamo un esempio di come può essere implementata la funzione rank in modo
    da effettuare accesso in tempo costante. Consideriamo il bitvector
    $B = 100101010010$ e scegliamo i valori di $l = 9$ e $k = 3$. Per eseguire
    la funzione rank in tempo costante si ottiene la struttura riportata in
    figura \ref{fig:rank}.
    \begin{figure}[!ht]
        \centering
        \includegraphics[width=0.5\textwidth]{img/Strutture Dati/rank.png}
        \caption{Esempio di implementazione della funzione rank}
        \label{fig:rank}
    \end{figure}
\end{esempio}
\begin{osservazione}
    Si può dimostrare che, con un procedimento abbastanza analogo a quello visto
    per la funzione di rank, è possibile costruire una struttura che necessita di
    $o(n)$ bit aggiuntivi e che permette di effettuare l'operazione di select in
    tempo costante.
\end{osservazione}
\section{Level Order Unary Degree Sequence}
Consideriamo in memoria la rappresentazione di un albero etichettato con $n$ nodi.
La classica rappresentazione attraverso puntatori richiede uno spazio pari a
$\mathcal{O}(n \log n)$.

Analizziamo ora una rappresentazione basata sulla visita in left-to-right
level-order di un albero, ovvero una visita in ampiezza. Utilizzando questa
visita, si salva per ogni nodo il suo grado e si memorizza la sequenza dei gradi
$D$ attraverso un prefix code binario. Quindi, per ogni nodo si aggiunge a un
bitvector $B$ i simboli della sequenza $1^d0$, con $d$ che rappresenta il grado.

In questa rappresentazione si considera un nodo detto \textbf{super-root}, il
quale viene aggiunto in modo che il numero di valori uguali a 1 presenti nel
bitvector sia uguale al numero di nodi dell'albero.

In questo modo si arriva a ottenere un bitvector di lunghezza $2n + 1$ dove si
ha un simbolo 1 associato a ogni nodo dell'albero. Con questa rappresentazione
posso indicare il nodo $m$ con l'indice relativo del corrispondente bit 1.

Utilizzando le operazioni rank e select definite per il bitvector, è possibile
implementare un set di operazioni per interrogare questa rappresentazione
dell'albero.
\begin{itemize}
    \item $is\_leaf(v) = T$ se e solo se $select_0(v) = select_0(v + 1) - 1$ in
          quanto per costruzione una foglia aggiunge solo uno 0 al bitvector $B$,
          quindi con $select_0(v)$ troviamo la posizione dello 0 posto in $B$ dal
          nodo antecedente a $v$ nella visita (quello antecedente perché abbiamo
          il 10 della super-root) e con $select_0(v + 1)$ la posizione dello 0
          relativo a $v$. Se sono in due posizioni adiacenti significa che $v$ ha
          aggiunto solo uno $0$ e quindi è una foglia.
    \item $first\_child(v) = rank_1(select_0(rank_1(select_1(v)))+1) = rank_1
              (select_0(v)+1)$:
          \begin{itemize}
              \item $k = select_0(v) + 1$ restituisce la posizione $k$ su $B$ del
                    primo "figlio" di $v$. In altri termini il $v$-esimo 0 mi
                    dice che ho finito di "visitare" la sotto-sequenza di bitvector
                    costruita per il nodo $v - 1$ e al bit successivo inizia la
                    sequenza del bitvector per $v$.
              \item $m = rank_1(k)$ restituisce il numero di nodo dell'albero in
                    posizione $k$ di $B$, quindi la label del primo "figlio" di
                    $v$.
              \item Imponiamo che $first\_child(v) = - 1$ se $is\_leaf (v) = T$
          \end{itemize}
    \item $last\_child(v) = rank_1(select_0(rank_1(select_1(v))+1)-1) = rank_1
              (select_0(v+1)-1)$:
          \begin{itemize}
              \item $k = select_0(v + 1)$ restituisce la posizione $k$ su $B$
                    dello $0$ inserito in visita level-order del nodo con label
                    $v$. In altri termini, il (v + 1)-esimo 0 mi dice che ho
                    finito di "visitare" la sotto-sequenza di bitvector costruita
                    per il nodo $v$.
              \item $w = k - 1$ restituisce l'indice dell'ultimo 1 inserito in
                    visita level-order del nodo con label j, quindi l'indice su
                    B dell'ultimo "figlio" di c. In altri termini, con la
                    precedente operazione si raggiunge lo 0 di $1^d 0$ e col -1
                    l'ultimo 1 di $1^d$
              \item $m = rank_1(w)$ restituisce il numero di nodo dell'albero in
                    posizione $w$ di B, quindi la label dell'ultimo "figlio" di v.
              \item Imponiamo che $last\_child(v) = -1$ se $is\_leaf (v) = T$
          \end{itemize}
    \item $parent(v) = rank_1(select_1(rank_0(select_1(v)))) = rank_0(select_1(v))$:
          \begin{itemize}
              \item $i = select_1(v)$ restituisce la posizione $i$ del nodo $v$
                    nel bitvector $B$ (identificando in quale $1^d 0$ è stato
                    aggiunto).
              \item $j = rank_0(i)$ restituisce il numero di sequenze che sono
                    state aggiunte al bitvector $B$ fino a quella relativa al
                    "genitore" del nodo in posizione $i$. Il numero di tali
                    sequenza è l'indice del nodo "genitore" per definizione di
                    vista level order e conseguente etichettatura dei nodi.
              \item Imponiamo che $parent(v) = -1$ se $v = 1$ (non considero la
                    super-root)
          \end{itemize}
    \item $degree(v) \ = \ last\_child(v) - first\_child(v) + 1$, imponiamo 
          $degree(v) = 0$ se $last\_child(v) = first\_child(v) = -1$
    \item $nth\_child(v, nth) = rank_1(select_1(first\_child(v)) + nth - 1)$,
          imponiamo $nth\_child(v, nth) = -1$ se $degree(v) < nth$
\end{itemize}
\begin{esempio}
    In figura \ref{fig:louds}, è riportata la costruzione di un e Level Order
    Unary Degree Sequence.
    \begin{figure}[!ht]
        \centering
        \includegraphics[width=0.5\textwidth]{img/Strutture Dati/LOUDS.png}
        \caption{Esempio di costruzione di un e Level Order Unary Degree Sequence}
        \label{fig:louds}
    \end{figure}
\end{esempio}
\section{Wavelet Tree}
Abbiamo visto come rank e select possono essere usati per interrogare un bitvector,
che ha un alfabeto fisso di grandezza 2. Siamo ora interessati a generalizzare
tali query ad un alfabeto di grandezza arbitraria. Per praticità assumiamo di
avere un alfabeto $\Sigma = [1 \dots s]$ il quale è ordinato nel seguente modo:
\begin{equation}
    \Sigma[i] \prec \Sigma[j] \iff i < j \dots
\end{equation}
Dato un generico alfabeto $\Sigma$ e una sequenza $T[1\dots n] \in \Sigma$ possiamo
ri-definire le funzioni di rank e select come:
\begin{itemize}
    \item $rank_{\sigma,T} (i)$ conta tutte le occorrenze del simbolo $\sigma
              \in \Sigma$ fino all'indice $i$ in $T$, $i \leq | T |$.
    \item $select_{\sigma,T} (i)$ ritorna la posizione dell'i-esima occorrenza
          del simbolo $\sigma \in \Sigma$ in $T$, $i \leq | T |$.
\end{itemize}
\begin{equation}
    rank_{\sigma,T} (select_{\sigma,T} (i)) = i, \ \forall \sigma \in \Sigma
    \land \forall i = 1 \dots n
\end{equation}
Una rappresentazione "naïve" consiste nel considerare come rappresentazione di
$T$ attraverso un insieme di $s$ ($| \Sigma | = s$) stringhe binarie $B_\sigma[1
        \dots n]$, $\forall \sigma \in \Sigma$, ovvero una per ogni simbolo
dell'alfabeto, si ha che:
\begin{equation}
    B_\sigma[i] = \begin{cases}
        1 \iff T[i] = \sigma \\
        0 \iff T[i] \neq \sigma
    \end{cases}
\end{equation}
Se per ogni bitvector $B_\sigma$ abbiamo calcolato la struttura a supporto vista
precedentemente si ha $rank_{\sigma,T} (i)$ in tempo costante $\mathcal{O}(1)$ e
uno spazio pari a $s \cdot (n + o(n))$ bit aggiuntivi.

Possiamo ottenere una rappresentazione più efficiente in memoria senza sacrificare
troppo i tempi di query. Per realizzare questa rappresentazione consideriamo un
\textbf{albero binario perfettamente bilanciato} dove ogni nodo corrisponde ad un
sottoinsieme di $\Sigma$.

I due figli di ogni nodo partizionano il corrispondente sottoinsieme di $\Sigma$
in due. A ogni nodo $v$ corrisponde una sequenza chiamata $R_v$, la quale è una
sotto-sequenza dell'input $T$ ed è anche una sotto-sequenza della sequenza con
cui è etichettato il nodo genitore di $v$. La root corrisponde la sequenza $R_v = T$.

A ogni nodo $v$ si associa un bitvector, denotato con $B_v$, che indica a quale
dei due figli del nodo $v$ ogni simbolo della sotto-sequenza $R_v$ appartiene.

Se consideriamo l'indice $j$, con $1 \leq j \leq |R_v|$, abbiamo che:
\begin{itemize}
    \item Se $B_v [j] = 0$, allora il carattere associato $R_v [j]$ appartiene
          alla sotto-sequenza rappresentata dal figlio di sinistra.
    \item Se $B_v [j] = 1$, allora il carattere associato a $R_v [j]$ appartiene
          alla sotto-sequenza rappresentata dal figlio di destra.
\end{itemize}
Le foglie dell'albero sono \textit{virtualmente} etichettate con i singoli
caratteri dell'alfabeto. In realtà ci basta la funzione $rank_1$ eseguita sui
bitvector che etichettano i genitori delle foglie per recuperare l'informazione.
\subsection{Operazioni}
Vediamo ora come realizzare le operazioni su questa struttura dati:
\begin{itemize}
    \item \textbf{rank}: il calcolo dell'operazione rank inizia nel nodo root,
          nel quale si determina a quale dei due figli appartiene $\sigma$.
          Questa operazione avviene utilizzando l'ordinamento dell'alfabeto. In
          particolare, nella radice se $B_v [j] = 0$ allora $R_v [j] = \Sigma
              \left[\left\lceil \frac{s}{2} \right\rceil\right] \lor R_v [j]
              \prec \Sigma \left[\left\lceil \frac{s}{2} \right\rceil\right]$,
          ovvero il carattere si trova nella prima metà dell'alfabeto. Altrimenti
          se $B_v[j] = 1$ allora il carattere che si cerca è nella seconda metà
          dell'alfabeto.

          A questo punto si prosegue nel seguente modo fino a raggiungere le foglie:
          \begin{itemize}
              \item Se $\sigma = \Sigma\left[\left\lceil \frac{s}{2} \right\rceil
                            \right] \lor \sigma \prec \Sigma\left[\left\lceil
                            \frac{s}{2} \right\rceil\right]$ (siamo nella prima
                    metà, dell'alfabeto indotto dal nodo), allora si prosegue
                    verso il figlio di sinistra, aggiornando il valore di $i$ con
                    $i = rank_{0,B_v}(i)$.
              \item Altrimenti, ovvero se siamo nella seconda metà, usiamo il
                    figlio di destra, e aggiorniamo il valore di $i$ con $i =
                        rank_{1,B_v}(i)$.
          \end{itemize}
          Nel nuovo nodo $v$ si procede nello stesso modo, considerando che ora
          $\Sigma = \Sigma \left[1 \dots \left\lceil \frac{s}{2} \right\rceil\right]$
          se si è andati a sinistra e $\Sigma = \Sigma\left[\left\lceil
                  \frac{s}{2} \right\rceil + 1 \dots s\right]$ se si è andati a
          destra.

          Si prosegue fino ad una foglia e $rank_{\sigma,T} (i) = i'$, dove $i'$
          è l'ultimo valore di i che si ottiene nei vari step.

          L'albero ha altezza $\lceil \log s\rceil$ quindi $rank_{\sigma,T} (i)$
          può essere calcolato in $\mathcal{O}(\log s)$, dove $s$ è la cardinalità
          dell'alfabeto.
    \item \textbf{random access}: in questo caso la scelta del figlio dipende
          unicamente da $B_v [i]$:
          \begin{itemize}
              \item Se $B_v [i] = 0$ proseguo verso il "figlio" di sinistra,
                    con $i = rank_{0,B_v}(i)$
              \item Se $B_v [i] = 1$ proseguo verso il "figlio" di destra,
                    con $i = rank_{1,B_v}(i)$
          \end{itemize}
          Si prosegue fino ad una foglia scegliendo il percorso da seguire in
          base al valore di $B_v [i]$.

          $T[i]$ è il simbolo che etichetta la foglia raggiunta alla fine della
          visita dato che il wavelet tree di una sequenza $T$ garantisce random
          access alla sequenza stessa possiamo sostituirla col suo wavelet tree.

          L'albero ha altezza $\lceil\log s \rceil$ quindi $access_{\sigma,T} (i)$
          può essere calcolato in $\mathcal{O}(\log s)$.
    \item \textbf{select}: analogamente ai bitvector si può dimostrare che anche
          $select_{\sigma,T}(i)$ può essere calcolato in $\mathcal{O}(\log s)$
\end{itemize}
\begin{nota}
    Nell'operazione di access ad ogni passo si effettua l'operazione di rank
    in base al valore del bit $B_v [i]$. Quindi per ogni nodo si effettua una
    scelta se effettuare una rank sul valore $0$ o sul valore $1$.
\end{nota}
\subsection{Costruzione}
Vediamo ora come costruire un wavelet tree livello per livello a partire dalla
root:
\begin{enumerate}
    \item Si etichetta la root con $R_v = T$ e un bitvector $B_v$ tale che
          $\forall i$, con $1 \leq i \leq |T|$:
          \begin{equation}
              B_v[i] = \begin{cases}
                  0 \iff T[i] = \Sigma\left[\left\lceil \frac{s}{2}
                      \right\rceil\right] \lor T[i] \prec \Sigma\left[
                  \left\lceil \frac{s}{2} \right\rceil\right] \\
                  1 \text{ altrimenti}
              \end{cases}
          \end{equation}
    \item Si estrae la sotto-sequenza $T'$ corrispondente ai valori 0 di $B_v$ e
          la si usa per etichettare il "figlio" di sinistra $v_1$ (che è relativo
          all'alfabeto $\Sigma = \Sigma\left[1 \dots \left\lceil \frac{s}{2}
                  \right\rceil\right]$) mentre quella corrispondente ai valori 1
          la si usa per etichettare il "figlio" di destra $v_2$ (che è relativo
          all'alfabeto $\Sigma = \Sigma\left[\left\lceil \frac{s}{2} \right\rceil
                  + 1 \dots s \right]$) per entrambe.
    \item Posso quindi cancellare $T$ e costruire i bitvector $B_{v1}$ e $B_{v2}$,
          con le rispettive strutture per la funzione rank e continuare
          ricorsivamente fino al raggiungimento delle foglie (quando si
          raggiungono alfabeti di cardinalità 1).
\end{enumerate}
Il processo di costruzione di un wavelet tree richiede un tempo pari a
$\mathcal{O}(n \log s)$.

In ogni momento della costruzione del wavelet tree abbiamo un bound in spazio
pari a $3n \log s + \mathcal{O}(s \log n)$, dato da:
\begin{itemize}
    \item 3 sotto-sequenze di $T$, quella del "genitore" e quelle dei due "figli".
    \item Tutti i bitvector finora computati che formano il wavelet tree.
    \item I puntatori che memorizzano la struttura ad albero.
\end{itemize}
Un ulteriore miglioramento in spazio si può ottenere concatenando tutti i bitvector
in un unico bitvector con una sola struttura a supporto della funzione rank. In
tal caso la struttura ad albero si può inferire dai partizionamenti dell'alfabeto
e dai bitvector. Questa variante è chiamata \textbf{levelwise wavelet tree}.

Un wavelet tree per una sequenza lunga $n$ costruita su alfabeto di cardinalità
$s$ occupa, avendo $\mathcal{O}(s \log n)$ per la topologia dell'albero:
\begin{equation}
    n \log s + o(n \log s) + \mathcal{O}(s \log n) \ bit
\end{equation}
Mentre, un levelwise wavelet tree richiede:
\begin{equation}
    n \log s + o(n \log s) \ bit
\end{equation}
\begin{esempio}
    Costruzione di un wavelet tree:
    \begin{figure}[!ht]
        \centering
        \includegraphics[width=0.7\textwidth]{img/Strutture Dati/wavelet tree.png}
        \caption{Esempio di costruzione di un wavelet tree su una sequenza semplice}
    \end{figure}
\end{esempio}
\section{Altre strutture dati succinte}
\begin{itemize}
    \item \textbf{Parentesi bilanciate}: si costruisce a partire dalla DFS
          dell'albero (preorder):
          \begin{itemize}
              \item "(" quando si raggiunge un nodo per la prima volta.
              \item ")" quando si è terminata la visita del sotto-albero.
          \end{itemize}
    \item Strutture dati succinte che supportano le \textbf{range minimum queries}.
    \item \textbf{Wavelet matrix}: nascono con l'idea di migliorare i levelwise
          wavelet tree nella gestione di larghi alfabeti:
          \begin{itemize}
              \item tempi di access dimezzati
              \item tempi di rank e select leggermente ridotti
          \end{itemize}

          L'idea è che i bit di un nodo "figlio" non sono più "allineati" al
          "genitore" ma si assume che passando da un livello all'altro, tutti gli
          zero vanno da una parte e gli uni dall'altra.

          Salvando ad ogni livello $l$ il numero totale di simboli $\sigma = 0$
          $z_l$, richiedendo in totale $\mathcal{O}(\log n \log s)$ bit, si ottiene
          lo stesso comportamento di un levelwise wavelet tree.

          Una wavelet matrix richiede $n \log s + o(n \log s)$ bit, può essere
          costruita in $\mathcal{O}(n \log s)$ e risponde alle stesse query di un
          (levelwise) wavelet tree in $\mathcal{O}(\log s)$
\end{itemize}
\begin{definizione}[\textbf{Range Minimum Query}]
    Dato un array $A[1\dots n]$ di numeri $n$ elementi da un universo totalmente
    ordinato la \textbf{Range Minimum Query} $RMQ_A(i, j)$, con $1 \leq i \leq
        j \leq n$, restituisce la posizione $k$ di un elemento minimo in $A[i
                \dots j]$:$$RMQ_A(i, j) = argmin_{i \leq k \leq j}\{A[k]\}$$

    Si può dimostrare che è possibile costruire una struttura dati succinta che
    richiede $2n + o(n)$ bit in memoria e che risponde in $\mathcal{O}(1)$.
\end{definizione}
\chapter{Strutture Dati Probabilistiche Hashing-Based}
Su strutture dati semplici come array, matrici, etc$\dots$ posso eseguire operazioni
di accesso, cancellazione di un elemento e inserimento di un elemento in tempo
$\mathcal{O}(1)$ data la chiave $k$, l'input $x$ e la posizione $k[x]$.

Con queste strutture dati si ha un problema in quanto è possibile che l'insieme
delle chiavi utilizzate sia molto minore rispetto alla dimensione della struttura
dati. Una soluzione a questo problema sono le \textbf{hash tables}.

Se con l'accesso diretto l'elemento con chiave $k$ era memorizzato in posizione
$k$, con l'hash è memorizzato in $h(k)$, dove $h$ è una funzione di hash.
\begin{definizione}[\textbf{Funzione di hash}]
    Una \textbf{funzione di hash} $h$ è definita come:
    \begin{equation}
        h: \mathcal{U} \to \{0, \dots, m - 1\}
    \end{equation}
    ovvero come una funzione definita dall'insieme universo $\mathcal{U}$
    all'insieme delle posizioni $\{0, \dots, m - 1\}$ di una tabella di hash $T$.
\end{definizione}
Le funzioni di hash possono avere input "\textit{scomodi}", ovvero input che
possono generare \textbf{collisioni}:
\begin{equation}
    h(k') = h(k'') \ \ \text{con} \ \ k' \neq k''
\end{equation}
Una possibile soluzione per ridurre le collisioni consiste nell'utilizzare una
famiglia di funzioni di hash al posto di una singola.
\begin{definizione}[\textbf{famiglia di funzioni hash}]
    Una \textbf{famiglia di funzioni hash} è un insieme $\mathcal{H}$ di funzioni
    hash con lo stesso dominio e codominio. La scelta di $h \in \mathcal{H}$ può
    essere fatta con un sampling uniforme su $\mathcal{H}$.
\end{definizione}
$\mathcal{H}$ è detta \textbf{universale} se e solo se, con $h : \mathcal{U} \to
    \{0, \dots, m - 1\}$ scelta casualmente da $\mathcal{H}$, si ha che:
\begin{equation}
    \mathcal{P}(h(x) = h(y)) \leq \frac{1}{m}
\end{equation}
ovvero la probabilità di collisioni è minore di $\frac{1}{m}$ dove $m$ è la
dimensione della tabella.

Un'altra possibile soluzione consiste nelle hash table dove le collisioni sono
risolte tramite concatenazione. Si mettono tutti gli elementi che collidono nella
stessa posizione dell'hash table in una lista concatenata. Chiamiamo $\alpha$ il
\textbf{fattore di carico}, ovvero il numero medio di elementi in queste liste
concatenate.

Il caso peggiore si ha quando tutte le $n$ chiavi "mappano" in una sola lista,
quindi i tempi di accesso richiedono un tempo pari a $\Theta(n)$ ma nella realtà
è difficile che accada quindi accesso in $\Theta(1 + \alpha)$ nel caso migliore,
con il numero di posizioni nella hash table proporzionale al numero di elementi
della tabella (quindi $\alpha \to 1$), ho accesso in tempo $\Theta(1)$.
\section{Membership problem}
\begin{definizione}[\textbf{Membership problem}]
    Il problema \textbf{membership problem} è definito come:
    \begin{itemize}
        \item \textbf{Input}:
              \begin{itemize}
                  \item Insieme universo $\mathcal{U}$, $|\mathcal{U}| = u$ che
                        per praticità assumiamo valori interi con ogni elemento
                        che occupa $w = \log u$ bit.
                  \item Insieme $S \subseteq \mathcal{U}$, $|S| = n$.
                  \item Un elemento $y \in \mathcal{U}$.
              \end{itemize}
        \item \textbf{Output}: $T$ se $y \in S$, $F$ altrimenti.
    \end{itemize}
\end{definizione}
Una prima soluzione per questo problema consiste nel creare una hash table per $S$
con le collisioni risolte tramite liste concatenate. Questa struttura ci permette
di ottenere una risposta in tempo pari a $\Theta(1)$ occupando uno spazio pari a
$\mathcal{O}(n \log u)$ bit.

È possibile ottenere una soluzione migliore se assumiamo di poter ammettere falsi
positivi ma comunque non falsi negativi. Nel caso in cui ammettiamo falsi positivi
ma non falsi negativi parliamo del problema di \textbf{approximate membership}.
In questo caso:
\begin{itemize}
    \item Se $y \in S$ voglio sempre ottenere $T$, quindi ho sempre l'informazione
          corretta in merito al fatto che un elemento $y$ sia in $S$.
    \item Se $y \notin S$ voglio ottenere $F$ con probabilità $\mathcal{P} \geq
              1 - \delta$, con $\delta \in \mathbb{R}^{+} \land \delta \to 0$.
\end{itemize}
Si assume quindi di avere errori sui falsi positivi, ovvero ottengo $T$ e non $F$
con probabilità $\mathcal{P} \leq \delta$

Per risolvere questo problema, creiamo una struttura con $\frac{n}{\delta}$ bit,
per $S$, dove $n = |S|$ insieme universo $\mathcal{U}$. Inoltre, sia $\mathcal{H}$
una famiglia universale di funzioni hash per $\mathcal{U}$ con:
\begin{equation}
    h_j : \mathcal{U} \to \{0, \dots, m - 1\}
\end{equation}
con $m = \frac{n}{\delta}$. Prendendo casualmente una funzione di hash $h \in
    \mathcal{H}$, popoliamo un bitvector $A$, $|A| = m = \frac{n}{\delta}$,
nel seguente modo:
\begin{equation}
    A[i] = \begin{cases}
        1 & \text{se} \ \exists k \in S \ \text{tale che} \ h(k) = i \\
        0 & \text{altrimenti}
    \end{cases}
\end{equation}
Sulla struttura dati appena creata è possibile eseguire le query per sapere se
$x \in S$ in tempo $\mathcal{O}(1)$ ($A[h(x)] = 1$). Inoltre:
\begin{itemize}
    \item Se $x \in S$ si ottiene sempre $T$.
    \item Se $x \notin S$ si ottiene lo stesso $T$ se e solo se $\exists k \in S$
          tale che $h(k) = h(x)$.
    \item $\mathcal{H}$ famiglia universale quindi $\mathcal{P}[h(k) = h(x)]
              \leq \frac{1}{m} = \frac{\delta}{n}$
    \item La probabilità che esista almeno una tale chiave $k$ è
          $(\mathcal{P}(A \cup B) \leq \mathcal{P}(A) + \mathcal{P}(B))$:
          \begin{equation}
              \sum_{k \in S} \mathcal{P}[(h(k) = h(x))] \leq \frac{n}{m} =
              \frac{(\delta \cdot n)}{n} = \delta
          \end{equation}
\end{itemize}
\section{Bloom Filter}
\begin{definizione} [\textit{Risultato solo teorico}]
    Data una funzione di hash $h : \mathcal{U} \to \{0, 1, \dots, m - 1\}$, per
    praticità $\mathcal{U} \subseteq \mathbb{N}$, questa è \textbf{ideale} se e
    solo se, $\forall k \in \mathcal{U}$, $h(k)$ vale indipendentemente un valore
    uniformemente distribuito su $[0 \dots m - 1]$. Quindi, $\forall k \in
        \mathcal{U}$, $h(k)$ vale un qualsiasi intero tra $0$ e $m - 1$ con la
    stessa probabilità e tale valore non dipende dal valore di hash delle altre
    chiavi.
\end{definizione}
Costruiamo ora una struttura dati per $S$, $|S| = n$. Si considera un bitvector
$A$, dove$|A| = m$, inoltre, si considera una famiglia di $l$ funzioni hash
ideali $\mathcal{H} = \{h_0, \dots, h_{l - 1}\}$ dove:
\begin{equation}
    h: \mathcal{U} \to \{0, \dots, m - 1\}, \ \forall h \in \mathcal{H}
\end{equation}
Il bitvector $A$ viene riempito nel seguente modo:
\begin{equation}
    \forall k \in S \ \text{e} \ \forall h_i \in \mathcal{H}: A[h_i(k)] = 1
\end{equation}
quindi per ogni $k \in S$ abbiamo fino a $l$ bit pari a $1$ in $A$. Si utilizza
il termine "fino a" perché alcune $h_i, h_j \in \mathcal{H}$ potrei avere
$h_i(k) = h_j(k)$ e se già $A[h_i(k)] = 1$ non avrò ulteriori modifiche in
posizione $h_i(k) = h_j(k)$.

Il bitvector $A$ è denominato \textbf{Bloom filter} di $S$.

Su questa struttura appena creata è possibile eseguire le query per l'approximate
membership problem. Dato $x \in \mathcal{U}$ si ha che $x \in S$ se e solo se:
\begin{equation}
    A[h_i(x)] = 1, \ \forall h_i \in \mathcal{H}
\end{equation}
\begin{figure}[!ht]
    \centering
    \includegraphics[width=0.5\textwidth]{img/hash/bloom.png}
    \caption{Esempio di query su Bloom Filter}
\end{figure}
\newpage
Una generalizzazione dei Bloom filter sono \textbf{Counting Bloom filter}. In
queste strutture dati si tiene conto anche di quante funzioni di hash mappano in
una certa posizione dato un elemento qualsiasi si verifica la presenza tramite
il counting Bloom filter tramite una threshold $\theta$.

Possiamo costruire questa struttura dati nel seguente modo:
\begin{equation}
    \forall k \in S \ text{e} \ \forall h_i \in \mathcal{H}: A[h_i(k)] += 1
\end{equation}
In fase di query, dato $x \in \mathcal{U}$ abbiamo:
\begin{itemize}
    \item Se $\exists h_i \in \mathcal{H}$ tale che $A[h_i(x)] = 0$ o
          $A[h_i(x)] \leq \theta$ allora $x \notin S$
    \item Se $\forall h_i \in \mathcal{H}$ $A[h_i(x)] > 0$ o $A[h_i(x)] > \theta$
          allora "probabilmente" $x \in S$, avendo $\mathcal{P}(FP) \neq 0$
    \item $A[h_i(x)]$ è una sovrastima del numero di elementi $x$ in $S$.
\end{itemize}
\section{Heavy Hitters Problem}
L'\textbf{heavy hitters problem} consiste nell'identificazione dell'elemento più
frequente o anche detto \textbf{heavy hitter}. Questo problema non può essere
risolto utilizzando un \textbf{Bloom Filter} in quanto richiederebbe troppo spazio
di memorizzazione.
\begin{definizione}[\textbf{Stream di dati}]
    Con \textbf{stream di dati} si intende una sequenza di dati passati uno ad
    uno alla struttura dati. Quindi, data una sequenza $S = s_0,s_1, \dots ,s_{n-1}$,
    prima si considera $s_0$, poi $s_1 \ \dots$ fino a $s_{n-1}$ si costruisce
    quindi una struttura dati che può essere interrogata con nuovi valori
    $x \in \mathcal{U}$.
\end{definizione}
Su questa struttura dati è possibile effettuare le seguenti query:
\begin{itemize}
    \item Quante volte appare $x$ nello stream:
          \begin{equation}
              | \{ i \in \{0, 1, \dots, n - 1\}| \ s_i = x \}|
          \end{equation}
    \item Quanti elementi distinti si hanno nello stream:
          \begin{equation}
              |\{s_i | i \in \{0, 1, \dots, n - 1\}\}|
          \end{equation}
\end{itemize}
\subsection{Count-min sketch}
Una struttura dati che ci permette di risolvere questo problema è il
\textbf{Count-min sketch}, la quale richiede poco spazio in memoria ed è
concettualmente simile a un Bloom filter.

Questa struttura dati è definita a partire da:
\begin{itemize}
    \item Insieme universo $\mathcal{U}$.
    \item Uno stream $S$ lungo $n$ costruito da elementi di $\mathcal{U}$.
    \item Due parametri d'errore $\delta$ e $\varepsilon$. Otterremo risposte
          alle query "sbagliate" entro un fattore aggiuntivo $\varepsilon$ con
          probabilità almeno $1 - \delta$
    \item $\mathcal{H}, \ |\mathcal{H}| = l = \lceil \ln \frac{1}{\delta} \rceil$,
          famiglia di funzioni hash universale per $\mathcal{U}$ si impone che:
          \begin{equation}
              h_i: \mathcal{U} \to \{0, \dots, m\}, \ \forall h_i \in \mathcal{H},
              \ \text{con} \ m = \left\lceil \frac{e}{\varepsilon} - 1 \right\rceil
          \end{equation}
\end{itemize}
Il \textbf{Count-min sketch} è costituito da una matrice bidimensionale $T$ con
$l$ righe, una per ogni $h_i \in \mathcal{H}$, e $m$ colonne. Si hanno quindi $l$
hash table indipendenti con $m$ entry ciascuna. La matrice $T$ viene inizializzata
con tutti gli elementi a $0$.

Il caricamento di $T$ avviene nel seguente modo:
\begin{itemize}
    \item Si considerano in ordine tutti gli $x_j \in S$, con $j = 0, 1, \dots,
              n - 1$.
    \item Sappiamo che ogni $h_i$ ha di fatto come codominio l'insieme degli indici
          di colonna. Quindi inserire $x_j$ in T vuol dire incrementare di 1
          $T_{h_i} [h_i(x_j)]$, $\forall h_i \in H$
\end{itemize}
Queste operazioni richiedono un tempo per essere eseguite pari a $\mathcal{O}(n)$,
dove $n$ è la lunghezza dello stream.

Su questa struttura dati è possibile eseguire la query per $q \in \mathcal{U}$
nel seguente modo:
\begin{itemize}
    \item Si applica ogni funzione di hash a $q$
    \item Si tiene traccia di ogni $T_{h_i} [h_i(q)]$, $\forall h_i \in \mathcal{H}$
    \item Si restituisce il minimo tra tali valori, che è una stima (una frequenza
          approssimata $\hat{a}_q$) di quante volte occorre $q$ in $S$:
          \begin{equation}
              \hat{a}_q = \min_{h_i \in \mathcal{H}} T_{h_i} [h_i(q)]
          \end{equation}
\end{itemize}
Una query si effettua in tempo $\mathcal{O}(l)$.

È possibile dimostrare che data $a_q$, ovvero la frequenza reale di $q$ in $S$,
si ha che:
\begin{equation}
    a_q \leq \hat{a}_q
\end{equation}
Si dimostra che $a_q \leq \hat{a}_q$ a causa delle collisioni si ottengono
sovrastime ma mai sottostime della frequenza.

Inoltre, è possibile dimostrare:
\begin{equation}
    \hat{a}_q \leq a_q + \varepsilon \cdot n
\end{equation}
con probabilità almeno $1 - \delta$.

Dato che la matrice bidimensionale $T$ è di dimensione $l \times m = \lceil \ln
    \frac{1}{\delta} \rceil \times \lceil \frac{e}{\varepsilon} \rceil$ con
valori che richiedono $\log n$ bit, allora la struttura dati occupa uno spazio
pari a:
\begin{equation}
    \left( \left\lceil \ln \frac{1}{\delta} \right\rceil \cdot \left\lceil
    \frac{e}{\varepsilon} \right\rceil \cdot \log n \right) \ \text{bit}
\end{equation}
\begin{figure}[!ht]
    \centering
    \begin{subfigure}[b]{0.45\textwidth}
        \centering
        \includegraphics[width=\textwidth]{img/hash/CMS1.png}
        \caption{Inizializzazione}
    \end{subfigure}
    \hfill
    \begin{subfigure}[b]{0.45\textwidth}
        \centering
        \includegraphics[width=\textwidth]{img/hash/CMS2.png}
        \caption{Inserimento del primo elemento}
    \end{subfigure}
    \hfill
    \begin{subfigure}[b]{0.45\textwidth}
        \centering
        \includegraphics[width=\textwidth]{img/hash/CMS3.png}
        \caption{Inserimento del secondo elemento}
    \end{subfigure}
    \hfill
    \begin{subfigure}[b]{0.45\textwidth}
        \centering
        \includegraphics[width=\textwidth]{img/hash/CMS4.png}
        \caption{Inserimento del terzo elemento}
    \end{subfigure}
    \caption{Esempio di riempimento Count-min sketch}
\end{figure}
\chapter{Pattern Matching su Stringhe}
\section{Introduzione}
Il problema del Pattern Matching consiste cercare un motivo all'interno di un oggetto più o meno complesso. In questo corso ci si concentrerà sul Pattern Matching su Stringhe, ovvero cercare all'interno di un testo $T$ le occorrenze di un pattern $P$.
\begin{definizione}[\textbf{Stringa}]
    Definiamo una \textbf{stringa} $X$ come una giustapposizione di simboli appartenenti a un alfabeto $\Sigma$.
    \begin{equation}
        X=x_1x_2\dots x_n \ \ x_i \in \Sigma \ \forall i = 1, \dots, n
    \end{equation}
    \begin{itemize}
        \item \textbf{Stringa nulla} $\varepsilon$ è una stringa composta da zero simboli.
        \item Simbolo in posizione $i$ si riferisce al simbolo in posizione $i$-esima $x_i = X[i]$
        \item \textbf{Sottostringa} da $i$ a $j$ è una porzione di stringa compresa tra gli indici $i$ e $j$. 
        \begin{equation}
            X[i]X[i+1]\dots X[j - 1]X[j]
        \end{equation}
        Posso esprimerlo attraverso la seguente notazione: $X[i, j] \lor X[i:j]$. Possiamo dire che una sottostringa $X[i, j]$ è:
        \begin{itemize}
            \item Propria se $i \neq 1$ e $j \neq |X|$
            \item Impropria altrimenti
        \end{itemize}
        \item \textbf{Prefisso} di lunghezza $j$ è una sottostringa $X[1, j]$. Anche in questo caso possiamo distinguere:
        \begin{itemize}
            \item Proprio se $j \neq |X|$
            \item Improprio altrimenti
        \end{itemize}
        Per il prefisso è possibile definire anche il prefisso nullo, ovvero il prefisso composto da zero caratteri ($X[1, j] \ \to \ j = 0$).
        \item \textbf{Suffisso} che inizia in $i$ è la sottostringa $X[i,|X|]$. Di questa sott-stringa posso calcolare la lunghezza del prefisso come: 
        \begin{equation}
            |X| - i + 1
        \end{equation}
        Anche in questo caso possiamo distinguere:
        \begin{itemize}
            \item Proprio se $i \neq 1$
            \item Improprio altrimenti
        \end{itemize}
        È possibile definire il suffisso nullo ovvero quello composto da zero caratteri ($X[i,|X|] \ \to \ i = |X| + 1$).
    \end{itemize}
\end{definizione}
\begin{nota}
    Una sequenza è invece una lista di elementi divisi da un particolare simbolo.
\end{nota}

Quando si parla di string matching possiamo definire due tipi. Dati un pattern $P$ e un testo $T$ possiamo definire lo String Matching:
\begin{enumerate}
    \item \textbf{Esatto}: consiste nel cercare le occorrenze esatte di $P$ in $T$.
    \item \textbf{Approssimato}: consiste nel cercare le occorrenze approssimate di $P$ in $T$.
\end{enumerate}
\subsection{String Matching Esatto}
Possiamo definire questo problema partendo da un input composto da un testo $T$ di lunghezza $n$ e un pattern $P$ di lunghezza $m$ definiti su un alfabeto $\Sigma$. L'output di tale problema consiste in tutte le occorrenze esatte di $P$ in $T$.
\begin{definizione}[\textbf{Occorrenza esatta}]
    Una posizione $i$ del testo $T$ tale che $T[i, i + m - 1] = P$ è un'\textbf{occorrenza esatta} di $P$ in $T$.
\end{definizione}
\begin{definizione}[\textbf{Match}]
    Dati due simboli $s_1, s_2 \in \Sigma$ si ha un \textbf{match} se $s_1 = s_2$.
\end{definizione}
\begin{definizione}[\textbf{Mismatch}]
    Dati due simboli $s_1, s_2 \in \Sigma$ si ha un \textbf{mismatch} se $s_1 \neq s_2$.
\end{definizione}

Avendo definito il problema in questo modo è possibile definire un semplice algoritmo che mi permette di calcolare l'output del problema. Questo algoritmo utilizza una finestra $W$, delle stesse dimensioni del pattern, che scorre sul testo. L'algoritmo semplice che permette di calcolare le occorrenze esatte è:
\begin{enumerate}
    \item Uso una finestra $W$ lunga $m$ che scorre lungo $T$ da sinistra a destra. La posizione iniziale di $W$ è $i = 1$.
    \item Si confronta ogni simbolo di $P$ con il corrispondente simbolo di $T$ all'interno di $W$ da sinistra verso destra.
    \begin{equation}
        P[j] = T[i + j - 1] \ \forall \ j \ \text{tale che} \ 1 \leq j \leq m \ \Rightarrow T[i, i + m - 1] = P
    \end{equation}
    \item $W$ viene spostata di una posizione verso destra e il confronto viene ripetuto.
    \item Ultima posizione di $W$ è $i = |T| - |P| + 1 = n - m + 1$
\end{enumerate}
\begin{algorithm}
  \begin{algorithmic}
    \Function{trivial\_exact\_occurrences}{$T, P$}
        \State $n\gets |T|$
        \State $m \gets |P|$
        \State $i\gets 1$
        \While {$i \leq n - m + 1$}
            \State $j \gets 1$
            \While {$P[j] = T[i + j - 1] \ \land \ j \leq m$}
                \State $j \leq j + 1$
            \EndWhile
            \If {$j = m + 1$}
                \State $\text{output } i$
            \EndIf
            \State $i \gets i + 1$
        \EndWhile
    \EndFunction
  \end{algorithmic}
  \caption{Algoritmo banale per String Matching Esatto}
\end{algorithm}

Questo algoritmo richiede un tempo pari a $\mathcal{O}(m \cdot n)$.
\begin{nota}
    Questo algoritmo può essere migliorato spostando la finestra alla posizione successiva al primo mismatch.
\end{nota}
\subsection{String Matching Approssimato}
Possiamo definire questo problema partendo da un input composto da un testo $T$ di lunghezza $n$, un pattern $P$ di lunghezza $m$ definiti su un alfabeto $\Sigma$ e una \textbf{soglia di errore} $k$. L'output di tale problema consiste in tutte le occorrenze approssimate di $P$ in $T$.

Introducendo una soglia di errore abbiamo bisogno di definire una metrica per calcolarlo. Per fare ciò si utilizza la \textit{distanza di edit} (ED) tra due stringhe. Tale distanza è definita come il minimo numero di operazioni di sostituzione, cancellazione, inserimento di un simbolo che trasformano una stringa nell'altra.
\begin{nota}
    \begin{equation}
        ED(X_1, X_2) \geq abs(|X_1| - |X_2|)
    \end{equation}
\end{nota}
\begin{definizione}[\textbf{Occorrenza approssimata}]
    Una posizione $i$ del testo $T$ tale che esista almeno una sottostringa $S = T[i - L + 1,i]$ tale che $ED(P, S) \leq k$, è un'occorrenza approssimata di $P$ in $T$.
\end{definizione}
\begin{nota}
    \begin{enumerate}
        \item Se $ED(P, S) \leq k$, allora $i$ è occorrenza approssimata.
        \item $ED(P, S) \geq abs(m - L) \ \Rightarrow \ $se $abs(m - L) > k$ allora $i$ non può essere occorrenza approssimata. 
    \end{enumerate}
\end{nota}

Avendo definito il problema in questo modo è possibile definire un semplice algoritmo che mi permette di calcolare l'output del problema. Questo algoritmo utilizza una finestra $W$, di dimensione variabile, che scorre sul testo.
\begin{enumerate}
    \item Uso una finestra $W$ di lunghezza variabile $\in [m - k, m + k]$ che scorre lungo il testo $T$ da sinistra a destra. La posizione iniziale di tale finestra è $i = m - k$ e la sua lunghezza iniziale è $m - k$.
    \item Se la distanza di edit tra $P$ e la sottostringa di $T$ compresa in $W$ è $\leq \ k$, allora $i$ è occorrenza approssimata di $P$ in $T$.
    \item $W$ viene spostata a destra di una posizione.
\end{enumerate}
\begin{algorithm}
  \begin{algorithmic}
    \Function{trivial\_approx\_occurrences}{$T, P, k$}
    \State $n\gets |T|$
    \State $m \gets |P|$
    \State $i\gets m - k$
    \While {$i \leq n$}
    \State $L \gets  m - k$
    \While {$L \leq m + k \ \land \ i - L + 1 \geq 1$}
        \If {$ED(P, T[i - L + i, i]) \leq k$}
            \State $\text{output } i$
        \EndIf
    \EndWhile
    \State $i \gets i + 1$
    \EndWhile
    \EndFunction
  \end{algorithmic}
  \caption{Algoritmo banale per String Matching Approssimato}
\end{algorithm}

Questo algoritmo mi permette di calcolare il matching approssimato in tempo $\mathcal{O}(n \cdot k \cdot m^2)$, dove $m^2$ è dovuto al calcolo della distanza di edit tra le due sottostringhe.
\section{Ricerca esatta con Automa a Stati Finiti}
Un automa è un modello di calcolo che riconosce un linguaggio, ovvero un insieme di stringhe che godono di una proprietà. Gli automi a stati finiti riconoscono un linguaggio regolare.
\begin{definizione} [\textbf{Automa a stati finiti}]
    Un Automa a Stati Finiti è formalmente una quintupla:
    \begin{equation}
        A = (Q, \Sigma, \delta, q_0, F)
    \end{equation}
    dove:
    \begin{itemize}
        \item $Q$, insieme finito di stati.
        \item $\Sigma$, alfabeto in input
        \item $\delta: Q \times \Sigma \to Q$, funzione di transizione. $\delta(q,\sigma)$ è lo stato di arrivo a partire da $q$ dopo la lettura di $\sigma$
        \item $q_0$, stato iniziale
        \item $F$ (sottoinsieme di $Q$), insieme degli stati accettanti.
    \end{itemize}
\end{definizione}
Gli automi a stati finiti possono essere rappresentati attraverso un diagramma di stato, ovvero attraverso una struttura a grafo dove i vertici sono gli stati. Esiste l'arco $(q_1,q_2)$ se almeno un simbolo $\sigma$ è tale per cui $\delta(q_1,\sigma) = q_2$. L'arco $(q_1,q_2)$ viene etichettato dalla lista di simboli che permettono la transizione da $q_1$ a $q_2$. Lo stato iniziale $q_0$ è indicato tramite un arco entrante che non esce da uno stato, mentre gli stati accettanti sono indicati da un doppio bordo.

È possibile rappresentare la funzione di transizione $\delta$ degli automi attraverso una matrice con $|Q|$ righe e $|\Sigma|$ colonne. Nella generica cella $(q, \sigma)$ sarà contenuto il valore di $\delta(q, \sigma)$, ovvero lo stato di arrivo a partire dallo stato $q$ attraverso il simbolo $\sigma$.
\begin{definizione}[\textbf{Bordo}]
    Il \textbf{bordo} di una stringa $X$ è il più lungo prefisso \textbf{proprio} di $X$ che occorre come suffisso di $X$. 
\end{definizione}
\begin{esempio}
    Esempi di bordo:
    \begin{itemize}
        \item $X = baaccbbaac$ il suo bordo sarà $B(X) = baac$
        \item $X = aaaccbbaac$ il suo bordo sarà $B(X) = \varepsilon$
        \item $X = abababa$ il suo bordo sarà $B(X) = ababa$
        \item $X = aaaaaaaa$ il suo bordo sarà $B(X) = aaaaaaa$
        \item $X = a$ il suo bordo sarà $B(X) = \varepsilon$
    \end{itemize}
\end{esempio}
\begin{nota}
    Il bordo di un simbolo è sempre vuoto.
\end{nota}
\begin{definizione}[\textbf{Concatenazione}]
    La \textbf{concatenazione} di un simbolo $\sigma$ con la stringa $X$ è la stringa $X\sigma$.
\end{definizione}
Possiamo definire ora l'automa a stati finiti per la ricerca esatta di un pattern $P$ di lunghezza $m$ definito su alfabeto $\Sigma$ è la quintupla $(Q, \Sigma, \delta, q_0, F)$ con:
\begin{itemize}
    \item $Q = \{0, 1, \dots, m\}$
    \item $\Sigma$ è l'alfabeto di definizione di $P$
    \item $\delta: Q \times \Sigma \to Q$ è la funzione di transizione
    \item $q_0 = 0$ è lo stato iniziale
    \item $F = \{m\}$ è lo stato accettante
\end{itemize}
A questo punto il processo di ricerca esatta attraverso un automa a stati finiti è composto da:
\begin{enumerate}
    \item Una parte di costruzione dell'automa per il pattern $P$, nel quale in calcolo della funzione di transizione $\delta$ avviene in tempo $\theta(m \cdot |\Sigma|)$.
    \item Uso dell'automa per riconoscere, in un testo $T$ definito su alfabeto $\Sigma$, tutte le occorrenze esatte di $P$. La scansione del testo $T$ avviene in tempo $\theta(n)$
\end{enumerate}
\subsection{Funzione di transizione}
La funzione di transizione $\delta$ per un pattern $P$ di lunghezza $m$ definito su un alfabeto $\Sigma$ è definita per ogni $(j, \sigma) \in Q \times \Sigma$ tale che $\delta(j, \sigma)$ è lo stato in cui si arriva da $j$ attraverso $\sigma$:
\begin{equation}
    \delta(j, \sigma) = \begin{cases}
        j + 1 & \text{se} \ j < m \ \land \ P[j + 1] = \sigma \\
        k & \text{se} \ j = m \ \lor \ P[j + 1] \neq \sigma
    \end{cases}
\end{equation}
dove $k$ è la lunghezza del bordo del prefisso di $P$ di lunghezza $1, j$ a cui è concatenato $\sigma$, ovvero:
\begin{equation}
    k = |B(P[1, j]\sigma)| \ \ k \leq j
\end{equation}

Dallo stato $0$ si arriva allo stato $0$ per qualsiasi simbolo diverso da P[1]. Dallo stato $0$ si arriva allo stato $1$ attraverso il simbolo $P[1]$. Dallo stato $j = m$ si arriva sempre a uno stato $k \leq m$, dallo stato $m$ si può giungere quindi di nuovo allo stato $m$.
\subsection{Scansione del testo}
La scansione del testo inizia da uno stato iniziale $j_0 = 0$. Partendo da questo stato, leggo il simbolo in posizione $i$ del testo ($T[i]$) e mi sposto nello stato $j_i$ attraverso la funzione di transizione $\delta(j_{i - 1}, T[i])$.
\begin{esempio}
     Consideriamo il testo $T$:
    \begin{table}[!ht]
    \centering
        \begin{tabular}{ccccccc}
            1 & 2 & 3 & 4 & 5 & 6 & 7 \\ \hline
            \multicolumn{1}{|c|}{c} & \multicolumn{1}{c|}{a} & \multicolumn{1}{c|}{b} & \multicolumn{1}{c|}{a} & \multicolumn{1}{c|}{c} & \multicolumn{1}{c|}{a} & \multicolumn{1}{c|}{b} \\ \hline
        \end{tabular}
    \end{table}
    
    e il pattern $P$:
    \begin{table}[!ht]
    \centering
        \begin{tabular}{ccccccc}
            1 & 2 & 3 & 4 \\ \hline
            \multicolumn{1}{|c|}{a} & \multicolumn{1}{c|}{c} & \multicolumn{1}{c|}{a} & \multicolumn{1}{c|}{c} \\ \hline
        \end{tabular}
    \end{table}
    
    Su cui è stata definita la seguente funzione di transizione $\delta$:
    \begin{table}[!ht]
        \centering
        \begin{tabular}{|>{\columncolor[HTML]{EFEFEF}}c |c|c|c|} \hline
            $\delta$ & \cellcolor[HTML]{EFEFEF}\textbf{a} & \cellcolor[HTML]{EFEFEF}\textbf{b} & \cellcolor[HTML]{EFEFEF}\textbf{c} \\ \hline
            \textbf{0} & 1 & 0 & 0 \\ \hline
            \textbf{1} & 1 & 0 & 2 \\ \hline
            \textbf{2} & 3 & 0 & 0 \\ \hline
            \textbf{3} & 1 & 0 & 4 \\ \hline
            \textbf{4} & 3 & 0 & 0 \\ \hline
        \end{tabular}
    \end{table}

    La scansione del testo $T$ cercando le occorrenze esatte del pattern $P$ mi permette di ottenere il risultato riportato in figura \ref{fig:scansione}
    \begin{figure}[!ht]
        \centering
        \includegraphics[width=0.5\textwidth]{img/pattern/ScansioneTesto.png}
        \caption{Risultato della scansione del testo}
        \label{fig:scansione}
    \end{figure}
\end{esempio}
A questo punto è necessario identificare un'occorrenza esatta del pattern $P$ nel testo $T$. Per risolvere questo, partiamo dalla successione di stati che abbiamo definito per la scansione del testo. Essa corrisponde a una successione di posizioni su $P$, o in altre parole, a una successione di lunghezze di prefissi del pattern $P$.
\begin{teorema}
    $j_i$, con $0 \leq i \leq n$, è la lunghezza del \textbf{più lungo} prefisso di $P$ uguale a una sottostringa di $T$ che finisce in posizione $i$.
\end{teorema}
\begin{dimostrazione}
    È possibile dimostrare il teorema precedente con una dimostrazione per induzione:
    \begin{enumerate}
        \item \textbf{Caso base}: per lo stato $j_0 = 0$ il teorema è banalmente dimostrabile, in quanto il prefisso di lunghezza $0$ è il prefisso nullo che è sottostringa di $T$.
        \item \textbf{Passo induttivo}: se $j_{i-1}$ è la lunghezza del più lungo prefisso di $P$ uguale alla sottostringa di $T$ che finisce in posizione $i-1$, allora $j_i$ è la lunghezza del più lungo prefisso di $P$ uguale alla sottostringa di $T$ che finisce in posizione $i$.

        Per dimostrare il passo induttivo ci basiamo sulla seguente ipotesi: $j_{i-1}$ è la lunghezza del più lungo prefisso di $P$ uguale a una sottostringa di $T$ che finisce in posizione $i-1$.
        \begin{itemize}
            \item \textbf{Caso 1}: $j_{i - 1} < m$ e $P[j_{i - 1} + 1] = T[i]$ questo implica $j_i = \delta(j_{i - 1}, T[i]) = j_{i - 1} + 1$
            \begin{itemize}
                \item $j_{i - 1} \neq 0$: la tesi è confermata
                \item $j_{i - 1} = 0$ vuol dire che il carattere corrisponde con il carattere iniziale del pattern
            \end{itemize}
            \item \textbf{Caso 2}: $j_{i - 1} = m$ oppure $P[j_{i - 1} + 1] \neq T[i]$ questo implica $j_i = \delta(j_{i - 1}, T[i]) = k$ dove $k$ è la lunghezza del bordo di $P[1, j_{i - 1}]T[i]$.
            \begin{itemize}
                \item $0 < j_{i - 1} < m$: in questo caso il valore di $k$ mi rappresenta una parte del testo per cui ho già verificato un'occorrenza esatta, questo mi viene garantito dalla definizione di bordo.
                \item $j_{i - 1} = 0$ vuol dire che sono rimasto nello stato 0.
                \item $j_{i - 1} = m$ ho trovato un’occorrenza esatta del pattern, inoltre per evitare di perdere delle occorrenze mi sposto in base alla lunghezza del bordo del pattern concatenato con il carattere successivo.
            \end{itemize}
        \end{itemize}
    \end{enumerate}
\end{dimostrazione}
Questo teorema mi fornisce la garanzia che non sto perdendo delle occorrenze. Inoltre, posso trovare la posizione di inizio dell'occorrenza esatta come $i - j + 1$. Nel caso in cui $j_i = m$ ho identificato un'occorrenza esatta del pattern $P$.

Possiamo riassumere la scansione del testo come:
\begin{enumerate}
    \item Si parte dallo stato iniziale $0$ e si effettua una scansione di $T$ dal primo all'ultimo simbolo.
    \item Per ogni posizione $i$ di $T$ si effettua la transizione dallo stato corrente $j_c$ al nuovo stato $j_f = \delta(j_c, T[i])$
    \item Ogni volta che lo stato $j_f$ è lo stato accettante ($m$), viene prodotta in output l'occorrenza $i - m + 1$
\end{enumerate}
\begin{algorithm}
  \begin{algorithmic}
    \Function{ASF\_exact\_occurrences}{$\delta, T, m$}
    \State $n \gets |T|$
    \State $j \gets 0$
    \For{$i \gets 1 \ \text{to} \ n$}
        \State $j \gets \delta(j, T[i])$
        \If{$j = m$}
            \State $\text{\textbf{Output}} \ i - m + 1$ 
        \EndIf
    \EndFor
    \EndFunction
  \end{algorithmic}
  \caption{Algoritmo per la ricerca esatta con Automa a Stati Finiti}
\end{algorithm}
Questo algoritmo viene eseguito in tempo $\theta(n)$.
\begin{esempio}
    Consideriamo il testo $T$:
    \begin{table}[!ht]
    \centering
        \begin{tabular}{ccccccccccccc}
            1 & 2 & 3 & 4 & 5 & 6 & 7 & 8 & 9 & 10 & 11 & 12 & 13 \\ \hline
            \multicolumn{1}{|c|}{c} & \multicolumn{1}{c|}{a} & \multicolumn{1}{c|}{b} & \multicolumn{1}{c|}{a} & \multicolumn{1}{c|}{c} & \multicolumn{1}{c|}{a} & \multicolumn{1}{c|}{c} & \multicolumn{1}{c|}{b} & \multicolumn{1}{c|}{a} & \multicolumn{1}{c|}{c} & \multicolumn{1}{c|}{a} & \multicolumn{1}{c|}{b} & \multicolumn{1}{c|}{a} \\ \hline
        \end{tabular}
    \end{table}
    
    e il pattern $P$:
    \begin{table}[!ht]
    \centering
        \begin{tabular}{ccccccc}
            1 & 2 & 3 & 4 & 5 & 6 & 7 \\ \hline
            \multicolumn{1}{|c|}{a} & \multicolumn{1}{c|}{c} & \multicolumn{1}{c|}{a} & \multicolumn{1}{c|}{c} & \multicolumn{1}{c|}{b} & \multicolumn{1}{c|}{a} & \multicolumn{1}{c|}{c} \\ \hline
        \end{tabular}
    \end{table}
    
    Su cui è stata definita la seguente funzione di transizione $\delta$:
    \begin{table}[!ht] 
    \centering
        \begin{tabular}{|>{\columncolor[HTML]{EFEFEF}}c |c|c|c|}\hline
            $\delta$ & \cellcolor[HTML]{EFEFEF}\textbf{a} & \cellcolor[HTML]{EFEFEF}\textbf{b} & \cellcolor[HTML]{EFEFEF}\textbf{c} \\ \hline
            \textbf{0} & 1 & 0 & 0 \\ \hline
            \textbf{1} & 1 & 0 & 2 \\ \hline
            \textbf{2} & 3 & 0 & 0 \\ \hline
            \textbf{3} & 1 & 0 & 4 \\ \hline
            \textbf{4} & 3 & 5 & 0 \\ \hline
            \textbf{5} & 6 & 0 & 0 \\ \hline
            \textbf{6} & 1 & 0 & 7 \\ \hline
            \textbf{7} & 3 & 0 & 0 \\ \hline
        \end{tabular}
    \end{table}

    Otteniamo la seguente esecuzione dell'algoritmo:
    \begin{figure}[!ht]
        \centering
        \includegraphics[width=0.3\textwidth]{img/pattern/ASF.png}
        \caption{Esecuzione dell'algoritmo per la ricerca esatta con Automa a Stati Finiti}
        \label{fig:enter-label}
    \end{figure}
\end{esempio}
\subsection{Calcolo della funzione di transizione $\delta$}
L'algoritmo più semplice che permette di calcolare i valori della funzione di transizione $\delta$ consiste nell’applicare la funzione di transizione $\delta$ per come è definita senza sfruttare i valori che sono già stati computati.
\begin{algorithm}
  \begin{algorithmic}
    \Function{Trivial-build-transition-function}{$P$}
    \State $m\gets |P|$
    \State $\delta \gets empty\_table (m + 1) \times | \Sigma|$
    \For{$j \gets 0 \ \text{to} \ m - 1$}
        \State $\delta(j, P[j + 1]) \gets j + 1$
    \EndFor
    \For{$j \gets 0 \ \text{to} \ m$}
        \For{$\sigma \in \Sigma$}
            \State $\delta(j, \sigma) \gets |B(P[1, j]\sigma)|$
        \EndFor
    \EndFor
    \State $\text{\textbf{return}} \ \delta$
    \EndFunction
  \end{algorithmic}
  \caption{Algoritmo banale per il calcolo della funzione di transizione $\delta$}
\end{algorithm}
Questo algoritmo richiede un tempo nel caso peggiore pari a $\mathcal{O}(m^3 | \Sigma|)$, questo è dovuto al fatto che per calcolare il bordo è necessario un tempo, nel caso peggiore pari a $\mathcal{O}(m^2)$.

Definiamo $\delta_j$ come la funzione di transizione di $P[1, j]$, ovvero come una funzione:
\begin{equation}
    \delta_j: \{0, 1, \dots, j\} \times \Sigma \to \{0, 1, \dots, j\}
\end{equation}
per questa funzione possiamo definire due casi particolari:
\begin{itemize}
    \item $\delta_0$ ovvero la funzione di transizione di $P[1, 0]$ che per definizione è $\varepsilon$, quindi il valore di tale funzione è sempre $0$.
    \item $\delta_m$ ovvero la funzione di transizione di $P[1, m]$ la quale corrisponde precisamente alla funzione di transizione del pattern $P$.
\end{itemize}
Possiamo definire il calcolo della funzione di transizione $delta$ per un pattern $P$ di lunghezza $m$ utilizzando l'induzione nel seguente modo:
\begin{itemize}
    \item Caso base: calcolo $\delta_0$
    \item Passo induttivo: calcolo $\delta_j$ da $\delta_{j - 1}$ in questo caso dobbiamo distinguere il caso in cui $j = 1$ e i restanti.
    \begin{itemize}
        \item Nel caso in cui $j = 1$, possiamo definire la funzione di transizione come:
        \begin{enumerate}
            \item Prendo il valore $0$ contenuto nella cella della riga $0$ di $\delta_0$ in corrispondenza del simbolo $P[1]$
            \item Sostituisco il valore $0$ con il valore $1$ (stato successivo a $0$).
            \item Aggiungo una nuova riga (corrispondente allo stato $1$)
            \item Copio la riga che corrisponde allo stato 0 nella riga che corrisponde allo stato $1$
            \item Rinomino $\delta_0$ in $\delta_1$
        \end{enumerate}
        \item Mentre nel caso in cui $j \neq 1$ possiamo definire la funzione di transizione come:
        \begin{enumerate}
            \item Prendo il valore $k$ contenuto nella cella della riga $j-1$ di $\delta_{j-1}$ in corrispondenza del simbolo $P[j]$
            \item Sostituisco il valore $k$ con il valore $j$ (stato successivo a $j - 1$)
            \item Aggiungo una nuova riga (corrispondente allo stato $j$)
            \item Copio la riga che corrisponde allo stato $k$ nella riga che corrisponde allo stato $j$
            \item Rinomino $\delta_{j-1}$ in $\delta_j$
        \end{enumerate}
    \end{itemize}
\end{itemize}
\begin{nota}
    Dimostrazione tramite esempi su slide.
\end{nota}
Con queste informazioni possiamo definire un algoritmo che mi permette di calcolare la funzione di transizione $\delta$ in tempo $\theta(m \cdot |\Sigma|)$. Tale algoritmo sfrutta le informazioni precedentemente calcolate.
\begin{algorithm}[!ht]
  \begin{algorithmic}
    \Function{Build-transition-function}{$P$}
    \State $m\gets |P|$
    \State $\delta \gets empty\_table (m + 1) \times | \Sigma|$
    \For{$\sigma \in \Sigma$}
        \State $\delta(0, \sigma) \gets 0$
    \EndFor
    \For{$j \gets 1 \ \text{to} \ m$}
        \State $k \gets \delta(j - 1, P[j])$
        \State $\delta(j - 1, P[j]) \gets j$
        \For{$\sigma \in \Sigma$}
            \State $\delta(j, \sigma) \gets \delta(k, \sigma)$
        \EndFor
    \EndFor
    \State $\text{\textbf{return}} \ \delta$
    \EndFunction
  \end{algorithmic}
  \caption{Algoritmo per il calcolo della funzione di transizione $\delta$}
\end{algorithm}
\newpage
\section{Algoritmo di Knuth-Morris-Pratt}
Questo algoritmo per la ricerca esatta è basato su un'analisi del pattern. Si hanno due fasi:
\begin{enumerate}
    \item \textbf{Preprocessing del pattern}: questa fase consiste nel calcolo della prefix function $\phi$, che è una funzione che associa ad ogni posizione del pattern la lunghezza del più lungo prefisso del pattern che è anche un suffisso del pattern. Questa funzione è calcolata in tempo lineare rispetto alla lunghezza del pattern $\mathcal{O}(m)$.
    \item \textbf{Scansione del testo}: questa fase consiste nel confrontare il pattern con il testo cercando di identificare le occorrenze esatte di esso. Questa fase è eseguita in tempo lineare rispetto alla lunghezza del testo $\mathcal{O}(n)$.
\end{enumerate}
La \textbf{prefix function} o funzione di fallimento $\phi$ è definita come segue:
\begin{equation}
    \phi: \{0, 1, \dots m\} \to \{-1, 0, \dots m-1\}
\end{equation}
Essa è definita come segue:
\begin{equation}
    \phi(j) = \begin{cases} |B(P[1, j])| & \text{se } 1 \leq j \leq m \\-1 & \text{se } j = 0 \end{cases}
\end{equation}
Questo algoritmo è un’evoluzione dell'algoritmo banale per la ricerca delle occorrenze di un pattern in un testo. In particolare, l'algoritmo KMP consiste in:
\begin{enumerate}
    \item Viene usata una finestra $W$ di lunghezza $m$ che scorre sul testo $T$ da sinistra a destra con posizione iniziale $i = 1$.
    \item Si confrontano i simboli di $P$ con i corrispondenti simboli di $T$ all'interno della finestra $W$ andando da sinistra a destra e partendo dal primo simbolo di $P$.
    \item Non appena si incontra un mismatch oppure ogni simbolo di $P$ ha un match con il corrispondente simbolo in $W$ (i è occorrenza esatta), $W$ viene spostata a destra nella posizione $p = i + j - \phi(j - 1) - 1$, dove $j$ è l'indice del simbolo di $P$ che ha causato il mismatch, mentre $i$ è la posizione iniziale di $W$.
    \item L'ultima posizione di $W$ è $n - m + 1$.
\end{enumerate}
Riassumendo:
\begin{itemize}
    \item $W$ viene spostata dalla posizione $i$ alla posizione $p$, dove: $$p = i + j - \phi(j - 1) - 1$$ con $j$ indice del simbolo di $P$ che ha causato il mismatch per $W$ in posizione $i$.
    \item Il confronto riparte dal simbolo di $P$ in posizione $j = \phi(j - 1) + 1$ e dal simbolo di $T$ in posizione $i + j - 1$.
\end{itemize}
Nel caso in cui $j = 1$, allora $p = i + 1$ e il confronto riparte dal primo simbolo di $P$ e dal simbolo di $T$ in posizione $i + 1$.
\begin{nota}
    Chiaramente il confronto riparte dalle posizioni $i + 1$ su $T$ e $1$ su P, ma dire che riparte dalla posizione $i$ su $T$ e dalla posizione $0$ su $P$ implicitamente fa riferimento ad un confronto iniziale fittizio tra $T[i]$ e $P[0]$ (simbolo inesistente) che di default viene considerato un match.
\end{nota}
Nel caso in cui $j = m + 1$, allora $p = i + m - \phi(m) - 1$ e il confronto riparte dal primo simbolo di $P$ e dal simbolo di $T$ in posizione $i + m - \phi(m)$.

\begin{table}[!ht]
    \centering
    \begin{tabular}{|l|l|l|}
    \hline
         & Automa a stati finiti & KMP \\ \hline
        Preprocessing di P & $\mathcal{O}(m | \Sigma |)$ & $\mathcal{O}(m)$ \\ \hline
        Scansione di T & $\mathcal{O}(n)$ & $\mathcal{O}(n)$ \\ \hline
        Spazio & $\mathcal{O}(m | \Sigma |)$ & $\mathcal{O}(m)$ \\ \hline
    \end{tabular}
\end{table}

Automa:
\begin{itemize}
    \item Efficiente per pattern piccoli.
    \item Richiede più tempo e memoria per pattern grandi.
    \item Ricerca di P in testi diversi
\end{itemize}
KMP:
\begin{itemize}
    \item Efficiente per pattern grandi
    \item Richiede più tempo per pattern piccoli
\end{itemize}

%% Mettere il codice solo se lo chiede
\section{Algoritmo di Baeza-Yates e Gonnet}
La ricerca esatta effettuata attraverso questo algoritmo effettua un confronto tra i simboli del pattern e del testo in maniera non esplicita, ovvero non confronta carattere per carattere. In questo algoritmo vengono effettuate in parallelo operazioni bit a bit su word di bit, viene anche chiamato \textit{algoritmo bit parallel}.

Questo algoritmo segue il paradigma \textbf{shift-and} ovvero compie fondamentalmente due sole operazioni:
\begin{itemize}
    \item \textbf{Shift} dei bit.
    \item \textbf{AND} logico tra i bit.
\end{itemize}
Come gli algoritmi visti fin ora, anche in questo caso possiamo descrivere il suo funzionamento attraverso due fasi:
\begin{itemize}
    \item Preprocessing del pattern $P$ nel quale vengono calcolate $|\Sigma|$ \textbf{words} ognuna di $m$ bit. Questa operazione viene eseguita in tempo $\theta(|\Sigma| + m)$.
    \item Scansione del testo $T$ per cercare le occorrenze esatte del pattern $P$. Questa operazione viene eseguita in tempo $\theta(n)$.
\end{itemize}
\subsection{Word e Operatori}
\begin{definizione}[\textbf{Word di bit}]
    Una \textbf{word di bit} è un gruppo di bit che viene trattato come un'unità la cui dimensione può variare e che rappresenta un valore di un certo tipo, come ad esempio un numero o un carattere. In una word, il bit più a destra è quello meno significativo, mentre quello più a sinistra è quello più significativo.
\end{definizione}
Sulle word si possono eseguire delle operazioni \textit{bit a bit}, ovvero si esegue un'operazione tra i bit corrispondenti di due o più words di bit della stessa lunghezza. Il valore restituito da queste operazioni è una nuova word in cui ogni bit è il risultato dell'operazione tra i bit corrispondenti nelle word in input. Tra queste operazioni abbiamo:
\begin{itemize}
    \item Congiunzione logica $\to$ AND. Questa operazione è implementata come:
    \begin{equation}
        w = w_1 \ \text{AND} \ w_2
    \end{equation}
    restituisce una word $w$ tale che:
    \begin{itemize}
        \item $w[j] = 1$ se e solo se $w_1[j] \ \text{AND} \ w_2[j] =  1$.
        \item $w[j] = 0$ altrimenti.
    \end{itemize}
    \item Disgiunzione logica (inclusiva )$\to$ OR. Questa operazione è implementata come:
    \begin{equation}
        w = w_1 \ \text{OR} \ w_2
    \end{equation}
    restituisce una word $w$ tale che:
    \begin{itemize}
        \item $w[j] = 1$ se e solo se $w_1[j] \ \text{OR} \ w_2[j] =  1$.
        \item $w[j] = 0$ altrimenti.
    \end{itemize}
    \item Shift dei bit di una posizione a destra con bit più significativo a 0 $\to$ RSHIFT. Questa operazione è implementata come:
    \begin{equation}
        w = \text{RSHIFT}(w_1)
    \end{equation}
    restituisce una word $w$ tale che:
    \begin{itemize}
        \item $w[j] = w_1[j - 1]$ se $j \geq 2$.
        \item $w[1] = 0$ altrimenti.
    \end{itemize}
    \item Shift dei bit di una posizione a destra con bit più significativo a 1 $\to$ RSHIFT1. Questa operazione viene implementata come un RSHIFT seguito da un OR con una maschera in cui nella prima posizione è presente 1 e nelle altre 0.
\end{itemize}
\subsection{Preprocessing}
Dato un pattern di lunghezza $m$ e $\sigma$ in $\Sigma$, $B_{\sigma}$ è una word di $m$ bit tale che:
\begin{equation}
    B_{\sigma}[j] = 1 \iff P[j] = \sigma
\end{equation}
Viene creata una word per ogni simbolo dell'alfabeto $\Sigma$ e viene memorizzata in una tabella $B$. Con questa rappresentazione posso effettuare le query del tipo: "il simbolo in posizione $j$ di $P$ è uguale a un certo simbolo $\sigma$".

Vediamo ora come calcolare la tabella $B$:
\begin{enumerate}
    \item tutte le word $B_{\sigma}$ vengono inizializzate a $m$ bit a $0$.
    \item viene creata una maschera $M$ di $m$ bit tutti uguali a $0$ tranne il più significativo che è uguale a $1$.
    \item si esegue una scansione di $P$ da sinistra a destra, e per ogni posizione $j$ vengono eseguite le due operazioni bit a bit:
    \begin{itemize}
        \item $B_{P[j]} = M \ \textbf{OR} \ B_{P[j]}$
        \item $M = \textbf{RSHIFT} (M)$
    \end{itemize}
\end{enumerate}
L'algoritmo per calcolare la tabella richiede un tempo pari a $\theta(|\Sigma| + m)$ ed è il seguente:
\begin{algorithm}
  \begin{algorithmic}
    \Function{Compute-table-B}{$P$}
        \State $m \gets |P|$
        \State $B \gets \text{empty table of} \ |\Sigma| \ \text{words} \ B_{\sigma}$
        \For{$\sigma \in \Sigma$}
            \State $B_{\sigma} \gets 00\dots0$
        \EndFor
        \State $M \gets 10\dots0$
        \For{$\sigma \in \Sigma$}
            \State $\sigma \gets P[j]$
            \State $B_{\sigma} \gets M \ \text{OR} B_{\sigma}$
            \State $M = \textbf{RSHIFT} (M)$
        \EndFor
        \State \textbf{return} $B$
    \EndFunction
  \end{algorithmic}
  \caption{Algoritmo per il calcolo della tabella $B$}
\end{algorithm}
\subsection{Scansione del testo}
La procedura per la scansione del testo è rappresentata come: 
\begin{enumerate}
    \item Il testo $T$ viene scandito dalla prima all'ultima posizione.
    \item Per ogni posizione $i$ del testo $T$ viene calcolata una word $D_i$ di $m$ bit.
    \item Ogni volta che in $D_i$ il bit meno significativo è uguale a $1$, allora $i - m + 1$ è occorrenza esatta di $P$ in $T$.
\end{enumerate}
Dobbiamo ora definire cosa si intende con la word $D_i$. Prima di fare ciò dobbiamo definire $P[1,j] = suff(T[1,i])$ ovvero $P[1,j]$ è uguale a un suffisso di $T[1,i]$.
\begin{definizione}[\textbf{Word $D_i$}]
    Dati $P$ lungo $m$ e $T$ lungo $n$, $D_i$ ($0 \leq i \leq n$) è una word di $m$ bit tale che:
    \begin{equation}
        D_i[j] = 1 \iff P[1,j] = suff(T[1,i])
    \end{equation}
    Inoltre, sappiamo per definizione che:
    \begin{itemize}
        \item $D_0 = 00\dots0$ dato che $P[1, j] \neq suff(T[1, 0]) \ \forall j$
        \item $D_i[m] = 1$ se e solo se $P[1, m] \neq suff(T[i-m + 1, i])$ ovvero si ha un’occorrenza esatta di $P$ in $T$ nella posizione $i - m + 1$.
    \end{itemize}
\end{definizione}
A questo punto possiamo definire la scansione del testo $T$ come:
\begin{itemize}
    \item Inizia dalla word $D_0 = 00\dots0$
    \item Per ogni $i$ da $1$ a $n$ calcola la word $D_i$ a partire dalla word $D_{i-1}$.
    \item Ogni volta che $D_i$ ha il bit meno significativo uguale a $1$, viene prodotta in output l'occorrenza $i- m + 1$.
\end{itemize}
Vediamo ora come si calcola il valore di $D_i$ a partire da $D_{i - 1}$:
\begin{itemize}
    \item Se $J =  1$ allora posso calcolarla come:
    \begin{equation}
        D_i[j] = 1 \text{\textbf{AND}} B_{T[i]}[1]
    \end{equation}
    Si aggiunge $1$ per semplificare le operazioni bit a bit.
    \item Se $j > 1$ allora posso calcolarla come:
    \begin{equation}
        D_i[j] = D_{i - 1}[j - 1] \text{\textbf{AND}} B_{T[i]}[j]
    \end{equation}
\end{itemize}
In questo modo stiamo ancora aggiornando un bit alla volta, ma sfruttando le operazioni bit a bit si possono aggiornare tutti in tempo costante nel seguente modo:
\begin{equation}
    D_i = \text{\textbf{RSHIFT1}}(D_{i - 1}) \ \text{\textbf{AND}} \ B_{T[i]}
\end{equation}
Vediamo ora come implementare l'algoritmo che effettua la scansione del testo in tempo $\theta(n)$
\begin{algorithm}
  \begin{algorithmic}
    \Function{BYG}{$B, T$}
        \State $n \gets |T|$
        \State $D \gets 00\dots0$
        \State $M \gets 00\dots01$
        \For{$i \gets 1 \ \text{to} \ n$}
            \State $\sigma \gets T[i]$
            \State $D \gets \text{\textbf{RSHIFT1}}(D_{i - 1}) \ \text{\textbf{AND}} \ B_{T[i]}$
            \If{$(D \ \text{\textbf{AND}} M) = M$}
                \State \text{output} $i - m + 1$
            \EndIf
        \EndFor
    \EndFunction
  \end{algorithmic}
  \caption{Algoritmo per la scansione del testo}
\end{algorithm}

\chapter{Appendice}
\section{Formulario}
\subsection{Pattern Matching algoritmo di Knuth-Morris-Pratt}
Notazione utilizzata:
\begin{itemize}
    \item $P$ = pattern
    \item $T$ = testo
    \item $m$ = lunghezza del pattern
    \item $n$ = lunghezza del testo
    \item $B(P[1, j])$ = bordo di $P[1, j]$
    \item $p$ = posizione della nuova finestra
    \item $i$ = posizione della finestra
    \item $j$ = posizione della scansione nel pattern
\end{itemize}
Formule utili:
\begin{itemize}
    \item Posizione della nuova finestra:
          \begin{equation}
              p = i + j - \phi(j - 1) - 1
          \end{equation}
    \item Posizione da cui riparte la scansione del testo:
          \begin{equation}
              i = i + j - 1
          \end{equation}
    \item Posizione da cui riparte la scansione del pattern:
          \begin{equation}
              j = \phi(j - 1) + 1
          \end{equation}
    \item Calcolo della prefix function:
          \begin{equation}
              \phi(j) = \begin{cases}
                  |B(P[1, j])| & \text{se } 1 \leq j \leq m \\
                  -1           & \text{se } j = 0
              \end{cases}
          \end{equation}
\end{itemize}
\subsection{Pattern Matching algoritmo di Baeza-Yates e Gonnet}
Notazione utilizzata:
\begin{itemize}
    \item $P$ = pattern
    \item $T$ = testo
    \item $m$ = lunghezza del pattern
    \item $n$ = lunghezza del testo
    \item $D_i$ = $i$-esima word
\end{itemize}
Formule utili:
\begin{itemize}
    \item Posizione dell'occorrenza esatta:
          \begin{itemize}
              \item $i = i - m + 1$
          \end{itemize}
    \item Calcolo della $i$-esima word:
          \begin{equation}
              D_i = \textbf{RSHIFT1}(D_{j - 1}) \textbf{AND} B_{\sigma}
          \end{equation}
\end{itemize}

\end{document}

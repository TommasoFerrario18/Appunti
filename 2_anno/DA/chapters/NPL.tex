\chapter{NPL}

NPL si occupa di introdurre la semantica e quindi le relazioni tra le parole.
Per esempio su un dataset di tweet possiamo analizzare le emozioni, capire la polarità
o il topic.

NPL la parte fondamentale è trovare una rappresentazione utilizzabile. Questo non è
una cosa banale:
\begin{itemize}
    \item spesso ci serve un contesto per rappresentare un testo che abbiamo
    \item In aggiunta ci può essere del rumore nei dati (dati errati o non fedeli)
    \item ambiguità nelle frasi, per esempio abbiamo modi di dire 
    (si risolve effettuando l'analisi sintattici, part-of-speech tagginxg)
    \item ambiguità sintattica part-tree-disambiguation
\end{itemize} 

Le librerie che forniscono sistemi di disambiguità restituiscono l'albero della 
frase più probabile. 

Gli NLP sono utili per:
\begin{itemize}
    \item sentiment analysis (hidden semantics)
    \item named-entity recognition, identificare le entità nel contesto
    \item named-entity linking, relazionare entità con la loro scheda nella base 
    di conoscienza (ex: entità con la pagina wikipedia)
\end{itemize}

\section{Sentiment analysis}

\begin{definizione}
    Analisi del testo per analizzare le emozioni che trasparano dalla frase.
\end{definizione}

I passi possono essere diversi, si può partire dal testo, suddividere in:
\begin{itemize}
    \item oggettiva: si sta esprimendo un fatto
    \item soggettiva (emozione): si sta esprimendo un'emozione 
    \begin{itemize}
        \item positivo
        \item negativo
        \item neutrale: complesso perché spesso è molto vicino ad una frase oggettiva o
        spesso si pesa la frase tra positiva e negativa, quindi se è bilanciata è neutrale.
    \end{itemize}
\end{itemize}
Un altro problema di riconoscere il senitment analysis è capire se c'è dell'ironia
o del sarcasmo nel testo.

Spesso si classifica il testo su 8 emozioni che sono graduali.

In ogni caso bisogna trattare la rappresentazione del testo, questa può essere 
fatta utilizzando un vocabolario e segnaliamo con un flag le parole che sono presenti
in una frase. Abbiamo quindi un dizionario con un array di bit. Questo problema 
è che il dizionario sarà sparso, enorme, perdiamo l'ordinamento delle parole nelle 
frasi.

Un ulterio metodo è  BagOfWorld usare la rappresentazione tramite vettore in uno spazio.
Il problema è che manca la composizionalità delle parole

Possiamo rappresentare usando una rappresentazione deep basata su word2vec, si rappresentare
il termine in base a quello che viene prima e dopo (simile a bagofworld ma permette 
di mantenere la composizionalità):
\begin{itemize}
    \item skip-gram: parte da una parola del documento e cerca di predire l'intorno della parola.
    Rappresenteremo la singola parola del documento con un one-hot-vector, un bitvector (lungo quanto il
    il vocabolario) con un 
    bit a $1$ nella posizione associata alla parola all'interno del vocabolario.
    Successivamente costruiamo una NN (Encoder Decoder) che preso in input il bitvector, in output
    dobbiamo avere un vettore lungo tanto quanto il vocabolario con un peso associato 
    ad ogni parola. La rappresentazione vettoriale coincide con la rappresentazione 
    nell'hidden layer, la sua dimensione è decisa in fase di construzione della 
    rete. Il documento viene rappresentato dal vettore media dei vettori dei termini
    presenti nel documento. La differenza dalle rappresentazioni più vecchie è che 
    abbiamo alla fine un vettore di dimensione $n$ selezionato dalla rete. Il
    problea è che non consideriamo il contesto del mondo.
    Esistono metodi più efficaci per rappresentare il testo come $USE$ e $BERT$. 
    \item cbow: cerca di predire la parola sulla base dell'intorno delle parole vicine
    (ex: fill in the blanks)
\end{itemize}
Questi permettono di avere parole con lo stesso significato o simini vine nello spazio.

Successivamente si deve identificare la polarità:
\begin{itemize}
    \item se usiamo BagOfWorld possiamo avere un dizionario di termini associati
    ai livelli di polarità, possiamo contare i vettori del testo che sono uguali ai termini
    associati alle emozioni, restiutisco l'emozione col conteggio maggiore. Se 
    le accezioni hanno lo stesso numero di vettori allora sono neutre. 
    Importante è scegliere bene il lessico e lo studio del lessico perché parole 
    possono essere positive o negative in base al contesto
    \item possiamo utilizzare l'approccio deeplearning  per classificare l'emozione
\end{itemize}


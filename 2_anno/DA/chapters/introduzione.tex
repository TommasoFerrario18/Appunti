\chapter{Introduzione}
Si tratterà di:
\begin{itemize}
    \item \textbf{network analysus}: analisi delle proprietà strutturali, centralità sui 
    grafi
    \item \textbf{natural language processing}: si studieranno modelli e tecniche per 
    processare il linguaggio naturale
\end{itemize}
Nel progetto non interessa come si scrive il codice sorgente, ma la parte di analisi.
L'obiettivo è di far capire ad una persona non del settore il risultato per prendere
delle decisioni correttamente.

Le attività di analytics vengono classificate rispetto: il valore del task e la 
sua difficoltà.
Categorizzeremo i task in:
\begin{itemize}
    \item \textbf{Descriptive Analytics}: clustering ex: segmentazione della clientela, generalmente 
    non supervisionati. 
    \item \textbf{Diagnostic Analytics}: perché succede qualcosa
    \item \textbf{Predicitve Analytics}: predizione e classificazione
    \item \textbf{Prescriptive Analytics}: il mondo evolve nel tempo e le decisioni nel 
    passato evolvono anche nel futuro.
\end{itemize} 

% TODO: aggiungere il grafico

\section*{Data pre-processing}
Quando dobbiamo modellare un dataset possiamo avere diversi problemi:
\begin{itemize}
    \item \textbf{valori mancanti}: dati non disponibili, non sono stati registrati,
    risultati di malfunzionamenti dei sensori, può essere stato cancellato. Importante
    gestirli perché rendono meno efficienti tutti i processi successivi, bias sui
    risultati perché introduciamo delle distorsioni nel dataset, complicazioni nel modello
    perché spesso non prevedono valori mancanti. Esistono 3 diverse categorie di valori
    mancanti:
    \begin{itemize}
        \item \textbf{MCAR}: il valore mancante non dipende dal dato osservato o
        da altri valori mancanti
        \item \textbf{MAR}: il dato può essere inferito dai dati osservati ma non dipende
        da quelli mancantil
        \item $\dots$
    \end{itemize}
    Per gestirli abbiamo tre metodi:
    \begin{itemize}
        \item ignorare gli esempi, approccio semplice ma applicabile solo quando 
        abbiamo pochi dati mancati e tanti dati osservati
        \item conversione dei valori mancanti in uno nuovo. Si assegna un significato
        al valore mancante e quindi un possibile valore di una variabile aleatoria
        allora si considera come un'informazione il fatto che manchi il valore.
        \item metodi di imputazione: assegno un valore basato sul resto delle osservazioni del dataset
        ma non è importate sapere che manchi e quindi meglio un valore che lo sostituisca. Utili per MCAR e MAR
        ma non utili per NMAR. Esistono diverse strategie di imputazione in base 
        alla tipologia del valore mancante:
        \begin{itemize}
            \item continuo: calcolo la media tra tutti gli altri (si assume che i dato sia distribuito normalmente)
            \item discreta o categorica: sostituiamo con un valore frequente
        \end{itemize}
        Possiamo pensare di sostituire con la media o il valore frequente ragionando
        per classe, ovvero media dei valori o il valore più frequente degli esempi 
        della stessa classe. Questo secondo metodo si usa in un contesto supervisionato.
        Altri metodi si basano su KNN (uguale all'algoritmo ML), predizione del 
        valore in base al valore medio o frequente dei k valori degli esempi vicini.
        Un altro metodo è utilizzare altri algoritmi di ML, ma significa 
        definire un modello per la previsione dei valori mancanti di ciascuna feature.
    \end{itemize}

    \item \textbf{rumorosità dei dati}: valori che possono essere outliers ma 
    in realtà spesso sono dovuti dalla mancanza di dati. Si utilizzano tecniche di
    discretizzazione dei valori continui (pu):
    \begin{itemize}
        \item \textbf{equal-width}: partiziona dell'intervallo dell'attributo in $N$ intervalli della stessa
        distanza. Mantiene sempre gli outliers.
        \item \textbf{equal-depth}: partiziona dell'intervallo dell'attributo in $N$ intervalli della stessa
        frequenza di dati. Genera una distribuzione uniforme ma non mantiene 
        gli outliers.
    \end{itemize}
    I due metodi si usano in base all'impatto che hanno sui modelli. Il secondo 
    metodo abbassa la varianza della feature nascondendo gli outlier, mentre il primo aumenta la varianza 
    e quindi evidenza gli outlier.
    %TODO: aggiungere il grafico delle distribuzioni dei due metodi
    \item  \textbf{dataset sblilanciati}
\end{itemize}

In quasi tutti i modelli non si possono gestire i valori mancanti, eccetto alcuni
casi particolari.
\chapter{Introduzione}
Durante questo corso si andranno a studiare:
\begin{itemize}
    \item \textbf{Network Analysis}: analisi delle proprietà strutturali delle
          reti, come ad esempio la centralità sui grafi.
    \item \textbf{Natural Language Processing} (\textbf{NLP}): si studieranno
          modelli e tecniche per processare il linguaggio naturale.
\end{itemize}
\begin{nota}
    L'obiettivo di questo corso è di capire come si analizzano i dati, non come
    si scrive il codice. Lo scopo del progetto è di far capire ad una persona non
    del settore il risultato per prendere delle decisioni correttamente.
\end{nota}

Una possibile classificazione per le attività di analytics si ottiene analizzando
il valore del task, ovvero quanto valore porta al business, e la sua difficoltà
nel risolverlo. Possiamo quindi distinguere tra:
\begin{itemize}
    \item \textbf{Descriptive Analytics}: si vuole descrivere un qualcosa come
          ad esempio la segmentazione della clientela, clustering. In generale
          sono risolti usando metodi non supervisionati.
    \item \textbf{Diagnostic Analytics}: si vuole analizzare il perché succede
          qualcosa. (reti bayesiane)
    \item \textbf{Predicitve Analytics}: si analizza cosa succede. Un esempio
          sono i compiti di previsione e classificazione.
    \item \textbf{Prescriptive Analytics}: il mondo evolve nel tempo e le
          decisioni nel passato evolvono anche nel futuro. Si vuole capire come
          far accadere qualcosa. (analisi di sistemi complessi tramite simulazioni)
\end{itemize}

% TODO: aggiungere il grafico

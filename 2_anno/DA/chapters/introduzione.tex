\chapter{Introduzione}
Si tratterà di:
\begin{itemize}
    \item \textbf{network analysus}: analisi delle proprietà strutturali, centralità sui 
    grafi
    \item \textbf{natural language processing}: si studieranno modelli e tecniche per 
    processare il linguaggio naturale
\end{itemize}
Nel progetto non interessa come si scrive il codice sorgente, ma la parte di analisi.
L'obiettivo è di far capire ad una persona non del settore il risultato per prendere
delle decisioni correttamente.

Le attività di analytics vengono classificate rispetto: il valore del task e la 
sua difficoltà. Ciascun task si categorizza in:
\begin{itemize}
    \item \textbf{Descriptive Analytics}: descrive qualcosa ex: segmentazione della clientela, clustering, generalmente 
    non supervisionati. 
    \item \textbf{Diagnostic Analytics}: perché succede qualcosa
    \item \textbf{Predicitve Analytics}: cosa succede, ex: predizione e classificazione
    \item \textbf{Prescriptive Analytics}: il mondo evolve nel tempo e le decisioni nel 
    passato evolvono anche nel futuro.
\end{itemize} 

% TODO: aggiungere il grafico

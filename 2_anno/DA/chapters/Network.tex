\chapter{Network}
\section{Proprietà dei grafi}
Quando le dimensioni dei grafi iniziano a crescere risulta difficile
rappresentarli in modo chiaro e comprensibile. Per questo motivo, quando si
lavora con grafi di grandi dimensioni, essi vengono rappresentati tramite
delle proprietà che ne descrivono la struttura.

Tra le proprietà più importanti troviamo:
\begin{itemize}
    \item \textbf{Lunghezza media dei cammini}: indica la lunghezza media dei
          cammini tra due nodi.
    \item \textbf{Diametro}: indica la lunghezza del cammino più lungo tra due
          nodi.
    \item \textbf{Grado di un nodo}: indica il numero di archi che partono o
          arrivano ad un nodo.
    \item \textbf{Coefficienti di clustering}: indicano la presenza di nodi
          vicini tra loro.
    \item \textbf{Centralità}: indica l'importanza di un nodo all'interno del
          grafo.
\end{itemize}
\subsection{Grado di un nodo}
\begin{definizione}[\textbf{Grado di un nodo}]
    Il \textbf{grado di un nodo} è il numero di archi che partono o arrivano ad
    un nodo. Nel caso di grafi diretti si parla di \textbf{grado uscente} o
    \textbf{out degree} e \textbf{grado entrante} o \textbf{in degree}.
\end{definizione}
Se si considera un grafo rappresentato attraverso una matrice di adiacenza, il
grado di un nodo può essere calcolato sommando i valori della riga o della
colonna corrispondente al nodo in questione. Nello specifico, se si considera
un grafo orientato, il grado uscente corrisponde alla somma dei valori della
riga corrispondente al nodo, mentre il grado entrante corrisponde alla somma dei
valori della colonna ad esso associata.

Più in generale possiamo riassumere il calcolo del grado di un nodo come segue:
\begin{equation}
    \text{Outdegree}(i) = \sum_{j=1}^{n} A_{ij} \quad \text{e} \quad
    \text{Indegree}(i) = \sum_{j=1}^{n} A_{ji}
\end{equation}
dove $A_{ij}$ è l'elemento della matrice di adiacenza corrispondente al nodo
$i$ e $j$. Nel caso di grafi non orientati, il grado di un nodo corrisponde
alla somma dei valori della riga o della colonna corrispondente al nodo in
questione.

Il grado di un nodo rappresenta una misura locale del grafo, nel caso in cui
si voglia ottenere una misura globale del grafo si può calcolare il grado
medio dei nodi. Tale misura può essere calcolata come segue:
\begin{equation}
    \langle k \rangle = \frac{1}{n} \sum_{i=1}^{n} \text{Grado}(i)
\end{equation}
dove $n$ rappresenta il numero di nodi del grafo. Anche in questo caso, se il
grafo è orientato dobbiamo distinguere tra grado uscente e grado entrante.
\begin{nota}
    Possiamo calcolare il grado medio anche come segue:
    \begin{itemize}
        \item Nel caso di grafi non orientati:
              \begin{equation}
                  \langle k \rangle = \frac{2E}{n}
              \end{equation}
        \item Nel caso di grafi orientati:
              \begin{equation}
                  \langle k \rangle = \frac{E}{n}
              \end{equation}
    \end{itemize}
    dove $E$ rappresenta il numero di archi del grafo e $n$ rappresenta il
    numero di nodi.
\end{nota}
Una misura più rappresentativa della media dei gradi dei nodi è la \textbf{distribuzione
    dei gradi}. Possiamo definire $p(k)$ come la probabilità che un nodo abbia
grado $k$. Tale distribuzione può essere calcolata come segue:
\begin{equation}
    p(k) = \frac{n_k}{n}
\end{equation}
dove $n_k$ rappresenta il numero di nodi con grado $k$ e $n$ rappresenta il
numero totale di nodi del grafo. La distribuzione dei gradi può essere
rappresentata tramite un istogramma, in cui sull'asse delle ascisse vengono
inseriti i valori dei gradi, mentre sull'asse delle ordinate sono presenti i
valori di $p(k)$.
\begin{figure}[!ht]
    \centering
    \includegraphics[width=0.7\textwidth]{./img/net/degreedist.png}
    \caption{Esempio di distribuzione dei gradi.}
    \label{fig:degree_distribution}
\end{figure}
\begin{nota}
    Questa distribuzione deve essere normalizzata, ovvero la somma di tutti i
    valori di $p(k)$ deve essere uguale a 1.
    \begin{equation*}
        \sum_{k=0}^{\infty} p(k) = 1
    \end{equation*}
\end{nota}
Il valore della distribuzione mi permette di rappresentare molti fenomeni dalla
robustezza della rete, alle sue vulnerabilità.
\subsection{Cammino e distanza}
\begin{definizione}[\textbf{Cammino}]
    Un \textbf{cammino} è una sequenza di nodi in cui ciascun nodo è adiacente
    al successivo.
\end{definizione}
\begin{definizione}[\textbf{Distanza}]
    La \textbf{distanza} tra due nodi è definita come il numero minimo di archi
    che devono essere attraversati per andare da un nodo all'altro. Se i due
    nodi non sono collegati, la distanza è infinita.
\end{definizione}
Un modo semplice per calcolare la distanza tra due nodi è quello di utilizzare
l'algoritmo BFS (Breadth First Search).
\begin{definizione}[\textbf{Diametro}]
    Il diametro di un grafo è la distanza massima tra due nodi nel grafo.
\end{definizione}
\begin{definizione}[\textbf{Lunghezza media dei cammini}]
    La lunghezza media dei cammini è la media delle distanze tra tutti i nodi
    del grafo. Tale misura può essere calcolata come segue:
    \begin{equation}
        \langle d \rangle = \frac{1}{2E_{max}} \sum_{i \neq j} d_{ij}
    \end{equation}
\end{definizione}
Mentre, nei grafi non orientati, la lunghezza media dei cammini può essere calcolata
come segue:
\begin{equation}
    \langle d \rangle = \frac{1}{E_{max}} \sum_{i \neq j} d_{ij}
\end{equation}
dove $d_{ij}$ rappresenta la distanza tra i nodi $i$ e $j$ e $E_{max}$ rappresenta
il numero massimo di archi presenti nel grafo.
\subsection{Coefficienti di clustering}
\begin{definizione}[\textbf{Coefficienti di clustering}]
    I \textbf{coefficienti di clustering} sono una misura della presenza di nodi
    vicini tra loro. In particolare, il coefficiente di clustering di un nodo è una
    misura della probabilità che i vicini di un nodo siano collegati tra loro.
    Dato un nodo $i$ con grado $k_i$, il coefficiente di clustering locale è
    definito come segue (grafo non orientato):
    \begin{equation}
        C_i = \frac{2E_i}{k_i(k_i - 1)}
    \end{equation}
    dove $E_i$ rappresenta il numero di archi tra i vicini del nodo $i$. Se il grafo
    è diretto allora si divide per $2$ il coefficiente.
\end{definizione}

Il valore di questo coefficiente può variare tra 0 e 1. Nel caso in cui il
coefficiente sia uguale a 1, significa che tutti i vicini del nodo $i$ sono
collegati tra loro. Nel caso in cui il coefficiente sia uguale a 0, significa
che nessun vicino del nodo $i$ è collegato ad un altro vicino.

Questa misura rappresenta la densità locale di un grafo. Per ottenere una
misura \textbf{globale} della densità del grafo, possiamo calcolare il
\textbf{coefficiente di clustering medio}. Tale misura può essere calcolata come
segue:
\begin{equation}
    \langle C \rangle = \frac{1}{n} \sum_{i=1}^{n} C_i
\end{equation}
dove $\langle C \rangle$ si può interpretare come la probabilità che due vicini
di un nodo, selezionato in modo casuale, siano collegati tra loro.
\begin{definizione}[\textbf{Hub}]
    Un \textbf{hub} è un nodo che ha un grado molto alto, ovvero è connesso a
    molti altri nodi e ha una probabilità molto bassa di occorrere nella rete.
\end{definizione}
\begin{esempio}
    Vediamo ora un esempio di come commentare lo studio di un grafo dal punto di
    vista delle statistiche descrittive. Supponiamo di avere ottenuto un
    i seguenti grafici:
    \begin{figure}[!ht]
        \centering
        \includegraphics[width=0.5\textwidth]{./img/net/esempio1.png}
        \caption{Esempio di grafici ottenuti dallo studio di un grafo.}
        \label{fig:graphstats}
    \end{figure}

    La distribuzione di grado, riportata nel grafico $b$ della figura \ref{fig:graphstats},
    ci permette di ottenere informazioni in più rispetto alla media. Questo
    perché la media è uno stimatore distorto, sarebbe meglio utilizzare la
    mediana. Inoltre la distribuzione di grado permette di studiare gli \textbf{hub}.

    Il grafico $c$ della figura \ref{fig:graphstats} ci permette di ottenere
    informazioni sul diametro del grafo, sulla lunghezza media dei cammini.

    Il coefficiente di clustering medio permette di capire quanto sono connessi
    i vicini di un nodo estratto casualmente. Il grafico del coefficiente di
    clustering, riportato in figura \ref{fig:graphstats} $d$ mostra le relazioni
    tra grado e clustering. In questo esempio, i nodi spoke hanno un coefficiente
    di clustering maggiore quindi il vicinato è molto connesso, sintomo
    organizzazione gerarchica.

    Con questi ragionamenti si riesce ad effettuare inferenze sul comportamento
    che si ottiene se si dovesse rimuovere un nodo.
\end{esempio}
\section{Centralità del grafo}
La centralità di un nodo è una misura dell'importanza di un nodo all'interno
del grafo. Questa misura può essere calcolata in diversi modi, in base a
diversi criteri. Tra le misure di centralità più comuni troviamo:
\begin{itemize}
    \item \textbf{Degree}
    \item \textbf{Betweeness}
    \item \textbf{Closeness}
    \item \textbf{Autovettori}
    \item \textbf{Pagerank}
\end{itemize}
Oltre alla misura di centralità di grado, possiamo calcolare la \textbf{centralizzazione}.
\begin{definizione}[\textbf{Centralizzazione}]
    La \textbf{centralizzazione} di un grafo è una misura della distribuzione
    della centralità dei nodi del grafo. In particolare, la centralizzazione
    di un grafo è massima quando un nodo ha una centralità molto più alta
    rispetto agli altri nodi del grafo.
\end{definizione}

La centralizzazione di un grafo può essere calcolata come usando la formula di Freeman:
\begin{equation}
    C_D = \frac{\sum_{i=1}^{n} C_D(n^\ast) - C_D(i)}{(n - 1)(n - 2)}
\end{equation}
dove:
\begin{itemize}
    \item $C_D(n^\ast)$: corrisponde alla centralità massima di un nodo del grafo.
    \item $C_D(i)$: corrisponde alla centralità del nodo $i$.
    \item $n$: corrisponde al numero di nodi del grafo.
\end{itemize}
\subsection{Degree}
Una delle prime possibilità di misurare la centralità di un nodo è quella di
utilizzare il grado del nodo. Questa misura è particolarmente utile nel caso di
reti sociali, in cui il grado di un nodo corrisponde al numero di amici che
il nodo ha.
\begin{equation}
    C_D(i) = \sum_{j=1}^{n} x_{ij}
\end{equation}
Nel caso di grafo orientato, possiamo distinguere tra grado entrante e grado
uscente. In particolare, il grado entrante di un nodo corrisponde al numero di
archi che arrivano al nodo, mentre il grado uscente corrisponde al numero di
archi che partono dal nodo.
\begin{equation}
    C_{D_{in}}(i) = \sum_{j=1}^{n} x_{ji} \quad \text{e} \quad
    C_{D_{out}}(i) = \sum_{j=1}^{n} x_{ij}
\end{equation}
La centralità massima per un nodo è $n - 1$.

Questa misura può essere normalizzata ottenendo la centralità di grado normalizzata
dividendo la centralità di grado per il numero di nodi meno uno.
\subsection{Betweeness}
Un altra misura di centralità è la \textbf{betweeness}, ovvero il nodi più
rilevanti sono quelli che fanno da ponte tra due gruppi di nodi. Posso anche
vederli come nodi che fanno da collo di bottiglia. La betweeness di un nodo
può essere calcolata come segue:
\begin{equation}
    C_B(i) = \sum_{i \neq j \neq k} \frac{g_{jk}(i)}{g_{jk}}
\end{equation}
dove:
\begin{itemize}
    \item $g_{ik}(i)$: corrisponde al numero di \textit{shortest path} tra i
          nodi $j$ e $k$ nella rete tale che il cammino passa per il nodo $i$.
    \item $g_{ik}$: corrisponde al numero di \textit{shortest path} tra i nodi
          $j$ e $k$.
\end{itemize}
In questo caso, la centralità è massima se un nodo ha alta capacità di unire
gruppi. Mentre la centralità tende a $0$ il nodo non collega nessuno.

Viene successivamente fatta una seconda normalizzazione, questa volta sui nodi,
per rendere indipendentemente dalla dimensione del grafo. Questo può essere
espresso nel caso di un grafo non orientato come segue:
\begin{equation}
    C_B(i) = \frac{C_B(i)}{(n - 1)(n - 2)}
\end{equation}
mentre nel caso di un grafo orientato come segue:
\begin{equation}
    C_B(i) = \frac{C_B(i)}{\frac{(n - 1)(n - 2)}{2}}
\end{equation}
\subsection{Closeness}
La centralità misurata come \textbf{closeness} è una misura della vicinanza di
un nodo agli altri nodi del grafo. Questa misura può essere calcolata come segue:
\begin{equation}
    C_C(i) = \frac{1}{\sum_{j=1}^{n} d_{ij}}
\end{equation}
Corrisponde alla misura di un vertice in merito alla sua efficienza nel
comunicare l'informazione, mentre la betweeness è una misura l'efficacia di un
nodo nell trasmettere un'informazione.

Questa misura può essere normalizzata come segue:
\begin{equation}
    C_C(i) = \frac{C_C(i)}{n - 1}
\end{equation}
\subsection{Autovettori}
La misura di centralità calcolata utilizzando il grado del nodo rappresenta una
misura locale del grafo. Inoltre, questa misura non tiene conto del ruolo dei
nodi con cui il nodo è collegato. Per ottenere una misura globale della
centralità del nodo, possiamo utilizzare la \textbf{centralità di autovettori}.

La centralità di autovettori è una misura della centralità di un nodo in base
al ruolo dei nodi con cui è collegato, ovvero un nodo è importante se è
collegato ad altri nodi importanti.

Questa misura può essere calcolata come segue:
\begin{equation}
    C_A(i) = \frac{1}{\lambda} \sum_{j=1}^{n} A_{ij} C_A(j)
\end{equation}
dove $A$ è la matrice di adiacenza del grafo e $\lambda$ è l'autovalore
corrispondente all'autovettore dominante della matrice di adiacenza.

A differenza della misura basata sul grado, un nodo che ha un alto grado ma è
collegato a nodi poco importanti avrà una centralità di autovettori bassa. Al
contrario, un nodo che ha un grado basso ma è collegato a nodi molto importanti
avrà una centralità di autovettori alta.
\subsection{Pagerank}
La centralità basata su \textbf{pagerank} è una specializzazione della centralità
di autovettori. Nello specifico questa misura è stata sviluppata per misurare
la centralità di un grafo orientato.

L'assunzione su cui si basa questa misura è che un arco tra il nodo $i$ e il
nodo $j$ rappresenta un voto del nodo $i$ al nodo $j$. Se i due nodi sono connessi
attraverso un arco, la probabilità che i due nodi siano in qualche modo collegati
è maggiore.

L'idea alla base di questa misura è che un nodo $i$ è influenzato dall'importanza
dai nodi che si collegano a lui, in altre parole, gli altri nodi votano per il
nodo $i$.

Il coefficiente di PageRank associato a un nodo $i$ può essere calcolato come:
\begin{equation}
    P(i) = \sum_{(j, i) \in E} \frac{P(j)}{O_j}
\end{equation}
dove $O_j$ rappresenta il numero di archi uscenti dal nodo $j$ e $P$ rappresenta
il vettore di PageRank.

Il vettore di PageRank può essere calcolato andando a risolvere un sistema
lineare di $n$ equazioni in $n$ variabili. In particolare, possiamo scrivere il
vettore di PageRank $P = (P(1), \dots, P(n))^T$ come segue:
\begin{equation}
    P = A^T \cdot P
\end{equation}
dove $A$ è la matrice di adiacenza del grafo definita come:
\begin{equation*}
    A_{ij} = \begin{cases}
        \frac{1}{O_j} & \text{se esiste un arco da $j$ a $i$} \\
        0             & \text{altrimenti}
    \end{cases}
\end{equation*}
A questo punto, la soluzione del problema può essere ottenuta utilizzando un
algoritmo iterativo che restituisce gli autovettori della matrice di adiacenza
normalizzata.

Il problema è che per garantire la convergenza di questo algoritmo è necessario
rispettare due condizioni:
\begin{itemize}
    \item Si ha un unico autovettore dominante.
    \item $P$ è il principale autovettore di $A$.
\end{itemize}

Un'altro possibile metodo per risolvere questo problema è quello di utilizzare
la \textbf{Markov Chain}. In questo caso ogni nodo rappresenta uno stato della
catena e ogni arco rappresenta una transizione che permette di passare da uno
stato all'altro con una certa probabilità.

Questa soluzione permette di simulare la navigazione dell'utente nel web
attraverso diverse pagine e di calcolare la probabilità che l'utente raggiunga
un nodo partendo da un altro nodo.

Per questa soluzione viene utilizzata la matrice di transizione di probabilità
$A$ definita in precedenza, dove ogni elemento $A_{ij}$ rappresenta la probabilità
che l'utente passi dal nodo $i$ al nodo $j$.

A questo punto, partendo da una probabilità iniziale $P_0 = (P_0(1), \dots, P_0(n))^T$
e la matrice di transizione $A$ abbiamo:
\begin{equation*}
    \sum_{i = 1}^n P_0(i) = 1
\end{equation*}
Inoltre, se la matrice $A$ soddisfa la seguente proprietà:
\begin{equation*}
    \sum_{j = 1}^n A_{ij} = 1
\end{equation*}
allora la matrice $A$ è stocastica.

Dal teorema della Markov Chain sappiamo che se la catena di Markov è definita da
una matrice di transizione stocastica allora è caratterizzata da un unica
distribuzione di probabilità stazionaria se $A$ è irriducibile e aperiodica.
\begin{nota}
    Una distribuzione di probabilità stazionaria garantisce che dopo un certo
    numero di passi la probabilità di raggiungere un nodo da tutti gli altri
    nodi non cambia.
\end{nota}
Per garantire questo devono essere verificate le seguenti condizioni:
\begin{itemize}
    \item La matrice $A$ deve essere \textbf{stocastica}, ovvero la somma di ogni riga
          deve essere uguale a 1.
          \begin{equation*}
              \sum_{j = 1}^n A_{ij} = 1
          \end{equation*}
          Un problema in questo caso è rappresentato da una pagina che non ha
          link uscenti, essa sarebbe rappresentata da una riga di $0$. Si hanno
          due possibili soluzioni:
          \begin{itemize}
              \item Aggiungere un arco uscente per ogni nodo che non rispetta la
                    condizione.
              \item Rimuovere tutte le pagine che non hanno link uscenti.
          \end{itemize}
    \item \textbf{Irriducibilità}: se il grafo è fortemente connesso.
    \item \textbf{Aperiodicità}: se tutti gli stati della catena sono aperiodici,
          ovvero se non fa parte di cicli.
\end{itemize}
Una soluzione alle problematiche relative alla irriducibilità e aperiodicità
si ottiene aggiungendo un arco da ogni nodo a tutti gli altri nodi del grafo con
una probabilità di transizione molto bassa controllata da un parametro $d$:
\begin{equation}
    A_{ij} = (1 - d) + d \cdot \sum_{(j, i) \in E}\frac{P(j)}{O_j}
\end{equation}

Il coefficiente di PageRank $P_i$ può essere calcolato risolvendo il seguente
sistema di equazioni:
\begin{equation}
    \left[
        \begin{array}{c}
            P_1    \\
            \vdots \\
            P_n
        \end{array}
        \right] = d \cdot \left[
        \begin{array}{ccc}
            A_{11} & \dots  & A_{1, n} \\
            \vdots & \ddots & \vdots   \\
            A_{n1} & \dots  & A_{nn}
        \end{array}
        \right] + (1 - d) \cdot \left[
        \begin{array}{c}
            1      \\
            \vdots \\
            1
        \end{array} \right]
\end{equation}
Sotto la seguente condizione:
\begin{equation*}
    \sum_{i}^n P_i = 1
\end{equation*}
\section{Community detection}
\begin{definizione}[\textbf{Community}]
    Le community sono dei sottogruppi di oggetti che condividono delle proprietà.
\end{definizione}
Ci interessa l'identificazione delle community per:
\begin{itemize}
    \item Scoprire funzionalità relative agli oggetti.
    \item Studiare interazioni tra gruppi.
    \item Inferire sui collegamenti che ci sono tra community ex: social Network.
    \item Inferire dei dati mancanti di un nodo dalla sua community.
\end{itemize}
\begin{esempio}
    Un esempio di community sono quelle presenti nei social network. In questo caso,
    le community sono degli insiemi di utenti che effettuano delle interazioni
    tra di loro, come ad esempio relazioni di amicizia, oppure mettere like a
    uno stesso post.
\end{esempio}
Non esiste una definizione unica di community, ma esistono diversi criteri per
identificarle. Ad esempio, se consideriamo la figura \ref{fig:community}, possiamo
identificare 3 community, ovvero quelle rappresentate dalle componenti connesse,
oppure 9 se le consideriamo come insiemi di nodi vicini tra loro, insiemi densi
di nodi.
\begin{figure}[!ht]
    \centering
    \includegraphics[width=0.5\textwidth]{./img/net/community.png}
    \caption{Esempio di community.}
    \label{fig:community}
\end{figure}

In una rete, le community che vengono identificate possono essere disgiunte,
oppure possono sovrapporsi (overlap).
\begin{figure}[!ht]
    \centering
    \includegraphics[width=0.5\textwidth]{./img/net/overlap.png}
    \caption{Esempio di community che si sovrappongono.}
    \label{fig:overlap}
\end{figure}
\subsection{Criteri per identificare le community}
Esistono diversi criteri per identificare le community, ognuno dei quali si adatta
a diversi tipi di studi che si vogliono effettuare. Questi criteri possono essere
suddivisi nelle seguenti categorie  ordinate dalla più stringente alla meno
stringente:
\begin{itemize}
    \item \textbf{Node-centric}: si creano delle community sulla base delle
          caratteristiche sui singoli nodi.
    \item \textbf{Group-centric}: si creano delle community dal punto di vista
          di un gruppo. Il gruppo deve rispettare un determinato criterio senza
          considerare i singoli nodi.
    \item \textbf{Network-centric}: si valuta il criterio nell'interezza della rete.
    \item \textbf{Hierarchy-centric}: si costruisce una struttura gerarchica delle
          community basata sulla topologia della rete.
\end{itemize}
\subsubsection{Node-centric}
Questo criterio si basa sulla ricerca di caratteristiche comuni tra i singoli
nodi. In particolare, i nodi che fanno parte della stessa community devono
rispettare delle proprietà comuni.

Con questo criterio possiamo identificare le community che soddisfano le seguenti
proprietà utilizzando diversi algoritmi:
\begin{itemize}
    \item \textbf{Complete Mutuality}: tutti i nodi della community sono connessi
          tra di loro. Questo problema può essere risolto utilizzando l'algoritmo
          di ricerca delle \textbf{clique}. Una clique è una sotto-componente
          fortemente connessa del grafo. Questo problema è NP-Hard, quindi richiede
          un tempo esponenziale per essere risolto. Possiamo distinguere due varianti
          di questo problema:
          \begin{enumerate}
              \item \textbf{Massima clique}: è la clique di grandezza massimale,
                    ovvero quella con il massimo numero di nodi.
              \item \textbf{All maximal clique}: è la clique con un numero minimo
                    di nodi.
          \end{enumerate}
          Data la complessità del problema, è possibile semplificare la ricerca
          introducendo un'operazione di \textbf{pruning}. Tale operazione
          consiste nell'eliminare i nodi che non possono far parte della clique.
          Se stiamo cercando una clique di grandezza $k$, allora tutti i nodi che
          fanno parte della clique devono avere almeno $k$ archi. Questo ci permette
          di non considerare i nodi il cui grado è minore di $k - 1$.

          In ogni caso le clique sono molto rare quindi rimane sempre pesante
          questa strategia.
    \item \textbf{Reachability of members}: tutti i nodi della community devono
          essere raggiungibili con un numero massimo di archi. Si possono utilizzare
          algoritmi come k-clique e k-club per svolgere questo compito.
          \begin{itemize}
              \item \textbf{k-clique}: si cercano sotto-grafi in cui la distanza
                    maggiore che si può avere tra due nodi è $\leq k$.
              \item \textbf{k-club}: si cercano sotto-grafi in cui il diametro è
                    $\leq k$.
          \end{itemize}
    \item \textbf{Node degree}: si cercano nodi con un grado minimo utilizzando
          algoritmi come $k-plex$ e $k-core$.
\end{itemize}
Questi metodi sono utili per reti di dimensioni ridotte a causa della loro
complessità.
\subsubsection{Group-centric}
Questo criterio si basa sul fatto che l'intero gruppo deve soddisfare un determinato
criterio, non si considerano i singoli nodi.

Un esempio di criterio utilizzato per questa categoria è la densità del gruppo.
In pratica si cercano delle \textbf{quasi-clique}, ovvero dei gruppi di nodi
che soddisfa il seguente vincolo:
\begin{equation}
    \frac{2 |E_s|}{|V_s|(|V_s| - 1)} \geq \delta
\end{equation}
Definito il vincolo, si possono definire diversi problemi di ottimizzazione per
trovare le quasi-clique:
\begin{itemize}
    \item \textbf{Massimizzazione}: si cercano i gruppi che massimizzano il criterio.
    \item \textbf{Threshold}: si cercano i gruppi che superano una determinata soglia.
\end{itemize}
Un'altra strategia per identificare le community è quella di utilizzare un approccio
\textbf{greedy}. In questo caso si parte da un nodo con grado elevato e si espande
la quasi-clique introducendo nodi che si pensa possano aiutare a costruire una
quasi-clique più grande.

Questo approccio è molto utile per reti di piccole dimensioni o quando si hanno
caratteristiche che devono essere rispettate dalle quasi-clique. Si tratta comunque
di un problema NP-Hard.
\subsubsection{Network-centric}
Si valuta la rete nella sua interezza. Un primo approccio è quello basato sulla
\textbf{Node similarity}.
\paragraph{Node similarity}
In questo caso, due nodi si considerano simili se hanno un pattern di interazione
simile, ovvero se sono connessi agli stessi nodi. Questo viene anche chiamato
\textbf{structural equivalence}. Questo approccio viene applicato utilizzando la
\textbf{vector similarity}, la quale può essere calcolata utilizzando diverse
funzioni di similarità, come ad esempio:
\begin{itemize}
    \item \textbf{Cosine similarity}: calcola la similarità tra due nodi, si
          utilizza solitamente quando il grafo è pesato.
          \begin{equation*}
              \cos(\theta) = \frac{A \cdot B}{||A|| \cdot ||B||}
          \end{equation*}
    \item \textbf{Jaccard similarity}: calcola la similarità tra due nodi
          considerando solamente se l'arco è presente o meno. Questa misura
          è molto utilizzata per grafi non pesati.
          \begin{equation*}
              J(A, B) = \frac{|A \cap B|}{|A \cup B|}
          \end{equation*}
\end{itemize}
Usando il valore di similarità, abbiamo che due nodi simili avranno un valore
di similarità alto, mentre due nodi dissimili avranno un valore di similarità
basso. Sfruttando questo valore, possiamo creare le community utilizzando
degli algoritmi di clustering, come ad esempio il \textbf{K-means}. 
\begin{nota}
    Usando l'algoritmo K-means è necessario considerare sempre tutti i problemi 
    che si possono verificare con questo algoritmo, come ad esempio la sensibilità
    all'inizializzazione dei centroidi, la presenza di minimi locali e la forma
    sferica dei cluster.
\end{nota}
\paragraph{Spectral clustering}
Un altro approccio per identificare le community è quello di utilizzare lo 
\textbf{spectral clustering}. Questo approccio rappresenta la similarità tra due 
grafi utilizzando le matrici e gli autovalori e autovettori. Questi ultimi 
rappresentano le informazioni globali sulla struttura del grafo.

Lo scopo di questo approccio è quello di portare i dati in uno spazio di dimensione
inferiore, in modo da poterli separare in maniera più semplice usando gli autovettori
e il grafo Laplaciano.

Il primo passo consiste nel rappresentare il grafo con una matrice di adiacenza
$A$. A questo punto, possiamo costruire la matrice Laplaciana del grafo, la quale
può essere calcolata come segue:
\begin{equation}
    L = D - A
\end{equation}
dove $D$ è la matrice diagonale che contiene la somma delle righe della matrice
di adiacenza $A$.

A questo punto possiamo calcolare gli autovalori e gli autovettori della matrice
Laplaciana. Usando uno o più autovettori, possiamo rappresentare il grafo in uno
spazio di dimensione inferiore. 

Una volta ottenuta la rappresentazione del grafo in uno spazio di dimensione 
inferiore, possiamo assegnare i nodi alle community.
\paragraph{Mudularity maximization}
Un altro approccio per identificare le community è quello di utilizzare la
\textbf{modularity maximization}. Questo approccio si basa sull'assunzione che 
un grafo ottenuto dal mondo reale ha una struttura molto lontana rispetto a 
quella di un grafo generato in modo casuale. Quindi, più il grafo in analisi è 
distante da uno casuale, più la sua struttura sarà particolare.

La \textbf{modularity} vuole fornire una misura di quanto sono diverse le 
interazioni tra i gruppi nel grafo che stiamo analizzando rispetto a delle 
interazioni casuali. In un grafo con $m$ archi, considerando due nodi con grado 
rispettivamente $d_i$ e $d_j$, le connessioni casuali tra i due nodi sarebbero
date da:
\begin{equation}
    \frac{d_i \cdot d_j}{2m}
\end{equation}
L'interazione tra due nodi in un gruppo è data dalla seguente formula:
\begin{equation}
    \sum_{i, j}\left(A_{ij} - \frac{d_i \cdot d_j}{2m}\right)
\end{equation}
La modularity può essere calcolata come segue:
\begin{equation}
    Q = \frac{1}{2m} \sum_{i, j} \left( A_{ij} - \frac{d_i \cdot d_j}{2m} \right) \delta(c_i, c_j)
\end{equation}
Più è grande la differenza tra $A_{ij} - \frac{d_i \cdot d_j}{2m}$, più forti 
sono le connessioni tra i nodi. Al contrario, se la differenza è piccola,
allora le connessioni tra i nodi sono casuali.
Un vantaggio di questo approccio è che il valore di modularity è compreso tra 
$-1$ e $1$. Inoltre, una modularity uguale a $0$ indica che tutti i nodi 
appartengono allo stesso gruppo.

Questo valore può essere calcolato attraverso gli autovettori della matrice di 
modularità, ovvero la matrice definita come segue:
\begin{equation}
    B_{ij} = A_{ij} - \frac{d_i \cdot d_j}{2m}
\end{equation}
\begin{nota}
    La limitazione di tutti questi approcci è che l'utente deve specificare il 
    numero di community che si vogliono identificare.
\end{nota}
\subsubsection{Hierarchy-centric}

Si costruisce una struttura gerarchica basate sulla topologia delle community.
Per costruire la struttura gerarchica del grafo si sfrutta la matrice delle distanze 
che viene aggiornata riducendo la dimensione (eliminando le colonne e righe dei nodi
uniti sotto una community) in modo da creare delle community. 


Altrimenti si può ragionare partizionando la rete. Si calcola la Betweeness sugli 
arichi e si rimuovono quelli con betweeness maggiore. In questo modo si partiziona
il grafo per definire le community. Si continua la partizione fino ai singoli nodi.

\subsection{Social media Analitics}
HUB = nodi con tanti collegamenti
Spoke = nodi con pochi collegamenti

Questo si applica anche nei social. In base al contesto risultano importanti i
collegamenti hub-hub o hub-spoke.

Possiamo stimare la probabilità che un nodo di un grado $k$ e un nodo di grado $k'$
che siano connessi, questo conto dipende dagli archi della rete. In pratica abbiamo
che gli hubs sono molto più facilmente collegati agli altri rispetto ai gradi di grado massimo.
Quindi se due nodi hanno molti archi allora si ha più probabilità che siano connessi.


La rete può essere:
\begin{itemize}
    \item assortativo: avremo una tendenza ad avere hub che si collegano tra di loro
    \item neutrale: rete che ha collegamenti tra due nodi con probabilità randomica
    \item disassortativo: avremo la tendenza ad avere spoke collegati agli hub
\end{itemize}

Possiamo plottare i collegamenti tra due nodi rispetto al loro grado, se il grafico 
mostra se la nuvola è sulla diagonale principale allora è assortativa, se non ha una 
direazione allora è neutrale, se ha una direzione sul'antidiagonale allora è disassortativa.

Possiamo definire la matrice di correlazione (la matrice plottata dal grafico prec.) 
permette di mostrare la correlazioene tra collegamenti di nodi di gradi differenti.
La matrice viene riempita contando il numero di archi che connette un nodo di 
grado $i$ con uno di grado $j$, poi la somma viene normalizzata.

Se il grafo è orientato allora non è simmetrica è il contrario. 

Possiamo determinare l'assortativa o sisassortatività può essere fatta controllando 
che $e_{ij} = q_iq_j$ dove $q_i$ e $q_j$ sono le probabilità che un nodo di grado 
$i$ sia collegato con un nodo di grado $j$ in una rete randomica. Se $=$ allora 
la rete è neutrale, altrimenti è assortativa o disassortativo. Possiamo utilizzare 
un coefficiente $-1<r<1$ che se $r<0$ allora disassortativa, $r=0$ neutrale e $r>0$ assortativa. 

La rimozione di un hub che è connesso ad altri hub o possiamo rimuovere un spoke.
In questo modo una volta che sappiamo la tipologia della rete (assortativa o disassortativa)
allora si può studiare la robustezza della rete.

\begin{definizione}
    Giant component è la più grande componente connessa di un grafo
\end{definizione}

Per studiare la robustessa si costruisce una rete a parte in cui si inseriscono 
randomicamente gli archi della rete di partenza nella rete randomica. quando 
aggiungiamo un arco allora i cammini esistenti tra quei nodi si espandono.
Se siamo in una rete assortativa, quando aggiungiamo un arco tra hub allora si avrà
un impatto notevole sui cammini della rete perché la Giant component si paleserà
prima nella costruzione. Se è disassortativa allora impiegheremo più tempo a palesare 
la giant component.

Quindi si può studiare la palesazione della Giant component in base alle caratteristiche 
della rete. Quindi si può studiare le proprietà di costruzione e distruzione della rete.

Per studiare la resistenza della rete si studia la rimozione dei nodi (site percolation)
o la rimozione degli archi (bond percolation). In questo caso studiamo quanto tempo 
impieghiamo a distruggere la Giant component. Se elimino un nodo allora cambio il 
grado della rete. Se l'assortativa si distrugge più lentamente.



\chapter{Data Pre-processing}
Nella realtà, i dati che vengono utilizzati per effettuare le varie analisi non
sono mai perfetti, anzi, sono spesso ``sporchi'', ovvero possono essere:
\begin{itemize}
    \item \textbf{Incompleti}: ovvero mancano valori per alcune feature.
    \item \textbf{Rumorosi}: ovvero contengono errori e/o outlier.
    \item \textbf{Inconsistenti}: sono presenti discrepanze nei dati.
\end{itemize}
Quando nei dati che si vogliono utilizzare per la creazione di un modello siamo
in presenza di queste condizioni, allora si devono compiere i seguenti passaggi:
\begin{enumerate}
    \item \textbf{Data cleaning}: in questa fase si vanno a risolvere i vari
          problemi di incompletezza, rumorosità e inconsistenza dei dati.
    \item \textbf{Data integration}: i dati possono essere presenti in più
          dataset, quindi si devono integrare insieme.
    \item \textbf{Data transformation}: in questa fase si vanno ad applicare
          operazioni di normalizzazione e aggregazione dei dati.
    \item \textbf{Data reduction}: selezione delle feature ed estrazione di
          sottoinsiemi di esse.
\end{enumerate}
\section{Data cleaning}
\subsection{Valori mancanti}
\begin{definizione} [\textbf{Valori mancanti}]
    Possiamo definire i \textbf{valori mancanti} come valori sconosciuti di una
    feature per una particolare istanza.
\end{definizione}
La presenza dei \textbf{valori mancanti} può essere dovuta a diversi fattori,
tra cui:
\begin{itemize}
    \item Dati non disponibili
    \item Dati non registrati
    \item Risultati di malfunzionamenti dei sensori
    \item Dati cancellati
    \item Dati sconosciuti
\end{itemize}
La gestione di valori mancanti è un problema molto importante ed è necessario
gestirli in modo corretto. La loro gestione è importante per le seguenti ragioni:
\begin{itemize}
    \item \textbf{Perdita di efficienza}: rendono meno efficienti i processi che
          vengono successivamente applicati ai dati. Inoltre la loro presenza può
          influenzare i risultati in quanto introduciamo delle distorsioni nel
          dataset.
    \item \textbf{Complicazioni}: si complicano i modelli perché non prevedono
          valori mancanti nelle istanze. Esistono modelli che sono in grado di
          superare questo problema ma sono più complessi.
    \item \textbf{Bias nei risultati}: si introducono delle differenze tra dati
          completi e dati incompleti.
\end{itemize}

Esistono 3 tipologie di dati mancanti:
\begin{itemize}
    \item \textbf{Missing Completely At Random} (\textbf{MCAR}): quando la
          mancanza del valore di un attributo non è dipendente dai valori
          osservati o dai valori mancanti.
    \item \textbf{Missing At Random} (\textbf{MAR}): quando la mancanza del
          valore di un attributo è dipendente dai valori osservati, ma non dai
          valori mancanti.
    \item \textbf{Not Missing At Random} (\textbf{NMAR}): quando la mancanza del
          valore di un attributo è dipendente dai valori mancanti.
\end{itemize}
In base alla tipologia del dato mancante si hanno diversi metodi per la loro
gestione di essi. I principali metodi sono i seguenti:
\begin{itemize}
    \item \textbf{Ignorare} l'istanza avente il valore dell'attributo mancante o
          ignorare l'attributo se molte istanze hanno valori mancanti per quella
          feature. Questo è l'approccio più semplice ma applicabile solo quando
          abbiamo pochi dati mancati rispetto a quelli dati osservati.
    \item \textbf{Conversione} dei valori mancanti. Si assegna un significato al
          valore mancante, quindi, in termini di modelli probabilistici, si
          aggiunge un ulteriore possibile valore alla variabile aleatoria e
          \textit{si considera come un'informazione} il fatto che manchi il valore.
    \item \textbf{Metodi di imputazione}, assegno un valore basato sul resto delle
          osservazioni del dataset, ma non è importate sapere che manchi e quindi
          meglio un valore che lo sostituisca.Questo metodo è applicabile per i
          dati di tipo \textit{MCAR} e \textit{MAR}, ma non per \textit{NMAR}.
\end{itemize}
\subsubsection{Metodi di imputazione}
A seconda della situazione e in base alla tipologia dell'attributo, si possono
usare diversi metodi di imputazione dei valori mancanti:
\begin{itemize}
    \item Una prima strategia è quella \textbf{Non ho segnato il Nome}, la quale
          fornisce due possibilità di imputazione:
          \begin{itemize}
              \item \textbf{Continuo}: calcolo la media tra tutti gli altri
                    valori delle features, in questa situazione si assume che il
                    dato sia distribuito normalmente.
              \item \textbf{Discreto} o \textbf{Categorico}: sostituiamo il valore
                    mancante con quello più frequente.
          \end{itemize}
    \item Un'altra strategia è quella \textbf{CMC}, la quale utilizza le stesse
          strategie di imputazione ma in base alla classe dell'istanza.
\end{itemize}
Il primo approccio si può utilizzare in un contesto non supervisionato, mentre il
secondo in un contesto supervisionato.

Gli approcci appena descritti sono molto semplici e veloci, ne esistono altri più
complessi che scomodano algoritmi di Machine Learning. Per esempio si può usare
\textbf{kNN} per predire il valore mancante in base al valore più frequente tra
i $k$ valori degli esempi più vicini. Un estensione di questo metodo è l'utilizzo
di algoritmi di Machine Learning per la predizione dei valori mancanti. Questo
significa definire un modello per la previsione dei valori mancanti per ciascuna
feature, il che lo rende computazionalmente molto costoso.
\subsection{Rumorosità dei dati}
\begin{definizione} [\textbf{Rumorosità dei dati}]
    La \textbf{Rumorosità dei dati} è l'insieme di dati che presentano dei
    valori che non hanno una relazione con gli altri valori, spesso dovuti ad
    errori di misurazione o errori nella scelta del campione.
\end{definizione}
Una possibilità per ridurre il rumore consiste nel discretizzare (\textbf{Binning})
i dati continui attraverso i seguenti metodi:
\begin{itemize}
    \item \textbf{Equal-width}: partiziona dell'intervallo dell'attributo in $N$
          intervalli della stessa larghezza. Mantiene sempre gli outliers e
          quindi la distribuzione non è uniforme.
          Ad esempio se abbiamo un intervallo $[0, 100]$ e vogliamo partizionarlo
          in 5 intervalli, otteniamo $[0, 20]$, $[20, 40]$, $[40, 60]$, $[60, 80]$,
          $[80, 100]$.
    \item \textbf{Equal-depth}: partiziona dell'intervallo dell'attributo in $N$
          intervalli in cui la frequenza dei dati è la stessa. Genera una
          distribuzione uniforme ma non mantiene gli outliers. Si riduce la
          varianza.
\end{itemize}

I due metodi si usano in base all'impatto che hanno sui modelli. Il secondo
metodo abbassa la varianza della feature nascondendo gli outlier, mentre il primo
aumenta la varianza e quindi evidenza gli outlier.
%TODO: aggiungere il grafico delle distribuzioni dei due metodi

\subsection{dataset sblilanciati}
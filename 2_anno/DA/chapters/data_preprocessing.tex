\chapter{Data Pre-processing}
I dati nella realtà sono spesso sporchi infatti possono essere:
\begin{itemize}
    \item \textbf{incompleti}: mancanza di dati
    \item \textbf{rumorosi}: contenti errori e outlier
    \item \textbf{inconsistenti}: contenenti discrepanze 
\end{itemize}

Quando abbiamo dei dati che devono essere utilizzati per un modello allora si 
devono compiere i seguenti passi:
\begin{enumerate}
    \item \textbf{data cleaning}: riempire i dati mancanti, rendere più omogenei
    i dati rumorosi, identificare e rimuovere gli outlier e risolvere le inconsistenze
    \item \textbf{data integration}: integrare diversi dataset insieme
    \item \textbf{data transformation}: applicare operazioni di normalizzazione e aggregazione
    \item \textbf{data reduction}: selezione delle feature ed estrazione.
\end{enumerate}


\section{Data cleaning}
\subsection{Valori mancanti}
\begin{definizione} [\textbf{Valori mancanti}]
    I \textbf{valori mancanti} sono i valori sconosciunti di una feature per una
    particolare istanza.
\end{definizione}
La presenza dei \textbf{valori mancanti} può essere dovuta da:
\begin{itemize}
    \item dati non disponibili
    \item dati non registrati
    \item risultati di malfunzionamenti dei sensori
    \item dati cancellati
    \item dati sconosciuti
\end{itemize}
Importante gestirli perché tre ragioni:
\begin{itemize}
    \item \textbf{perdita di efficienza}: rendono meno efficienti tutti i processi successivi
    risultati perché introduciamo delle distorsioni nel dataset, compromettendo
    i risultati.
    \item \textbf{complicazioni}: si complicano i modelli perché non prevedono
    valori mancanti nelle istanze
    \item \textbf{bias nei risultati}: si introducono delle differenze tra dati 
    completi e dati incompleti.  
\end{itemize}

Esistono 3 tipologie di dati mancanti:
\begin{itemize}
    \item Missing Completely At Random (\textbf{MCAR}): quando la mancanza del
    valore di un attributo non è dipendente dai valori osservati o dai valori mancanti
    \item Missing At Random (\textbf{MAR}): quando la mancanza del
    valore di un attributo è dipendente dai valori osservati, ma non dai valori mancanti
    \item Not Missing At Random (\textbf{NMAR}): quando la mancanza del
    valore di un attributo è dipendente dai valori mancanti
\end{itemize}
In base alla tipologia del dato mancante si hanno diversi metodi per la loro gestione.
In generale si hanno tre metodi:
\begin{itemize}
    \item \textbf{ignorare} l'istanza avente il valore dell'attributo mancante o 
    ignorare l'attributo se molte istanze hanno valori mancanti per quella feature.
    Approccio più semplice ma applicabile solo quando abbiamo pochi dati mancati 
    e tanti dati osservati.
    \item \textbf{conversione} dei valori mancanti. Si assegna un significato al
    valore mancante, quindi, intermini di modelli probabilistici, si aggiunge 
    un ulteriore possibile valore alla variabile aleatoria e si considera come 
    un'informazione il fatto che manchi il valore.
    \item \textbf{metodi di imputazione}, assegno un valore basato sul resto delle 
    osservazioni del dataset, ma non è importate sapere che manchi e quindi meglio 
    un valore che lo sostituisca. Utili per MCAR e MAR, ma inutili per NMAR. 
\end{itemize}
Esistono diversi metodi di imputazione dei valori mancanti in base alla tipologia
dell'attributo:
\begin{itemize}
    \item \textbf{continuo}: calcolo la media tra tutti gli altri (si assume che i dato sia distribuito normalmente)
    \item \textbf{discreto} o \textbf{categorico}: sostituiamo il valore mancante con
    quello più frequente
\end{itemize}
Possiamo utilizzare questi metodi seguento un ragionamento differente, ovvero
ragionando per classe, quindi calcolare la media e il valore frequente rispetto
le istanze etichettate con la stessa classe dell'istanza col valore mancante.
Questo secondo metodo si usa principalmente in un contesto supervisionato.
Gli approcci appena descritti sono molto semplici e veloci, ne esistono altri più
complessi che scomodano algoritmi di ML. Per esempio si può usare KNN per predire
il valore mancante in base al valore medio o più frequente tra i $k$ valori degli 
esempi più vicini. Un estensione di questo metodo è l'utilizzo di algoritmi di ML
per la predizione dei valori mancanti, ma significa definire un modello per la 
previsione dei valori mancanti per ciascuna feature.


\subsection{Rumorosità dei dati}
\begin{definizione} [\textbf{Rumorosità dei dati}]
    La \textbf{Rumorosità dei dati} è l'insieme di dati che presentanto dei 
    valori che non hanno una relazione con gli altri valori, spesso dovuti ad errori di misurazione,
    o errori nella scelta del campione. 
\end{definizione}
Spesso si discretizzano i dati continui attraverso i seguenti metodi:
\begin{itemize}
    \item \textbf{equal-width}: partiziona dell'intervallo dell'attributo in $N$ intervalli della stessa
    distanza. Mantiene sempre gli outliers.
    \item \textbf{equal-depth}: partiziona dell'intervallo dell'attributo in $N$ intervalli della stessa
    frequenza di dati. Genera una distribuzione uniforme ma non mantiene 
    gli outliers.
\end{itemize}

I due metodi si usano in base all'impatto che hanno sui modelli. Il secondo 
metodo abbassa la varianza della feature nascondendo gli outlier, mentre il primo aumenta la varianza 
e quindi evidenza gli outlier.
%TODO: aggiungere il grafico delle distribuzioni dei due metodi

\subsection{dataset sblilanciati}
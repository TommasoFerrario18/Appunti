\chapter{Chemistry and Reactions}
\begin{definition}[\textbf{Mixture}]
    In chemistry the forms of matter are called \textbf{mixtures}. Some of them
    are \textbf{homogeneous} and others are \textbf{solution} with a \textbf{solvent}
    and a \textbf{solute}.

    To separate the components of a mixture we can use different methods like
    \textbf{filtration}, \textbf{distillation}, \textbf{chromatography} and
    \textbf{evaporation}.
\end{definition}
\begin{definition}[\textbf{Compound}]
    Certain substances are not separable by simple physical methods. These are
    called \textbf{compounds} and are formed by the union of two or more elements.
\end{definition}
\begin{definition}[\textbf{Reactions}]
    Pure substances cannot be further separated by means of physical interventions,
    but can be modified by means of (bio)chemical reactions.

    A reaction involves a certain number of reactants, yielding a number of products.
    \begin{equation}
        R_1 \bigoplus R_2 \bigoplus \ldots \bigoplus R_n \rightarrow P_1
        \bigoplus P_2 \bigoplus \ldots \bigoplus P_m
    \end{equation}

    As we all know, there are bits of matter that cannot be modified at the
    chemical level. These are called \textbf{elements}.
\end{definition}
\begin{definition}[\textbf{Atoms}]
    Atoms are composed by nuclei (protons and neutrons) and electrons on (quantized)
    orbits. Every orbit can contain up to 8 electrons (except for the first,
    which can hold only 2) this fact is known as the octect rule.

    The most external orbit is, except for the “noble gases”, incomplete; the
    number of electrons that occupy them determine the atom valence.
\end{definition}

The configuration of the external orbits of atoms, allows them to bind together
in compounds. Molecule structure depends on the organization of the electron
sharing in the most external orbits.

Atoms with valence up to 4 tend to be electron donors to atoms with a valence
from 5 to 7; atoms are therefore grouped in donors and receptors. Receptors atoms
are said to have an electronegative tendency.

There are several ways atoms bind to each other, the most common are:
\begin{itemize}
    \item \textbf{Ionic bindings} happen between atoms with very different valence.
    \item \textbf{Covalent bindings} happen between atoms of similar valence.
\end{itemize}

Molecules have a polarity, depending on the electronegativity of each participant
atom and their spatial configuration. Polar molecules tend to attract each other,
while non-polar molecules are relatively.

The forces that create bonds between atoms are also responsible of the attraction
between atoms and molecules. An intermediate attraction force is that resulting
from \textbf{hydrogen bonds} which keep together the DNA double helix.
\section{Reactions and Metabolism}
Biochemical reactions modify properties of various compounds and we can classify
them in two main categories:
\begin{itemize}
    \item \textbf{Reversible reactions} are those that can be reversed by changing
          the conditions of the system. They can be analyze at the equilibrium.
    \item \textbf{Irreversible reactions} are those that cannot be reversed.
\end{itemize}
The quantitative ratio among compounds in a reaction is called the reaction
\textbf{stoichiometry}.
The translation of stoichiometric ratios into physical quantities and vice-versa
requires the introduction of new units. So, chemists have defined the \textbf{mole}
as the quantity of substances comprising about $6.022 \times 10^{23}$ molecules.

This analysis is performed by looking at concentrations of a substance, measured
in mole/liter; this dimension is called the molarity of a solution.

Every reaction has its own reaction rate also called velocity. Given a reaction,
when the concentration of a reactant is very low, or the reaction rate is very
slow, then the reaction is said to be kinetically impaired.

A reaction can happen only when a positive $\Delta G$ is present and that is
enough to exceed the so-called activation barrier; such amount of energy is known
as activation energy.

Energy contained in ATP phosphate bonds is sufficient for the activation of
several biochemical reaction. This energy is however not enough to endanger several
of the other bond types that are present in an organism.

The metabolism is the activity that allows an organism to survive. This process
is mostly carried out in the cytoplasm of the cells and we can divide it in two
main categories:
\begin{itemize}
    \item \textbf{Catabolism} is the set of reactions that decompose various
          complexes, mostly acquired from the environment.
    \item \textbf{Anabolism} is the set of reactions that synthesize complexes.
\end{itemize}

Many of the basic reactions are shared, without much variation, by most living
organisms this is known as core metabolism. The metabolism that is specialized
for a class of organisms is dubbed secondary metabolism.

A very important process is the one that allows an organism to “load” ADP molecules
with a phosphate group, thus generating ATP. To obtain this result, organisms use
chains of reactions, called \textbf{metabolic pathways}.

Several of these reactions have a rather high activation energy or a quite slow
reaction rate. To speed up such reactions, organisms use the catalysis machinery.
A catalyzer speeds up a reaction or lowers its activation energy without being
consumed. Catalyzers are called enzymes, acting on substrates.
\section{Mathematical Modeling of Reactions}
Much of Computational Biology is \textbf{modeling reaction networks} which can be
subdivided in two main categories:
\begin{itemize}
    \item \textbf{Metabolic networks} are the set of reactions that take place
          in a cell. These reactions are mostly catalyzed by enzymes.
    \item \textbf{Regulatory networks} are the set of reactions that regulate
          the activity of the metabolic networks.
\end{itemize}
\subsection{Law of Mass-Action}
Collision between two chemical compounds $A$ and $B$ happen with a rate $k$ and
produce a compound $C$.
\begin{equation*}
    A + B \xrightarrow{k} C
\end{equation*}

We can rewrite the previous equation as:
\begin{equation}
    \frac{\delta [C]}{\delta t} = k [A][B]
\end{equation}
where with $[A]$ we denote the concentration of compound $A$. As said before,
we have also reaction that can be reversed, so we can write:
\begin{equation}
    A + B \mathrel{\mathop{\rightleftarrows}^{\mathrm{k_1}}_{\mathrm{k_2}}} C
\end{equation}
and the rate of the reaction is:
\begin{equation}
    \frac{\delta [A]}{\delta t} = k_2 [C] - k_1 [A][B]
\end{equation}
When a system is at equilibrium, the concentration don't change, therefore the
following condition holds:
\begin{equation}
    \frac{k_1}{k_2} = K_{eq} = \frac{[A]_{eq} [B]_{eq}}{[C]_{eq}}
\end{equation}
where $K_{eq}$ is the equilibrium constant of the reaction. If $K_{eq}$ is small,
then this is an indication that, at equilibrium, the concentration of $A$ and $B$
are effectively combined to form $C$.

Enzymatic reactions, which are non elementary reactions, can be modeled by the
Michaelis-Menten equation:
\begin{equation}
    S + E \mathrel{\mathop{\rightleftarrows}^{\mathrm{k_1}}_{\mathrm{k_2}}} catalysis \xrightarrow{k_3} P + E
\end{equation}
where $S$ is the substrate, $E$ is the enzyme, $P$ is the product and catalysis
and $C$ is the intermediate product.

Let's now consider the total enzyme concentration to be $[E]_0 = [E] + [C]$ and
also that all the reaction rate are constant. We can define the velocity at which
we increase the product concentration is:
\begin{equation}
    \frac{\delta [P]}{\delta t} = k_3 [C] \approx [E][S]
\end{equation}
so we can say that the velocity is proportional to the concentration of $[E]$ and
$[S]$.

The Michaelis-Menten system can be analytically solved by an equilibrium
approximation where we assume that the substrate $S$ and the complex $C$ are in
instantaneous equilibrium. This assumption allows us to write the following:
\begin{equation}
    k_1[S][E] = k_2[C] + k_3[C]
\end{equation}
Since $[E]_0 = [C] + [E]$, we can rewrite the previous equation as:
\begin{equation}
    [S][E]_0 - [S][C] = [C]\left( \frac{k_1 - k_3}{k_2}\right)
\end{equation}
We now introduce the Michaelis constant $K_M = \frac{k_2 + k_3}{k_1}$ which is
the combination of the rate constants of the reaction. We can rewrite the previous
equation as:
\begin{equation}
    [C] = \frac{[S][E]_0}{K_M + [S]}
\end{equation}
The velocity of the creation of $P$ is then:
\begin{equation}
    V = \frac{\delta [P]}{\delta t} = k_3[C]
\end{equation}
At $V_{max}$ we have that all the enzyme $E_0$ must be bound in the complex $C$,
this means that:
\begin{equation}
    V_{max} = k_3[E]_0
\end{equation}

If we now put everything together, we can write the Michaelis-Menten equation as:
\begin{equation}
    V = \frac{V_{max}[S]}{K_M + [S]}
\end{equation}
We can say that the Michaelis-Menten constant is a concentration ($\approx [S]$)
at which the velocity is half of $V_{max}$. In formula:
\begin{equation}
    V = \frac{V_{max}[S]}{K_M + [S]} = \frac{V_{max}}{2}
\end{equation}
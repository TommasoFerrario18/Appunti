\chapter{Operatori sulle immagini}
Per ogni pixel possiamo definire due tipologie di vicinato:
\begin{itemize}
    \item \textbf{vicinato di Von Neumann}: se il pixel è $x,y$ il vicinato è
          \begin{equation*}
              (x-1, y) \ (x, y-1) \ (x+1, y) \ (x, y+1)
          \end{equation*}
    \item \textbf{vicinato di Moore}: se il pixel è $x,y$ il vicinato è:
          \begin{equation*}
              (x-1, y) \ (x-1, y-1) \ (x, y-1) \ (x+1, y-1) \ (x+1, y) \ (x+1, y+1)
              \ (x, y+1) \ (x-1, y+1)
          \end{equation*}
\end{itemize}
Possiamo definire il \textbf{percorso digitale} tra due pixel come la sequenza di
pixel che separano i due di interesse. La \textbf{lunghezza del cammino} sarà il
numero di edge del percorso, ricorda che la scelta del vicinato fa variare i vari
cammini disponibili. Si definisce \textbf{cammino chiuso} come il percorso da un
pixel a se stesso.

Dato $S$ un insieme di pixel allora due pixel $p$, $q$ sono \textbf{connessi} in
$S$ se esiste un percorso in $S$ tra i due formato solo da pixel in $S$. L'insieme
dei pixel connessi in $S$ si chiama \textbf{componente connessa}, se esiste una
sola componente ed è connessa allora $S$ è chiamato \textbf{insieme connesso}.

Sia $R$ un sottoinsieme di pixel dell'immagine, $R$ è una \textbf{regione} dell'immagine
se è un insieme connesso. Date due regioni $R_1,R_2$, se sono \textbf{adiacenti}
allora la loro unione forma un insieme connesso altrimenti sono disconnessi.

\begin{nota}
    Le definizioni variano in base alla vicinanza.
\end{nota}

Date delle regioni disgiunte allora, ogni singola regione può essere chiamata
\textbf{foreground} mentre il complemento è il \textbf{background}. Definite le
regioni possiamo definire due bordi:
\begin{itemize}
    \item bordo interno sono i pixel del foreground adiacenti ai pixel del background
    \item bordo esterno sono i pixel del background adiacenti ai pixel del foreground
\end{itemize}

Possiamo definire diverse distanze tra i pixel:
\begin{itemize}
    \item \textbf{euclidea}:
          \begin{equation}
              D_e(p,q) = \sqrt{(x_p-x_q)^2+(y_p-y_q)^2}
          \end{equation}
    \item \textbf{city-block}:
          \begin{equation}
              D_4(p,q) = |x_p-x_q|+|y_p-y_q|
          \end{equation}
    \item \textbf{chessboard}:
          \begin{equation}
              D_8(p,q) = max(|x_p-x_q|,|y_p-y_q|)
          \end{equation}
\end{itemize}

Definiremo diverse tipologie di operazioni:
\begin{itemize}
    \item \textbf{Operazioni element-wise}: operazioni puntuali tra due matrici
    \item \textbf{Operazioni matriciali}: operazioni tra matrici
    \item \textbf{Operazioni lineari}: operatori $\mathcal{H}[f(x,y)] = g(x,y)$
          che soddisfano le seguenti proprietà:
          \begin{itemize}
              \item omogeneità: dominio uguale al codominio
              \item additività: $\mathcal{H}[\alpha f_1(x,y) + \beta f_2(x,y)] =\alpha  \mathcal{H}[f_1(x,y)] + \beta  \mathcal{H}[f_2(x,y)] $
          \end{itemize}
\end{itemize}

Si possono definire \textbf{operazioni aritmetiche} tra immagini ovvero $+, -, \cdot, \%$
che sono element-wise e applicabili su immagini delle stesse dimensioni.

\begin{nota}
    Se da una immagine rumorosa applichiamo una media della immagine con la stessa con
    diversi noise di media nulla allora riduciamo la varianza del noise sull'immagine.
\end{nota}
\begin{nota}
    Possiamo comparare le immagini mediante la loro differenza.
\end{nota}
\begin{nota}
    Possiamo correggere le ombre mediante le operazioni di prodotto e divisione.
\end{nota}

Ogni volta che applichiamo operazioni sulle immagini rischiamo di uscire dalla
quantizzazione utilizzata quindi dobbiamo sempre \textbf{clippare} e \textbf{scalare}.
Per farlo usiamo la seguente funzione:
\begin{equation}
    g_s(x,y) = K\frac{g(x,y) -g_{min}}{g_{max} g_{min}}
\end{equation}
dove $K$ è la massima intensità rappresentabile nella quantizzazione scelta e
$g_{min}, g_{max}$ sono rispettivamente la minima e la massima intensità riscontrata
nell'immagine.

Si possono definire le \textbf{operazioni insiemistiche} sulle immagini. Per
prima cosa bisogna definire le immagini secondo insiemi:
\begin{equation}
    A = \{ (x,y,z) \}
\end{equation}
dove $x,y$ sono le coordinate spaziali mentre $z$ è l'intensità, quindi le operazioni
si baseranno su questa definizione, per esempio:
\begin{equation}
    A^c=\{(x,y, K-z) |(x,y, K-z) \in A\}
\end{equation}
\begin{equation}
    A\cup B=\{max_z(a,b) |a \in A, b\in B\}
\end{equation}

Possiamo applicare anche \textbf{operazioni logiche} che si possono applicare solo
su immagini binarie, ovvero $\land, \lor, \lnot$.

Possiamo applicare anche \textbf{operazioni spaziali}, le quali si dividono in:
\begin{itemize}
    \item \textbf{single-pixel}: modificano i pixel individualmente mediante una
          trasformazione $T$ (funzione continua) dove $z$ è l'intensità iniziale
          del pixel e $s$ è quella trasformata:
          \begin{equation}
              s=T(z)
          \end{equation}
    \item \textbf{neighborhood}: operano sull'intensità basandosi sull'intensità
          dei pixel vicini (filtri)
    \item \textbf{geometric transformation}: non cambiano l'intensità ma trasformano
          le coordinate geometriche dei pixel attraverso \textbf{trasformazioni
              affini} dello spazio, quindi applicano traslazioni, scaling,
          rotazioni e shearing. Le trasformazioni affini preservano punti, linee
          e piani. Dal momento che le trasformazioni sono affini allora possiamo
          concatenare una serie di trasformazioni applicando un prodotto tra matrici.
          Dopo ciascuna trasformazione spaziale si applica sempre una interpolazione
          dell'intensità in modo da applicare un antialiasing, per fare questo
          passo si usano i seguenti metodi:
          \begin{itemize}
              \item nearest neighbor: si prende il pixel più vicino agli angoli e si usa
                    il suo colore.
              \item bilinear interpolation: servono $4$ vicini
              \item bicubic interpolation: servono $16$ vicini
          \end{itemize}
\end{itemize}

Spesso quando si devono risolvere dei task si passano le immagini ad uno
\textbf{spazio trasformato}, in generale significa che abbiamo una funzione
$r(x,y,u,v)$ chiamata \textbf{forward trasnformation kernel} che trasforma ciascun
pixel:
\begin{equation}
    T(u,v) = \sum_r \sum_c f(r,c) r(r,c,u,v)
\end{equation}
Per ogni trasformazione esiste sempre l'inversa per tornare a quella originaria
\begin{equation}
    f(r,c) = \sum_u \sum_v T(u,v) r^{-1}(r,c,u,v)
\end{equation}

Parleremo di forward trasnformation kernel \textbf{separabili} se:
\begin{equation}
    r(x,y,u,v) = r_1(x,u) \cdot r_2(y,v)
\end{equation}
Dove $x,y$ e $u,v$ sono rispettivamente le coordinate nel dominio spaziale e trasformato.
Inoltre parleremo di kernel \textbf{simmetrico} se:
\begin{equation}
    r(x,y,u,v) = r_1(x,u) \cdot r_1(y,v)
\end{equation}

Possiamo calcolare la \textbf{distribuzione di intensità} dei pixel:
\begin{equation}
    p(z_k) = \frac{n_k}{MN}
\end{equation}
dove $n_k$ numero di pixel di intensità $z_k$ e $MN$ è il numero di pixel totali.
Definendo la distribuzione possiamo calcolare:
\begin{itemize}
    \item \textbf{media}:
          \begin{equation}
              m = \sum_{k} z_k \cdot p(z_k)
          \end{equation}
    \item \textbf{varianza}:
          \begin{equation}
              \sigma^2 = \sum_{k} (z_k -m)^2 \cdot p(z_k)
          \end{equation}
    \item in generale \textbf{n-momento centrale}:
          \begin{equation}
              \mu_n(z) = \sum_{k} (z_k -m)^n \cdot p(z_k)
          \end{equation}
\end{itemize}
\begin{nota}
    $\mu_0 = 1$, $\mu_1= 0$ e $\mu_2=\sigma^2$
\end{nota}

Nel dominio spaziale abbiamo 2 tipi di trasformazioni:
\begin{itemize}
    \item \textbf{intensity trasformation}: per esempio miglioramento dell'immagine
    \item \textbf{spatial filtering}: per esempio denoising
\end{itemize}

\section{Intensity trasformation}
Trasformazioni di intensità applicate sui singoli pixels, sono indipendenti dalla
location e sono determinati solo dai valori di intensità. Si utilizzano funzioni
di trasformazioni, le operazioni non si fanno su quella originale ma su una nuova.

Abbiamo anche intensity trasformation basate sul vicinato, si usano filtri spaziali,
per esempio è la media.

L'esempio più semplice è \textbf{image inversion}:
\begin{equation}
    s = (L-1)-r
\end{equation}
Dove $L-1$ è la massima intensità rappresentabile, utile per invertire il colore.

Un'altra è \textbf{log trasformation} utilizzata per migliorare le aree scure e
comprime quelle chiare.
\begin{equation}
    s=c\log (1+r)
\end{equation}
Si somma $1$ perché così si evita di calcolare il log su numeri piccoli che
ritornano valori negativi. $c$ è la costante che varia in base all'applicazione.

Abbiamo poi la \textbf{gamma correction} che corrisponde:
\begin{equation}
    s=cr^\gamma \equiv s=c(r+\epsilon)^\gamma
\end{equation}
$c$ è la costante che varia all'applicazione, $r$ è il valore di intensità e in
base al valore di $\gamma$ otteniamo risultati diversi:
\begin{itemize}
    \item $\gamma < 1$: simile al log
    \item $\gamma > 1$: migliora le aree chiare
    \item $\gamma =c = 1$: è la funzione identità
\end{itemize}
La gamma è utile per migliorare il contrasto.

In generale per le trasformazioni sull'intensità abbiamo:
\begin{itemize}
    \item identità: mantiene tutto invariato
    \item negative: funzioni negative invertono il colore
    \item funzioni log e root, quindi curve convesse: migliorano le immagini scure
    \item funzioni inverse log e potenza, quindi curve concave: migliorano le immagini chiare
\end{itemize}
La gamma è versatile per tutti.

Altre trasformazioni che modellano il contrasto sono le \textbf{contrast stretching}
si definiscono 2 punti $r_1,s_1$ e $r_2,s_2$ e la funzione collega questi punti,
in questo modo si può corregge il contrasto.

Possiamo avere una \textbf{intensity-level function} utilizzata per incrementare l'intensità
di alcuni valori rispetto agli altri.

Possiamo decomporre l'immagine mediante un insieme di immagini binarie ognuna
rappresenta un bit della rappresentazione dell'intensità originale (\textbf{bit-plane
    slicing}), per esempio possiamo pensare ad una immagine a $1$ byte, avremo
$8$ immagini ognuna avente $1$ per i colori con codifica a $1$ sul bit associato
all'immagine. Questo ci permette di estrarre delle feature per le quali sono più
significative sulle immagini dei bit pià significativi. Possiamo utilizzare
questo metodo per comprimere l'immagine considerando solo i bit più significativi.

\section{Histogram processing}
Per ogni immagine possiamo calcolare l'istogramma delle intensità dell'immagine,
questo può essere normalizzato o meno e coincide con il count dei pixel di ogni
intensità. L'istogramma normalizzato è la distribuzione delle intensità dell'immagine.

Gli istogrammi sono utili come indicatori del contrasto:
\begin{itemize}
    \item basso contrasto: i bins sono tutti concentrati in una zona
    \item alto contrasto: i bins sono tutti
\end{itemize}
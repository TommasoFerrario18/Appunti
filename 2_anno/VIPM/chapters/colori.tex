\chapter{Colori}
I colori possono essere descritti in termini:
\begin{itemize}
    \item \textbf{fisici}: attraverso la distribuzione spettrale di energia,
          fattore di riflessione, gloss\dots
    \item \textbf{sensoristica}: stimolazione dei fotorecettori dell'occhio attraverso
          le radiazioni elettromagnetiche.
    \item \textbf{psicologica}: impressione soggettiva del colore condizionata
          dalla situazione dell'osservatore e i materiali degli oggetti.
\end{itemize}

Il colore è frutto della combinazioni di diverse componenti:
\begin{itemize}
    \item distribuzione spettrale di energia illuminante: specifica quale colore
          viene emesso dalla luce.
    \item distribuzione spettrale di riflettanza: specifica quali colori vengono
          assorbiti dall'oggetto e quali colori vengono riflessi dall'oggetto.
    \item quantità di fotoni assorbiti dall'occhio umano.
    \item sensibilità dei coni che assorbono i singoli colori della luce.
\end{itemize}

L'occhio umano è composto da:
\begin{itemize}
    \item iride: si occupa di far passare la luce nell'occhio
    \item cristallino: si occupa di regolare il fuoco
    \item retina: si occupa di assorbire il colore
\end{itemize}

Sulla retina sono presenti:
\begin{itemize}
    \item rods: si occupano della vista scotopica utile per vedere i movimenti (funzionano con scarsa luce)
    \item cones: si occupano di recepire il colore (funzionano con una buona luce)
\end{itemize}

I coni degli sono sensibili alle lunghezze d'onda dei colori blu, rosso e verde.
Le curve rosse e verdi sono più sovrapposte perché in questo modo si possono
riconoscere più sfumature composte da questi colori.

Dal momento che gli esseri umani hanno una percezione del colore a tre dimensioni
questo viene detto \textbf{tristimulus color} e i singoli componenti vengono chiamati:
\begin{itemize}
    \item \textbf{short}: percepisce le onde corte e quindi il blu:
          \begin{equation*}
              S = \int_{\lambda} \phi(\lambda) S(\lambda)d\lambda
          \end{equation*}
    \item \textbf{medium}: percepisce le onde medie e quindi il verde
          \begin{equation*}
              M = \int_{\lambda} \phi(\lambda) M(\lambda)d\lambda
          \end{equation*}
    \item \textbf{long}: percepisce le onde lunghe e quindi il rosso
          \begin{equation*}
              L = \int_{\lambda} \phi(\lambda) L(\lambda)d\lambda
          \end{equation*}
\end{itemize}

Le formule specificate precedentemente specificano la quantità massima di fotoni
percepibili dai coni di un essere umano.

Quindi per assorbire il colore, data una distribuzione spettrale di energia e
la sensibilità dei tre coni, si calcolano $3$ distribuzioni colore ottenute dall'intersezione
tra l'area di ciascun cono e quella della distribuzione del colore. Il colore è
data dalla tripla $R,G,B$  ottenute dall'integrale delle $3$ distribuzioni definite
precedentemente.

\begin{nota}
    I coni effettuano una discretizzazione del colore in $3$ canali
\end{nota}

In aggiunta tutti gli spettri che stimolano la stessa risposta dei coni sono
indistinguibili e in caso si chiamano \textbf{metameric matches}. In realtà
abbiamo 3 diverse metameric match:
\begin{itemize}
    \item \textbf{Color metamerism}: due spettri sono identici sotto lo stesso
          osservatore e la stessa luce.
    \item \textbf{Illumination metamerism}: due spettri sono identici sotto la
          stessa illuminazione.
    \item \textbf{Observer metamerism}: due spettri sono identici sotto lo stesso
          osservatore.
\end{itemize}
\section{Colorimetry}
\begin{definizione}[Colorimetry]
    La \textbf{colorimetry} è la branca della scienza del colore che si occupa
    di specificare numericamente il colore in modo che:
    \begin{itemize}
        \item se il colore viene visto da un osservatore normale, sotto le stesse
              condizioni deve essere uguale.
        \item colori uguali devono avere la stessa rappresentazione.
        \item le codifiche devono essere associate a funzioni fisiche.
    \end{itemize}
\end{definizione}

L'obiettivo è di mappare tutti i colori visibili in codifiche standard basate sulla
combinazione di colori primari.

\begin{definizione}[Grassman's law]
    La legge di \textbf{Grassman's law} è la legge che specifica che un colore può
    essere ottenuto da una combinazione lineare di una serie di colori primari.
\end{definizione}

\subsection{CIE RGB}
Una delle prime codifiche dei colori è la \textbf{CIE RGB} la quale si basa su
uno studio in cui i partecipanti correggere dei sensori modificando la quantità
dei colori primari per ottenere il colore target.

Risulta importante notare che questo standard ammette anche contributi negativi,
ovvero si ha la possibilità di sottrarre il contributo di un particolare colore.
Questo viene fatto aggiungendo al colore target la parte che viene sottratta.
\begin{nota}
    La codifica appena presentata rispetta la legge di Grassman.
\end{nota}
\subsection{CIE XYZ}
Il problema della codifica \textit{CIE RGB} è che si hanno contributi negativi,
di conseguenza, si è passati a un nuovo standard \textbf{CIE XYZ}. Sfruttando il
fatto che lo spazio di color matching è lineare, si è passati alla codifica \textbf{CIE XYZ}
inserendo i seguenti vincoli:
\begin{itemize}
    \item Il canale Y rappresenta la luminanza.
    \item Nelle curve che rappresentano i colori non ci sono valori negativi.
    \item Il colore bianco è ottenuto come: $1/3, 1/3, 1/3$
\end{itemize}
Il problema di questa codifica è che i primari non sono colori che esistono nella
realtà perciò non sono visibili. In aggiunta, la codifica rispetta anche il terzo
principio della \textbf{colorimetry}, ovvero le codifiche dipendono da una
rappresentazione fisica infatti:
\begin{equation*}
    \begin{array}{c}
        X = \int_{380}^{780} l(\lambda) \overline{x}(\lambda)d \lambda \\\\
        Y = \int_{380}^{780} l(\lambda) \overline{y}(\lambda)d \lambda \\\\
        Z = \int_{380}^{780} l(\lambda) \overline{z}(\lambda)d \lambda \\\\
    \end{array}
\end{equation*}

Dove $380-780$ sono  le frequenze visibili all'occhio umano e $\overline{x},
    \overline{y},\overline{z}$ sono i colori primari.

I colori dello spazio XYZ sono sul piano di intersezione tra i tre colori primari
e nel piano non tutti i punti sono colori reali.

Visto che i colori sono rappresentanti su un piano è stata definita la codifica
\textbf{CIE xyY} che specifica il colore usando una componente per la
\textbf{luminanza} e due componenti per la \textbf{cromaticità}.

Lo spazio colore xyY ha i colori puri sulla frontiera e il bianco è nella parte
centrale.

\begin{definizione}[Gamma cromatica]
    La \textbf{gamma cromatica} è l'insieme di tutti i colori rappresentabili
    dalla combinazione lineare di un insieme di colori primari.
\end{definizione}

Nel diagramma di cromaticità xyY, specificando un numero finito di colori primari
si può formare una forma convessa la quale contiene tutti i colori rappresentabili
da quelli primari specificati. Ciò risulta utile trovare un modo per comparare i
colori, questo potrebbe essere risolto attraverso una misura di distanza, il
problema è che sullo spazio XYZ o xyY le distanze normali non corrispondono alla
similarità del colore percepita.

Una dimostrazione di ciò sono gli \textbf{ellissi MacAdam}, ovvero ellissi che
racchiudono colori che sono percepibili simili. Questi non sono uniformi su tutto
il piano quindi le normali distanze non funzionano.
\subsection{CIE-LAB}
Per risolvere questo problema si passa a allo spazio colore con colori opponenti,
ovvero la prima componente deve essere quella \textbf{acromatica} (lucentezza) (0/100),
\textbf{rosso-verde} (-100/100) e infine \textbf{giallo-blu} (-100/100). Nella
trasformazione del canale di lucentezza si hanno due casi in base al fatto se si
ha tanta luce o meno per modellare il fatto che, a luce soffusa, il contributo
al colore viene dato da coni e dai bastoncini.

Questo standard viene chiamato \textbf{CIE-LAB} e permette l'utilizzo della norma
euclidea per confrontare i colori. L'unico problema rimanente è trovare la threshold
sulla distanza per dire che due colori sono uguali, questo dipende interamente dal
dominio.

\textbf{CIE} ha elaborato degli standard per le sorgenti luminose e la loro
temperatura come $D65$ e $D50$.

Nell'ambito dei colori è utile avere degli strumenti che li misurano:
\begin{itemize}
    \item Spettro-radiometro: misura la distribuzione spettrale di energia.
    \item Colorimetro: misura i valori di tristimolo assumendo la luce D65 e D50
          come reference del bianco.
    \item Spettrofotometro: misura la curva di riflettanza dell'oggetto
\end{itemize}
Per ogni strumento dobbiamo abbiamo due caratteristiche:
\begin{itemize}
    \item \textbf{Precisione}: quante volte misurando lo stesso oggetto otteniamo
          la stessa misura
    \item \textbf{Accurato}: quanto si avvicina la misura al valore vero.
\end{itemize}

\section{Acquisizione dell'immagine e sensibilità}
L'immagine viene generata dalla combinazione della sorgente di illuminazione e
dalle rifrazioni degli oggetti nella scena. Quando dobbiamo acquisire un'immagine
in digitale dovremmo generare una matrice di pixel che rappresenta la luce nella
scena. Per creare l'immagine digitalizzata è necessario effettuare due passi:
\begin{itemize}
    \item \textbf{sampling}: discretizzazione dei pixel
    \item \textbf{quantizzazione}: discretizzazione dei colori
\end{itemize}

Definiremo in seguito la \textbf{risoluzione spaziale} ovvero la dimensione dei
singoli pixel o anche il numero di pixel un'unità di distanza ed è dipendente dal
sampling rate.

Definiremo la \textbf{risoluzione di intensità} ovvero il numero di bit usati per
la quantizzazione.

Le risoluzioni producono degli artefatti:
\begin{itemize}
    \item risoluzione spaziale troppo piccola produce contorni spigolosi
    \item risoluzione di intensità troppo bassa produce dei falsi contorni
\end{itemize}

La prima risoluzione è sensibile alle variazione della forma mentre la seconda è
sensibile alle variazioni dell'intensità.

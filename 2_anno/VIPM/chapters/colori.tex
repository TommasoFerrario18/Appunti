\chapter{Colori}
I colori possono essere descritti in termini:
\begin{itemize}
    \item \textbf{fisici}: attraverso la distribuzione spettrale di energia, fattore 
    di riflessione, gloss\dots
    \item \textbf{sensoristica}: stimolazione dei fotorecettori dell'occhio attraverso
    le radiazioni elettromagnetiche
    \item \textbf{psicologica}: impressione soggetiva del colore condizionata 
    dalla situazione dell'osservatore e i materiali degli oggetti
\end{itemize}

Il colore è frutto della combinazioni di diverse componenti:
\begin{itemize}
    \item distribuzione spettrale di energia illuminante: specifica quale colore 
    viene emesso dalla luce
    \item distribuzione spettrale di riflettanza: specifica quali colori vengono 
    assorbiti dall'oggetto e quali colori vengono riflessi dall'oggetto.
    \item quantità di fotoni assorbiti dall'occhio umano
    \item sensibilità dei coni che assorbono i singoli colori della luce 
\end{itemize}

L'occhio umano è composto da:
\begin{itemize}
    \item iride: si occupa di far passare la luce nell'occhio
    \item cristallino: si occupa di regolare il fuoco
    \item retina: si occupa di assorbire il colore
\end{itemize}

Sulla retina sono presenti:
\begin{itemize}
    \item rods: si occupano della vista scotopica utile per vedere i movimenti (funzionano con scarsa luce)
    \item cones: si occupano di recepire il colore (funzionano con una buona luce)
\end{itemize}

I coni degli sono sensibili alle lunghezze d'onda dei colori blu, rosso e verde. Le curve 
rosse e verdi sono sovrapposte perché in questo modo si possono riconoscere più 
sfumature di colori tra rosso e verde. Dal momento che gli esseri umani hanno una 
percezione del colore a tre dimensioni questo viene detto \textbf{tristimulus color} 
e i singoli componenti vengono chiamati:
\begin{itemize}
    \item \textbf{short}: percepisce le onde corte e quindi il blu
    $$S = \int_{\lambda} \phi(\lambda) S(\lambda)d\lambda$$
    \item \textbf{medium}: percepisce le onde medie e quindi il verde
    $$M = \int_{\lambda} \phi(\lambda) M(\lambda)d\lambda$$
    \item \textbf{long}: percepisce le onde lunghe e quindi il rosso
    $$L = \int_{\lambda} \phi(\lambda) L(\lambda)d\lambda$$
\end{itemize}

Le formule specificate precedentemente specificano la quantità massima di fotoni
percepibili dai coni di un essere umano.

Quindi per assorbire il colore, data una distribuzione spettrale di energia e 
la sensibilità dei tre coni, si calcolano $3$ distribuzioni colore ottenute dall'intersezione 
tra l'area di ciascun cono e quella della distribuzione del colore. Il colore è 
data dalla tripla $R,G,B$  ottenute dall'integrale delle $3$ distribuzioni definite 
precedentemente. 

\begin{nota}
I coni effettuano una discretizzazione del colore in $3$ canali
\end{nota}

In aggiunta tutti gli spettri che stimulano la stessa risposta dei coni sono indistinguibili 
e in caso si chiamano \textbf{metameric matches}. In realtà abbiamo 3 diverse metameric 
match:
\begin{itemize}
    \item \textbf{color metamerism}: due spettri sono identici sotto lo stesso osservatore 
    e la stessa luce
    \item \textbf{illumination metamerism}: due spettri sono identici sotto la stessa illuminazione
    \item \textbf{observer metamerism}: due spettri sono identici sotto lo stesso osservatore
\end{itemize} 

\begin{definizione}
    La \textbf{colorimetry} è la branca della scienza del colore che si occupa 
    di specificare numericamente il colore in modo che:
    \begin{itemize}
        \item se il colore viene visto da un osservatore normale, sotto le stesse 
        condizioni deve essere uguale
        \item colori uguali devono avere la stessa rappresentazione
        \item le codifiche devono essere associate a funzioni fisiche
    \end{itemize}
\end{definizione}

L'obiettivo è di mappare tutti i colori visibili in codifiche standard basate sulla 
combinazione di colori primari.

\begin{definizione}
    La legge di \textbf{Grassman's law} è la legge che specifica che un colore può 
    essere ottenuto da una combinazione lineare di una serie di colori primari.
\end{definizione}

La prima codifica è \textbf{CIE RGB} la quale si è basata su uno studio composto da 
$17$ persone normali, messe davanti ad uno spiraglio con un campo visivo di 2 gradi,
nel quale vengono mostrati due colori, uno quello target, l'altro quello composto 
da rosso, verde e blu. Lo studio si basava sul fatto che le persone dovevamo modificare 
la quantità dei colori primari per ottenere il colore target. In questo modo si 
potevano ottenere delle codifiche dei colori variando le intensità dei colori primari. 
Da notare che questo standard ammette anche contributi negativi che assumono la rimozione 
di un colore nelle altre componenti. Le codifiche sono ottenute attraverso la media 
delle risposte

\begin{nota}
    Questa codifica rispetta la legge di grassman.
\end{nota}

Il problema di questa codifica è che si hanno contributi negativi quindi è stato trovato un nuovo 
standard \textbf{CIE XYZ} che corrisponde ad un cambiamento di base dei coni in modo 
che:
\begin{itemize}
    \item Y è la luminanza
    \item valori positivi nelle matching curve 
    \item il bianco è $1/3, 1/3, 1/3$
\end{itemize}
Il problema di questa codifica è che i primari non sono colori che esistono nella realtà
perciò non sono visibili. In aggiunta, la codifica rispetta anche il terzo principio 
della \textbf{colorimetry}, ovvero le codifiche dipendono da una rappresentazione 
fisica infatti:
$$\begin{array}{c}
    X = \int_{380}^{780} l(\lambda) \overline{x}(\lambda)d \lambda\\\\
    Y = \int_{380}^{780} l(\lambda) \overline{y}(\lambda)d \lambda\\\\
    Z = \int_{380}^{780} l(\lambda) \overline{z}(\lambda)d \lambda\\\\
\end{array}$$

Dove $380-780$ sono  le frequenze visibili all'occhio umano e $\overline{x},\overline{y},\overline{z}$
sono i colori primari.

I colori dello spazio XYZ sono sul piano di intersezione tra i tre colori primari
e nel piano non tutti i punti sono colori reali. 

Visto che i colori sono rappresentanti su un piano è stata definita la codifica 
\textbf{CIE xyY} che specifica il colore secondo le componenti di \textbf{luminanza} e \textbf{cromaticità}.

Lo spazio colore xyY ha i colori puri sulla frontiera e il bianco è nella parte 
centrale.

\begin{definizione}
    La \textbf{gamma} è l'insieme di tutti i colori rappresentabili dalla combinazione
    lineare di un insieme di colori primari.
\end{definizione}

Nel diagramma di cromaticità xyY, specificando un numero finito di colori primari
si può formare una forma convessa la quale contiene tutti i colori rappresentabili 
da quelli primari specificati. Ciò risulta utile trovare un modo per comparare i colori, questo potrebbe 
essere risolto attraverso una misura di distanza, il problema è che sullo spazio 
XYZ o xyY le distanze normali non corrispondono alla similarità del colore percepita. 
Una dimostrazione di ciò sono gli \textbf{ellissi MacAdam}, ovvero ellissi che 
racchiudono colori che sono precibilmente simili. Questi non sono uniformi su tutto il 
piano quindi le normali distanze non funzionano. 

Per risolvere questo problema si cambia spazio colore nel seguente modo, ovvero 
la prima componente deve essere quella \textbf{acromatica} (lucentezza) (0/100), \textbf{rosso-verde} (-100/100)
e infine \textbf{giallo-blu} (-100/100). Nella trasformazione del canale di lucentezza si hanno 
due casi in base al fatto se si ha tanta luce o meno per modellare il fatto che, 
a luce soffusa, il contributo al colore viene dato da coni e dai bastoncini. Questo 
standard viene chiamato \textbf{CIE-LAB} e permette l'utilizzo  della norma euclidea 
per confrontare i colori. L'unico problema rimanente è trovare la threshold sulla 
distanza per dire che due colori sono uguali, questo dipende interamente dal 
dominio.

\textbf{CIE} ha elaborato degli standard per le sorgenti liminose e la loro temperatura come $D65$ e $D50$.

Nell'ambito dei colori è utile avere degli strumenti che li misurano:
\begin{itemize}
    \item spettroradiometro: misura la distribuzione spettrale di energia
    \item colorimetro: misura i valori di tristimolo assumendo la luce D65 e D50
    come reference del bianco
    \item spettrofotometro: misura la curva di riflettanza dell'oggetto
\end{itemize}
Per ogni strumento dobbiamo abbiamo due caratteristiche:
\begin{itemize}
    \item \textbf{precisione}: quante volte misurando lo stesso oggetto otteniamo 
    la stessa misura
    \item \textbf{accurato}: quanto si avvicina la misura al valore vero. 
\end{itemize}


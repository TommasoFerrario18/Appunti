\documentclass[a4paper, oneside]{book}
\usepackage[italian]{babel}
\usepackage[utf8]{inputenc}
\usepackage[a4paper,top=2.5cm,bottom=2.5cm,left=2cm,right=2cm]{geometry}
\usepackage{amssymb}
\usepackage{amsthm}
\usepackage{graphics}
\usepackage{amsfonts}
\usepackage{amsmath}
\usepackage{amstext}
\usepackage{engrec}
\usepackage{rotating}
\usepackage[safe,extra]{tipa}
\usepackage{multirow}
\usepackage{hyperref}
\usepackage{enumerate}
\usepackage{braket}
\usepackage{marginnote}
\usepackage{pgfplots}
\usepackage{cancel}
\usepackage{polynom}
\usepackage{booktabs}
\usepackage{enumitem}
\usepackage{algorithm}
\usepackage{algpseudocode}
\usepackage{framed}
\usepackage{pdfpages}
\usepackage{pgfplots}
\usepackage{fancyhdr}
\usepackage{caption}
\usepackage{subcaption}
\usepackage{setspace}
\usepackage{hyperref}

\usepackage{tikz}\usetikzlibrary{er}\tikzset{multi  attribute /.style={attribute
    ,double  distance =1.5pt}}\tikzset{derived  attribute /.style={attribute
    ,dashed}}\tikzset{total /.style={double  distance =1.5pt}}\tikzset{every
  entity /.style={draw=orange , fill=orange!20}}\tikzset{every  attribute
  /.style={draw=MediumPurple1, fill=MediumPurple1!20}}\tikzset{every
  relationship /.style={draw=Chartreuse2,
    fill=Chartreuse2!20}}\newcommand{\key}[1]{\underline{#1}}
\usetikzlibrary{arrows.meta}
\usetikzlibrary{decorations.markings}
\usetikzlibrary{arrows,shapes, shapes.geometric,backgrounds,petri}
\tikzset{
  place/.style={
    circle,
    thick,
    draw=black,
    minimum size=6mm,
  },
  transition/.style={
    rectangle,
    thick,
    fill=black,
    minimum width=8mm,
    inner ysep=2pt
  },
  transitionv/.style={
    rectangle,
    thick,
    fill=black,
    minimum height=8mm,
    inner xsep=2pt
  }
}
\tikzset{elliptic state/.style={draw,ellipse}}

\usetikzlibrary{automata,positioning, calc}

\pagestyle{fancy}
\fancyhead[L,RO]{\slshape \rightmark}
\fancyfoot[C]{\thepage}

\title{Sistemi complessi e incerti}
\author{Tommaso Ferrario (\href{https://github.com/TommasoFerrario18}{@TommasoFerrario18}) \\\\
Telemaco Terzi (\href{https://github.com/Tezze2001}{@Tezze2001})}
\date{October 2023}

\pgfplotsset{compat=1.13}

\begin{document}

\maketitle

\newtheorem{teorema}{Teorema}
\newtheorem{dimostrazione}{Dimostrazione}
\newtheorem{definizione}{Definizione}
\newtheorem{esempio}{Esempio}
\newtheorem{osservazione}{Osservazione}
\newtheorem{nota}{Nota}
\newtheorem{corollario}{Corollario}
\tableofcontents
\renewcommand{\chaptermark}[1]{
  \markboth{\chaptername
    \ \thechapter.\ #1}{}}
\renewcommand{\sectionmark}[1]{\markright{\thesection.\ #1}}
\chapter{Sistemi incerti}
Si studieranno sistemi incerti che possono essere:
\begin{itemize}
    \item interpretazione dei testi
    \item incertezza dei mercati
    \item valutare il valore degli oggetti
    \item valutare un rischio
    \item prendere decisioni, l'incertezza può avere un'accezione positiva
    \item etc$\dots$
\end{itemize}

Lo studio dell'incertezza è fondamentale in machine learning, dal momento che 
gli algoritmi hanno bisogno di dati e questi possono avere un certo livello di 
incertezza. Fondamentale sarà capire:
\begin{itemize}
    \item l'\textbf{oggetto} dell'incertezza e la sua fonte
    \item il \textbf{livello} e la \textbf{magnitudine} dell'incertezza
    \item \textbf{come} comunicare l'incertezza, ovvero in che forma ed espressione
    \item \textbf{preché} comunicare l'incertezza, ovvero lo scopo di questa comunicazione
\end{itemize}

\begin{definizione}
    L'\textbf{incertezza}  è l'imperfezione dell'informazione dovuta a diverse cause
\end{definizione}
\begin{definizione}
    \textbf{Agenti} sono i soggetti che esprimono l'incertezza
\end{definizione}
\begin{definizione}
    \textbf{Oggetti} sono gli oggetti su cui un agente esprime incertezza
\end{definizione}
\begin{definizione}
    \textbf{Proprietà} sono le proprietà dell'oggetto su cui un agente ha incertezza
\end{definizione}
Ogni agente ha un'opinione su una proprietà degli oggetti.

L'incertezza può essere dovuta principalemnte da:
\begin{itemize}
    \item \textbf{informazioni incomplete} dovuto a diverse cause:
    \begin{itemize}
        \item \textbf{valori mancanti}: ex: voglio stabilire se l'auto è gialla ma
        non conosco il colore dell'auto
        \item \textbf{valori imprecisi}: ex: l'auto ha tra i 10 e 15 anni ma non 
        conosco l'anno preciso
        \item \textbf{proprietà mancanti}: conosciamo i sintomi del paziente ma non 
        ho abbastanza sintomi per determinare l'incertezza. ex: nel caso prendessimo come oggetti i
        pazienti e come proprietà i sintomi, non possiamo sapere se un paziente 
        ha una particolare malattia se ha dei particolari sintomi.
    \end{itemize}
    \item \textbf{randomicità}: non possiamo stabilire la 
    veridicità di un'informazione, ex: domani piove (non è detto)
    \item  \textbf{vaghezza}/\textbf{gradualità}: ex: l'auto è veloce, cosa significa veloce?
    (dipende dal contesto)
    \item \textbf{informazione inaffidabile}: si possono avere informazioni
    contrastanti dovuti da agenti che hanno opinioni diverse, elementi a favore
    o contro ed esempi e controesempi. In aggiunta, si possono avere informazioni
    inaffidabili perché possiamo non avere la conpleta fiducia nei dati a disposizione.
    Ex: L'auto di Giulia è rossa o nera, sicuramente non è blu.
\end{itemize}

Esistono diverse forme di incertezza e generalmente si classificano all'interno
di una tassonomia. Le classificaizoni si dividino in:
\begin{itemize}
    \item \textbf{scope}: il campo del sapere che copre quella classificazione
    \item \textbf{criteria}: sorgente dell'incertezza
\end{itemize}

Una tassonomia ha lo scopo di essere universale per raccogliere tutti i tipi di
incertezze indipendentemente dal particolare campo. Le tassonomie sono composte
da alberi i quali hanno sempre come radice \textbf{ignorance} (simula la non conoscienza),
nessuna classificazione appare in tutte le tipologie di tassonomie, ma al massimo in alcune e non 
nella stessa posizione, le tassonomie possono essere più specifiche e possono avere delle intersezioni.

Non esiste e non esisterà un unico modo per classificare l'incertezza e l'ignoranza, 
però sono utilili per essere studiati per analizzare le caratteristiche comuni.

Esistono anche tassonomie che sono legate ad un campo, ex: ecologia o ML. Per
queste tassonomie più piccolo è il dominio e più specifica è la terminologia della
classificazione. L'ontologia del dominio (semantica) può essere utile per capire dove l'incertezza
è e come azionarsi per gestirla. Quando costruiamo una classificazione di dominio
deve essere chiara tutte le parti di incertezza.

Le classificazioni sono utili per identificare l'incertezza per poi capire come 
gestirla. Le classificazioni generali tornano utili per specificare una prima classificazione 
specifica dell'incertezza nel mio dominio.

I criteri per definire una classificazione sono i seguenti:
\begin{itemize}
    \item \textbf{source}:  classificano in base alla sorgente (incertezza 
    sui dati e sugli utenti, temporale, spaziale e linguistica$\dots$)
    \item \textbf{manifestazione concrete}: classificano in base a cosa implica l'incertezza
    \item \textbf{personalità in cui risiede l'incertezza}: classifichiamo in base
    al punto di vista che vede l'incertezza
\end{itemize}

Possiamo avere classificazioni a più dimensioni, come matrici, e quindi sono più
complete e complesse. 

Essendoci diversi tipi di incertezza, si hanno diversi tipi di strumenti per 
modellare l'incertezza:
\begin{itemize}
    \item estensione della probabilità:
     \begin{itemize}
        \item Belief functions
        \item Imprecise probabilities
    \end{itemize}
    \item alternative alla probabilità:
    \begin{itemize}
        \item Possibility Theory/Logic
        \item Fuzzy Sets/Logic
        \item Modal Logic
        \item Interval Sets
        \item Rough Sets
    \end{itemize}
\end{itemize}

\section{Logica a tre valori di verità}
Spono logiche che introducono un terzo valore di verità che può assumere diversi 
significati in base a due caratteristiche:
\begin{itemize}
    \item ontologico (riguardano un qualcosa di vero): il valore può essere usato per assegnare sfumature di verità (mezzo vero)
    oppure si può assegnare un significato indefinito oppure per specificare il 
    concetto di irrilevante.
    \item epistemico (riguarda un problema di conoscienza): il valore può essere 
    usato per assegnare il valore unknown oppure può essere usato per specificare 
    inconsistenze  oppure per assegnare ad affermazioni che possono essere possibilmente 
    vere ma si scoprirà sono in futuro.
\end{itemize}


\section{Algebra booleana}
\begin{definizione}
    Una relazione d'ordine parziale $R$ è una relazione binaria:
    \begin{itemize}
        \item riflessiva
        \item antisimmetrica
        \item transitiva
    \end{itemize}
\end{definizione}

\begin{esempio}
    Possiamo definire delle relazioni d'ordine parziale per le funzioni caratteristiche 
    e una funzione è minore dell'altra se la prima mappa un sottoinsieme della seconda 
    funzione caratteristica.
\end{esempio}

Possiamo utilizzare il diagramma di Hasse è un diagramma che permette di rappresentare 
un insieme parzialmente ordinato senza esprimere la riflessività e transitività.

\begin{definizione}
    Dato un poset $ \langle\mathcal{P}, \le \rangle$, due elementi 
    $a,b\in \mathcal{P}$ si dicono \textbf{comparabili} sse $a\le b$ o $b\le a$. Altrimenti 
    si diranno \textbf{incomparabili}.
\end{definizione}

\begin{definizione}
    Data una relazione di ordine parziale $\le$ su un poset $\mathcal{P}$, se $\forall b,a\in \mathcal{P}$
    $a\le b$ o $b\le a$ e $b\ne a$ allora $\le$ è una \textbf{relazione di ordine parziale ristretta} e 
    si segnerà con $<$.
\end{definizione}

\begin{definizione}
    Data una relazione di ordine parziale $\le$ su un poset $\mathcal{P}$, se $\forall b,a\in \mathcal{P}$
    $a\le b$ o $b\le a$ allora $\le$ è una \textbf{relazione di ordine totale} (ovvero quando 
    tutti gli elementi sono comparabili)
\end{definizione}

Per rappresentare un poset possiamo utilizzare dei diagrammi di Hasse, ovvero diagrammi 
che mostrano le relazioni di antisimmetria, riflessività e transitività attraverso 
un ordine topologico e collegamenti non orientati.

\begin{definizione}
    Dato un Poset $\mathcal{P}$, definiremo $0\in \mathcal{P}$ il \textbf{minimo}
    se $\forall p \in \mathcal{P}\implies 0 \le p$ 
\end{definizione}
\begin{definizione}
    Dato un Poset $\mathcal{P}$, definiremo $1\in \mathcal{P}$ il \textbf{massimo}
    se $\forall p \in \mathcal{P}\implies p\le 1$ 
\end{definizione}

\begin{definizione}
    Un poset con un minimo e un massimo è un \textbf{poset limitato} e si scrive 
    $\langle \mathcal{P},\le , 0, 1\rangle$
\end{definizione}

\begin{nota}
    elemento minimo e massimo di un insieme parzialemente ordinato POSET non sempre 
    esistono
\end{nota}

\begin{definizione}
    Sia $S\ne \emptyset$ tale che $S\subseteq \mathcal{P}$, definiremo:
    \begin{itemize}
        \item $x\in \mathcal{P}$ \textbf{least upper bound} ($x=\sup(S)$, $x=\lor S$) sse:
        \begin{itemize}
            \item $\forall s\in S, s\le x$
            \item se $c\in \mathcal{P}$ soddisfa $s\le c,\forall s\in S$ allora $x\le c$
        \end{itemize}
        Se vale solo la prima condizione allora si dirà \textbf{upper bound} di $S$
        \item $y\in \mathcal{P}$ \textbf{greatest lower bound} ($y=\inf(S)$, $y=\land S$) sse:
        \begin{itemize}
            \item $\forall s\in S, y\le s$
            \item se $d\in \mathcal{P}$ soddisfa $d\le s,\forall s\in S$ allora $d\le y$
        \end{itemize}
        Se vale solo la prima condizione allora si dirà \textbf{lower bound} di $S$
    \end{itemize}
\end{definizione}

\begin{osservazione}
    Dato il poset $\langle \mathcal{P},\le\rangle$ allora:
    \begin{itemize}
        \item sia $Y\subseteq\mathcal{P}$ allora $\sup (Y)$ e $\inf (Y)$ se esistono
        sono unici
        \item sia $Y_1,Y_2\subseteq\mathcal{P}$ tale che $Y_1\le Y_2$ (ovvero tutti 
        gli elementi di $Y_1$ sono minori di tutti gli elementi di $Y_2$) allora entrambi
        $\inf(Y_1)$ e $\sup (Y_2)$ esistono, sono unici e $\inf(Y_1)\le \sup(Y_2)$
        \item sia $x\le y\iff x\land y = x \land x\lor y=y$ 
    \end{itemize}
\end{osservazione}

\begin{nota}
    Dato il poset $\langle \mathcal{P},\le\rangle$ che ha un $\lor \mathcal{P}$ e 
    un $\land\mathcal{P}$ e quindi è limitato, allora 
    $$\land\mathcal{P}\le x\le \lor \mathcal{P}, \forall x\in \mathcal{P}$$ 
\end{nota}


\begin{definizione}
    Un \textbf{reticolo} è un poset $\langle \mathcal{L}, \le \rangle$ tale che 
    $\forall x,y\in\mathcal{L}$ esiste il $\sup(\{x,y\})$ e $\inf(\{x,y\})$
\end{definizione}
nei reticoli il $\inf \equiv \text{meet}$ ($x\land y$) e  $\sup \equiv \text{join}$ ($x\lor y$)

Più precisamente si avrà:
\begin{itemize}
    \item $a\land b \le \{a,b\}$ e se $x\le  \{a,b\}\implies x\le a\land b$
    \item $\{a,b\} \le a\lor b$ e se $\{a,b\}\le x \implies a\lor b \le x$
\end{itemize}
\begin{definizione}
    Un \textbf{reticolo} è \textbf{completo} sse $\inf(S)$ e $\sup(S)$ esistono 
    sempre $\forall S \ne \emptyset \subseteq\mathcal{L}$
\end{definizione}

Il $\sup$ e l'$\inf$ nei reticoli possono essere visti come delle operazioni binarie
dal momento che esistono sempre generando la seguente struttura algebrica $\langle \mathcal{L}, \land,\lor \rangle$:
$$\land : \mathcal{L} \times \mathcal{L} \to \mathcal{L}, (x,y) \to x\land y :=\inf(\{x,y\})$$
$$\lor : \mathcal{L} \times \mathcal{L} \to \mathcal{L}, (x,y) \to x\lor y :=\sup(\{x,y\})$$


\begin{teorema}
    Dato il seguente reticolo $\langle \mathcal{L}, \le \rangle$ allora è associata 
    la struttura algebrica $\langle \mathcal{L}, \lor, \land\rangle$ che contiene le 
    operazioni di meet e join sul reticolo, allora le operazioni rispettano 
    le seguenti proprietà $\forall x,y,z\in \mathcal{L}$:
    \begin{itemize}
        \item \textbf{idempotenza}:
        $$x\land x = x \quad x\lor x = x$$ 
        \item \textbf{commutativa}:
        $$x\land y = y\land x \quad x\lor y = y\lor x$$ 
        \item \textbf{associativa}:
        $$x\land (y \land z) = (x \land y)\land z \quad x\lor (y \lor z) =( x\lor y)\lor z$$ 
        \item \textbf{assorbimento}:
        $$x\land (x \lor y) =x\quad x\lor (x \land y) =x$$  
        \item inoltre:
        $$x\le y \equiv x=x\land y\quad x\le y \equiv y=x\lor y$$ 
    \end{itemize} 
\end{teorema}

Possiamo partire da un insieme generico definire due operazioni che sono commutative associative 
e assorbimento, con queste operazioni possiamo definire un reticolo.

\begin{teorema}
    Data la seguente struttura algebrica $\langle \mathcal{L}, \lor, \land\rangle$
    dove:
    \begin{itemize}
        \item $\mathcal{L}\ne \emptyset$
        \item $\land$ e $\lor$ sono due operazioni su $\mathcal{L}$ commutative, 
        associative e con l'assorbimento
    \end{itemize}
    Se definiamo una relazione binaria $\le\subseteq \mathcal{L}\times \mathcal{L}$
    tale che 
    $$x\le y \iff y=x\lor y \equiv x\le y \iff x= x\land y$$
    allora $\langle \mathcal{L},\le\rangle$ è un reticolo tale che 
    $$x\lor y = \sup \{x,y\} \quad x\land y = \inf \{x,y\}$$
\end{teorema}

Quindi a partire da un reticolo poset possiamo formare una struttura algebrica alla quale 
sarà associato un reticolo che è lo stesso di quello di partenza. 

Al contrario da una struttura algebrica di un reticolo possiamo formare un reticolo 
il quale sarà effettivamente lo stesso del reticolo poset associato.

esistono reticoli che non sono distributivi e reticoli che sono distributivi.

\begin{nota}
    Sia $\langle \mathcal{L}, \lor, \land\rangle$ una struttura algebrica che formula 
    un reticolo allora:
    \begin{itemize}
        \item Se l'elemento neutro dell'operazione $\lor$ esiste (quindi $0$), allora 
        questo elemento è unico ed è il minimo elemento rispetto alla relazione $\le$.
        \item Se l'elemento neutro dell'operazione $\land$ esiste (quindi $1$), allora 
        questo elemento è unico ed è il massimo elemento rispetto alla relazione $\le$.
    \end{itemize}
\end{nota}

\begin{teorema}
    Sia $\langle \mathcal{L}, \le, 0\rangle$ un reticolo limitato inferiormente allora 
    $\lor$ rispetta le seguenti proprietà:
    \begin{itemize}
        \item \textbf{idempotenza}:
        $$ x\lor x = x$$ 
        \item \textbf{commutativa}:
        $$x\lor y = y\lor x$$ 
        \item \textbf{associativa}:
        $$x\lor (y \lor z) =( x\lor y)\lor z$$ 
        \item \textbf{elemento neutro}:
        $$x\lor 0 =x$$  
    \end{itemize}
    Quindi $\langle \mathcal{L}, \lor, 0\rangle$ è un monoide commmutativo. Sotto queste 
    ipotesi allora possiamo dire che dal monoide commutativo possiamo ottenere un 
    reticolo dove $0$ è l'elemento minimo.
\end{teorema}

\begin{definizione}
    Dati due reticoli $\mathcal{L}_1, \mathcal{L_2}$ sono detti isomorfi se esiste 
    una mappa biettiva $\phi:\mathcal{L}_1\times \mathcal{L}_2$ tale che 
    per un arbitrario $x,y\in \mathcal{L}_1$
    $$\phi(x\land_1 y) = \phi(x)\land_2\phi(y) \quad \phi(x\lor_1 y) = \phi(x)\lor_2\phi(y)$$
    Se $\phi$ è un isomorfismo di reticoli allora $\phi^{-1}$ è anch'esso un isomorfismo
    di reticoli. L'isomorfismo tra reticoli preserva anche l'ordinamento quindi è anche 
    un isomorfismo di poset.
\end{definizione}

\begin{definizione}
    Se $\mathcal{L}$ è un reticolo distributivo se le operazioni rispettano la proprietà
    distributiva:
    \begin{itemize}
        \item $\forall x,y,z\in \mathcal{L}, x\lor (y\land z)= (x\lor y) \land (x\lor z)$
        \item $\forall x,y,z\in \mathcal{L}, x\land (y\lor z)= (x\land y) \lor (x\land z)$
    \end{itemize}
\end{definizione}

\begin{teorema}
    Sia $\langle \mathcal{L}, \land, \lor \rangle$ una struttura algebrica e si ha:
    \begin{itemize}
        \item $\mathcal{L}\ne \emptyset$
        \item $\land$ e $\lor$ sono due operazioni binarie che soddisfano le seguenti 
        proprietà:
        \begin{itemize}
            \item $x=x\land(x\lor y)$
            \item $x\land (y\lor z) = (z\land x) \lor (y\land x)$
        \end{itemize}
    \end{itemize}
    Allora $\mathcal{L}$ è un reticolo distributivo.
\end{teorema}


\begin{definizione}
    Se $\mathcal{L}$ è un reticolo limitato, allora identifichiamo $y\in\mathcal{L}$
    il complemento di $x\in \mathcal{L}$ tale che 
    $$x\lor y = 1 \quad x\land y = 0$$
\end{definizione}
se $x$ è il complemento di $y$ allora $y$ è il complemento di $x$.
Inoltre $0\lor 1= 1$ e $0\land 1 = 0$.

Nei reticoli distribuiti il complemento di un elemento, se esiste, è unico. Per 
reticoli non distributivi allora la proprietà non è unica. 

\begin{definizione}
    Un reticolo booleano è una struttura $\langle \mathcal{B}, \land,\lor, ',0,1\rangle$ 
    dove:
    \begin{itemize}
        \item $\mathcal{B}$ è un insieme contenente due elementi distinti $1,0$.
        \item $\land$ e $\lor$ sono due operazionin binarie su $\mathcal{B}$ dove 
        la struttura algebrica $\langle \mathcal{B}, \land,\lor, 0,1\rangle$  è 
        un reticolo distribuito limitato dagli elementi $0,1$, quidni $\forall x\in \mathcal{B}$
        $0\le x\le 1$.
        \item $\forall x \in \mathcal{B}, \exists x'\in \mathcal{B}$ tale che 
        $x\land x' = 0$ e $x\lor x' = 1$, quindi $x'$ è il complemento.
    \end{itemize}
\end{definizione}

Quindi un'algebra booleana è un reticolo distribuito, complementabile e limitato.
Avremo quindi l'operazione di complemento ovvero $':\mathcal{B}\to \mathcal{B}$
 tale che restituisce l'elemento complemento che è unico.

\begin{esempio}
    Un reticolo booleano è la famosa tabella di verità dell'and e or, il complemento 
    è rappresentat dall'operazione complemneto.
\end{esempio}

\begin{definizione}
    Un'algebra booleana è una struttura $\langle \mathcal{B}, \land,\lor, ',0,1\rangle$ 
    dove:
    \begin{itemize}
        \item $\mathcal{B} = \{0,1\}$
        \item $\land$ e $\lor$ sono due operazioni binarie su $\mathcal{B}$
        \item $':\mathcal{B}\to \mathcal{B}$ è l'operazione di complemento
    \end{itemize}
    tale che $\forall x,y,z\in \mathcal{B}$ valgono i seguenti assiomi:
    \begin{itemize}
        \item $x\lor y = y\lor x \quad x\land y = y\land x $
        \item $x\land (y\lor z) = (x\land y) \lor (x\land z) \quad x\lor (y\land z) = (x\lor y) \land (x\lor z)$
        \item $x\lor 0 = x\quad x\land 1 = x$ 
        \item $x\lor x' = 1\quad x\land x' = 0$ 
    \end{itemize}
\end{definizione}

\begin{definizione}
    Un'algebra booleana può essere definita da una struttura $\langle \mathcal{B}, \land,\lor, ',0,1\rangle$ 
    tale che $\forall a,b,c\in \mathcal{B}$ valgono i seguenti assiomi:
    \begin{itemize}
        \item $a\land b = (a'\lor b')'$
        \item $a \lor b = b \lor a$
        \item $a\lor (b\lor c) = (a\lor b)\lor c$ 
        \item $(a\land b)\lor (a\land b') = a$ 
    \end{itemize}
\end{definizione}

\begin{teorema}
    Ogni algebra booleanda è possibile prova le  seguenti affermazioni:
    \begin{itemize}
        \item Proprietà del reticolo:
        \begin{itemize}
            \item \textbf{idempotenza}:
            $$x\land x = x \quad x\lor x = x$$ 
            \item \textbf{commutativa}:
            $$x\land y = y\land x \quad x\lor y = y\lor x$$ 
            \item \textbf{associativa}:
            $$x\land (y \land z) = (x \land y)\land z \quad x\lor (y \lor z) =( x\lor y)\lor z$$ 
            \item \textbf{assorbimento}:
            $$x\land (x \lor y) =x\quad x\lor (x \land y) =x$$  
        \end{itemize} 
        \item Proprietà distributive:
        \begin{itemize}
            \item $\forall x,y,z\in \mathcal{L}, x\lor (y\land z)= (x\lor y) \land (x\lor z)$
            \item $\forall x,y,z\in \mathcal{L}, x\land (y\lor z)= (x\land y) \lor (x\land z)$
        \end{itemize}
        \item Elementi neutri e unità:
        \begin{itemize}
            \item $x\lor 0 = x\quad x\lor 1 = 1$
            \item $x\land 0 = 0\quad x\land 1 =x$
        \end{itemize}
        \item condizioni di ortocomplementazione:
        \begin{itemize}
            \item \textbf{involuzione}: $(x')'= x$
            \item \textbf{leggi di De Morgan}: $(x\land y)' = x'\lor y' \quad (x\lor y)' = x'\land y'$
            \item $x\lor x' = 1$
            \item $x\land x' = 0$
        \end{itemize}

    \end{itemize}
\end{teorema}

\begin{definizione}
    Sia $\langle \mathcal{B}, \land,\lor\rangle$ una struttura algebrica dove:
    \begin{itemize}
        \item $\mathcal{L}\ne \emptyset$
        \item $\land$ e $\lor$ sono due operazioni su  $\mathcal{L}$ che soddisfano:
        \begin{itemize}
            \item $x= x\land (x\lor y)$
            \item $x\land (y\lor z) = (z\land x) \lor (y\land x)$
            \item $x\land(y\lor y') = x\lor(y\land y')$
        \end{itemize}
    \end{itemize}
    allora $\mathcal{L}$ è un reticolo booleano.
\end{definizione}


\begin{definizione}
    Date due algebre $\mathcal{B}_1$ e $\mathcal{B}_2$ sono isomorfe se esiste un
    isomorfismo tra reticoli $\phi:\mathcal{B}_1\to \mathcal{B}_2$ che preserva 
    l'elemento complemento, $\forall x\in \mathcal{B}_1, \phi(x') = \phi(x)'$
\end{definizione}


\end{document}
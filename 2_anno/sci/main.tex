\documentclass[a4paper, oneside]{book}
\usepackage[italian]{babel}
\usepackage[utf8]{inputenc}
\usepackage[a4paper,top=2.5cm,bottom=2.5cm,left=2cm,right=2cm]{geometry}
\usepackage{amssymb}
\usepackage{amsthm}
\usepackage{graphics}
\usepackage{amsfonts}
\usepackage{amsmath}
\usepackage{amstext}
\usepackage{engrec}
\usepackage{rotating}
\usepackage[safe,extra]{tipa}
\usepackage{multirow}
\usepackage{hyperref}
\usepackage{enumerate}
\usepackage{braket}
\usepackage{marginnote}
\usepackage{pgfplots}
\usepackage{cancel}
\usepackage{polynom}
\usepackage{booktabs}
\usepackage{enumitem}
\usepackage{algorithm}
\usepackage{algpseudocode}
\usepackage{framed}
\usepackage{pdfpages}
\usepackage{pgfplots}
\usepackage{fancyhdr}
\usepackage{caption}
\usepackage{subcaption}
\usepackage{setspace}
\usepackage{hyperref}

\usepackage{tikz}\usetikzlibrary{er}\tikzset{multi  attribute /.style={attribute
    ,double  distance =1.5pt}}\tikzset{derived  attribute /.style={attribute
    ,dashed}}\tikzset{total /.style={double  distance =1.5pt}}\tikzset{every
  entity /.style={draw=orange , fill=orange!20}}\tikzset{every  attribute
  /.style={draw=MediumPurple1, fill=MediumPurple1!20}}\tikzset{every
  relationship /.style={draw=Chartreuse2,
    fill=Chartreuse2!20}}\newcommand{\key}[1]{\underline{#1}}
\usetikzlibrary{arrows.meta}
\usetikzlibrary{decorations.markings}
\usetikzlibrary{arrows,shapes, shapes.geometric,backgrounds,petri}
\tikzset{
  place/.style={
    circle,
    thick,
    draw=black,
    minimum size=6mm,
  },
  transition/.style={
    rectangle,
    thick,
    fill=black,
    minimum width=8mm,
    inner ysep=2pt
  },
  transitionv/.style={
    rectangle,
    thick,
    fill=black,
    minimum height=8mm,
    inner xsep=2pt
  }
}
\tikzset{elliptic state/.style={draw,ellipse}}

\usetikzlibrary{automata,positioning, calc}

\pagestyle{fancy}
\fancyhead[L,RO]{\slshape \rightmark}
\fancyfoot[C]{\thepage}

\title{Sistemi complessi e incerti}
\author{Tommaso Ferrario (\href{https://github.com/TommasoFerrario18}{@TommasoFerrario18}) \\\\
Telemaco Terzi (\href{https://github.com/Tezze2001}{@Tezze2001})}
\date{October 2023}

\pgfplotsset{compat=1.13}

\begin{document}

\maketitle

\newtheorem{teorema}{Teorema}
\newtheorem{dimostrazione}{Dimostrazione}
\newtheorem{definizione}{Definizione}
\newtheorem{esempio}{Esempio}
\newtheorem{osservazione}{Osservazione}
\newtheorem{nota}{Nota}
\newtheorem{corollario}{Corollario}

\tableofcontents
\renewcommand{\chaptermark}[1]{
  \markboth{\chaptername
    \ \thechapter.\ #1}{}}
\renewcommand{\sectionmark}[1]{\markright{\thesection.\ #1}}
\chapter{Sistemi Complessi}
\section{Introduzione}
Verranno trattati degli stumenti per modellare e descrivere fenomeni naturali e reali
che variano nel tempo
\begin{definizione}[\textbf{Sistema complesso}]
    Un \textbf{Sistema complesso} è un insieme di unità semplici che cooperano tra
    di loro facendo emergere dei comportamenti complessi.
\end{definizione}
Verranno analizzati modelli discreti che sono facilmente implementabili utilizzati
nelle simulazioni, un sesempio sono:
\begin{itemize}
    \item \textbf{Automi cellulari}: paridigma di calcolo locale, parallelo e omogeneo
    \item \textbf{Subshift}: modelli per definire sistemi di condifica basata su
          parole proibite
    \item \textbf{Tiling}: modelli basati sul problema del tiling, ovvero ricoprire
          una superficie con delle piastrelle in modo tale che combaciano i colori sui
          lati di due piastrelle omogenee (problema NP-Hard).
\end{itemize}

Si studieranno le prorpietà dei fenomeni reali utilizzando i modelli e si mapperanno
le domande sulle proprietà dei modelli, in questo modo si potrà rispondere alle
domande analizzando le proprietà dei modelli.



\section{Automi cellulari (alfabeto finito)}
Gli automi cellulari sono delle reti di automi, gli automi, non per forza a stati
finiti, di cui sono composte possono essere uguali (automi cellulari uniformi) o differenti (automi cellulari non
uniformi). Lo stato dell'automa cellurare in un particolare momento dell'esecuzione
sarà l'insieme degli stati dei singoli automi nel particolare momento.

\begin{nota}
    Si parla di insieme degli stati del singolo automa perché può essere una
    sequenza nel caso di CA 1D, una matrice nel caso CA 2D, ecc$\dots$
\end{nota}

La transizione dell'automa cellulare dipende direttamente dalle transizioni dei singoli automi,
i quali, quando scattano, leggono il loro stato interno e gli stati degli automi
vicini e cambiano stato di conseguenza. Il cambio della configurazione dell'automa
cellulare coincide col cambiamento di stato di tutti gli automi della rete contemporaneamente.
Per questo gli automi cellulari sono un paradigma di calcolo:
\begin{itemize}
    \item locale
    \item parallelo
    \item uniforme (non uniforme)
\end{itemize}

\begin{nota}
    Gli automi cellulari sono dei paradigmi di calcolo universale perché si
    può convertire una qualsiasi Macchina di Turing in un automa cellulare
\end{nota}

\begin{nota}
    Gli automi cellulari sono delle reti di automi, un esempio di rete di automi
    è la rete neurale, in questo caso:
    \begin{itemize}
        \item non è uniforme perché ogni neurone può avere una funzione differente
        \item non è locale perché ogni neurone può non dipendere solo dai neuroni
              vicini
        \item è parallelo perché i neuroni scattano contemporaneamente
    \end{itemize}
\end{nota}

\begin{definizione} [\textbf{sequenze bi-infinite}]
    Dato un alfabeto $A$, chiameremo \textbf{sequenze bi-infinite} sull'alfabeto
    $A$ tutte quelle sequenze che sono infinite a destra ed infinite a sinistra.
\end{definizione}

\begin{esempio}
    Sia $A = \left\{0,1\right\}$ allora:
    \begin{itemize}
        \item $01101001110011$: è una sequenza finita
        \item $01101001110011\dots$: è una sequenza infinita a destra
        \item $\dots01101001110011$: è una sequenza infinita a sinistra
        \item $\dots01101001110011\dots$: è una sequenza bi-infinita
    \end{itemize}
\end{esempio}

\begin{definizione}[\textbf{spazio delle configurazioni}]
    Lo \textbf{spazio delle configurazioni} è
    $$A^\mathbb{Z}= \left\{x|x:\mathbb{Z}\rightarrow A\right\}$$
    Coincide con l'insieme di ttte le sequenze \textbf{bi-infinite} sull'alfabeto
    $A$
\end{definizione}
Lo spazio delle configurazioni è un insieme di funzioni, ciascuna funzione rappresenta
una sequenza \textbf{bi-infinita} che associa ad ogni indice in $\mathbb{Z}$ della
sequenza un carattere dell'alfabeto.
\begin{esempio}
    Sia $x\in A^\mathbb{Z}$, si può rappresentare nel sequente modo
    $$x = \left(\dots, x_{-2}, x_{-1},x_{0},x_{1},x_{2},\dots\right), \ \forall x_i\in A$$
    Con
    $$x(i) = x_i$$
\end{esempio}

Si studiano automi cellulari infiniti perché possono catturare tutti i comportamenti
del sistema, cosa che non posso fare con gli automi finiti. Essendo infiniti,
le singole configurazioni dell'automa devono essere infite, ecco perché si utilizzano
\textbf{sequenze bi-infinite}.

\begin{definizione} [\textbf{sottosequenza}]
    Possiamo definire le \textbf{sottosequenza} di una \textbf{sequenza bi-infinita}
    nel seguente modo.

    $\forall x \in A^{\mathbb{Z}}$ e $\forall h,k\in \mathbb{Z}$ con $h\le k$
    abbiramo:
    $$x_{[h,k]}= x_hx_{h+1}\dots x_k\in A^{k-h+1}$$
    La sottosequenza sarà finita e di lunghezza $k-h+1$.
\end{definizione}

Per studiare alcune proprietà degli automi cellulari si utilizzano i concetti di
distanza.
\begin{definizione}[\textbf{distanza su un generico insieme}]
    Dato un insieme $X$, una \textbf{funzione distanza} è una qualsiasi funzione
    $d:X\times X \rightarrow \mathbb{R}_+$ che soddisfa le sequenti proprietà:
    \begin{itemize}
        \item \textbf{non degenerazione}: $\forall x,y\in X, d(x,y) = 0\iff x=y$
        \item \textbf{simmetrica}: $\forall x,y\in X, d(x,y) = d(y,x)$
        \item \textbf{disuguaglianza triangolare}: $\forall x,y,z\in X, d(x,y) \le d(x,z)+d(z,y)$
    \end{itemize}
\end{definizione}

\begin{esempio}[distanza triviale]
    Un esempio di distanza è la seguente
    $$d(x,y) =\begin{cases}
            0 & x=y    \\
            1 & x\ne y \\
        \end{cases}$$
\end{esempio}

\begin{esempio}[distanza euclidea]
    Un esempio di distanza è la seguente
    $$d(x,y) =|x-y|$$
    Generalizzabile su $\mathbb{R}^n$ con la $|.|_2$$\dots$
\end{esempio}

A noi servirà una distanza tra le configurazioni di CA per poter confrontare
la vicinanza tra di loro e studiare le varie proprietà, utilizzeremo la distanza di
\textbf{Tychonoff}.
\begin{definizione} [\textbf{distanza su $A^\mathbb{Z}$}]
    $d:A^\mathbb{Z}\times A^\mathbb{Z} \rightarrow \mathbb{R}_+$ definita nel
    seguente modo:
    $$d(x,y) = \begin{cases}
            0             & x=y        \\
            \frac{1}{2^n} & altrimenti
        \end{cases}$$
    Dove $n= \min\left\{k\in \mathbb{N} | x_{[-k,k]} \ne y_{[-k,k]}\right\}$
\end{definizione}

La distanza si può anche non centrare nello $0$ ma in un altro indice differente,
ovviamente il calcolo cambia. Sia $d_i$ la distanza con il calcolo di $n$ centrato
in $i$ allora per ottenere $n$ si prende il valore $n'$ centrato in $i$ e si somma
l'offset dall'indice $0$.

\begin{nota}
    $\forall x,y\in A^\mathbb{Z}$, $\forall n\in \mathbb{N}$ vale che
    $$d(x,y)< \frac{1}{2^n}\iff x_{[-n, n]}=_{[-n, n]}$$
\end{nota}

\begin{nota} [Proprietà $1$ della distanza] \label{prop:dist}
    $\forall x,y\in A^\mathbb{Z}$, $\forall n\in \mathbb{N}$ vale che
    $$d(x,y)< \frac{1}{2^n}\iff x_{[-n, n]}=y_{[-n, n]}$$
\end{nota}

\begin{definizione}
    $(A^\mathbb{Z}, d)$ è lo \textbf{spazio metrico} delle configurazioni dell'automa
\end{definizione}

\begin{nota} [Proprietà $2$ della distanza]
    $(A^\mathbb{Z}, d)$ è un \textbf{spazio metrico compatto} cioè ogni successione
    di eleemnti di $A^\mathbb{Z}$ ammette una sottosuccessione convergente in $A^\mathbb{Z}$.
\end{nota}
\begin{definizione} [ \textbf{Automa cellulare 1D} ]
    Un \textbf{Automa cellulare 1D}  è una tripla $\langle A, r, f\rangle$
    dove:
    \begin{itemize}
        \item $A$: insieme finito di caratteri, corrisponde all'alfabet dell'automa e
              quindi agli stati degli automi interni.
        \item $r\in \mathbb{N}$: raggio degli automi interni, specifica il raggio
              di dipendenza degli automi interni da quelli vicini. La transizione degli
              automi interni sarà dipendente dagli stati degli automi nel raggio
              di vicinanza $r$.
        \item $f: A^{2r+1}\rightarrow A$: regola locale di aggiornamento dello
              stato degli automi interni.
    \end{itemize}
\end{definizione}
Ad ogni $CA = \langle A, r, f\rangle$ viene associata una $F:A^\mathbb{Z}\rightarrow A^\mathbb{Z}$
chiamata \textbf{regola globale} che coincide con la specifica del cambiamento
di configurazione del $CA$.
\begin{definizione} [\textbf{Regola globale per CA}]
    $F:A^\mathbb{Z}\rightarrow A^\mathbb{Z}$, dati $\forall x \in A^\mathbb{Z}, i \in \mathbb{Z}$
    si definisce
    $$F(x)_i = f(x_{[i-r, i+r]})$$
    quindi
    \begin{table}[!h]
        \centering
        \begin{tabular}{ccccccccccc}
            $x$    & $=$ & $($ & $\dots$                     & $x_{i-r}$ & $\dots$                     & $x_i$ & $\dots$ & $x_{i+r}$ & $\dots$ & $)$ \\
            $F(x)$ & $=$ & $($ & \multicolumn{3}{c}{$\dots$} & $F(x)_i$  & \multicolumn{3}{c}{$\dots$} & $)$
        \end{tabular}
    \end{table}
\end{definizione}
\begin{definizione} [ \textbf{Automa cellulare 1D elementare} ]
    Gli \textbf{Automi cellulari 1D elementari} sono \textbf{Automi cellulari 1D}
    con $A=\{0,1\}$ e $r=1$
\end{definizione}

Più precisamente per rappresentare un CA  $\langle A, r, f\rangle$ ci sono 3 modi:
\begin{itemize}
    \item \textbf{tabella}: si scrive la tabella che definisce la regola locale
    \item \textbf{numero decimale}: si utilizza un numero decimale che condifica
          la regola locale, si può ricavare facilmente dalla tabella
          $$n_f = \{0,\dots, |A|^{|A|^{2r+1}-1}\}$$
    \item \textbf{grafo di De Bruijn}: si utilizza un grafo orientato
\end{itemize}


\begin{esempio} [Rappresentazione tabellare]
    Per rappresentare la regola locale in tabella, basta creare una tabella di
    $|A|^{2r+1}$ righe e $2$ colonne. Nella prima colonna si mettono per riga e in ordine
    tutte le combinazioni di input, nella seconda colonna si mette il carattere
    finale.
\end{esempio}

\begin{esempio} [Rappresentazione numero decimale]
    Per rappresentare la regola locale in numero decimale, basta rappresentare
    la regola locale in tabella e poi codificare in decimale il numero scritto in
    verticale dal basso verso l'alto nella seconda colonna. Ricorda che questo numero
    è in base $|A|$.
\end{esempio}

\begin{nota}
    Per gli CA elementari si ha che la regola locale è defina nel seguente modo
    $$f:\{0,1\}^3\rightarrow \{0,1\}$$
    Quindi si possono avere un totale di $256 = |\{0,1\}^3|
        = |C|^{|D|}$ regole locali che
    corrispondono quindi a $256$ CA 1D elementari e ogni automa è rappresentato da un
    numero $x\in [0, 255]$ che corrisponde alla conversione in decimale della regola
    locale.
\end{nota}

\begin{definizione} [\textbf{Grafo di De Bruijn}]
    Il \textbf{Grafo di De Bruijn} associato a CA $\langle A, r, f\rangle$ è $\langle V,E,l\rangle$
    dove:
    \begin{itemize}
        \item $V=A^{2r}$
        \item $E = \{(u,v)\in V\times V | u=u_1\dots u_{2r},  v=v_1\dots v_{2r} \text{ con } u_2\dots u_{2r} = v_1\dots v_{2r-1}\}$
        \item $l:E\rightarrow \{0,1\}$ funzione che etichetta gli archi nel seguente
              modo $$\forall (u,v)\in E, l(u,v) = f(u,v_{2r})=f(u_1,v)$$
    \end{itemize}
\end{definizione}

La rappresentazione a grafo è quella più riusabile perché se dovessimo cambiare
la regola locale allora basta solo aggiornare l'etichette degli archi. In aggiunta
data una regola locale generica $f$, possiamo risriverla nel segunete modo.
\begin{equation}
    f: A\times A^{2r} \rightarrow A \equiv \delta: Q\times \sigma \rightarrow Q
\end{equation}
Con $Q= A$ e $\sigma = A^{2r} $, così è possibile sottolineare la funzione di transizione
di ogni automa nel CA.
\subsection{ Mappa di Shift e teorema di Hedlund}
\begin{definizione} [\textbf{Mappa di shift}]
    La \textbf{mappa di shift} è la funzione $\sigma:A^\mathbb{Z}\rightarrow A^\mathbb{Z}$
    definita nel seguente modo
    $$\forall x\in A^\mathbb{Z}, \forall i \in Z, \sigma(x)_i=x_{i+1}$$
    \begin{table}[!h]
        \centering
        \begin{tabular}{ccccccccc}
            $x$    & $=$ & $($ & $\dots$ & $x_{i-1}$ & $x_i$     & $x_{i+1}$ & $\dots$ & $)$ \\
            $\sigma(x)$ & $=$ & $($ & $\dots$ & $x_{i}$   & $x_{i+1}$ & $x_{i+2}$ & $\dots$ & $)$
        \end{tabular}
    \end{table}
\end{definizione}

\begin{nota}
    La mappa di shift è la regola globale del CA $\langle A,r=1,f\rangle$ dove 
    $f:A^3\rightarrow A$ è definita come $f(a,b,c) = c$. Con $|A|=2$ allora $f\equiv 170$

\end{nota}

\begin{teorema} [\textbf{Hedlund}] \label{th:hedlund}
    Sia $F:A^\mathbb{Z}\rightarrow A^\mathbb{Z}$ una qualunque funzione.
    $F$ è la regola globale di un CA \textbf{sse} entrambe le seguenti affermazioni
    sono vere:
    \begin{itemize}
        \item $F$ continua
        \item $F$ commuta con $\sigma$, ovvero $ F\circ\sigma =\sigma \circ F$
    \end{itemize}
    \begin{proof}
        Dimostriamo entrambe le implicazioni:
        \begin{itemize}
            \item $\implies$: partiamo col dimostrare che $F$ sia continua
            $$\forall x \in A^\mathbb{Z}, \forall \epsilon > 0 \exists \delta> 0 \text{ t.c } \forall y \in A^\mathbb{Z}: d(y, x) < \delta \implies d(F(y), F(x))< \epsilon$$
            Bene, scegliamo arbitrariamente $x\in  A^\mathbb{Z}$ e $\epsilon >0$.
            Sia $n\in \mathbb{N}$ tale che $\frac{1}{2^n} < \epsilon $, facciamo vedere
            che $\exists \delta>0$ tale che 
            $$\forall y\in A^\mathbb{Z}, d(y,x) < \delta \implies d(F(y), F(x))< \frac{1}{2^n}$$
            con $\delta=\frac{1}{2^{n+r}}$ è vera. Perché per la prima proprietà 
            della distanza e per la sua definizione (vedi nota \ref{prop:dist}) abbiamo che $d(F(y),F(x)) < \frac{1}{2^{n}}$ 
            quindi significa che $F(x)_{[-n,n]} = F(y)_{[-n,n]}$ quindi per
            $\delta=\frac{1}{2^{n+r}}$ è vero che $d(y,x) < \frac{1}{2^{n+r}}$
            che significa sempre per la stessa proprietà che $x_{[-n-r,n+r]} = y_{[-n-r,n+r]}$.
            Tutto viene spiegato dal fatto che se non fossero uguali le sottosequenze
            di $x$ e $y$ allora non possono essere uguali le sottosequenze di $F(x), F(y)$.
            Abbiamo dimostrato la continuità.

            Successivamente dimostriamo $F\circ \sigma = \sigma \circ F\implies F(\sigma(x)) = \sigma(F(x)), \forall x \in A^\mathbb{Z}$. 
            l'uguaglianza precedente può essere riscritta come $\forall x \in A^\mathbb{Z}, \forall i \in \mathbb{Z}, (F(\sigma(x)))_i =(\sigma(F(x)))_i$. 
            Possiamo notare che 
            $$(F(\sigma(x)))_i = f(\sigma(x)_{[i-r,i+r]})=f(x_{[i-r+1,i+r+1]})$$
            Inoltre 
            $$(\sigma(F(x)))_i =F(x)_{i+1} =f(x)_{[i+1-r,i+1+r]}$$
            Abbiamo dimostrato che commutano.
            \item $\impliedby$: dobbiamo dimostrare che esiste la tripla che definisce il CA.
            Per prima cosa conosciamo l'alfabeto che è $A$ ricavato da $F$. Successivamente
            datto che $F$ è continua e $A^\mathbb{Z}$ è compatto allora $F$ è \textbf{uniformemente 
            continua}, ossia $\forall\epsilon >0,\exists \delta > 0 \text{ t.c. } \forall x,y \in A^\mathbb{Z}$
            abbiamo $d(y,x)<\delta \implies d(F(y), F(x)) < \epsilon$. Scegliamo $\epsilon = 1\implies \frac{1}{2^0}\implies n=0\implies F(y)_0= F(x)_0$
            sicuramente sappiamo che $\exists \delta >0$ tale che $$\forall x,y \in A^\mathbb{Z} d(y,x)<\delta \implies d(F(y), F(x)) < 1$$
            Ovvero $$\forall x,y \in A^\mathbb{Z} d(y,x)<\delta \implies F(y)_0= F(x)_0$$
            Sia $r\in \mathbb{N}$ il più piccolo numero t.c. $\frac{1}{2^r}<\delta$.
            Allora $\forall x,y \in A^\mathbb{Z}$
            $$d(y,x)<\frac{1}{2^r} \implies F(y)_0= F(x)_0$$
            Ovvero $\forall x,y \in A^\mathbb{Z}$
            $$y_{[-r,r]} =x_{[-r,r]}\implies F(y)_0= F(x)_0$$
            Quindi abbiamo trovato $r$.

            Facciamo vedere che $F$ sia la regola globale, ovvero determiniamo $f$, cioè
            $\forall x\in A^\mathbb{Z},\forall i\in\mathbb{Z}$ 
            $$F(x)_i = f(x_{[i-r,i+r]})$$
            Sia $f: A^{2r+1}\rightarrow A$ definita come $\forall u \in A^{2r+1}$
            $$f(u) = F(z)_0$$ dove $z$ è una qualunque configurazione di 
            $A^\mathbb{Z}$ tale che $z_{[-r,r]} = u$. Bisogna però controllare che 
            la definizione sia ben posta per ogni valore $z$. Lo è dal momento che 
            $\forall z',z''\in A^\mathbb{Z}: z'_{[-r,r]} =  z''_{[-r,r]}\implies F(z')_0 = F(z'')_0$.
            Ora mostriamo che sia vera $\forall x\in A^\mathbb{Z}$ 
            $$F(x)_i = f(x_{[i-r,i+r]}), i = 0$$
            Questo è vero perché discende dalla definizione di $f$ con $z = x$.
            Ora mostriamo che sia vera $\forall x\in A^\mathbb{Z}$ 
            $$F(x)_i = f(x_{[i-r,i+r]}), i \ne 0$$
            $$F(x)_i = (\sigma^i(F(x)))_0=(F(\sigma^i(x)))_0=f(\sigma^i(x)_{[-r,r]}) = f(x_{[i-r,i+r]}), i \ne 0$$
        \end{itemize}
    \end{proof}
\end{teorema}

\begin{definizione}
    Un elemento $x\in A^\mathbb{Z}$ è detto \textbf{configurazione spazialmente 
    periodica} sse $\exists k>0 $ t.c. $\sigma^k(x) = x$, o equivalentemente, sse 
    $\exists u\in A^k$ con $k>0, x = ^\infty u ^\infty$. Cioè $x$ è una ripetizione
    bi-infinita di $u$.
\end{definizione}

Il teorema \ref{th:hedlund} ha le seguenti conseguenze conseguenze:
\begin{itemize}
    \item se $x\in A^\mathbb{Z}$ è \textbf{spazialmente periodica} allora $y=F(x)$
    è  \textbf{spazialmente periodica}
    \begin{proof}
        Se $x\in A^\mathbb{Z}$  è spazialmente periodica $\implies\exists k>0$ t.c.
        $\sigma^k(x)=x \implies F(\sigma^k(x)) = F(x)\implies \sigma^k(F(x))$.
        Si noti che se $x$ è la ripetizione di una parola di lunghezza $k>0$ allora
        anche $F(x)$ è la ripetizione di una parola di lunghezza $k$. Perciò dopo
        un numero finito di applicazioni consecutive di $F$ (al più $k$), si riuscirà
        a riottenere la stringa già ottenuta dalle applicazioni precedenti dal momento
        che si effettua una permutazione di una stringa $k$. Infatti l'\textbf{evoluzione 
        periodica} poiché $A^k$ è un insieme finito.
    \end{proof}
    \item Dato un $CA = <A,r,f>$ con regola globale $F$, è possibile definire un  $CA '= <A,2r,f'>$ 
    con regola globale $F'$ tale che $F'=F^2$ con $f'$ definita in questo modo.
    $$f':A^{4r+1}\rightarrow A$$
    Dove $\forall u=u_1\cdots u_{4r+1}\in A^{4r+1}$
    $$f'(u) = f'(f(u_1\cdots u_{2r+1})f(u_{2}\cdots u_{2r+2})\cdots f(u_{2r+1}\cdots u_{4r+1}))$$
    Per dimostrare che $F^2$ sia una regola globale possiamo utilizzare il teorema di 
    hedlund:
    \begin{itemize}
        \item \textbf{continuità}: si perché $F$ è continua
        \item \textbf{commutatività}: $F^2\circ \sigma = F\circ (F\circ \sigma ) = F\circ (\sigma\circ F  ) = \sigma\circ (F\circ F  )  = \sigma \circ F^2$
    \end{itemize}
    Quindi è una regola globale.
    \item Siano $F$ e $G$ due regole globali di due $CA$ differenti sullo stesso 
    alfabeto $A$. $F\circ G$ è una regola globale di un CA? Possiamo dimostrarlo con
    il teorema:
    \begin{itemize}
        \item \textbf{continuità}: si perché è una composizione di funzioni continue
        \item \textbf{commutatività}: $F\circ (G\circ \sigma) = F\circ (\sigma\circ G) = (F\circ \sigma)\circ G =(\sigma\circ F)\circ G = \sigma\circ (F\circ G)$
    \end{itemize}
    \item Sia $F$ la regola globale per CA, supponiamo che $F$ sia invertibile, 
    allora $F^{-1}$ è una regola globale per CA? Dimostriamolo:
    \begin{itemize}
        \item \textbf{continuità}: si perché $A^\mathbb{Z}$ è compatto
        \item \textbf{commutatività}: $$F\circ \sigma = \sigma \circ F $$
        $$ F^{-1} \circ F\circ \sigma \circ F^{-1}= F^{-1}\circ \sigma \circ F \circ F^{-1}$$ 
        $$ (F^{-1} \circ F)\circ \sigma \circ F^{-1}= F^{-1}\circ \sigma \circ (F \circ F^{-1})$$
        $$ \sigma \circ F^{-1}= F^{-1}\circ \sigma$$
    \end{itemize}
\end{itemize}
\chapter{Sistemi incerti}
Si studieranno sistemi incerti che possono essere:
\begin{itemize}
    \item interpretazione dei testi
    \item incertezza dei mercati
    \item valutare il valore degli oggetti
    \item valutare un rischio
    \item prendere decisioni, l'incertezza può avere un'accezione positiva
    \item etc$\dots$
\end{itemize}

Lo studio dell'incertezza è fondamentale in machine learning, dal momento che 
gli algoritmi hanno bisogno di dati e questi possono avere un certo livello di 
incertezza. Fondamentale sarà capire:
\begin{itemize}
    \item l'\textbf{oggetto} dell'incertezza e la sua fonte
    \item il \textbf{livello} e la \textbf{magnitudine} dell'incertezza
    \item \textbf{come} comunicare l'incertezza, ovvero in che forma ed espressione
    \item \textbf{preché} comunicare l'incertezza, ovvero lo scopo di questa comunicazione
\end{itemize}

\begin{definizione}
    L'\textbf{incertezza}  è l'imperfezione dell'informazione dovuta a diverse cause
\end{definizione}
\begin{definizione}
    \textbf{Agenti} sono i soggetti che esprimono l'incertezza
\end{definizione}
\begin{definizione}
    \textbf{Oggetti} sono gli oggetti su cui un agente esprime incertezza
\end{definizione}
\begin{definizione}
    \textbf{Proprietà} sono le proprietà dell'oggetto su cui un agente ha incertezza
\end{definizione}
Ogni agente ha un'opinione su una proprietà degli oggetti.

L'incertezza può essere dovuta principalemnte da:
\begin{itemize}
    \item \textbf{informazioni incomplete} dovuto a diverse cause:
    \begin{itemize}
        \item \textbf{valori mancanti}: ex: voglio stabilire se l'auto è gialla ma
        non conosco il colore dell'auto
        \item \textbf{valori imprecisi}: ex: l'auto ha tra i 10 e 15 anni ma non 
        conosco l'anno preciso
        \item \textbf{proprietà mancanti}: conosciamo i sintomi del paziente ma non 
        ho abbastanza sintomi per determinare l'incertezza. ex: nel caso prendessimo come oggetti i
        pazienti e come proprietà i sintomi, non possiamo sapere se un paziente 
        ha una particolare malattia se ha dei particolari sintomi.
    \end{itemize}
    \item \textbf{randomicità}: non possiamo stabilire la 
    veridicità di un'informazione, ex: domani piove (non è detto)
    \item  \textbf{vaghezza}/\textbf{gradualità}: ex: l'auto è veloce, cosa significa veloce?
    (dipende dal contesto)
    \item \textbf{informazione inaffidabile}: si possono avere informazioni
    contrastanti dovuti da agenti che hanno opinioni diverse, elementi a favore
    o contro ed esempi e controesempi. In aggiunta, si possono avere informazioni
    inaffidabili perché possiamo non avere la conpleta fiducia nei dati a disposizione.
    Ex: L'auto di Giulia è rossa o nera, sicuramente non è blu.
\end{itemize}

Esistono diverse forme di incertezza e generalmente si classificano all'interno
di una tassonomia. Le classificaizoni si dividino in:
\begin{itemize}
    \item \textbf{scope}: il campo del sapere che copre quella classificazione
    \item \textbf{criteria}: sorgente dell'incertezza
\end{itemize}

Una tassonomia ha lo scopo di essere universale per raccogliere tutti i tipi di
incertezze indipendentemente dal particolare campo. Le tassonomie sono composte
da alberi i quali hanno sempre come radice \textbf{ignorance} (simula la non conoscienza),
nessuna classificazione appare in tutte le tipologie di tassonomie, ma al massimo in alcune e non 
nella stessa posizione, le tassonomie possono essere più specifiche e possono avere delle intersezioni.

Non esiste e non esisterà un unico modo per classificare l'incertezza e l'ignoranza, 
però sono utilili per essere studiati per analizzare le caratteristiche comuni.

Esistono anche tassonomie che sono legate ad un campo, ex: ecologia o ML. Per
queste tassonomie più piccolo è il dominio e più specifica è la terminologia della
classificazione. L'ontologia del dominio (semantica) può essere utile per capire dove l'incertezza
è e come azionarsi per gestirla. Quando costruiamo una classificazione di dominio
deve essere chiara tutte le parti di incertezza.

Le classificazioni sono utili per identificare l'incertezza per poi capire come 
gestirla. Le classificazioni generali tornano utili per specificare una prima classificazione 
specifica dell'incertezza nel mio dominio.

I criteri per definire una classificazione sono i seguenti:
\begin{itemize}
    \item \textbf{source}:  classificano in base alla sorgente (incertezza 
    sui dati e sugli utenti, temporale, spaziale e linguistica$\dots$)
    \item \textbf{manifestazione concrete}: classificano in base a cosa implica l'incertezza
    \item \textbf{personalità in cui risiede l'incertezza}: classifichiamo in base
    al punto di vista che vede l'incertezza
\end{itemize}

Possiamo avere classificazioni a più dimensioni, come matrici, e quindi sono più
complete e complesse. 

Essendoci diversi tipi di incertezza, si hanno diversi tipi di strumenti per 
modellare l'incertezza:
\begin{itemize}
    \item estensione della probabilità:
     \begin{itemize}
        \item Belief functions
        \item Imprecise probabilities
    \end{itemize}
    \item alternative alla probabilità:
    \begin{itemize}
        \item Possibility Theory/Logic
        \item Fuzzy Sets/Logic
        \item Modal Logic
        \item Interval Sets
        \item Rough Sets
    \end{itemize}
\end{itemize}

\section{Logica a tre valori di verità}
Spono logiche che introducono un terzo valore di verità che può assumere diversi 
significati in base a due caratteristiche:
\begin{itemize}
    \item ontologico (riguardano un qualcosa di vero): il valore può essere usato per assegnare sfumature di verità (mezzo vero)
    oppure si può assegnare un significato indefinito oppure per specificare il 
    concetto di irrilevante.
    \item epistemico (riguarda un problema di conoscienza): il valore può essere 
    usato per assegnare il valore unknown oppure può essere usato per specificare 
    inconsistenze  oppure per assegnare ad affermazioni che possono essere possibilmente 
    vere ma si scoprirà sono in futuro.
\end{itemize}


\section{Algebra booleana}
\begin{definizione}
    Una relazione d'ordine parziale $R$ è una relazione binaria:
    \begin{itemize}
        \item riflessiva
        \item antisimmetrica
        \item transitiva
    \end{itemize}
\end{definizione}

\begin{esempio}
    Possiamo definire delle relazioni d'ordine parziale per le funzioni caratteristiche 
    e una funzione è minore dell'altra se la prima mappa un sottoinsieme della seconda 
    funzione caratteristica.
\end{esempio}

Possiamo utilizzare il diagramma di Hasse è un diagramma che permette di rappresentare 
un insieme parzialmente ordinato senza esprimere la riflessività e transitività.

\begin{definizione}
    Dato un poset $ \langle\mathcal{P}, \le \rangle$, due elementi 
    $a,b\in \mathcal{P}$ si dicono \textbf{comparabili} sse $a\le b$ o $b\le a$. Altrimenti 
    si diranno \textbf{incomparabili}.
\end{definizione}

\begin{definizione}
    Data una relazione di ordine parziale $\le$ su un poset $\mathcal{P}$, se $\forall b,a\in \mathcal{P}$
    $a\le b$ o $b\le a$ e $b\ne a$ allora $\le$ è una \textbf{relazione di ordine parziale ristretta} e 
    si segnerà con $<$.
\end{definizione}

\begin{definizione}
    Data una relazione di ordine parziale $\le$ su un poset $\mathcal{P}$, se $\forall b,a\in \mathcal{P}$
    $a\le b$ o $b\le a$ allora $\le$ è una \textbf{relazione di ordine totale} (ovvero quando 
    tutti gli elementi sono comparabili)
\end{definizione}

Per rappresentare un poset possiamo utilizzare dei diagrammi di Hasse, ovvero diagrammi 
che mostrano le relazioni di antisimmetria, riflessività e transitività attraverso 
un ordine topologico e collegamenti non orientati.

\begin{definizione}
    Dato un Poset $\mathcal{P}$, definiremo $0\in \mathcal{P}$ il \textbf{minimo}
    se $\forall p \in \mathcal{P}\implies 0 \le p$ 
\end{definizione}
\begin{definizione}
    Dato un Poset $\mathcal{P}$, definiremo $1\in \mathcal{P}$ il \textbf{massimo}
    se $\forall p \in \mathcal{P}\implies p\le 1$ 
\end{definizione}

\begin{definizione}
    Un poset con un minimo e un massimo è un \textbf{poset limitato} e si scrive 
    $\langle \mathcal{P},\le , 0, 1\rangle$
\end{definizione}

\begin{nota}
    elemento minimo e massimo di un insieme parzialemente ordinato POSET non sempre 
    esistono
\end{nota}

\begin{definizione}
    Sia $S\ne \emptyset$ tale che $S\subseteq \mathcal{P}$, definiremo:
    \begin{itemize}
        \item $x\in \mathcal{P}$ \textbf{least upper bound} ($x=\sup(S)$, $x=\lor S$) sse:
        \begin{itemize}
            \item $\forall s\in S, s\le x$
            \item se $c\in \mathcal{P}$ soddisfa $s\le c,\forall s\in S$ allora $x\le c$
        \end{itemize}
        Se vale solo la prima condizione allora si dirà \textbf{upper bound} di $S$
        \item $y\in \mathcal{P}$ \textbf{greatest lower bound} ($y=\inf(S)$, $y=\land S$) sse:
        \begin{itemize}
            \item $\forall s\in S, y\le s$
            \item se $d\in \mathcal{P}$ soddisfa $d\le s,\forall s\in S$ allora $d\le y$
        \end{itemize}
        Se vale solo la prima condizione allora si dirà \textbf{lower bound} di $S$
    \end{itemize}
\end{definizione}

\begin{osservazione}
    Dato il poset $\langle \mathcal{P},\le\rangle$ allora:
    \begin{itemize}
        \item sia $Y\subseteq\mathcal{P}$ allora $\sup (Y)$ e $\inf (Y)$ se esistono
        sono unici
        \item sia $Y_1,Y_2\subseteq\mathcal{P}$ tale che $Y_1\le Y_2$ (ovvero tutti 
        gli elementi di $Y_1$ sono minori di tutti gli elementi di $Y_2$) allora entrambi
        $\inf(Y_1)$ e $\sup (Y_2)$ esistono, sono unici e $\inf(Y_1)\le \sup(Y_2)$
        \item sia $x\le y\iff x\land y = x \land x\lor y=y$ 
    \end{itemize}
\end{osservazione}

\begin{nota}
    Dato il poset $\langle \mathcal{P},\le\rangle$ che ha un $\lor \mathcal{P}$ e 
    un $\land\mathcal{P}$ e quindi è limitato, allora 
    $$\land\mathcal{P}\le x\le \lor \mathcal{P}, \forall x\in \mathcal{P}$$ 
\end{nota}


\begin{definizione}
    Un \textbf{reticolo} è un poset $\langle \mathcal{L}, \le \rangle$ tale che 
    $\forall x,y\in\mathcal{L}$ esiste il $\sup(\{x,y\})$ e $\inf(\{x,y\})$
\end{definizione}
nei reticoli il $\inf \equiv \text{meet}$ ($x\land y$) e  $\sup \equiv \text{join}$ ($x\lor y$)

Più precisamente si avrà:
\begin{itemize}
    \item $a\land b \le \{a,b\}$ e se $x\le  \{a,b\}\implies x\le a\land b$
    \item $\{a,b\} \le a\lor b$ e se $\{a,b\}\le x \implies a\lor b \le x$
\end{itemize}
\begin{definizione}
    Un \textbf{reticolo} è \textbf{completo} sse $\inf(S)$ e $\sup(S)$ esistono 
    sempre $\forall S \ne \emptyset \subseteq\mathcal{L}$
\end{definizione}

Il $\sup$ e l'$\inf$ nei reticoli possono essere visti come delle operazioni binarie
dal momento che esistono sempre generando la seguente struttura algebrica $\langle \mathcal{L}, \land,\lor \rangle$:
$$\land : \mathcal{L} \times \mathcal{L} \to \mathcal{L}, (x,y) \to x\land y :=\inf(\{x,y\})$$
$$\lor : \mathcal{L} \times \mathcal{L} \to \mathcal{L}, (x,y) \to x\lor y :=\sup(\{x,y\})$$


\begin{teorema}
    Dato il seguente reticolo $\langle \mathcal{L}, \le \rangle$ allora è associata 
    la struttura algebrica $\langle \mathcal{L}, \lor, \land\rangle$ che contiene le 
    operazioni di meet e join sul reticolo, allora le operazioni rispettano 
    le seguenti proprietà $\forall x,y,z\in \mathcal{L}$:
    \begin{itemize}
        \item \textbf{idempotenza}:
        $$x\land x = x \quad x\lor x = x$$ 
        \item \textbf{commutativa}:
        $$x\land y = y\land x \quad x\lor y = y\lor x$$ 
        \item \textbf{associativa}:
        $$x\land (y \land z) = (x \land y)\land z \quad x\lor (y \lor z) =( x\lor y)\lor z$$ 
        \item \textbf{assorbimento}:
        $$x\land (x \lor y) =x\quad x\lor (x \land y) =x$$  
        \item inoltre:
        $$x\le y \equiv x=x\land y\quad x\le y \equiv y=x\lor y$$ 
    \end{itemize} 
\end{teorema}

Possiamo partire da un insieme generico definire due operazioni che sono commutative associative 
e assorbimento, con queste operazioni possiamo definire un reticolo.

\begin{teorema}
    Data la seguente struttura algebrica $\langle \mathcal{L}, \lor, \land\rangle$
    dove:
    \begin{itemize}
        \item $\mathcal{L}\ne \emptyset$
        \item $\land$ e $\lor$ sono due operazioni su $\mathcal{L}$ commutative, 
        associative e con l'assorbimento
    \end{itemize}
    Se definiamo una relazione binaria $\le\subseteq \mathcal{L}\times \mathcal{L}$
    tale che 
    $$x\le y \iff y=x\lor y \equiv x\le y \iff x= x\land y$$
    allora $\langle \mathcal{L},\le\rangle$ è un reticolo tale che 
    $$x\lor y = \sup \{x,y\} \quad x\land y = \inf \{x,y\}$$
\end{teorema}

Quindi a partire da un reticolo poset possiamo formare una struttura algebrica alla quale 
sarà associato un reticolo che è lo stesso di quello di partenza. 

Al contrario da una struttura algebrica di un reticolo possiamo formare un reticolo 
il quale sarà effettivamente lo stesso del reticolo poset associato.

esistono reticoli che non sono distributivi e reticoli che sono distributivi.

\begin{nota}
    Sia $\langle \mathcal{L}, \lor, \land\rangle$ una struttura algebrica che formula 
    un reticolo allora:
    \begin{itemize}
        \item Se l'elemento neutro dell'operazione $\lor$ esiste (quindi $0$), allora 
        questo elemento è unico ed è il minimo elemento rispetto alla relazione $\le$.
        \item Se l'elemento neutro dell'operazione $\land$ esiste (quindi $1$), allora 
        questo elemento è unico ed è il massimo elemento rispetto alla relazione $\le$.
    \end{itemize}
\end{nota}

\begin{teorema}
    Sia $\langle \mathcal{L}, \le, 0\rangle$ un reticolo limitato inferiormente allora 
    $\lor$ rispetta le seguenti proprietà:
    \begin{itemize}
        \item \textbf{idempotenza}:
        $$ x\lor x = x$$ 
        \item \textbf{commutativa}:
        $$x\lor y = y\lor x$$ 
        \item \textbf{associativa}:
        $$x\lor (y \lor z) =( x\lor y)\lor z$$ 
        \item \textbf{elemento neutro}:
        $$x\lor 0 =x$$  
    \end{itemize}
    Quindi $\langle \mathcal{L}, \lor, 0\rangle$ è un monoide commmutativo. Sotto queste 
    ipotesi allora possiamo dire che dal monoide commutativo possiamo ottenere un 
    reticolo dove $0$ è l'elemento minimo.
\end{teorema}

\begin{definizione}
    Dati due reticoli $\mathcal{L}_1, \mathcal{L_2}$ sono detti isomorfi se esiste 
    una mappa biettiva $\phi:\mathcal{L}_1\times \mathcal{L}_2$ tale che 
    per un arbitrario $x,y\in \mathcal{L}_1$
    $$\phi(x\land_1 y) = \phi(x)\land_2\phi(y) \quad \phi(x\lor_1 y) = \phi(x)\lor_2\phi(y)$$
    Se $\phi$ è un isomorfismo di reticoli allora $\phi^{-1}$ è anch'esso un isomorfismo
    di reticoli. L'isomorfismo tra reticoli preserva anche l'ordinamento quindi è anche 
    un isomorfismo di poset.
\end{definizione}

\begin{definizione}
    Se $\mathcal{L}$ è un reticolo distributivo se le operazioni rispettano la proprietà
    distributiva:
    \begin{itemize}
        \item $\forall x,y,z\in \mathcal{L}, x\lor (y\land z)= (x\lor y) \land (x\lor z)$
        \item $\forall x,y,z\in \mathcal{L}, x\land (y\lor z)= (x\land y) \lor (x\land z)$
    \end{itemize}
\end{definizione}

\begin{teorema}
    Sia $\langle \mathcal{L}, \land, \lor \rangle$ una struttura algebrica e si ha:
    \begin{itemize}
        \item $\mathcal{L}\ne \emptyset$
        \item $\land$ e $\lor$ sono due operazioni binarie che soddisfano le seguenti 
        proprietà:
        \begin{itemize}
            \item $x=x\land(x\lor y)$
            \item $x\land (y\lor z) = (z\land x) \lor (y\land x)$
        \end{itemize}
    \end{itemize}
    Allora $\mathcal{L}$ è un reticolo distributivo.
\end{teorema}


\begin{definizione}
    Se $\mathcal{L}$ è un reticolo limitato, allora identifichiamo $y\in\mathcal{L}$
    il complemento di $x\in \mathcal{L}$ tale che 
    $$x\lor y = 1 \quad x\land y = 0$$
\end{definizione}
se $x$ è il complemento di $y$ allora $y$ è il complemento di $x$.
Inoltre $0\lor 1= 1$ e $0\land 1 = 0$.

Nei reticoli distribuiti il complemento di un elemento, se esiste, è unico. Per 
reticoli non distributivi allora la proprietà non è unica. 

\begin{definizione}
    Un reticolo booleano è una struttura $\langle \mathcal{B}, \land,\lor, ',0,1\rangle$ 
    dove:
    \begin{itemize}
        \item $\mathcal{B}$ è un insieme contenente due elementi distinti $1,0$.
        \item $\land$ e $\lor$ sono due operazionin binarie su $\mathcal{B}$ dove 
        la struttura algebrica $\langle \mathcal{B}, \land,\lor, 0,1\rangle$  è 
        un reticolo distribuito limitato dagli elementi $0,1$, quidni $\forall x\in \mathcal{B}$
        $0\le x\le 1$.
        \item $\forall x \in \mathcal{B}, \exists x'\in \mathcal{B}$ tale che 
        $x\land x' = 0$ e $x\lor x' = 1$, quindi $x'$ è il complemento.
    \end{itemize}
\end{definizione}

Quindi un'algebra booleana è un reticolo distribuito, complementabile e limitato.
Avremo quindi l'operazione di complemento ovvero $':\mathcal{B}\to \mathcal{B}$
 tale che restituisce l'elemento complemento che è unico.

\begin{esempio}
    Un reticolo booleano è la famosa tabella di verità dell'and e or, il complemento 
    è rappresentat dall'operazione complemneto.
\end{esempio}

\begin{definizione}
    Un'algebra booleana è una struttura $\langle \mathcal{B}, \land,\lor, ',0,1\rangle$ 
    dove:
    \begin{itemize}
        \item $\mathcal{B} = \{0,1\}$
        \item $\land$ e $\lor$ sono due operazioni binarie su $\mathcal{B}$
        \item $':\mathcal{B}\to \mathcal{B}$ è l'operazione di complemento
    \end{itemize}
    tale che $\forall x,y,z\in \mathcal{B}$ valgono i seguenti assiomi:
    \begin{itemize}
        \item $x\lor y = y\lor x \quad x\land y = y\land x $
        \item $x\land (y\lor z) = (x\land y) \lor (x\land z) \quad x\lor (y\land z) = (x\lor y) \land (x\lor z)$
        \item $x\lor 0 = x\quad x\land 1 = x$ 
        \item $x\lor x' = 1\quad x\land x' = 0$ 
    \end{itemize}
\end{definizione}

\begin{definizione}
    Un'algebra booleana può essere definita da una struttura $\langle \mathcal{B}, \land,\lor, ',0,1\rangle$ 
    tale che $\forall a,b,c\in \mathcal{B}$ valgono i seguenti assiomi:
    \begin{itemize}
        \item $a\land b = (a'\lor b')'$
        \item $a \lor b = b \lor a$
        \item $a\lor (b\lor c) = (a\lor b)\lor c$ 
        \item $(a\land b)\lor (a\land b') = a$ 
    \end{itemize}
\end{definizione}

\begin{teorema}
    Ogni algebra booleanda è possibile prova le  seguenti affermazioni:
    \begin{itemize}
        \item Proprietà del reticolo:
        \begin{itemize}
            \item \textbf{idempotenza}:
            $$x\land x = x \quad x\lor x = x$$ 
            \item \textbf{commutativa}:
            $$x\land y = y\land x \quad x\lor y = y\lor x$$ 
            \item \textbf{associativa}:
            $$x\land (y \land z) = (x \land y)\land z \quad x\lor (y \lor z) =( x\lor y)\lor z$$ 
            \item \textbf{assorbimento}:
            $$x\land (x \lor y) =x\quad x\lor (x \land y) =x$$  
        \end{itemize} 
        \item Proprietà distributive:
        \begin{itemize}
            \item $\forall x,y,z\in \mathcal{L}, x\lor (y\land z)= (x\lor y) \land (x\lor z)$
            \item $\forall x,y,z\in \mathcal{L}, x\land (y\lor z)= (x\land y) \lor (x\land z)$
        \end{itemize}
        \item Elementi neutri e unità:
        \begin{itemize}
            \item $x\lor 0 = x\quad x\lor 1 = 1$
            \item $x\land 0 = 0\quad x\land 1 =x$
        \end{itemize}
        \item condizioni di ortocomplementazione:
        \begin{itemize}
            \item \textbf{involuzione}: $(x')'= x$
            \item \textbf{leggi di De Morgan}: $(x\land y)' = x'\lor y' \quad (x\lor y)' = x'\land y'$
            \item $x\lor x' = 1$
            \item $x\land x' = 0$
        \end{itemize}

    \end{itemize}
\end{teorema}

\begin{definizione}
    Sia $\langle \mathcal{B}, \land,\lor\rangle$ una struttura algebrica dove:
    \begin{itemize}
        \item $\mathcal{L}\ne \emptyset$
        \item $\land$ e $\lor$ sono due operazioni su  $\mathcal{L}$ che soddisfano:
        \begin{itemize}
            \item $x= x\land (x\lor y)$
            \item $x\land (y\lor z) = (z\land x) \lor (y\land x)$
            \item $x\land(y\lor y') = x\lor(y\land y')$
        \end{itemize}
    \end{itemize}
    allora $\mathcal{L}$ è un reticolo booleano.
\end{definizione}


\begin{definizione}
    Date due algebre $\mathcal{B}_1$ e $\mathcal{B}_2$ sono isomorfe se esiste un
    isomorfismo tra reticoli $\phi:\mathcal{B}_1\to \mathcal{B}_2$ che preserva 
    l'elemento complemento, $\forall x\in \mathcal{B}_1, \phi(x') = \phi(x)'$
\end{definizione}


\end{document}
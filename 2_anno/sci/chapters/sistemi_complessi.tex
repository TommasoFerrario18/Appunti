\chapter{Sistemi Complessi}
\section{Introduzione}
Verranno trattati degli stumenti per modellare e descrivere fenomeni naturali e reali
che variano nel tempo
\begin{definizione}[\textbf{Sistema complesso}]
    Un \textbf{Sistema complesso} è un insieme di unità semplici che cooperano tra
    di loro facendo emergere dei comportamenti complessi.
\end{definizione}
Verranno analizzati modelli discreti che sono facilmente implementabili utilizzati
nelle simulazioni, un sesempio sono:
\begin{itemize}
    \item \textbf{Automi cellulari}: paridigma di calcolo locale, parallelo e omogeneo
    \item \textbf{Subshift}: modelli per definire sistemi di condifica basata su
    parole proibite
    \item \textbf{Tiling}: modelli basati sul problema del tiling, ovvero ricoprire
    una superficie con delle piastrelle in modo tale che combaciano i colori sui
    lati di due piastrelle omogenee (problema NP-Hard).
\end{itemize}

Si studieranno le prorpietà dei fenomeni reali utilizzando i modelli e si mapperanno
le domande sulle proprietà dei modelli, in questo modo si potrà rispondere alle 
domande analizzando le proprietà dei modelli.

\begin{nota}
    Gli automi cellulari sono dei paradigmi di calcolo universale perché si 
    può convertire una qualsiasi Macchina di Turing in un automa cellulare
\end{nota}

\begin{nota}
    Gli automi cellulari sono delle reti di automi, un esempio di rete di automi
    è la rete neurale
\end{nota}
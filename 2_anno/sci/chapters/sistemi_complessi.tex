\chapter{Sistemi Complessi}
\section{Introduzione}
Verranno trattati degli stumenti per modellare e descrivere fenomeni naturali e reali
che variano nel tempo
\begin{definizione}[\textbf{Sistema complesso}]
    Un \textbf{Sistema complesso} è un insieme di unità semplici che cooperano tra
    di loro facendo emergere dei comportamenti complessi.
\end{definizione}
Verranno analizzati modelli discreti che sono facilmente implementabili utilizzati
nelle simulazioni, un sesempio sono:
\begin{itemize}
    \item \textbf{Automi cellulari}: paridigma di calcolo locale, parallelo e omogeneo
    \item \textbf{Subshift}: modelli per definire sistemi di condifica basata su
          parole proibite
    \item \textbf{Tiling}: modelli basati sul problema del tiling, ovvero ricoprire
          una superficie con delle piastrelle in modo tale che combaciano i colori sui
          lati di due piastrelle omogenee (problema NP-Hard).
\end{itemize}

Si studieranno le prorpietà dei fenomeni reali utilizzando i modelli e si mapperanno
le domande sulle proprietà dei modelli, in questo modo si potrà rispondere alle
domande analizzando le proprietà dei modelli.



\section{Automi cellulari (alfabeto finito)}
Gli automi cellulari sono delle reti di automi, gli automi, non per forza a stati
finiti, di cui sono composte possono essere uguali (automi cellulari uniformi) o differenti (automi cellulari non
uniformi). Lo stato dell'automa cellurare in un particolare momento dell'esecuzione
sarà l'insieme degli stati dei singoli automi nel particolare momento.

\begin{nota}
    Si parla di insieme degli stati del singolo automa perché può essere una
    sequenza nel caso di CA 1D, una matrice nel caso CA 2D, ecc$\dots$
\end{nota}

La transizione dell'automa cellulare dipende direttamente dalle transizioni dei singoli automi,
i quali, quando scattano, leggono il loro stato interno e gli stati degli automi
vicini e cambiano stato di conseguenza. Il cambio della configurazione dell'automa
cellulare coincide col cambiamento di stato di tutti gli automi della rete contemporaneamente.
Per questo gli automi cellulari sono un paradigma di calcolo:
\begin{itemize}
    \item locale
    \item parallelo
    \item uniforme (non uniforme)
\end{itemize}

\begin{nota}
    Gli automi cellulari sono dei paradigmi di calcolo universale perché si
    può convertire una qualsiasi Macchina di Turing in un automa cellulare
\end{nota}

\begin{nota}
    Gli automi cellulari sono delle reti di automi, un esempio di rete di automi
    è la rete neurale, in questo caso:
    \begin{itemize}
        \item non è uniforme perché ogni neurone può avere una funzione differente
        \item non è locale perché ogni neurone può non dipendere solo dai neuroni
              vicini
        \item è parallelo perché i neuroni scattano contemporaneamente
    \end{itemize}
\end{nota}

\begin{definizione} [\textbf{sequenze bi-infinite}]
    Dato un alfabeto $A$, chiameremo \textbf{sequenze bi-infinite} sull'alfabeto
    $A$ tutte quelle sequenze che sono infinite a destra ed infinite a sinistra.
\end{definizione}

\begin{esempio}
    Sia $A = \left\{0,1\right\}$ allora:
    \begin{itemize}
        \item $01101001110011$: è una sequenza finita
        \item $01101001110011\dots$: è una sequenza infinita a destra
        \item $\dots01101001110011$: è una sequenza infinita a sinistra
        \item $\dots01101001110011\dots$: è una sequenza bi-infinita
    \end{itemize}
\end{esempio}

\begin{definizione}[\textbf{spazio delle configurazioni}]
    Lo \textbf{spazio delle configurazioni} è
    $$A^\mathbb{Z}= \left\{x|x:\mathbb{Z}\rightarrow A\right\}$$
    Coincide con l'insieme di ttte le sequenze \textbf{bi-infinite} sull'alfabeto
    $A$
\end{definizione}
Lo spazio delle configurazioni è un insieme di funzioni, ciascuna funzione rappresenta
una sequenza \textbf{bi-infinita} che associa ad ogni indice in $\mathbb{Z}$ della
sequenza un carattere dell'alfabeto.
\begin{esempio}
    Sia $x\in A^\mathbb{Z}$, si può rappresentare nel sequente modo
    $$x = \left(\dots, x_{-2}, x_{-1},x_{0},x_{1},x_{2},\dots\right), \ \forall x_i\in A$$
    Con
    $$x(i) = x_i$$
\end{esempio}

Si studiano automi cellulari infiniti perché possono catturare tutti i comportamenti
del sistema, cosa che non posso fare con gli automi finiti. Essendo infiniti,
le singole configurazioni dell'automa devono essere infite, ecco perché si utilizzano
\textbf{sequenze bi-infinite}.

\begin{definizione} [\textbf{sottosequenza}]
    Possiamo definire le \textbf{sottosequenza} di una \textbf{sequenza bi-infinita}
    nel seguente modo.

    $\forall x \in A^{\mathbb{Z}}$ e $\forall h,k\in \mathbb{Z}$ con $h\le k$
    abbiramo:
    $$x_{[h,k]}= x_hx_{h+1}\dots x_k\in A^{k-h+1}$$
    La sottosequenza sarà finita e di lunghezza $k-h+1$.
\end{definizione}

Per studiare alcune proprietà degli automi cellulari si utilizzano i concetti di
distanza.
\begin{definizione}[\textbf{distanza su un generico insieme}]
    Dato un insieme $X$, una \textbf{funzione distanza} è una qualsiasi funzione
    $d:X\times X \rightarrow \mathbb{R}_+$ che soddisfa le sequenti proprietà:
    \begin{itemize}
        \item \textbf{non degenerazione}: $\forall x,y\in X, d(x,y) = 0\iff x=y$
        \item \textbf{simmetrica}: $\forall x,y\in X, d(x,y) = d(y,x)$
        \item \textbf{disuguaglianza triangolare}: $\forall x,y,z\in X, d(x,y) \le d(x,z)+d(z,y)$
    \end{itemize}
\end{definizione}

\begin{esempio}[distanza triviale]
    Un esempio di distanza è la seguente
    $$d(x,y) =\begin{cases}
            0 & x=y    \\
            1 & x\ne y \\
        \end{cases}$$
\end{esempio}

\begin{esempio}[distanza euclidea]
    Un esempio di distanza è la seguente
    $$d(x,y) =|x-y|$$
    Generalizzabile su $\mathbb{R}^n$ con la $|.|_2$$\dots$
\end{esempio}

A noi servirà una distanza tra le configurazioni di CA per poter confrontare
la vicinanza tra di loro e studiare le varie proprietà, utilizzeremo la distanza di
\textbf{Tychonoff}.
\begin{definizione} [\textbf{distanza su $A^\mathbb{Z}$}]
    $d:A^\mathbb{Z}\times A^\mathbb{Z} \rightarrow \mathbb{R}_+$ definita nel
    seguente modo:
    $$d(x,y) = \begin{cases}
            0             & x=y        \\
            \frac{1}{2^n} & altrimenti
        \end{cases}$$
    Dove $n= \min\left\{k\in \mathbb{N} | x_{[-k,k]} \ne y_{[-k,k]}\right\}$
\end{definizione}

La distanza si può anche non centrare nello $0$ ma in un altro indice differente,
ovviamente il calcolo cambia. Sia $d_i$ la distanza con il calcolo di $n$ centrato
in $i$ allora per ottenere $n$ si prende il valore $n'$ centrato in $i$ e si somma
l'offset dall'indice $0$.

\begin{nota}
    $\forall x,y\in A^\mathbb{Z}$, $\forall n\in \mathbb{N}$ vale che
    $$d(x,y)< \frac{1}{2^n}\iff x_{[-n, n]}=_{[-n, n]}$$
\end{nota}

\begin{nota} [Proprietà $1$ della distanza] \label{prop:dist}
    $\forall x,y\in A^\mathbb{Z}$, $\forall n\in \mathbb{N}$ vale che
    $$d(x,y)< \frac{1}{2^n}\iff x_{[-n, n]}=y_{[-n, n]}$$
\end{nota}

\begin{definizione}
    $(A^\mathbb{Z}, d)$ è lo \textbf{spazio metrico} delle configurazioni dell'automa
\end{definizione}

\begin{nota} [Proprietà $2$ della distanza]
    $(A^\mathbb{Z}, d)$ è un \textbf{spazio metrico compatto} cioè ogni successione
    di eleemnti di $A^\mathbb{Z}$ ammette una sottosuccessione convergente in $A^\mathbb{Z}$.
\end{nota}
\begin{definizione} [ \textbf{Automa cellulare 1D} ]
    Un \textbf{Automa cellulare 1D}  è una tripla $\langle A, r, f\rangle$
    dove:
    \begin{itemize}
        \item $A$: insieme finito di caratteri, corrisponde all'alfabet dell'automa e
              quindi agli stati degli automi interni.
        \item $r\in \mathbb{N}$: raggio degli automi interni, specifica il raggio
              di dipendenza degli automi interni da quelli vicini. La transizione degli
              automi interni sarà dipendente dagli stati degli automi nel raggio
              di vicinanza $r$.
        \item $f: A^{2r+1}\rightarrow A$: regola locale di aggiornamento dello
              stato degli automi interni.
    \end{itemize}
\end{definizione}
Ad ogni $CA = \langle A, r, f\rangle$ viene associata una $F:A^\mathbb{Z}\rightarrow A^\mathbb{Z}$
chiamata \textbf{regola globale} che coincide con la specifica del cambiamento
di configurazione del $CA$.
\begin{definizione} [\textbf{Regola globale per CA}]
    $F:A^\mathbb{Z}\rightarrow A^\mathbb{Z}$, dati $\forall x \in A^\mathbb{Z}, i \in \mathbb{Z}$
    si definisce
    $$F(x)_i = f(x_{[i-r, i+r]})$$
    quindi
    \begin{table}[!h]
        \centering
        \begin{tabular}{ccccccccccc}
            $x$    & $=$ & $($ & $\dots$                     & $x_{i-r}$ & $\dots$                     & $x_i$ & $\dots$ & $x_{i+r}$ & $\dots$ & $)$ \\
            $F(x)$ & $=$ & $($ & \multicolumn{3}{c}{$\dots$} & $F(x)_i$  & \multicolumn{3}{c}{$\dots$} & $)$
        \end{tabular}
    \end{table}
\end{definizione}
\begin{definizione} [ \textbf{Automa cellulare 1D elementare} ]
    Gli \textbf{Automi cellulari 1D elementari} sono \textbf{Automi cellulari 1D}
    con $A=\{0,1\}$ e $r=1$
\end{definizione}

Più precisamente per rappresentare un CA  $\langle A, r, f\rangle$ ci sono 3 modi:
\begin{itemize}
    \item \textbf{tabella}: si scrive la tabella che definisce la regola locale
    \item \textbf{numero decimale}: si utilizza un numero decimale che condifica
          la regola locale, si può ricavare facilmente dalla tabella
          $$n_f = \{0,\dots, |A|^{|A|^{2r+1}-1}\}$$
    \item \textbf{grafo di De Bruijn}: si utilizza un grafo orientato
\end{itemize}


\begin{esempio} [Rappresentazione tabellare]
    Per rappresentare la regola locale in tabella, basta creare una tabella di
    $|A|^{2r+1}$ righe e $2$ colonne. Nella prima colonna si mettono per riga e in ordine
    tutte le combinazioni di input, nella seconda colonna si mette il carattere
    finale.
\end{esempio}

\begin{esempio} [Rappresentazione numero decimale]
    Per rappresentare la regola locale in numero decimale, basta rappresentare
    la regola locale in tabella e poi codificare in decimale il numero scritto in
    verticale dal basso verso l'alto nella seconda colonna. Ricorda che questo numero
    è in base $|A|$.
\end{esempio}

\begin{nota}
    Per gli CA elementari si ha che la regola locale è defina nel seguente modo
    $$f:\{0,1\}^3\rightarrow \{0,1\}$$
    Quindi si possono avere un totale di $256 = |\{0,1\}^3|
        = |C|^{|D|}$ regole locali che
    corrispondono quindi a $256$ CA 1D elementari e ogni automa è rappresentato da un
    numero $x\in [0, 255]$ che corrisponde alla conversione in decimale della regola
    locale.
\end{nota}

\begin{definizione} [\textbf{Grafo di De Bruijn}]
    Il \textbf{Grafo di De Bruijn} associato a CA $\langle A, r, f\rangle$ è $\langle V,E,l\rangle$
    dove:
    \begin{itemize}
        \item $V=A^{2r}$
        \item $E = \{(u,v)\in V\times V | u=u_1\dots u_{2r},  v=v_1\dots v_{2r} \text{ con } u_2\dots u_{2r} = v_1\dots v_{2r-1}\}$
        \item $l:E\rightarrow \{0,1\}$ funzione che etichetta gli archi nel seguente
              modo $$\forall (u,v)\in E, l(u,v) = f(u,v_{2r})=f(u_1,v)$$
    \end{itemize}
\end{definizione}

La rappresentazione a grafo è quella più riusabile perché se dovessimo cambiare
la regola locale allora basta solo aggiornare l'etichette degli archi. In aggiunta
data una regola locale generica $f$, possiamo risriverla nel segunete modo.
\begin{equation}
    f: A\times A^{2r} \rightarrow A \equiv \delta: Q\times \sigma \rightarrow Q
\end{equation}
Con $Q= A$ e $\sigma = A^{2r} $, così è possibile sottolineare la funzione di transizione
di ogni automa nel CA.
\subsection{ Mappa di Shift e teorema di Hedlund}
\begin{definizione} [\textbf{Mappa di shift}]
    La \textbf{mappa di shift} è la funzione $\sigma:A^\mathbb{Z}\rightarrow A^\mathbb{Z}$
    definita nel seguente modo
    $$\forall x\in A^\mathbb{Z}, \forall i \in Z, \sigma(x)_i=x_{i+1}$$
    \begin{table}[!h]
        \centering
        \begin{tabular}{ccccccccc}
            $x$    & $=$ & $($ & $\dots$ & $x_{i-1}$ & $x_i$     & $x_{i+1}$ & $\dots$ & $)$ \\
            $\sigma(x)$ & $=$ & $($ & $\dots$ & $x_{i}$   & $x_{i+1}$ & $x_{i+2}$ & $\dots$ & $)$
        \end{tabular}
    \end{table}
\end{definizione}

\begin{nota}
    La mappa di shift è la regola globale del CA $\langle A,r=1,f\rangle$ dove 
    $f:A^3\rightarrow A$ è definita come $f(a,b,c) = c$. Con $|A|=2$ allora $f\equiv 170$

\end{nota}

\begin{teorema} [\textbf{Hedlund}] \label{th:hedlund}
    Sia $F:A^\mathbb{Z}\rightarrow A^\mathbb{Z}$ una qualunque funzione.
    $F$ è la regola globale di un CA \textbf{sse} entrambe le seguenti affermazioni
    sono vere:
    \begin{itemize}
        \item $F$ continua
        \item $F$ commuta con $\sigma$, ovvero $ F\circ\sigma =\sigma \circ F$
    \end{itemize}
    \begin{proof}
        Dimostriamo entrambe le implicazioni:
        \begin{itemize}
            \item $\implies$: partiamo col dimostrare che $F$ sia continua
            $$\forall x \in A^\mathbb{Z}, \forall \epsilon > 0 \exists \delta> 0 \text{ t.c } \forall y \in A^\mathbb{Z}: d(y, x) < \delta \implies d(F(y), F(x))< \epsilon$$
            Bene, scegliamo arbitrariamente $x\in  A^\mathbb{Z}$ e $\epsilon >0$.
            Sia $n\in \mathbb{N}$ tale che $\frac{1}{2^n} < \epsilon $, facciamo vedere
            che $\exists \delta>0$ tale che 
            $$\forall y\in A^\mathbb{Z}, d(y,x) < \delta \implies d(F(y), F(x))< \frac{1}{2^n}$$
            con $\delta=\frac{1}{2^{n+r}}$ è vera. Perché per la prima proprietà 
            della distanza e per la sua definizione (vedi nota \ref{prop:dist}) abbiamo che $d(F(y),F(x)) < \frac{1}{2^{n}}$ 
            quindi significa che $F(x)_{[-n,n]} = F(y)_{[-n,n]}$ quindi per
            $\delta=\frac{1}{2^{n+r}}$ è vero che $d(y,x) < \frac{1}{2^{n+r}}$
            che significa sempre per la stessa proprietà che $x_{[-n-r,n+r]} = y_{[-n-r,n+r]}$.
            Tutto viene spiegato dal fatto che se non fossero uguali le sottosequenze
            di $x$ e $y$ allora non possono essere uguali le sottosequenze di $F(x), F(y)$.
            Abbiamo dimostrato la continuità.

            Successivamente dimostriamo $F\circ \sigma = \sigma \circ F\implies F(\sigma(x)) = \sigma(F(x)), \forall x \in A^\mathbb{Z}$. 
            l'uguaglianza precedente può essere riscritta come $\forall x \in A^\mathbb{Z}, \forall i \in \mathbb{Z}, (F(\sigma(x)))_i =(\sigma(F(x)))_i$. 
            Possiamo notare che 
            $$(F(\sigma(x)))_i = f(\sigma(x)_{[i-r,i+r]})=f(x_{[i-r+1,i+r+1]})$$
            Inoltre 
            $$(\sigma(F(x)))_i =F(x)_{i+1} =f(x)_{[i+1-r,i+1+r]}$$
            Abbiamo dimostrato che commutano.
            \item $\impliedby$: dobbiamo dimostrare che esiste la tripla che definisce il CA.
            Per prima cosa conosciamo l'alfabeto che è $A$ ricavato da $F$. Successivamente
            datto che $F$ è continua e $A^\mathbb{Z}$ è compatto allora $F$ è \textbf{uniformemente 
            continua}, ossia $\forall\epsilon >0,\exists \delta > 0 \text{ t.c. } \forall x,y \in A^\mathbb{Z}$
            abbiamo $d(y,x)<\delta \implies d(F(y), F(x)) < \epsilon$. Scegliamo $\epsilon = 1\implies \frac{1}{2^0}\implies n=0\implies F(y)_0= F(x)_0$
            sicuramente sappiamo che $\exists \delta >0$ tale che $$\forall x,y \in A^\mathbb{Z} d(y,x)<\delta \implies d(F(y), F(x)) < 1$$
            Ovvero $$\forall x,y \in A^\mathbb{Z} d(y,x)<\delta \implies F(y)_0= F(x)_0$$
            Sia $r\in \mathbb{N}$ il più piccolo numero t.c. $\frac{1}{2^r}<\delta$.
            Allora $\forall x,y \in A^\mathbb{Z}$
            $$d(y,x)<\frac{1}{2^r} \implies F(y)_0= F(x)_0$$
            Ovvero $\forall x,y \in A^\mathbb{Z}$
            $$y_{[-r,r]} =x_{[-r,r]}\implies F(y)_0= F(x)_0$$
            Quindi abbiamo trovato $r$.

            Facciamo vedere che $F$ sia la regola globale, ovvero determiniamo $f$, cioè
            $\forall x\in A^\mathbb{Z},\forall i\in\mathbb{Z}$ 
            $$F(x)_i = f(x_{[i-r,i+r]})$$
            Sia $f: A^{2r+1}\rightarrow A$ definita come $\forall u \in A^{2r+1}$
            $$f(u) = F(z)_0$$ dove $z$ è una qualunque configurazione di 
            $A^\mathbb{Z}$ tale che $z_{[-r,r]} = u$. Bisogna però controllare che 
            la definizione sia ben posta per ogni valore $z$. Lo è dal momento che 
            $\forall z',z''\in A^\mathbb{Z}: z'_{[-r,r]} =  z''_{[-r,r]}\implies F(z')_0 = F(z'')_0$.
            Ora mostriamo che sia vera $\forall x\in A^\mathbb{Z}$ 
            $$F(x)_i = f(x_{[i-r,i+r]}), i = 0$$
            Questo è vero perché discende dalla definizione di $f$ con $z = x$.
            Ora mostriamo che sia vera $\forall x\in A^\mathbb{Z}$ 
            $$F(x)_i = f(x_{[i-r,i+r]}), i \ne 0$$
            $$F(x)_i = (\sigma^i(F(x)))_0=(F(\sigma^i(x)))_0=f(\sigma^i(x)_{[-r,r]}) = f(x_{[i-r,i+r]}), i \ne 0$$
        \end{itemize}
    \end{proof}
\end{teorema}

\begin{definizione}
    Un elemento $x\in A^\mathbb{Z}$ è detto \textbf{configurazione spazialmente 
    periodica} sse $\exists k>0 $ t.c. $\sigma^k(x) = x$, o equivalentemente, sse 
    $\exists u\in A^k$ con $k>0, x = ^\infty u ^\infty$. Cioè $x$ è una ripetizione
    bi-infinita di $u$.
\end{definizione}

Il teorema \ref{th:hedlund} ha le seguenti conseguenze conseguenze:
\begin{itemize}
    \item se $x\in A^\mathbb{Z}$ è \textbf{spazialmente periodica} allora $y=F(x)$
    è  \textbf{spazialmente periodica}
    \begin{proof}
        Se $x\in A^\mathbb{Z}$  è spazialmente periodica $\implies\exists k>0$ t.c.
        $\sigma^k(x)=x \implies F(\sigma^k(x)) = F(x)\implies \sigma^k(F(x))$.
        Si noti che se $x$ è la ripetizione di una parola di lunghezza $k>0$ allora
        anche $F(x)$ è la ripetizione di una parola di lunghezza $k$. Perciò dopo
        un numero finito di applicazioni consecutive di $F$ (al più $k$), si riuscirà
        a riottenere la stringa già ottenuta dalle applicazioni precedenti dal momento
        che si effettua una permutazione di una stringa $k$. Infatti l'\textbf{evoluzione 
        periodica} poiché $A^k$ è un insieme finito.
    \end{proof}
    \item Dato un $CA = <A,r,f>$ con regola globale $F$, è possibile definire un  $CA '= <A,2r,f'>$ 
    con regola globale $F'$ tale che $F'=F^2$ con $f'$ definita in questo modo.
    $$f':A^{4r+1}\rightarrow A$$
    Dove $\forall u=u_1\cdots u_{4r+1}\in A^{4r+1}$
    $$f'(u) = f'(f(u_1\cdots u_{2r+1})f(u_{2}\cdots u_{2r+2})\cdots f(u_{2r+1}\cdots u_{4r+1}))$$
    Per dimostrare che $F^2$ sia una regola globale possiamo utilizzare il teorema di 
    hedlund:
    \begin{itemize}
        \item \textbf{continuità}: si perché $F$ è continua
        \item \textbf{commutatività}: $F^2\circ \sigma = F\circ (F\circ \sigma ) = F\circ (\sigma\circ F  ) = \sigma\circ (F\circ F  )  = \sigma \circ F^2$
    \end{itemize}
    Quindi è una regola globale.
    \item Siano $F$ e $G$ due regole globali di due $CA$ differenti sullo stesso 
    alfabeto $A$. $F\circ G$ è una regola globale di un CA? Possiamo dimostrarlo con
    il teorema:
    \begin{itemize}
        \item \textbf{continuità}: si perché è una composizione di funzioni continue
        \item \textbf{commutatività}: $F\circ (G\circ \sigma) = F\circ (\sigma\circ G) = (F\circ \sigma)\circ G =(\sigma\circ F)\circ G = \sigma\circ (F\circ G)$
    \end{itemize}
    \item Sia $F$ la regola globale per CA, supponiamo che $F$ sia invertibile, 
    allora $F^{-1}$ è una regola globale per CA? Dimostriamolo:
    \begin{itemize}
        \item \textbf{continuità}: si perché $A^\mathbb{Z}$ è compatto
        \item \textbf{commutatività}: $$F\circ \sigma = \sigma \circ F $$
        $$ F^{-1} \circ F\circ \sigma \circ F^{-1}= F^{-1}\circ \sigma \circ F \circ F^{-1}$$ 
        $$ (F^{-1} \circ F)\circ \sigma \circ F^{-1}= F^{-1}\circ \sigma \circ (F \circ F^{-1})$$
        $$ \sigma \circ F^{-1}= F^{-1}\circ \sigma$$
    \end{itemize}
\end{itemize}

\subsection{Proprietà di suriettività}

\begin{definizione} [\textbf{Estensione della regola locale}]
    L'\textbf{estensione della regola locale} per parole di lunghezza maggiore di 
    $2r+1$. coincide con la regola globale ma 
    applicata ad una stringa finita. Formalmente, $f^\ast: A^\ast\rightarrow A^\ast$
    tale che $\forall u\in A^\ast$:
    \begin{equation*}
        f^\ast(u) = \begin{cases}
            \epsilon & |u| < 2r+1\\
            f(u_1\dots u_{2r+1})f(u_2\dots u_{2r+2})\dots f(u_{n-2r}\dots u_{n})& altrimenti
        \end{cases}
    \end{equation*}  
\end{definizione}

\begin{nota}
    Se $A^\ast\equiv A^\mathbb{Z}\implies F \equiv f^\ast$. 
\end{nota}

\begin{nota}
    Per la suriettività di $F$ sarà importante conoscere se $f$ è bilanciata ovvero
    se $|f^{-1}(0)| =|f^{-1}(1)|$, dove $  f^{-1}(y)$ coincide con l'insieme delle 
    preimmagini di $y$.
\end{nota}

\begin{nota}
    Sia un CA $\langle A,r,f\rangle$ e $G$ il suo grafo di De Bruijn associato.
    I cammini bi-infiniti sui vertici di $G$ sono in corrispondenza biunivoca 
    con gli elementi di $A^\mathbb{Z}$. Inoltre, $\forall x\in A^\mathbb{Z}$ ($ \forall u\in A^\ast$)
    $F(x)$ ($f^\ast(u)$) è dato dalle etichette degli archi del cammino sui vertici
    corrispondente a $x$ ($u$).
    \begin{proof}
        Deriva dalla definizione del grafo di De Bruijn associato ad un automa cellulare.
    \end{proof}
\end{nota}

Questo è vero perché due nodi consecutivi in un cammino corrispondono ad un elemento
del dominio di $f$, mentre l'etichetta dell'arco corrisponde all'immagine dell'elemento 
stesso.

\begin{nota}
    Il grafo di De Bruijn $G$ di un CA $\langle A,r,f\rangle$ può essere considerato
    come un \textbf{automa a stati finiti non deterministico} in cui tutti i nodi sono sia stati iniziali, sia 
    stati finali. Questi particolari automi sono chiamati \textbf{automi di Fisher}
\end{nota}

Questi automi sono importanti perché permettono di accettare il linguaggio composto
da parole bi-infinite che rappresentano tutte le possibili configurazioni dell'automa
cellulare, sia bi-infinite, sia parziali.

Da notare che dal grafo di De Bruijn otteniamo degli automi non deterministici, 
per la teoria degli automi possiamo facilmente ricavare l'automa deterministico
calcolando l'insieme potenza degli stati.

\begin{definizione} [\textbf{Suriettività}]
    La definizione di \textbf{suriettività} di un automa cellulare è quando la sua 
    \textbf{regola globale} è \textbf{suriettiva} ovvero:
    \begin{itemize}
        \item $\forall y\in A^\mathbb{Z},\exists x\in A^\mathbb{Z}: F(x) = y$
        \item $F(A^\mathbb{Z}) = A^\mathbb{Z}$
        \item $\forall y \in A^\mathbb{Z}, F^{-1}(y) \ne \emptyset $
    \end{itemize}
    Dove $F^{-1}(y)$ è l'insieme delle preimmagini di $y$.
\end{definizione}

La suriettività è una proprietà di \textbf{raggiungibilità debole}

\begin{nota}
    Il significato di $f^{\ast}$ bilanciato significa che $\forall n\in \mathbb{N}, n>0$
    ogni parola di $u\in A^n$ ha lo stesso numero di preimmagini. Questo significa
    chiedere che ogni parola del dominio di $f^{\ast}$ ($|A|^{n+2r}$) abbia lo stesso 
    numero di immagini ($|A|^n$). Questo significa che il numero di preimmagini
    sarà
    \begin{equation*}
        |A|^{2r} = \frac{|A|^{n+2r}}{|A|^n} = \frac{|Dom|}{|CoDom|}
    \end{equation*}
\end{nota}

\begin{teorema}
    Sia $\langle A,r,f\rangle$ un CA e sia $F: A^\mathbb{Z}\rightarrow A^\mathbb{Z}$
    la regola globale. Sia $G$ il grafo di De Bruijn associato all'automa cellulare.
    Sia $\mathcal{L}$ il linguaggio associato all'automa ottenuto dal grafo di De Bruijn.
    Allora le seguenti affermazioni sono equivalentementi:
    \begin{enumerate}
        \item \label{cond:1}$F$ è \textbf{suriettiva}
        \item \label{cond:2}$f^\ast$ è \textbf{bilanciata} ossia $\forall u \in A^+ (u\ne \epsilon), |f^{\ast^{-1}}(u)|= |A^{2r}|$
        \item \label{cond:3}$\mathcal{L} = A^\ast$
    \end{enumerate}
    \begin{proof}
        Dimostriamo le singole implicazioni:
        \begin{itemize}
            \item $\ref{cond:1}\implies \ref{cond:2}$: la dimostrazione si basa 
            sul dimostrare $\lnot \ref{cond:2}\implies \lnot \ref{cond:1}$.

            Ipotizziamo per assurdo che $\exists u\in A^n$ t.c. $|f^{\ast^{-1}}|\ne |A|^{2r}$.
            Senza perdere di generalità assumiamo che $|f^{\ast^{-1}}(u)|< |A|^{2r}$,
            non perdiamo generalità perché se prendessimo $u$ tale che $|f^{\ast^{-1}}(u)|> |A|^{2r}$
            allora esisterà un'altra stringa $u'$ tale che $|f^{\ast^{-1}}(u')|< |A|^{2r}$,
            in caso avremmo preso quella.

            Sia $s = |f^{\ast^{-1}}(u)|$, sia $k>1$ un numero arbitrario allora costruiamo
            i seguenti insiemi di parole:
            \begin{itemize}
                \item $V$: sarà la giustapposizione di $k$ preimmagini qualunque 
                di $u$. Quindi $|V| = s^k$
                \item $W$: contenente parole del tipo $w=(uA^{2r})^{k-1}u$. Quindi $|W| = |A|^{(k-1)2r}$
            \end{itemize}

            Per costruzione sappiamo che $\forall v\in V, f^{\ast}(v)\in W$, ma $\exists
            w'\in W\text{ t.c. } f^{\ast^{-1}} (w')= \emptyset$? (si sta controllando se non è suriettiva)
            
            Sicuramente $w'$ esiste se $|V|<|W|$, ovvero  $s^k<(|A|^{2r})^{(k-1)}$.
            Sapendo che $s^k < |A|^{2r}$ allora $\exists k$ tale che  $|V|<|W|$.
            Quindi $\exists w \in W: f^{\ast^{-1}} (w) =\emptyset$, perciò 
            $\exists y \in A^{\mathbb{Z}}: F^{-1}(y) = \emptyset$, basta scegliere 
            $y$ che contiene $w$. Concludiamo che $F$ non è surettiva.
            
            \item $\ref{cond:2}\implies \ref{cond:3}$: occorre mostrare che $\mathcal{L} \subseteq A^\ast$
            e allo stesso tempo $ A^\ast \subseteq\mathcal{L}$. La prima deriva 
            dalla definizione di $\mathcal{L}$, quindi ci basta dimostrare 
            $A^\ast \subseteq\mathcal{L}$. Scegliamo arbitrariamente $v\in  A^\ast$,
            mostriamo che $v\in \mathcal{L}$, ossia che $v$ si ottiene dalla concatenazione
            delle etichette degli archi di un cammino sui vertici di $v$.
            
            Siccome $|f^{\ast^{-1}}(v)| = |A|^{2r}$, in particolare $\exists u \in A^\ast$
            t.c $f^\ast(u) = v$. Per definizione di $G$, siccome $u$ corrisponde 
            ad un cammino finito sui vertici di $G$, $f(u)=v$ si ottiene dalle etichette
            degli archi di quel cammino.
            \item $\ref{cond:3}\implies \ref{cond:1}$: scelto arbitrariamente $y\in A^\mathbb{Z}$,
            occorre mostrare che $\exists x\in A^\mathbb{Z} \text{ t.c. } F(x) =y$.

            Sia $y\in A^\mathbb{Z}$ scelto arbitrariamente. $\forall n\in \mathbb{N}$ 
            sia $v^{(n)} = y_{[-n,n]}\in A^{2n+1}$.

            Per ipotesi sappiamo $\exists u^{(n)}\in f^{\ast^{-1}}(v^{(n)})\subseteq A^{2n+2r+1}$.
            Quindi $f^{\ast^{-1}}(u^{(n)}) = v^{(n)}$.

            Sia $x^{(n)}\in A^\mathbb{Z}$ una qualunque configurazione tale che 
            $$x^{(n)}_{[-n-r,n+r]} = u^{(n)}$$

            Non è detto che $F(x^{(n)})=y$, vale sicuramente che $\forall n \in \mathbb{N},F(x^{(n)})_{[-n,n]}=y_{[-n,n]}$,
            ossia che $d(F(x^{(n)})_{[-n,n]},y_[-n,n])< \frac{1}{2^n}$.
            
            Inoltre, $\{x^{(n)}\}_{n\in \mathbb{N}}$ ammette una sottosuccessione
            $\{x^{(n_h)}\}_{h\in \mathbb{N}}$ convergente (per la compattezza di $A^{\mathbb{N}}$).
            Ovvero sia $x= \lim\limits_{h\rightarrow +\infty}x ^{(n_h)}$ otteniamo  che $$ F(x)= F(\lim\limits_{h\rightarrow +\infty}x ^{(n_h)}) = \lim\limits_{h\rightarrow +\infty} F(x^{(n_h)})=y$$,
            quindi $\exists x \in A^\mathbb{Z}$ t.c. $y = F(x)$
        \end{itemize}
    \end{proof}
\end{teorema}
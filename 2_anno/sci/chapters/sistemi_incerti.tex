\chapter{Sistemi incerti}
Si studieranno sistemi incerti che possono essere:
\begin{itemize}
    \item interpretazione dei testi
    \item incertezza dei mercati
    \item valutare il valore degli oggetti
    \item valutare un rischio
    \item prendere decisioni
    \item etc$\dots$
\end{itemize}

Lo studio dell'incertezza è fondamentale in machine learning, dal momento che 
gli algoritmi hanno bisogno di dati e questi possono avere un certo livello di 
incertezza. Fondamentale sarà capire:
\begin{itemize}
    \item l'\textbf{oggetto} dell'incertezza e la sua fonte
    \item il \textbf{livello} e la \textbf{magnitudine} dell'incertezza
    \item \textbf{come} comunicare l'incertezza, ovvero in che forma ed espressione
    \item \textbf{preché} comunicare l'incertezza, ovvero lo scopo di questa comunicazione
\end{itemize}

\begin{definizione}
    L'\textbf{incertezza}  è l'imperfezione dell'informazione dovuta a diverse cause
\end{definizione}
\begin{definizione}
    \textbf{Agenti} sono i soggetti che esprimono l'incertezza
\end{definizione}
\begin{definizione}
    \textbf{Oggetti} sono gli oggetti su cui un agente esprime incertezza
\end{definizione}
\begin{definizione}
    \textbf{Proprietà} sono le proprietà dell'oggetto su cui un agente ha incertezza
\end{definizione}
Ogni agente ha un'opinione su una proprietà degli oggetti.

L'incertezza può essere dovuta principalemnte da:
\begin{itemize}
    \item \textbf{informazioni incomplete}:
    \begin{itemize}
        \item \textbf{valori mancanti}: ex: voglio stabilire se l'auto è gialla ma
        non conosco il colore dell'auto
        \item \textbf{valori imprecisi}: ex: l'auto ha tra i 10 e 15 anni
        \item \textbf{proprietà mancanti}: ex: nel caso prendessimo come oggetti i
        pazienti e come proprietà i sintomi, non possiamo sapere se un paziente 
        ha una particolare malattia se ha dei particolari sintomi.
    \end{itemize}
    \item \textbf{randomicità}: non possiamo stabilire la 
    veridicità di un'informazione, ex: domani piove (non è detto)
    \item \textbf{gradualità}: ex: l'auto è veloce, cosa significa veloce?
    \item \textbf{informazione inaffidabile}: si possono avere informazioni
    contrastanti dovuti da agenti che hanno opinioni diverse, elementi a favore
    o contro ed esempi e controesempi. In aggiunta, si possono avere informazioni
    inaffidabili perché possiamo non avere la conpleta fiducia nei dati a disposizione.
\end{itemize}

Esistono diverse forme di incertezza e generalmente si classificano all'interno
di una tassonomia. Le classificaizoni si dividino in:
\begin{itemize}
    \item \textbf{scope}: il campo che copre quella classificazione
    \item \textbf{criteria}: sorgente dell'incertezza
\end{itemize}

Una tassonomia ha lo scopo di essere universale per raccogliere tutti i tipi di
incertezze indipendentemente dal particolare campo. Le tassonomie sono composte
da alberi i quali hanno sempre come radice \textbf{ignorance}, nessuna classificazione
appare in tutte le tipologie di tassonomie, ma al massimo in alcune, le tassonomie
possono essere più specifiche e possono avere delle intersezioni.

Esistono anche tassonomie che sono legate ad un campo, ex: ecologia o ML. Per
queste tassonomie più piccolo è il dominio e più specifica è la terminologia della
classificazione. L'ontologia del dominio (semantica) può essere utile per capire dove l'incertezza
è e come azionarsi per gestirla. Quando costruiamo una classificazione di dominio
deve essere chiara tutte le parti di incertezza.

I criteri per la classificazione si basano su:
\begin{itemize}
    \item \textbf{source}: il criterio principale che dipende dal dominio (incertezza 
    sui dati e sugli utenti, temporale, spaziale e linguistica$\dots$)
    \item \textbf{mmanifestazione concrete}: dove puoi vewdere l'effetto dell'incertezza
    \item \textbf{personalità in cui risiede l'incertezza}: l'agente
\end{itemize}

Per esprimere l'incertezza, oltre alla probabilità abbiamo:
\begin{itemize}
    \item estensione della probabilità:
     \begin{itemize}
        \item Belief functions
        \item Imprecise probabilities
    \end{itemize}
    \item alternative alla probabilità:
    \begin{itemize}
        \item Possibility Theory/Logic
        \item Fuzzy Sets/Logic
        \item Modal Logic
        \item Interval Sets
        \item Rough Sets
    \end{itemize}
\end{itemize}
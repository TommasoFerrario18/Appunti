\chapter{Sistemi incerti}
Si studieranno sistemi incerti che possono essere:
\begin{itemize}
    \item interpretazione dei testi
    \item incertezza dei mercati
    \item valutare il valore degli oggetti
    \item valutare un rischio
    \item prendere decisioni, l'incertezza può avere un'accezione positiva
    \item etc$\dots$
\end{itemize}

Lo studio dell'incertezza è fondamentale in machine learning, dal momento che 
gli algoritmi hanno bisogno di dati e questi possono avere un certo livello di 
incertezza. Fondamentale sarà capire:
\begin{itemize}
    \item l'\textbf{oggetto} dell'incertezza e la sua fonte
    \item il \textbf{livello} e la \textbf{magnitudine} dell'incertezza
    \item \textbf{come} comunicare l'incertezza, ovvero in che forma ed espressione
    \item \textbf{preché} comunicare l'incertezza, ovvero lo scopo di questa comunicazione
\end{itemize}

\begin{definizione}
    L'\textbf{incertezza}  è l'imperfezione dell'informazione dovuta a diverse cause
\end{definizione}
\begin{definizione}
    \textbf{Agenti} sono i soggetti che esprimono l'incertezza
\end{definizione}
\begin{definizione}
    \textbf{Oggetti} sono gli oggetti su cui un agente esprime incertezza
\end{definizione}
\begin{definizione}
    \textbf{Proprietà} sono le proprietà dell'oggetto su cui un agente ha incertezza
\end{definizione}
Ogni agente ha un'opinione su una proprietà degli oggetti.

L'incertezza può essere dovuta principalemnte da:
\begin{itemize}
    \item \textbf{informazioni incomplete} dovuto a diverse cause:
    \begin{itemize}
        \item \textbf{valori mancanti}: ex: voglio stabilire se l'auto è gialla ma
        non conosco il colore dell'auto
        \item \textbf{valori imprecisi}: ex: l'auto ha tra i 10 e 15 anni ma non 
        conosco l'anno preciso
        \item \textbf{proprietà mancanti}: conosciamo i sintomi del paziente ma non 
        ho abbastanza sintomi per determinare l'incertezza. ex: nel caso prendessimo come oggetti i
        pazienti e come proprietà i sintomi, non possiamo sapere se un paziente 
        ha una particolare malattia se ha dei particolari sintomi.
    \end{itemize}
    \item \textbf{randomicità}: non possiamo stabilire la 
    veridicità di un'informazione, ex: domani piove (non è detto)
    \item  \textbf{vaghezza}/\textbf{gradualità}: ex: l'auto è veloce, cosa significa veloce?
    (dipende dal contesto)
    \item \textbf{informazione inaffidabile}: si possono avere informazioni
    contrastanti dovuti da agenti che hanno opinioni diverse, elementi a favore
    o contro ed esempi e controesempi. In aggiunta, si possono avere informazioni
    inaffidabili perché possiamo non avere la conpleta fiducia nei dati a disposizione.
    \item \textbf{non affidabilità}: non abbiamo completa fiducia nei dati a disposizione.
    Ex: L'auto di Giulia è rossa o nera, sicuramente non è blu.
\end{itemize}

Esistono diverse forme di incertezza e generalmente si classificano all'interno
di una tassonomia. Le classificaizoni si dividino in:
\begin{itemize}
    \item \textbf{scope}: il campo del sapere che copre quella classificazione
    \item \textbf{criteria}: sorgente dell'incertezza
\end{itemize}

Una tassonomia ha lo scopo di essere universale per raccogliere tutti i tipi di
incertezze indipendentemente dal particolare campo. Le tassonomie sono composte
da alberi i quali hanno sempre come radice \textbf{ignorance} (simula la non conoscienza),
nessuna classificazione appare in tutte le tipologie di tassonomie, ma al massimo in alcune e non 
nella stessa posizione, le tassonomie possono essere più specifiche e possono avere delle intersezioni.

Non esiste e non esisterà un unico modo per classificare l'incertezza e l'ignoranza, 
però sono utilili per essere studiati per analizzare le caratteristiche comuni.

Esistono anche tassonomie che sono legate ad un campo, ex: ecologia o ML. Per
queste tassonomie più piccolo è il dominio e più specifica è la terminologia della
classificazione. L'ontologia del dominio (semantica) può essere utile per capire dove l'incertezza
è e come azionarsi per gestirla. Quando costruiamo una classificazione di dominio
deve essere chiara tutte le parti di incertezza.

Le classificazioni sono utili per identificare l'incertezza per poi capire come 
gestirla. Le classificazioni generali tornano utili per specificare una prima classificazione 
specifica dell'incertezza nel mio dominio.

I criteri per definire una classificazione sono i seguenti:
\begin{itemize}
    \item \textbf{source}:  classificano in base alla sorgente (incertezza 
    sui dati e sugli utenti, temporale, spaziale e linguistica$\dots$)
    \item \textbf{manifestazione concrete}: classificano in base a cosa implica l'incertezza
    \item \textbf{personalità in cui risiede l'incertezza}: classifichiamo in base
    al punto di vista che vede l'incertezza
\end{itemize}

Possiamo avere classificazioni a più dimensioni, come matrici, e quindi sono più
complete e complesse. 

Essendoci diversi tipi di incertezza, si hanno diversi tipi di strumenti per 
modellare l'incertezza:
\begin{itemize}
    \item estensione della probabilità:
     \begin{itemize}
        \item Belief functions
        \item Imprecise probabilities
    \end{itemize}
    \item alternative alla probabilità:
    \begin{itemize}
        \item Possibility Theory/Logic
        \item Fuzzy Sets/Logic
        \item Modal Logic
        \item Interval Sets
        \item Rough Sets
    \end{itemize}
\end{itemize}

\section{Logica a tre valori di verità}
Spono logiche che introducono un terzo valore di verità che può assumere diversi 
significati in base a due caratteristiche:
\begin{itemize}
    \item ontologico (riguardano un qualcosa di vero): il valore può essere usato per assegnare sfumature di verità (mezzo vero)
    oppure si può assegnare un significato indefinito oppure per specificare il 
    concetto di irrilevante.
    \item epistemico (riguarda un problema di conoscienza): il valore può essere 
    usato per assegnare il valore unknown oppure può essere usato per specificare 
    inconsistenze  oppure per assegnare ad affermazioni che possono essere possibilmente 
    vere ma si scoprirà sono in futuro.
\end{itemize}


\section{Algebra booleana}
\begin{definizione}
    Una relazione d'ordine parziale $R$ è una relazione binaria:
    \begin{itemize}
        \item riflessiva
        \item antisimmetrica
        \item transitiva
    \end{itemize}
\end{definizione}

\begin{esempio}
    Possiamo definire delle relazioni d'ordine parziale per le funzioni caratteristiche 
    e una funzione è minore dell'altra se la prima mappa un sottoinsieme della seconda 
    funzione caratteristica.
\end{esempio}

Possiamo utilizzare il diagramma di Hasse è un diagramma che permette di rappresentare 
un insieme parzialmente ordinato senza esprimere la riflessività e transitività.

\begin{definizione}
    elementi comparabili e non comparabili
\end{definizione}
\begin{definizione}
    elemento minimo e massimo di un insieme parzialemente ordinato POSET non sempre 
    esistono
\end{definizione}
\begin{definizione}
    LUB, dato $S\subseteq P$ sarà l'elemto più piccolo di tutti quelli più grandi di $S$
    GLB, dato $S\subseteq P$ sarà l'elemto più grande di tutti quelli più piccoli di $S$

    Gli elementi o esistono e sono unici o non esistono. 
\end{definizione}
con due elementi il sup lo indichiamo con $\lor$ mentre inf con $\land$.

\begin{definizione}
    reticolo
\end{definizione}
\begin{definizione}
    Operazioni su reticoli di inf e sup:
    \begin{itemize}
        \item idempotente
        \item commutativa
        \item associativa
        \item assorbimento
    \end{itemize} 
\end{definizione}

Possiamo partire da un insieme generico definire ue operazioni che sono commutative associative 
e assorbimento, con questo operazioni possiamo definire una relazione d'ordine parziale.

esistono reticoli che non sono distributivi e reticoli che sono distributivi.

Reticolo booleano

\begin{esempio}
    Un reticolo booleano è la famosa tabella di verità dell'and e or, il complemento 
    è rappresentat dall'operazione complemneto.
\end{esempio}

Siamo partiti dai POSET, siamo passati ai lattice e poi siamo passati al boolean lattice 
e l'algebra booleana permette di definire il reticolo booleano.

poi abbiamo un isomorfismo da algebre booleane.
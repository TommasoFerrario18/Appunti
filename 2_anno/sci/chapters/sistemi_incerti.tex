\chapter{Sistemi incerti}
Si studieranno sistemi incerti che possono essere:
\begin{itemize}
    \item interpretazione dei testi
    \item incertezza dei mercati
    \item valutare il valore degli oggetti
    \item valutare un rischio
    \item prendere decisioni, l'incertezza può avere un'accezione positiva
    \item etc$\dots$
\end{itemize}

Lo studio dell'incertezza è fondamentale in machine learning, dal momento che 
gli algoritmi hanno bisogno di dati e questi possono avere un certo livello di 
incertezza. Fondamentale sarà capire:
\begin{itemize}
    \item l'\textbf{oggetto} dell'incertezza e la sua fonte
    \item il \textbf{livello} e la \textbf{magnitudine} dell'incertezza
    \item \textbf{come} comunicare l'incertezza, ovvero in che forma ed espressione
    \item \textbf{preché} comunicare l'incertezza, ovvero lo scopo di questa comunicazione
\end{itemize}

\begin{definizione}
    L'\textbf{incertezza}  è l'imperfezione dell'informazione dovuta a diverse cause
\end{definizione}
\begin{definizione}
    \textbf{Agenti} sono i soggetti che esprimono l'incertezza
\end{definizione}
\begin{definizione}
    \textbf{Oggetti} sono gli oggetti su cui un agente esprime incertezza
\end{definizione}
\begin{definizione}
    \textbf{Proprietà} sono le proprietà dell'oggetto su cui un agente ha incertezza
\end{definizione}
Ogni agente ha un'opinione su una proprietà degli oggetti.

L'incertezza può essere dovuta principalemnte da:
\begin{itemize}
    \item \textbf{informazioni incomplete} dovuto a diverse cause:
    \begin{itemize}
        \item \textbf{valori mancanti}: ex: voglio stabilire se l'auto è gialla ma
        non conosco il colore dell'auto
        \item \textbf{valori imprecisi}: ex: l'auto ha tra i 10 e 15 anni ma non 
        conosco l'anno preciso
        \item \textbf{proprietà mancanti}: conosciamo i sintomi del paziente ma non 
        ho abbastanza sintomi per determinare l'incertezza. ex: nel caso prendessimo come oggetti i
        pazienti e come proprietà i sintomi, non possiamo sapere se un paziente 
        ha una particolare malattia se ha dei particolari sintomi.
    \end{itemize}
    \item \textbf{randomicità}: non possiamo stabilire la 
    veridicità di un'informazione, ex: domani piove (non è detto)
    \item  \textbf{vaghezza}/\textbf{gradualità}: ex: l'auto è veloce, cosa significa veloce?
    (dipende dal contesto)
    \item \textbf{informazione inaffidabile}: si possono avere informazioni
    contrastanti dovuti da agenti che hanno opinioni diverse, elementi a favore
    o contro ed esempi e controesempi. In aggiunta, si possono avere informazioni
    inaffidabili perché possiamo non avere la conpleta fiducia nei dati a disposizione.
    Ex: L'auto di Giulia è rossa o nera, sicuramente non è blu.
\end{itemize}

Esistono diverse forme di incertezza e generalmente si classificano all'interno
di una tassonomia. Le classificaizoni si dividino in:
\begin{itemize}
    \item \textbf{scope}: il campo del sapere che copre quella classificazione
    \item \textbf{criteria}: sorgente dell'incertezza
\end{itemize}

Una tassonomia ha lo scopo di essere universale per raccogliere tutti i tipi di
incertezze indipendentemente dal particolare campo. Le tassonomie sono composte
da alberi i quali hanno sempre come radice \textbf{ignorance} (simula la non conoscienza),
nessuna classificazione appare in tutte le tipologie di tassonomie, ma al massimo in alcune e non 
nella stessa posizione, le tassonomie possono essere più specifiche e possono avere delle intersezioni.

Non esiste e non esisterà un unico modo per classificare l'incertezza e l'ignoranza, 
però sono utilili per essere studiati per analizzare le caratteristiche comuni.

Esistono anche tassonomie che sono legate ad un campo, ex: ecologia o ML. Per
queste tassonomie più piccolo è il dominio e più specifica è la terminologia della
classificazione. L'ontologia del dominio (semantica) può essere utile per capire dove l'incertezza
è e come azionarsi per gestirla. Quando costruiamo una classificazione di dominio
deve essere chiara tutte le parti di incertezza.

Le classificazioni sono utili per identificare l'incertezza per poi capire come 
gestirla. Le classificazioni generali tornano utili per specificare una prima classificazione 
specifica dell'incertezza nel mio dominio.

I criteri per definire una classificazione sono i seguenti:
\begin{itemize}
    \item \textbf{source}:  classificano in base alla sorgente (incertezza 
    sui dati e sugli utenti, temporale, spaziale e linguistica$\dots$)
    \item \textbf{manifestazione concrete}: classificano in base a cosa implica l'incertezza
    \item \textbf{personalità in cui risiede l'incertezza}: classifichiamo in base
    al punto di vista che vede l'incertezza
\end{itemize}

Possiamo avere classificazioni a più dimensioni, come matrici, e quindi sono più
complete e complesse. 

Essendoci diversi tipi di incertezza, si hanno diversi tipi di strumenti per 
modellare l'incertezza:
\begin{itemize}
    \item estensione della probabilità:
     \begin{itemize}
        \item Belief functions
        \item Imprecise probabilities
    \end{itemize}
    \item alternative alla probabilità:
    \begin{itemize}
        \item Possibility Theory/Logic
        \item Fuzzy Sets/Logic
        \item Modal Logic
        \item Interval Sets
        \item Rough Sets
    \end{itemize}
\end{itemize}

\section{Logica a tre valori di verità}
Spono logiche che introducono un terzo valore di verità che può assumere diversi 
significati in base a due caratteristiche:
\begin{itemize}
    \item ontologico (riguardano un qualcosa di vero): il valore può essere usato per assegnare sfumature di verità (mezzo vero)
    oppure si può assegnare un significato indefinito oppure per specificare il 
    concetto di irrilevante.
    \item epistemico (riguarda un problema di conoscienza): il valore può essere 
    usato per assegnare il valore unknown oppure può essere usato per specificare 
    inconsistenze  oppure per assegnare ad affermazioni che possono essere possibilmente 
    vere ma si scoprirà sono in futuro.
\end{itemize}


\section{Algebra booleana}
\begin{definizione}
    Una relazione d'ordine parziale $R$ è una relazione binaria:
    \begin{itemize}
        \item riflessiva
        \item antisimmetrica
        \item transitiva
    \end{itemize}
\end{definizione}

\begin{esempio}
    Possiamo definire delle relazioni d'ordine parziale per le funzioni caratteristiche 
    e una funzione è minore dell'altra se la prima mappa un sottoinsieme della seconda 
    funzione caratteristica.
\end{esempio}

Possiamo utilizzare il diagramma di Hasse è un diagramma che permette di rappresentare 
un insieme parzialmente ordinato senza esprimere la riflessività e transitività.

\begin{definizione}
    Dato un poset $ \langle\mathcal{P}, \le \rangle$, due elementi 
    $a,b\in \mathcal{P}$ si dicono \textbf{comparabili} sse $a\le b$ o $b\le a$. Altrimenti 
    si diranno \textbf{incomparabili}.
\end{definizione}

\begin{definizione}
    Data una relazione di ordine parziale $\le$ su un poset $\mathcal{P}$, se $\forall b,a\in \mathcal{P}$
    $a\le b$ o $b\le a$ e $b\ne a$ allora $\le$ è una \textbf{relazione di ordine parziale ristretta} e 
    si segnerà con $<$.
\end{definizione}

\begin{definizione}
    Data una relazione di ordine parziale $\le$ su un poset $\mathcal{P}$, se $\forall b,a\in \mathcal{P}$
    $a\le b$ o $b\le a$ allora $\le$ è una \textbf{relazione di ordine totale} (ovvero quando 
    tutti gli elementi sono comparabili)
\end{definizione}

Per rappresentare un poset possiamo utilizzare dei diagrammi di Hasse, ovvero diagrammi 
che mostrano le relazioni di antisimmetria, riflessività e transitività attraverso 
un ordine topologico e collegamenti non orientati.

\begin{definizione}
    Dato un Poset $\mathcal{P}$, definiremo $0\in \mathcal{P}$ il \textbf{minimo}
    se $\forall p \in \mathcal{P}\implies 0 \le p$ 
\end{definizione}
\begin{definizione}
    Dato un Poset $\mathcal{P}$, definiremo $1\in \mathcal{P}$ il \textbf{massimo}
    se $\forall p \in \mathcal{P}\implies p\le 1$ 
\end{definizione}

\begin{definizione}
    Un poset con un minimo e un massimo è un \textbf{poset limitato} e si scrive 
    $\langle \mathcal{P},\le , 0, 1\rangle$
\end{definizione}

\begin{nota}
    elemento minimo e massimo di un insieme parzialemente ordinato POSET non sempre 
    esistono
\end{nota}

\begin{definizione}
    Sia $S\ne \emptyset$ tale che $S\subseteq \mathcal{P}$, definiremo:
    \begin{itemize}
        \item $x\in \mathcal{P}$ \textbf{least upper bound} ($x=\sup(S)$, $x=\lor S$) sse:
        \begin{itemize}
            \item $\forall s\in S, s\le x$
            \item se $c\in \mathcal{P}$ soddisfa $s\le c,\forall s\in S$ allora $x\le c$
        \end{itemize}
        Se vale solo la prima condizione allora si dirà \textbf{upper bound} di $S$
        \item $y\in \mathcal{P}$ \textbf{greatest lower bound} ($y=\inf(S)$, $y=\land S$) sse:
        \begin{itemize}
            \item $\forall s\in S, y\le s$
            \item se $d\in \mathcal{P}$ soddisfa $d\le s,\forall s\in S$ allora $d\le y$
        \end{itemize}
        Se vale solo la prima condizione allora si dirà \textbf{lower bound} di $S$
    \end{itemize}
\end{definizione}

\begin{osservazione}
    Dato il poset $\langle \mathcal{P},\le\rangle$ allora:
    \begin{itemize}
        \item sia $Y\subseteq\mathcal{P}$ allora $\sup (Y)$ e $\inf (Y)$ se esistono
        sono unici
        \item sia $Y_1,Y_2\subseteq\mathcal{P}$ tale che $Y_1\le Y_2$ (ovvero tutti 
        gli elementi di $Y_1$ sono minori di tutti gli elementi di $Y_2$) allora entrambi
        $\inf(Y_1)$ e $\sup (Y_2)$ esistono, sono unici e $\inf(Y_1)\le \sup(Y_2)$
        \item sia $x\le y\iff x\land y = x \land x\lor y=y$ 
    \end{itemize}
\end{osservazione}

\begin{nota}
    Dato il poset $\langle \mathcal{P},\le\rangle$ che ha un $\lor \mathcal{P}$ e 
    un $\land\mathcal{P}$ e quindi è limitato, allora 
    $$\land\mathcal{P}\le x\le \lor \mathcal{P}, \forall x\in \mathcal{P}$$ 
\end{nota}


\begin{definizione}
    Un \textbf{reticolo} è un poset $\langle \mathcal{L}, \le \rangle$ tale che 
    $\forall x,y\in\mathcal{L}$ esiste il $\sup(\{x,y\})$ e $\inf(\{x,y\})$
\end{definizione}
nei reticoli il $\inf \equiv \text{meet}$ ($x\land y$) e  $\sup \equiv \text{join}$ ($x\lor y$)

Più precisamente si avrà:
\begin{itemize}
    \item $a\land b \le \{a,b\}$ e se $x\le  \{a,b\}\implies x\le a\land b$
    \item $\{a,b\} \le a\lor b$ e se $\{a,b\}\le x \implies a\lor b \le x$
\end{itemize}
\begin{definizione}
    Un \textbf{reticolo} è \textbf{completo} sse $\inf(S)$ e $\sup(S)$ esistono 
    sempre $\forall S \ne \emptyset \subseteq\mathcal{L}$
\end{definizione}

Il $\sup$ e l'$\inf$ nei reticoli possono essere visti come delle operazioni binarie
dal momento che esistono sempre generando la seguente struttura algebrica $\langle \mathcal{L}, \land,\lor \rangle$:
$$\land : \mathcal{L} \times \mathcal{L} \to \mathcal{L}, (x,y) \to x\land y :=\inf(\{x,y\})$$
$$\lor : \mathcal{L} \times \mathcal{L} \to \mathcal{L}, (x,y) \to x\lor y :=\sup(\{x,y\})$$


\begin{teorema}
    Dato il seguente reticolo $\langle \mathcal{L}, \le \rangle$ allora è associata 
    la struttura algebrica $\langle \mathcal{L}, \lor, \land\rangle$ che contiene le 
    operazioni di meet e join sul reticolo, allora le operazioni rispettano 
    le seguenti proprietà $\forall x,y,z\in \mathcal{L}$:
    \begin{itemize}
        \item \textbf{idempotenza}:
        $$x\land x = x \quad x\lor x = x$$ 
        \item \textbf{commutativa}:
        $$x\land y = y\land x \quad x\lor y = y\lor x$$ 
        \item \textbf{associativa}:
        $$x\land (y \land z) = (x \land y)\land z \quad x\lor (y \lor z) =( x\lor y)\lor z$$ 
        \item \textbf{assorbimento}:
        $$x\land (x \lor y) =x\quad x\lor (x \land y) =x$$  
        \item inoltre:
        $$x\le y \equiv x=x\land y\quad x\le y \equiv y=x\lor y$$ 
    \end{itemize} 
\end{teorema}

Possiamo partire da un insieme generico definire due operazioni che sono commutative associative 
e assorbimento, con queste operazioni possiamo definire un reticolo.

\begin{teorema}
    Data la seguente struttura algebrica $\langle \mathcal{L}, \lor, \land\rangle$
    dove:
    \begin{itemize}
        \item $\mathcal{L}\ne \emptyset$
        \item $\land$ e $\lor$ sono due operazioni su $\mathcal{L}$ commutative, 
        associative e con l'assorbimento
    \end{itemize}
    Se definiamo una relazione binaria $\le\subseteq \mathcal{L}\times \mathcal{L}$
    tale che 
    $$x\le y \iff y=x\lor y \equiv x\le y \iff x= x\land y$$
    allora $\langle \mathcal{L},\le\rangle$ è un reticolo tale che 
    $$x\lor y = \sup \{x,y\} \quad x\land y = \inf \{x,y\}$$
\end{teorema}

Quindi a partire da un reticolo poset possiamo formare una struttura algebrica alla quale 
sarà associato un reticolo che è lo stesso di quello di partenza. 

Al contrario da una struttura algebrica di un reticolo possiamo formare un reticolo 
il quale sarà effettivamente lo stesso del reticolo poset associato.

esistono reticoli che non sono distributivi e reticoli che sono distributivi.

\begin{nota}
    Sia $\langle \mathcal{L}, \lor, \land\rangle$ una struttura algebrica che formula 
    un reticolo allora:
    \begin{itemize}
        \item Se l'elemento neutro dell'operazione $\lor$ esiste (quindi $0$), allora 
        questo elemento è unico ed è il minimo elemento rispetto alla relazione $\le$.
        \item Se l'elemento neutro dell'operazione $\land$ esiste (quindi $1$), allora 
        questo elemento è unico ed è il massimo elemento rispetto alla relazione $\le$.
    \end{itemize}
\end{nota}

\begin{teorema}
    Sia $\langle \mathcal{L}, \le, 0\rangle$ un reticolo limitato inferiormente allora 
    $\lor$ rispetta le seguenti proprietà:
    \begin{itemize}
        \item \textbf{idempotenza}:
        $$ x\lor x = x$$ 
        \item \textbf{commutativa}:
        $$x\lor y = y\lor x$$ 
        \item \textbf{associativa}:
        $$x\lor (y \lor z) =( x\lor y)\lor z$$ 
        \item \textbf{elemento neutro}:
        $$x\lor 0 =x$$  
    \end{itemize}
    Quindi $\langle \mathcal{L}, \lor, 0\rangle$ è un monoide commmutativo. Sotto queste 
    ipotesi allora possiamo dire che dal monoide commutativo possiamo ottenere un 
    reticolo dove $0$ è l'elemento minimo.
\end{teorema}

\begin{definizione}
    Dati due reticoli $\mathcal{L}_1, \mathcal{L_2}$ sono detti isomorfi se esiste 
    una mappa biettiva $\phi:\mathcal{L}_1\times \mathcal{L}_2$ tale che 
    per un arbitrario $x,y\in \mathcal{L}_1$
    $$\phi(x\land_1 y) = \phi(x)\land_2\phi(y) \quad \phi(x\lor_1 y) = \phi(x)\lor_2\phi(y)$$
    Se $\phi$ è un isomorfismo di reticoli allora $\phi^{-1}$ è anch'esso un isomorfismo
    di reticoli. L'isomorfismo tra reticoli preserva anche l'ordinamento quindi è anche 
    un isomorfismo di poset.
\end{definizione}

\begin{definizione}
    Se $\mathcal{L}$ è un reticolo distributivo se le operazioni rispettano la proprietà
    distributiva:
    \begin{itemize}
        \item $\forall x,y,z\in \mathcal{L}, x\lor (y\land z)= (x\lor y) \land (x\lor z)$
        \item $\forall x,y,z\in \mathcal{L}, x\land (y\lor z)= (x\land y) \lor (x\land z)$
    \end{itemize}
\end{definizione}

\begin{teorema}
    Sia $\langle \mathcal{L}, \land, \lor \rangle$ una struttura algebrica e si ha:
    \begin{itemize}
        \item $\mathcal{L}\ne \emptyset$
        \item $\land$ e $\lor$ sono due operazioni binarie che soddisfano le seguenti 
        proprietà:
        \begin{itemize}
            \item $x=x\land(x\lor y)$
            \item $x\land (y\lor z) = (z\land x) \lor (y\land x)$
        \end{itemize}
    \end{itemize}
    Allora $\mathcal{L}$ è un reticolo distributivo.
\end{teorema}


\begin{definizione}
    Se $\mathcal{L}$ è un reticolo limitato, allora identifichiamo $y\in\mathcal{L}$
    il complemento di $x\in \mathcal{L}$ tale che 
    $$x\lor y = 1 \quad x\land y = 0$$
\end{definizione}
se $x$ è il complemento di $y$ allora $y$ è il complemento di $x$.
Inoltre $0\lor 1= 1$ e $0\land 1 = 0$.

Nei reticoli distribuiti il complemento di un elemento, se esiste, è unico. Per 
reticoli non distributivi allora la proprietà non è unica. 

\begin{definizione}
    Un reticolo booleano è una struttura $\langle \mathcal{B}, \land,\lor, ',0,1\rangle$ 
    dove:
    \begin{itemize}
        \item $\mathcal{B}$ è un insieme contenente due elementi distinti $1,0$.
        \item $\land$ e $\lor$ sono due operazionin binarie su $\mathcal{B}$ dove 
        la struttura algebrica $\langle \mathcal{B}, \land,\lor, 0,1\rangle$  è 
        un reticolo distribuito limitato dagli elementi $0,1$, quidni $\forall x\in \mathcal{B}$
        $0\le x\le 1$.
        \item $\forall x \in \mathcal{B}, \exists x'\in \mathcal{B}$ tale che 
        $x\land x' = 0$ e $x\lor x' = 1$, quindi $x'$ è il complemento.
    \end{itemize}
\end{definizione}

Quindi un'algebra booleana è un reticolo distribuito, complementabile e limitato.
Avremo quindi l'operazione di complemento ovvero $':\mathcal{B}\to \mathcal{B}$
 tale che restituisce l'elemento complemento che è unico.

\begin{esempio}
    Un reticolo booleano è la famosa tabella di verità dell'and e or, il complemento 
    è rappresentat dall'operazione complemneto.
\end{esempio}

\begin{definizione}
    Un'algebra booleana è una struttura $\langle \mathcal{B}, \land,\lor, ',0,1\rangle$ 
    dove:
    \begin{itemize}
        \item $\mathcal{B} = \{0,1\}$
        \item $\land$ e $\lor$ sono due operazioni binarie su $\mathcal{B}$
        \item $':\mathcal{B}\to \mathcal{B}$ è l'operazione di complemento
    \end{itemize}
    tale che $\forall x,y,z\in \mathcal{B}$ valgono i seguenti assiomi:
    \begin{itemize}
        \item $x\lor y = y\lor x \quad x\land y = y\land x $
        \item $x\land (y\lor z) = (x\land y) \lor (x\land z) \quad x\lor (y\land z) = (x\lor y) \land (x\lor z)$
        \item $x\lor 0 = x\quad x\land 1 = x$ 
        \item $x\lor x' = 1\quad x\land x' = 0$ 
    \end{itemize}
\end{definizione}

\begin{definizione}
    Un'algebra booleana può essere definita da una struttura $\langle \mathcal{B}, \land,\lor, ',0,1\rangle$ 
    tale che $\forall a,b,c\in \mathcal{B}$ valgono i seguenti assiomi:
    \begin{itemize}
        \item $a\land b = (a'\lor b')'$
        \item $a \lor b = b \lor a$
        \item $a\lor (b\lor c) = (a\lor b)\lor c$ 
        \item $(a\land b)\lor (a\land b') = a$ 
    \end{itemize}
\end{definizione}

\begin{teorema}
    Ogni algebra booleanda è possibile prova le  seguenti affermazioni:
    \begin{itemize}
        \item Proprietà del reticolo:
        \begin{itemize}
            \item \textbf{idempotenza}:
            $$x\land x = x \quad x\lor x = x$$ 
            \item \textbf{commutativa}:
            $$x\land y = y\land x \quad x\lor y = y\lor x$$ 
            \item \textbf{associativa}:
            $$x\land (y \land z) = (x \land y)\land z \quad x\lor (y \lor z) =( x\lor y)\lor z$$ 
            \item \textbf{assorbimento}:
            $$x\land (x \lor y) =x\quad x\lor (x \land y) =x$$  
        \end{itemize} 
        \item Proprietà distributive:
        \begin{itemize}
            \item $\forall x,y,z\in \mathcal{L}, x\lor (y\land z)= (x\lor y) \land (x\lor z)$
            \item $\forall x,y,z\in \mathcal{L}, x\land (y\lor z)= (x\land y) \lor (x\land z)$
        \end{itemize}
        \item Elementi neutri e unità:
        \begin{itemize}
            \item $x\lor 0 = x\quad x\lor 1 = 1$
            \item $x\land 0 = 0\quad x\land 1 =x$
        \end{itemize}
        \item condizioni di ortocomplementazione:
        \begin{itemize}
            \item \textbf{involuzione}: $(x')'= x$
            \item \textbf{leggi di De Morgan}: $(x\land y)' = x'\lor y' \quad (x\lor y)' = x'\land y'$
            \item $x\lor x' = 1$
            \item $x\land x' = 0$
        \end{itemize}

    \end{itemize}
\end{teorema}

\begin{definizione}
    Sia $\langle \mathcal{B}, \land,\lor\rangle$ una struttura algebrica dove:
    \begin{itemize}
        \item $\mathcal{L}\ne \emptyset$
        \item $\land$ e $\lor$ sono due operazioni su  $\mathcal{L}$ che soddisfano:
        \begin{itemize}
            \item $x= x\land (x\lor y)$
            \item $x\land (y\lor z) = (z\land x) \lor (y\land x)$
            \item $x\land(y\lor y') = x\lor(y\land y')$
        \end{itemize}
    \end{itemize}
    allora $\mathcal{L}$ è un reticolo booleano.
\end{definizione}


\begin{definizione}
    Date due algebre $\mathcal{B}_1$ e $\mathcal{B}_2$ sono isomorfe se esiste un
    isomorfismo tra reticoli $\phi:\mathcal{B}_1\to \mathcal{B}_2$ che preserva 
    l'elemento complemento, $\forall x\in \mathcal{B}_1, \phi(x') = \phi(x)'$
\end{definizione}

\begin{esempio}
L'esempio classico di algebra booleana è $0, 1$ e gli operatori saranno $\land, \lor,\lnot$.
In questa algebra si può definire $\implies$.
\end{esempio}

\section{Logica proposizionale booleana}
Per definire le logiche bisogna definire singolarmente la \textbf{sintassi} e 
\textbf{semantica}. Successivamente su queste bisogna definire un sistema deduttivo 
per poter effettuare inferenza nella nostra logica.

In generalem per definire la sintassi sobbiamo definire un \textbf{alfabeto} $\mathcal{A}$
e un insieme di formule $\mathcal{F}$ su di esso, più precisamente l'alfabeto sarà
$\mathcal{A} =\{V, L, U\}$ dove:
\begin{itemize}
    \item $V$ è un insieme infinito di variabili proposizionali
    \item $L\ne \emptyset$ è un insieme finito di connettivi logici unari e binari
    \item $U$ è un insieme di simboli ausiliari come le parentesi e la punteggiatura
\end{itemize}

Per una generica logica booleana possiamo pensare di definire l'alfabeto in questo modo
$$\mathcal{A} = \{\{p,q,r,s,\dots\}, \{\lnot, \land, \lor, \supset \}, \{(,),\top, \perp\}\}$$

Dall'alfabeto si definisce l'insieme di formule, spesso chiamate FbF perché definite 
in modo induttivo.

Nel caso generale le formule ben formate $\mathcal{F}$ saranno:
\begin{itemize}
    \item $\forall v\in V, v\in \mathcal{F}$
    \item $p,q\in FbF: \lnot p, p\land q, p\lor q, p \supset q\in \mathcal{F}$
    \item Nient'altro è una FbF
\end{itemize}

Definiremo $L= \langle \mathcal{A},\mathcal{F}\rangle$ il linguaggio della logica classica 
proposizionale booleana.

Per definire una \textbf{semantica} dobbiamo passare prima da una specifica 
su come interpretare una particolare variabile dell'alfabeto.

Per interpretare le variabili definiremo una mappa $v:V\to \mathcal{T}$ una funzione 
che associa una variabile ad un possibile valore di verità. L'interpretazione
verrà esotesa successivamente a tutte le formule $\mathcal{F}$, definendo la mappa 
$v^\ast:\mathcal{F}\to \mathcal{T}$ nel seguente modo:
\begin{itemize}
    \item $v^\ast(p) = v(p), \forall p\in V$
    \item $v^\ast(\lnot p) = v^\ast(p)', \forall p\in \mathcal{F}$
    \item $v^\ast(p\land q) = v^\ast(p) \land v^\ast(q), \forall p,q\in \mathcal{F}$
    \item $v^\ast(p\lor q) = v^\ast(p) \lor v^\ast(q), \forall p,q\in \mathcal{F}$
    \item $v^\ast(p\supset q) = v^\ast(p) \implies v^\ast(q), \forall p,q\in \mathcal{F}$
\end{itemize}

Dove $\lnot, \land, \lor, \supset$ coincidono con gli operatori dell'algebra booleana
$', \land, \lor, \implies$.

\begin{nota}
    Il valore di verità della formula dipende dal valore di verità delle singole componenti
    elementari e questa proprietà si chiama \textbf{truth-functionality}.
\end{nota}

Successivamente si deve definire il concetto di soddisfacibilità.

\begin{definizione} 
    Se $v$ è la funzione di valutazione per un linguaggio formale $L$, se $p\in \mathcal{F}$
    e $v(p) = 1$ allora $v$ \textbf{soddisfa} $p$, $v\vDash p$.
\end{definizione}

Se $v$ non soddisfa $p$ allora $v(p) = 0$ e si scrive $v\not\vDash p$.

Diremo che una formula $p$ è \textbf{soffisfacibile} se $\exists v$ interpretazione 
tale che $v\vDash p$. Diremo che una formula $p$ è una \textbf{tautologia} se $\forall v$ interpretazioni
tale che $v\vDash p$. Diremo che una formula $p$ è una \textbf{contraddizione} se $\forall v$ interpretazioni
tale che $v\not \vDash p$.

\begin{nota}
    Se la formula $p$ è una tautologia allora si può direttamente scrivere $\vDash p$.
\end{nota}

\begin{definizione} [\textbf{Sistema deduttivo}]
    Un \textbf{sistema deduttivo} è una tripla $\langle L, \mathcal{U}, \{r_1,\dots, r_k\}\rangle$+
    dove:
    \begin{itemize}
        \item $L$ è il linguaggio formale
        \item $\mathcal{U}$ è l'insieme di assiomi (tautologie)
        \item $\{r_1,\dots, r_k\}$ sono le regole di inferenza.
    \end{itemize}
\end{definizione}

L'insieme di assiomi della logica booleana sono:
\begin{itemize}
    \item $p\supset (q \supset p)$
    \item $(p\supset (q \supset r))\supset ((p\supset q)\supset (p\supset r))$
    \item $\lnot p \supset(p\supset q)$
    \item $(p\supset q) \supset((\lnot p\supset q)\supset q)$
\end{itemize}

L'unica regola di inferenza sarà il \textbf{modus pones}
$$\frac{p,p\supset q}{q}$$

Ovviamente gli assiomi possono essere scritti anche secondo $\land,\lor$ dal momento 
che 
$$p\land q \equiv \lnot(p\supset \lnot q) \qquad p\lor q \equiv (p\supset q )\supset q$$

Inoltre si può utilizzare anche $\iff$ che coincide con
$$p\iff q \equiv (p\supset q) \land (q\supset p)$$


\begin{definizione} [\textbf{Deduzione}]
    Dato un sistema deduttivo, la \textbf{deduzione} di $q$ a partire da un insieme
    di proposizioni $A= (p_1,\dots, p_n)$ (premesse) è quando possiamo arrivare 
    dalle premesse alla formula $q$ utilizzando le regole di inferenza. In qìuesto 
    caso avremo 
    $$p_1,\dots p_n \vdash q \equiv A\vdash q$$
\end{definizione}

Il concetto di soddisfacibilità deriva dalla semantica mentre il concetto di 
deduzione deriva dalla sintassi.

\begin{definizione}
    Due formule $p,q$ sono logicamente equivalenti sse $p\vDash q$ e $q\vDash p$
    e le riscriviamo $p\equiv q$.
    
    $$p\equiv q \iff \forall v \begin{cases}
        v(p)=1 \implies v(q)=1\\
        v(q)=1 \implies v(p)=1\\
    \end{cases}$$
\end{definizione}
\begin{teorema}
    La sequenza $\langle \mathcal{F}, \land, \lor, \lnot, 0, 1\rangle$ è una 
    algebra booleana
\end{teorema}

\begin{teorema}
    Sia una formula $p$ è provabile nella logica bolleana, ex: $\vdash p$ sse 
    $p$ è una tautologia  nella logica, ovvero se $\forall v, v\vDash p$.
\end{teorema}


\section{Logiche a più valori di verità}

Logiche 3 valori di verità spesso assegnano un valore epistemico (sconosciuto) al
terzo valore di verità. Più spesso si l'interpretazione del terzo è una verità 
sfumata (mezzo vero), oppure per identificare un valore sconosciuto, oppure per 
identificare un valore irrilevante.

Il valore epistemico del terzo valore porta ad un problema di vero funzionale,
dove se $p=u$ ma anche $\lnot p =u$ quindi $p\land \lnot p$ dovrebbe essere falso 
e $p\lor p$ dovrebbe essere vero ma in realtà sono entrambi $u$.



\subsection{Logica di Kleene}
La logica di Kleene viene sviluppata dallo studio delle \textbf{funzioni parziali ricorsive},
ovvero $f:\mathbb{N}^n\to \mathbb{N}$ che sono definite su un $D\subseteq \mathbb{N}^n$.
Durante il loro studio Kleene definisce 2 tipologie di logiche:
\begin{itemize}
    \item \textbf{strong}: logica alla base della gestione del valore null in SQL,
    il terzo valore di verità assume la semantica di \textbf{unknown}.
    \item \textbf{weak}
\end{itemize} 

L'insieme dei valori di verità è $\{F,T,u\}$  quindi $T=\{t\}$, $F=\{f\}$
mentre $u=\{t,f\}$ dove $t,f$ sono i valori di verità della logica booleana data 
tutta la conoscienza nota.

Per definire i connettivi possiamo utilizzare le operazioni tra insiemi:
\begin{itemize}
    \item intersezione $\sqcap$ ovvero applica l'operatore booleano $\land$ su 
    tra tutte le coppie di valori di verità
    \item unione $\sqcup$ ovvero applica l'operatore booleano $\lor$ su 
    tra tutte le coppie di valori di verità
    \item negazione $'$ coincide con il complemento degli insiemi di valori di verità.
    \item implica $\implies$ ovvero coincide con la seguente formulazione 
    $A \implies B \equiv (\lnot A) \lor B \equiv A' \sqcup B $
\end{itemize}

\begin{esempio}
    Ecco un paio di esempi:
    \begin{itemize}
        \item $F\sqcap T = \{f\} \sqcap \{t\} = \{f\land t\} = \{f\} = F$ 
        \item $F\sqcap u = \{f\} \sqcap \{f,t\} = \{f\land t, f\land f\} = \{f, f\} =\{f\}= F$
        \item $T\sqcap u = \{t\} \sqcap \{f,t\} = \{t\land t, t\land f\} = \{f,t\} = u$
        \item $T\sqcup u = \{t\} \sqcup \{f,t\} = \{t\lor t, t\lor f\} = \{t,t\}= \{t\}= T$
        \item $F\sqcup u = \{f\} \sqcup \{f,t\} = \{f\lor t, f\lor f\} = \{f,t\}= u$
    \end{itemize}
\end{esempio}

Le tabelle di verità sono le seguenti.

\begin{table}[b]
    \centering
    \begin{subtable}[t]{0.49 \textwidth}
        \begin{tabular}{c|c|c|c}
            $\sqcap$ & F & u & T \\
            \hline
            F        & F & F & F \\
            u        & F & u & u \\
            T        & F & u & T
        \end{tabular}
    \end{subtable}
    \hspace{\fill}
    \begin{subtable}[t]{0.49 \textwidth}
        
        \begin{tabular}{c|c|c|c}
            $\sqcup$ & F & u & T \\
            \hline
            F        & F & u & T \\
            u        & u & u & T \\
            T        & T & T & T
        \end{tabular}
    \end{subtable}
    \bigskip

    \begin{subtable}[t]{0.49 \textwidth}
        \begin{tabular}{c|c}
            $'$& \\
            \hline
            F  & T \\
            u  & u \\
            T  & F 
        \end{tabular}
    \end{subtable}


    \begin{subtable}[t]{0.49 \textwidth}
        
        \begin{tabular}{c|c|c|c}
            $\implies$ & F & u & T \\
            \hline
            F        & T & T & T \\
            u        & u & u & T \\
            T        & F & u & T
        \end{tabular}
    \end{subtable}
    \bigskip
    
\end{table}


Vale il modus ponens ma non esistono tautologie perché se associamo ad ogni variabile 
della formula il valore sconosciuto allora i connettivi restituiscono sconosciuto.

Alla logica di Kleene possiamo sempre associare una struttura algebrica, prima 
però dobbiamo introdurre il \textbf{reticolo di de Morgan}.

\begin{definizione}[\textbf{Reticolo di De Morgan}]
    Il \textbf{reticolo di De Morgan} è $\langle L, \land, \lor,',0,1\rangle$ che è un reticolo limitato $0,1$
    dove $1=0'$ e l'operatore $'$ soddisfa solo:
    \begin{itemize}
        \item $a=a''$
        \item leggi di de Morgan: $a'\land b' = (a\lor b)'\qquad a'\lor b' = (a\land b)'$
    \end{itemize}
    (quindi non $b\lor b' = 1$ che vale per le algebre booleane)
    Se vale la distributività allora si parta di algebra di De Morgan.
\end{definizione}

\begin{nota}
    Nel reticolo di De Morgan i seguenti punti sono equivalenti:
    \begin{itemize}
        \item $a'\land b' = (a\lor b)'$
        \item $a'\lor b'= (a \land b)'$
        \item $a\le b \implies b'\le a'$
        \item $a'\le b' \implies b\le a$
    \end{itemize}
\end{nota}

\begin{definizione}[\textbf{Reticolo/algebra di Kleene}]
    Un \textbf{reticolo/algebra di Kleene} è un reticolo/algebra di De Morgan tale che 
    $$a\land a' \le b\lor b'$$
\end{definizione}
\begin{esempio}
    esistono reticoli di De Morgan che non sono di Kleene.
\end{esempio}

Successivamente dobbiamo definire il sistema deduttivo della logica, per la logica 
di Kleene non esistono più assiomi, ma si hanno solo le regole di inferenza.
Ciascuna regola rappresenta una proprietà della nostra logica:
\begin{itemize}
    \item $\frac{a\land b}{a}, \frac{a\land b}{b}$
    \item $\frac{a, b}{a\land b}$
    \item $\frac{a}{a\lor b}$
    \item $\frac{a\lor b}{b\lor a}$
    \item $\frac{a\lor a}{ a}$
    \item $\frac{a\lor (b\lor c)}{ (a\lor b)\lor c}$
    \item $\frac{a\lor (b\land c)}{ (a\lor b) \lor (a\lor c)}$
    \item $\frac{ (a\lor b) \lor (a\lor c)}{a\lor (b\land c)}$
    \item $\frac{ a\lor c}{\lnot \lnot a\lor c}$
    \item $\frac{ \lnot(a\lor b)\lor c}{(\lnot a\land \lnot b)\lor c}$
    \item $\frac{ \lnot(a\land b)\lor c}{(\lnot a\lor \lnot b)\lor c}$
    \item $\frac{\lnot \lnot a\lor c}{ a\lor c}$
    \item $\frac{(\lnot a\land \lnot b)\lor c}{ \lnot(a\lor b)\lor c}$
    \item $\frac{(\lnot a\lor \lnot b)\lor c}{ \lnot(a\land b)\lor c}$
    \item $\frac{a\lor (b\land \lnot b)}{ a}$
\end{itemize}
Le regole di inferenza esprimono tutte le proprietà della logica,  ex: distributività, doppia negazione.

Anche qui abbiamo l'inferenza sintattica ovvero $\Gamma \vdash p$ con $\Gamma \subseteq \mathcal{F}$
allora significa che $p$ si può derivare dell'insieme di formule $\Gamma$ mediante 
l'applicazione di regole di inferenza.

Questa algebra permette di modellare le query con null nei dbms relazionali.

Ai valori di verità possiamo associare dei valori reali $F=0, T=1,u=\frac{1}{2}$.
Questo mi permette di definire le operazioni con delle operazioni matematiche:
\begin{itemize}
    \item $a\land b = \min \{a,b\}$
    \item $a\lor b = \max \{a,b\}$
    \item $a' = 1-a$
\end{itemize}

Questo è uno dei possibili modi per estendere all'intervallo $[0,1]$ le operazioni.
Il problema del verità funzionale è che si considera $u$ come coflitto tra il vero 
e falso ma non tra quello noto come vero e noto come falso. Altre logiche separano 
queste operazioni.  

\textbf{DA SISTEMARE}
Possiamo modificare la congiunzione\dots (deve essere monotona, compatibile con la logica booleana).
La congiunzione non sempre è communtativa. Quelle con i nomi sono commutative e sono 
le uniche associative.

(tnorma, uninorma sono modi per definire AND su intervallo [0,1])

Possiamo definire lanche l'implicazione\dots Questo introduce delle ocndizioni di contorno
ex: $T\implies x= x$ e la proprietà d'ordine. Anche quin si ottengono altre logiche.
tnorme quelle azzurre.

La negazione può essere definita come implicazione e in base all'implicazione 
che andiamo ad usare allora il significato di $\frac{1}{2}'$ cambia.

Quindi abbiamo 14 congiunzioni e disgiunzioni e 


\subsection{Logica di prist}
Logica paraconsistente, permette di associare il valore $u$ quando si hanno inconsistenze.
Quindi in questo caso i valori di verità considerati come veri come $1,\frac{1}{2}$
la conseguenza è che si hanno tautologie, tutte le tautologie booleane vengono preservate 
però perdo il modusponens.
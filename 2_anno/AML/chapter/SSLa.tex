\chapter{Self supervised learning}
Generally we train a pretrained model on a big dataset.

PErformance of NN depends on capacity and dimension of dataset.
capacity = number of parameters. We can expect that high capacity and large dataset 
to improve performance. 

The problem on large dataset is the labelling because it is too expencive. Another 
problem is labelling video.

To avoid this problem, we can train the model to solve a pretext task and this 
help the model to learn features, morover, it labels the data. Then we use this model 
and apply transfert learninig to train the supervised downstream model

terminology:
\begin{itemize}
    \item \textbf{human-annotated label}: label created by humans
    \item \textbf{pretext task}: pre-design tasks for networks to solve and visual
    features  are learnined by learning objective functions of pretext tasks
    \item \textbf{pseudo label}: label used in pretext task 
    \item \textbf{Supervised learning}: reders to leanining methods that use 
    human annotated labels
    \item \textbf{semi- supervised learning}: small amount of labeld data combined 
    with high amount of unlabelled data
    \item \textbf{weakly- supervised learning}: refers to learningi methods to learn whith 
    coarse grained labels or inaccurate labels. Weak-annotation is cheaper than the fine-grained labels 
    \item \textbf{unsupervised learning}: no labels
    \item \textbf{self-supervised learning}: we extract labels from data and we use to train models
\end{itemize}

pro:
\begin{itemize}
    \item we don't have human annotation and can be use for dataset of any size
    \item 
\end{itemize}

there are many pretext tasks:
\begin{itemize}
    \item foreground segmentation
    \item image inpainting 
    \item clustering 
    \item image colorization
    \item temporal order verification
    \item visual audio correspondece verification
\end{itemize}

learning visual features:
\begin{itemize}
    \item coloring a real image grey scale (convert from grey to rgb). The 
    input is the gray scale image and the label is the original image
    \item there are many
\end{itemize}
we can cathegorize approches in 4 groups:
\begin{itemize}
    \item generation: involved video or image generation. Visual features are learning 
    image colorzation, image super resolution , image impainting , image generation GAN.
    Visual features are learned in video generation with GAN and video prediction.
    Label is original image. For GAN we train the network and we use only the discriminator 
    and we take features from last layers.
    \item context-based: employs the context features. It can be done by context similarity, 
    spatial/temporal context structure. We want learn features by learn structure of 
    the features. We can do bu predictive task or contrastive task, for both we 
    can clusterize data in the way that data in the same grous are data with similar
    context and the claster is the label. In this way the model train the invariance features.
    For contrastive we want that a trasformed image is similar to the original but
    differ by others.  In the spatial context we can divide image in patched and 
    than the model has to learn the order of patches to reconstruct the original
    image (model learn spatial features). The problem of this approach is the 
    combiantory explosion, we can try to limit considering a subset of total number of 
    permutation, we need to define a task not too difficult and not to easy other ways
    it can't learn. 
    \item free semantic label: train network with some semantic label that are automatic 
    generated. The labels are not extracted by the features. Ex: we can generate 
    image using a game engine because we can access to depth data, segmentation, lightining.
    We create a syntetic dataset but this has a problem because game engine render
    images at high resolution, while in the reality we use low resolution image.
    We can generate data by using hard code programs, by using standard algorithms
    \item cross modal based methods: multiple multimodal information, 1 channel for 
    each type information ex: audio and video. The goal is matching audio and video. 
\end{itemize}

we can compare performance of image feature training by evaluate the classificator on 
the same dataset.

Self supervised learning add robustness to model.

\chapter{Feed Forward Neural Network}
\textbf{Feed Forward Neural Network} is a neural network with a bounch of layers 
each one direct connected to the next one without loop. Each layer represent
a vector function and the entire network is a composition of each function on each 
layer. For example: if a FFNN is an approximization of $f(x)$ and has $3$ hidden layer,
so we have 
$$f(x) = f^3(f^2(f^1(x)))$$
Where $f^i$ is a function of $i$-layer. The overall lenght of hidden layers is the \textbf{depth} of the model.

During the training process we want to learn a function $f(x)$ that generates our 
real data by approximizate a $f^*(x)$, called target function, using a neural network.   
$f^*(x)$ will be learned using training set, in the case of supervised problems
we will try to get for all $x$ training example: $y\sim f^*(x)$. It's important 
that is $y\sim f^*(x)$ and not $y=f^*(x)$ because we want a model that generalizes.

The trainig process starts from a random initialization of all weights, that for each 
training example $(x, y)$ the network compute $f^*(x)$ than we compute the error
$E(y, f^*(x))$ and we adjust all weights by using gradient approximate with 
backpropagation. 

\begin{nota}
During trainig phase weights are adjusted every $n$ examples, where $n$ is the batch 
size. Generaly we use batch because if we adjust weights at every examples, we risk 
to enter in overfitting.
\end{nota}

If $f(x)$ is \textbf{non linear} than we can teoretical approximate with an hidden 
layer, the problem is to find weights.

The essence of supervised machine learning is the creation of functions that 
can look at examples (instances) and produce generalizations (predict data never seen).
This is a searching task over oll possible functions, this is complex problem to solve
so generaly we restrict the set of all possible functions to a specific families,
this is called \textbf{hypothesis classes}. 

An example is to approximate a non linear function using only a linear function, 
this is possible by modifing the rappresentation, in this way we reduce the number 
of candidates to only linear function family.

To modify rappresentation of data we apply \textbf{non linear transformation} $\phi$.
Given an instance $x$, $\phi(x)$ is a new set of features describing $x$. There are 
a lot of mode to define a transformation:
\begin{itemize}
    \item use a predefine $\phi$ called \textbf{kernel}
    \item define manually $\phi$ but it isn't convinient
    \item we can learn it  but it isn't a general transformation
\end{itemize}

$Kernel$ and leaned transformation are automatically defined. In deep learning 
we learn a transformation, while in SVM we use a Kernel. So a deep learning model 
can be described mathematically by the following formula
$$f^*(x;\theta, \omega, b) = \phi(x; \theta)^T \omega + b$$

Where $f^*$ is the approximation of the target, $\phi$ is the transformation of 
the input, the underlying model is linear, $\omega$ are weights, $\theta$ are 
parameters of the transformation. $\phi$ is defines a hidden layer, so for complex 
network we have a composition of $\phi$ defining a series of hidden layer. 

The prediction model is generally a high linear function $f': wx + b$, here we have a listing:
\begin{itemize}
    \item \textbf{sign function}: this function is used for binary classification
    $$\hat{y} = sign(f'(x))$$
    \item \textbf{sigmoid function}: this function is used for binary classification
    when we need a confidence on decision for log-linear binary classification.
    $$\hat{y} = \frac{1}{1+e^{f'(x)}}$$
    \item \textbf{softmax}: this is used for multi-class classification, suppose 
    to have $k$ classes
    $$\hat{y} =softmax(f'(x)) = \frac{e^{f'(x)_i}}{\sum_{j=1}^k e^{f'(x)_j}}$$
\end{itemize}

\textbf{Pros} for linear models are that can be fit efficiently and reliably, because 
gradient on linear model are easy to compute.

\textbf{Cons} models are restricted to linear fucntions so they works for data linear
separable, solved by a transformation.

\section{Kernel trick}
Operations in higher dimensional space may be computationally expensive, to solve 
this problem we use kernel trick that operate to original feaure space without computing 
the transformations, generally kernel trick is used by SVM.

A general linear model would be
$$w^T x + b = b+ \sum_{i=1}^m \alpha_ix^Tx^{(i)}$$

Where $b$ is the byas, $\alpha_i$ is a coefficient, $x$ is a point, $x^(i)$ is 
$i$-instance of training set.
To use this model on not linear separable data we apply a transformations 
$$ b+ \sum_{i=1}^m \alpha_ix^Tx^{(i)} = b+ \sum_{i=1}^m \alpha_i\phi (x^T)\phi(x^{(i)}) $$
The kernel trick override $\phi (x^T)\phi(x^{(i)})$ with $k(x^T, x^(i))$ without 
computing $\phi(x)$ and it's more computational efficient compared to $\phi$. Moreover
kernel trick allows us to learn models that are nonlinear as a function of x using 
a convex optimization techniques that are guaranteed to converge efficiently, this is 
possible because $\phi $ is fixed and we optimaze only $\alpha_i$.
SVM uses only support vector and $\alpha_i$ regulates this, infact $alpha_i= 0$
if and only if $i$ isn't a support vector.

Each algorithm that uses kernel is called \textbf{kernel machines} or \textbf{kernel 
methods}

\section{Gradient optimization}
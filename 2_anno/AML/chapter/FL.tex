\chapter{Federated Learning}
The standard setting in Machine Learning (ML) considers a centralized dataset
processed in a tightly integrated system, but in the real world data is often
decentralized across many parties. One challenge is that collecting these data
at a single point is not always feasible for various reasons. Transmitting the
data might be prohibitively expensive, such as when dealing with large volumes
of data, or it may be constrained by the limited bandwidth or power capacity of
wireless devices.

Another challenge is that the data might be believed too sensitive. In recent
years, there has been increasing public awareness and stricter regulations
concerning data privacy. Additionally, maintaining control over data can offer a
competitive edge in both business and research contexts.

One possible solution could be to create a separate model for each data source,
utilizing the data available locally. However, this approach has significant
drawbacks. The amount of data at each source may be too limited, leading to
overfitting or decisions that lack statistical significance. Moreover, the
local dataset may fail to accurately represent the overall target distribution,
further compromising the model's effectiveness.

\textbf{Federated Learning} (FL) aims to collaboratively train a ML model while
keeping the data decentralized. The methodology for implementing federated
learning is as follows:
\begin{enumerate}
    \item The server initialize the model weights and send a copy to all the parties;
    \item Each party make an update using its local dataset;
    \item Parties share local updates for aggregation;
    \item Server aggregates updates and send back to parties;
    \item Parties update their copy of the model and iterate.
\end{enumerate}

We would like the final model to be as good as the centralized solution (ideally),
or at least better than what each party can learn on its own.

\begin{note}
    The assumption is that starting from the weights we cannot obtain the data,
    otherwise we could have just send the data.
\end{note}

More broadly, we can categorize federated learning (FL) into different
classifications. The first classification differentiates between:
\begin{itemize}
    \item \textbf{Cross-device FL}: this scenario involves a large number of
          participants, each possessing a small dataset. Additionally, it is
          important to account for the fact that some participants may not always
          be reachable, and some may have malicious intentions, posing further
          challenges.
    \item \textbf{Cross-silo}: in this scenario we have a limited number of parties
          with medium large dataset. Parties are always available and are
          typically honest.
\end{itemize}

Another type of classification is:
\begin{itemize}
    \item \textbf{Server orchestrated FL}: where we have a server that performs
          global coordination and aggregation. The server is the bottleneck and
          a single point of failure.
    \item \textbf{Device-to-device communication}: peer to peer communication
          between the parties. This approach can scale better.
\end{itemize}

\section{Distributed Learning}
A different training paradigm is \textbf{distributed learning}. In this approach,
unlike federated learning (FL), the data is initially centralized and then distributed
later to accelerate model training. The key distinction from FL lies in how the
data is distributed: in distributed learning, the data is typically distributed
uniformly at random across workers under controlled conditions.

In contrast, FL involves data that is naturally distributed and generated locally.
This data is often not independent and identically distributed (non-i.i.d.) and
is usually imbalanced. FL also introduces additional challenges that span various
fields of computer science, such as enforcing privacy constraints, addressing
the potentially limited reliability and availability of participants, ensuring
robustness against malicious actors, and more.
\section{Baseline algorithm: FedAvg}
Let's start by introducing some notation. Let's consider a set of $K$ parts
(clients). Each part $k$ holds a data set $\mathcal{D}_k$ composed of $n_k$
instances. Let $\mathcal{D} = \mathcal{D}_1 \cup \dots \cup \mathcal{D}_K$ be the
total dataset and $n = \sum_k n_k$ the total number of instance.

In this situation, we want to solve the problems of the form:
\begin{equation}
    \min_{\theta \in \mathbb{R}^p} F(\theta; \mathcal{D})
\end{equation}
where:
\begin{equation}
    F(\theta; \mathcal{D}) = \sum_{k = 1}^K \frac{n_k}{n} F_k(\theta; \mathcal{D}_k) \,\,\, \text{ and }
    \,\,\, F_k(\theta; \mathcal{D}_k) = \sum_{d \in \mathcal{D}_k} f(\theta; d)
\end{equation}
where $\theta \in \mathbb{R}^p$ are model parameters such as neural network weights.

\begin{multicols}{2}
    \noindent
    \textbf{Algorithm: FedAvg (server-side)} \\
    \textbf{Parameters:} client sampling rate $\rho$ \\
    \begin{algorithmic}[1]
        \State Initialize $\theta$
        \For{each round $t = 0, 1, \dots$}
        \State $S_t \gets$ random set of $m = \lceil \rho K \rceil$ clients
        \For{each client $k \in S_t$ in parallel}
        \State $\theta_k \gets \text{ClientUpdate}(k, \theta)$
        \EndFor
        \State $\theta \gets \sum_{k \in S_t} \frac{n_k}{n} \theta_k$
        \EndFor
    \end{algorithmic}

    \vfill

    \columnbreak

    \noindent
    \textbf{Algorithm: ClientUpdate($k, \theta$)} \\
    \textbf{Parameters:} batch size $B$, number of local steps $L$, learning rate $\eta$ \\
    \begin{algorithmic}[1]
        \For{each local step $l = 1, \dots, L$}
        \State $B \gets$ mini-batch of $b$ examples from $\mathcal{D}_k$
        \State $\theta \gets \theta - \frac{\eta}{B} \sum_{d \in B} \nabla \ell(\theta; d)$
        \EndFor
        \State \textbf{send} $\theta$ to server
    \end{algorithmic}
\end{multicols}

\textbf{Notes:}
\begin{itemize}
    \item For $L = 1$ and $\rho = 1$, it is equivalent to classic parallel SGD:
          updates are aggregated, and the model is synchronized at each step.
    \item For $L > 1$, each client performs multiple local SGD steps before communicating.
\end{itemize}

\textbf{FedAvg} with $L > 1$ allows to reduce the number of communication rounds,
which is often the bottleneck in FL. It empirically achieves better generalization
than parallel SGD with large mini-batch. Convergence to the optimal model can be
guaranteed for i.i.d. data, but issues arise in strongly non-i.i.d. cases.

Learning from non-i.i.d. data is difficult/slow because each party wants the model
to go in a particular direction. If data distributions are very different, learning
a single model which performs well for all parties may require a very large number
of parameters.

Another direction to deal with non-i.i.d. data is thus to lift the requirement
that the learned model should be the same for all parties. Instead, we can allow
each party $k$ to learn a (potentially simpler) personalized model $\theta_k$
but design the objective so as to enforce some kind of collaboration.

Several strategies have been developed to manage the combination of this last
idea, including:
\begin{itemize}
    \item A first approach propose to regularize personalized models to their mean:
          \begin{equation}
              F(\theta_1, \dots, \theta_k; \mathcal{D}) = \frac{1}{K} \sum_{k = 1}^K F_k(\theta_k; \mathcal{D}) + \frac{\lambda}{2 \cdot K} \sum_{k = 1}^K \left\|\theta_k - \frac{1}{K} \sum_{i = i}^K \theta_i\right\|
          \end{equation}
    \item Another approach, inspired by \textbf{meta-learning}, propose to learn
          a global model which easily adapts to each party:
          \begin{equation}
              F(\theta; \mathcal{D}) = \frac{1}{K} \sum_{k = 1}^K F_k(\theta - \alpha \cdot \nabla F_k(\theta); \mathcal{D})
          \end{equation}
\end{itemize}

We can remove outliers using median at the place of mean.
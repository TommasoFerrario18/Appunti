\chapter{Counterfactuals}
To make any sense about responsibility, we must be able to compare what did happen
with what would have happened under some alternative hypothesis.
\section{Introduction to counterfactuals}
We start by introducing a very useful concept for this part, namely \textbf{Hypothetical
    condition} or \textbf{Antecedent}.  These terms refer to the decision you
have made in the counterfactuals world.

\textbf{Counterfactuals} are used to compare two outcomes under the exact same
conditions, differing only in antecedent.

The fact that we know the outcome of our actual decision is important, because
my estimand after seeing the consequences of my actual decision may be totally
different from my estimate prior to seeing the consequence.

We cannot express this estimate using the do-expression because the do-operator
is too crude to distinguish between the \textbf{actual value} and the
\textbf{hypothetical one}. In this section, we are trying to compare between the
results of different possibilities after we had choose one of them and observe
the results.

We need this distinction in order to let the actual value inform our assessment
of the hypothetical value. The do-operator is not capable to respond at this question.

Below, we will use the following notation, which allows us to express the possible
results of $Y$ knowing that we have chosen $X = x$:
\begin{equation*}
    Y_{X = x} = Y_x = Y(X = x)
\end{equation*}

We can formulare it as follow:
\begin{equation*}
    \mathbb{E}[Y_1 | X = 0, Y = Y_0]
\end{equation*}
where $Y_1$ is the \textbf{hypothetical condition} and $X = 0,Y = Y_0$ are the
\textbf{observed variables}.

\begin{note}
    Randomized control experiments cannot help us because they give us the information
    on hypothetical condition after the intervention:
    \begin{equation*}
        \mathbb{E}[Y_1 | do(X = 0)]
    \end{equation*}
    but it doesn't add information of the fact that we had observed $Y_0$.
\end{note}

\section{Defining and computing counterfactual}
Consider a fully specified \textit{structural causal model} $M=\langle U, V,F \rangle$
for which we know both the functions $F$ and the values of all exogenous variables $U$.

In such a deterministic model every assignment $U = u$ to the exogenous variables
corresponds to a single member of, or a unit in a population, or to a situation
in nature.

The reason for this correspondence is as follows: each assignment $U = u$ uniquely
determines the value of all variables in $V$. This is given by the definition of $M$.
Also, the characteristic of each unit in a population have unique values, depending
on that individual's identity.

Consider now the following \textbf{counterfactual sentence}:
\begin{center}
    "$Y$ would be $y$ had $X$ been $x$, in situation $U=u$"
\end{center}
which can be summarize in $Y_x(u) = y$

The phrase part ``had $X$ been $x$'' need to be interpreted as an instruction to
make a minimal modification in the \textbf{current model} so as to establish the
antecedent condition $X = x$ which is likely to conflict with the observed valve
$X$, $X(u)$. Such a minimal modification amounts to replacing the equation for
$X$ with a constant $x$, which may be thought as an external intervention $do(X = x)$
not necessary by a human experimenter.

This replacement permits the constant $x$ to differ from the actual value of $X$
($X(u)$) without rendering the system of equations inconsistent, and in this way,
it allows all variables to serve as antecedent to other variables.

Each structural causal model encodes within it many counterfactuals, corresponding
to the various values that its variable can take.

Counterfactuals are different than the ordinary interventions, captured by the
do-operator. Now, we want to work at individual level: for each situation $U = u$,
we obtained a definite number, $Y_x (u)$, which stands for than hypothetical value
of $Y$ in that situation.

The do-operator is defined exclusively for probability distributions and always
produces probabilistic outcomes, such as $\mathbb{E}[Y|do(X = x)]$.It operates at
the population level. In contrast, counterfactuals focus on individual units
$u$ within the population $U$, where we compute the outcome $Y_x(u)$. This represents
the hypothetical value of $Y$ for the specific unit $u$ in a scenario where $X = x$.

We are now ready to generalize the concept of counterfactuals to any structural
model $M$. Consider any arbitrary two variables $X$ and $Y$, not necessarily
connected by a single equation.

We can pass from a model $M$ to $M_x$ by replacing the equation of $X$ with $X = x$,
so we can introduce the formal definition of counterfactuals $Y_x (u)$:
\begin{equation}
    Y_x (u) = Y_{M_x}(u)
\end{equation}
The results is the counterfactual obtained after the modification of the model
$M$ in $M_x$. So each model $M$ has a graph $G$ and $M_x$ has a graph $G_X$, where
$G_X$ is the graph of $M$ ($G$) where all incoming edges to $X$ have been removed.

The same definition can be also apply when $X$ and $Y$ are sets of variables.

Counterfactuals enable us to use our scientific model of reality, $M$, to answer
a vast array of hypothetical questions. The values assigned to these counterfactuals
are not arbitrary; they must be internally consistent and align with the underlying model.

\begin{note}
    If we observed $X(u) = 1$ and $Y(u) = 0$ then the counterfactual:
    \begin{equation*}
        Y_{X=1}(u) = 0
    \end{equation*}
    is coherent with the world, because you are watching the actual world.
\end{note}

\begin{definition}[\textbf{Counterfactual consistency rule}]
    If $X = x$ then $Y_x = Y$.

    If $X$ is binary then we have the convenient form:
    \begin{equation}
        Y = X \cdot Y_1 + (1 - X) \cdot Y_0
    \end{equation}
\end{definition}
This is coherent because:
\begin{itemize}
    \item $Y_1$ is equal to observed value of $Y$ whenever $X$ takes the value of $1$
    \item $Y_0$ is equal to observed value of $Y$ whenever $X$ takes the value of $0$
\end{itemize}

Any \textbf{deterministic counterfactuals} is computed by the following steps:
\begin{itemize}
    \item \textbf{Abduction}: use evidence $E = e$ to determine the value of $U$.
    \item \textbf{Action}: modify the model, $M$, by removing the structural
          equations for the variables in $X$ and replacing them with the appropriate
          functions $X = x$, to obtain the modified model $M_x$.
    \item \textbf{Prediction}: use the modified model $M_x$ and the values of $U$
          to compute the value of $Y$, the consequence of the counterfactuals.
\end{itemize}

The three steps will solve any deterministic counterfactual,that is, counterfactuals
pertaining to a single unit of the population in which we know the value of every
relevant variable.

Structural equation models are able to answer counterfactual queries of this nature
because each equation represents the mechanism by which the variable obtains its
values.

If we know these mechanism, we should also be able to predict what values would be
obtained had some of these mechanisms been altered, given alterations.

\subsection{Probabilistic counterfactual}
Counterfactual can also be probabilistic, pertaining to a class of units within
the population.

This probabilistic counterfactuals differ from do-operator interventions because
they restrict the set of individual intervened upon, which do-expression cannot do.
We do not have information on all variables (we can't compute $u$) and probabilistic
counterfactual allows us to ask question about probabilities and expectations of
counterfactuals.

Nondeterminism enters causal models by assigning probabilities $P(U = u)$
(\textbf{exogenous probability}) over the exogenous variables
\begin{center}
    Uncertainty = Probability distribution
\end{center}

Exogenous probability induce a unique probability distribution on the endogenous
variables.
\begin{equation*}
    P(U = u) \implies P(V = v)
\end{equation*}
where $v$ is the value of the endogenous variables determined by $u$. This allow
us to compute the joint distribution of all combinations of observed and
counterfactuals variables.

A typical query will be
\begin{center}
    Given that we observe feature $E = e$ for e given individual, what would we
    expect the value of $Y$ for that individual to be if $X$ had been $x$?
\end{center}
This expectation is given by the conditional expectation:
\begin{equation*}
    \mathbb{E}[Y_{X = x} | E = e]
\end{equation*}
where we allow $E = e$ to conflict which the antecedent $X = x$.

Given an arbitrary counterfactual of the form:
\begin{equation*}
    \mathbb{E}[Y_{X = x} | E = e]
\end{equation*}
the three-step process for probabilistic counterfactual:
\begin{itemize}
    \item \textbf{Abduction}: update $P(U)$ by the evidence to obtain $P(U| E = e)$;
    \item \textbf{Action}: modify the model, $M$, by removing the structural equations for
          the variables in $X$ and replacing them with the appropriate function
          $X = x$ to obtain the modified model $M_X$;
    \item \textbf{Prediction}: use the modify model $M_X$ and the updated probabilities
          over the $U$ variables, $P(U| E = e)$, to compare the expectation of $Y$,
          the consequence of the counterfactuals.
\end{itemize}
\section{Nondeterministic Counterfactuals}
As already mentioned above, to go to insert the concept of non-determinism in
counterfactuals we assign a probability to each value of $U$.

\textbf{Cross-world probabilities} are the probabilities of counterfactuals, that
is, the probabilities of the form $P(Y_x = y)$, where $Y_x$ is the counterfactual
value of $Y$ when $X = x$.

Cross-world probabilities are as simple to derive as \textbf{intra-world probabilities}:
we simply identify the rows in which the specified combination is true and sum
up the probabilities assigned to those rows. This allows us to compute conditional
probabilities among counterfactuals and defining notions such as dependence and
conditional independence among counterfactuals.

Joint probabilities over multiple-world counterfactuals can be computed from any
structural model. They cannot however be expressed using the $do(x)$ notation,
because the latter delivers just one probability for each intervention $X = x$.

Thus, in general, the do-expression will not capture our counterfactual question:
\begin{equation*}
    \mathbb{E}[Y| do(X = 1), Z = 1] \neq \mathbb{E}[Y_{X = 1} | Z = 1]
\end{equation*}
The $do(x)$ notation cannot capture this difference, because:
\begin{itemize}
    \item $X = 1$ refers to preintervention world;
    \item $Z = 1$ refers to postintervention world.
\end{itemize}

\textbf{Retrospective Reasoning} concerning dependence on the unrealized past, is 
not shown explicitly in the graph structure that we had use upon. To facilitate 
such reasoning, we need to devise means of representing counterfactual variables 
directly in the graph.
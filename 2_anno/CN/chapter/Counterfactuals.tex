\chapter{Counter Factuals}
To make any sense on responsability, we must be able to compare what did happend
with what would have happened under some alternative hypothesis.
\section{Introduction to counterfactuals}


\textbf{Hypochetical condision} or \textbf{Antecendent} is the kind of decision
we could have taken in the counter factual worlds.


Counter facuals are used to compare two outcomes under the exact same conditions,
differing only in antecedent.

The fact that we know the outcome of our actual decision is important, because my estimand
after seeing the consequences of my actual decision may be totally different from
my etimate prior to seeing the consequence.

We cannot express this estimate using the do-expression because the do-operator
is too crude To distinguish: The \textbf{actual value} and the \textbf{hypothetical one}.
We are trying to compare between two choice after being choose one.

We need this distinction in order to let the actual value inform our assessment
of the hypothetical value. The do-operator is not capable to respond at this question.

We will use potential outcame frameworks:
$$Y_{X=x} = Y_x = Y(X=x)$$
Because we need to discover the values of $Y$ befour the choise of $X=x$.


We can formulare it as follow:

$$\mathbb{E}[Y_1|X=0,Y=Y_0]$$

Where $Y_1$ is the \textbf{hypothetical condition} and $X=0,Y=Y_0$ are the observed
variables.

\begin{note}
    Randomized control experiments cannot help us because they give us the information
    on hypothetical condition after the intervention:
    $$\mathbb{E}[Y_1|do(X=x)]$$
    but it doesn't add information of the fact that we had observed $Y_0$.
\end{note}

\section{Defining and computing counterfactual}

Consider a fully specified model SCM $M=<U, V,F>$ for which we know both the
functions $F$ and the values of all exogenous variables $U$.

In such a deterministic model every assignment $U=u$ to the exogenous variables
corresponds to a single member of, or a unit in a population, or to a situation in nature.
Each assignment $U=u$ uniquely determines the value of all variables in $V$. This
is given by the definintion of $M$.

The characteristic of each unit in a population have unique values, depending on
that individual's identity.

Consider now the \textbf{counterfactual sentence}
\begin{center}
    "$Y$ would be $y$ had $X$ been $x$, in situation $U=u$"
\end{center}
Which can be summarize in $Y_x(u) = y$


had $X$ been $x$ need to be interpreted as an istruction to more a minimal modification
in the \textbf{current model} so as to establish the antecedent condition $X = x$ which is
likely to conflict with the observed valve $X$, $X(u)$. Such a minimal modification
amounts to replacing the equation for $X$ with a constant $x$, which may be thought
as an external intervention $do(X=x)$ not necessary by a human experimenter.

This replacement permits the constant $x$ to differ from the actual value of $X$ ($X(u)$)
without rendering the system of equations inconsistent, and in this way, it allows
all variables to serve as antecedent to other variables.

Each SCM encodes within it many counterfactuals, corresponding to the various values
that its variable can take.

Counterfactuals are different than the ordinary inverventions, captured by the do-operacor.
We want to work at individual level: for each sizuation $U =u$, we obtained a definite
number, $Y_x (u)$, which stands for than hypothesical value of $Y$ in that situation.

The do-operator is only defined on probability distributions and always delivers
probabilistic results such as $\mathbb{E}[Y| do(x)]$. We are working on a population
level, while with counter factuals we are working on a unit $u$ of the population $U$
computing the outcome $Y_x(u)$, we are computing the hypothetical value of $Y$
in the situation of $X=x$ for the unit $u$.

We are now ready to generalize the concept of counterfactuals to any structural
model $M$. Consider any arbitrary two variables $X$ and $Y$, not necessarily connected
by a single equation.

We can pass from a model $M$ to $M_x$ by replacing the equation of $X$ with $X=x$,
so we can introduce the formal definition of counterfactuals $Y_x (u)$:
$$Y_x (u) = Y_{M_x}(u)$$
The counterfactual is the counterfactual obtained after the modification of the model
$M$ in $M_x$. So each model $M$ has a graph $G$ and $M_x$ has a graph $G_X$, where
$G_X$ is the graph of $M$ ($G$) where all incoming edges to $X$ have been removed.

The same definition is applicable when $X$ and $Y$ are sets of variables.

Counterfactuals allow us to take our scientific conception of reality, $M$, and
use it to generate answers to an enormous number of hypothetical question.
The values than these counterfacuals recieve are not totally arbitrary, but must
cohere with each other to be consistent with an underlying model.

\begin{note}
    If we observed $X(u)=1$ and $Y(u) = 0$ then the counterfactual:
    $$Y_{X=1}(u)=0$$
    is coeherent with the world, because you are whatching the actual world
\end{note}

\begin{definition}[\textbf{Counterfactual consistency rule}]
    If $X= x$ then $Y_x=Y$.

    If $X$ is binary then we have the convenient form
    $$Y = X_1 + (1-X)Y_0 $$
\end{definition}
This is coherent because:
\begin{itemize}
    \item $Y_1$ is equal to observed value of $Y$ whenever $X$ takes the value of $1$
    \item $Y_0$ is equal to observed value of $Y$ whenever $X$ takes the value of $0$
\end{itemize}

Any \textbf{deterministic counterfactuals} is computed by the following steps:
\begin{itemize}
    \item \textbf{Abduction}: use evidence $E = e$ to derermine the value of $U$
    \item \textbf{Action}: modify the model, $M$, by removing the structural equations for
          the variables in $X$ and replacing them with the appropriate functions $X = x$,
          to obrain the modified model $M_x$
    \item \textbf{Prediction}: use the modified model $M_x$ and the values of $U$ to compute
          the value of $Y$, the consequence of the counterfacuals.
\end{itemize}

So to compute counterfactual we need a unit and each observed variables of that unit.
Structural equation models are able to answer counterfactual queries of this nature because
each equation represents the mechanism by which the variable obtains its values.

If we know these mechanism, we should also be able to predict what values would be
obtained had some of these mechanisms been altered, given alterations.

\subsection{Probabilistic counterfactual}
Counterfactual can also be probabilistic, pertaining to a class of units within the population (ex: $Y<2$).
This probabilistic counterfactuals differ from do-operator intervencions because
they restrict the set of individual intervened upon, which do-expression cannot do.
We do not have information on all variables (we can't compute $u$) and probabilistic counterfactual allows
us to ask question about probabilisties and expectations of counterfactuals.

Nondeterminism enters causal models by assigning probabilizies $P(U=u)$ (\textbf{exogenous probability}) over the
exogenous variables
\begin{center}
    Uncertaty = Probability distribution
\end{center}

Exogenous probability induce a unique probability distriburion on the endogenous
variables.

$$P(U=u) \implies P(V=v)$$

This allow us to compute the joint distribuzion of all combinations of
observed and counterfactuals variables.

A typical query will be
\begin{center}
    Given that we observe feature $E=e$ for e given individual, what would we
    expect the value of $Y$ for that individual to be if $X$ had been $x$?
\end{center}

This expected is denoted $\mathbb{E}[Y_{X=x}|E=e]$
where we allow $E=e$ to conflict wich the antecedent $X=x$

The three-step process for probabilistic counterfactual:
\begin{itemize}
    \item \textbf{Abduction}: Update $P(U)$ by the envidence to obtain $P(U |E=e)$
    \item \textbf{Action}: modify the model, $M$, by removing the structural equations for
          the variables in $X$ and replacing them with the appropiate function $X=x$ to
          obtain the modified model $M_X$
    \item \textbf{Prediction}: use the modify model $M_X$ and the updated probabilities
          over $U$ variables, $P(U|E=e)$, to compure the expectation of $Y$, the consequence
          of the countertactuals
\end{itemize}

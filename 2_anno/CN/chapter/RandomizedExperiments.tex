\chapter{Randomized Experiments}
\textbf{Randomized experiments} are noticeably different from observational studies.
In randomized experiments, the experimenter has complete control over the treatment
assignment mechanism.

This complete control over how treatment is chosen is what distinguishes randomized
experiments from observational studies. In this simple experimental setup, the
treatment isn't a function of covariates at all, it's assigned randomly. While,
in observational studies, the treatment is almost always a function of some of
the covariate(s). This difference is key to whether or not confounding is present
in our data.

In randomized experiments, \textit{association} = \textit{causation} because they
guarantee that there is no confounding. This can be seen from the following:
\begin{equation}
    \mathbb{E}[Y(1)] -  \mathbb{E}[Y(0)] = \mathbb{E}[Y| X = 1] - \mathbb{E}[Y | X = 0]
\end{equation}

Since the treatment and control groups would be the same, in all aspects, except
for treatment $X$, any difference in the outcomes $Y$ of the treatment and control
groups is due to the treatment $X$.
\begin{definition}[\textbf{Covariate balance}]
    We have \textbf{covariate balance} if the distribution of covariates $Z$ is
    the same across treatment groups.
    \begin{equation}
        P(Z|X = 1) \stackrel{d}{=} P(Z|X = 0)
    \end{equation}
\end{definition}

Randomization implies covariate balance, across all covariates $Z$, even unobserved
ones. Intuitively, this is because the treatment is chosen at random, regardless
of $Z$, so the treatment and control groups should look very similar. This is
confirmed by the fact that in randomized experiment, we chose randomly how to assign
treatment so we are removing dependencies between the confounding variables
and the treatment, so there aren't any backdoor paths from $X$ to $Y$ and $\mathbf{S} =\{\emptyset\}$
is a sufficient adjustment to block any association flowing from $X$ to $Y$ except
for causal one.

We will now prove that association is equal to causation in randomized experiments
by proving that:
\begin{equation*}
    P(Y|do(x)) = P(Y|x)
\end{equation*}
For the proof, the main property we utilize is that covariate balance implies
$Z$ and $X$ are independent.

First, let $\mathbf{S} = \{\emptyset\}$ be a \textbf{sufficient adjustment set}
that potentially contains unobserved variables. Such an \textbf{adjustment set}
$\mathbf{S}$ must exist because we allow it to contain any variables, observed
or unobserved.

Then, from the backdoor adjustments we get:
\begin{equation}
    \begin{array}{ll}
        P(y|do(x)) & = \sum_{\mathbf{s}} P(y|x, \mathbf{S} = \mathbf{s})P(\mathbf{S} = \mathbf{s})                \\
                   & = \sum_{\mathbf{s}} \frac{P(y|x, \mathbf{s})P(x|\mathbf{s})P(\mathbf{s})}{P(x | \mathbf{s})} \\
                   & = \sum_{\mathbf{s}} \frac{P(y, x, \mathbf{s})}{P(x | \mathbf{s})}                            \\
                   & = \sum_{\mathbf{s}} P(y, \mathbf{s}| x)                                                      \\
                   & = P(y | x)
    \end{array}
\end{equation}
Thise means that all associations are causation.

\textbf{Exchangeability} gives us another perspective on why randomization makes
causation equal to association. If the groups are exchangeable, we could exchange
these groups, and the average outcomes would remain the same.

The final perspective that we'll look at to see why association is causation in
randomized experiments is that of \textbf{graphical causal models}.

In regular observational data, there is almost always confounding. However, if we
randomize it doesn't depend on anything other than the output of a coin toss.
Since the node has no incoming edges and thus no backdoor paths exist.
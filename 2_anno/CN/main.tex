\documentclass[a4paper, oneside]{book}
\usepackage[italian]{babel}
\usepackage[utf8]{inputenc}
\usepackage[a4paper,top=2.5cm,bottom=2.5cm,left=2cm,right=2cm]{geometry}
\usepackage{amssymb}
\usepackage{amsthm}
\usepackage{graphics}
\usepackage{amsfonts}
\usepackage{amsmath}
\usepackage{amstext}
\usepackage{engrec}
\usepackage{rotating}
\usepackage[safe,extra]{tipa}
\usepackage{multirow}
\usepackage{hyperref}
\usepackage{enumerate}
\usepackage{braket}
\usepackage{marginnote}
\usepackage{pgfplots}
\usepackage{cancel}
\usepackage{polynom}
\usepackage{booktabs}
\usepackage{enumitem}
\usepackage{algorithm}
\usepackage{algpseudocode}
\usepackage{framed}
\usepackage{pdfpages}
\usepackage{pgfplots}
\usepackage{fancyhdr}
\usepackage{caption}
\usepackage{subcaption}
\usepackage{setspace}
\usepackage{hyperref}
\pagestyle{fancy}
\fancyhead[L,RO]{\slshape \rightmark}
\fancyfoot[C]{\thepage}

\title{Causal Networks}
\author{Tommaso Ferrario (\href{https://github.com/TommasoFerrario18}{@TommasoFerrario18})}
\date{Ottobre 2024}

\pgfplotsset{compat=1.13}

\begin{document}

\maketitle
\newtheorem{teorema}{Teorema}
\newtheorem{dimostrazione}{Dimostrazione}
\newtheorem{definition}{Definition}
\newtheorem{esempio}{Esempio}
\newtheorem{osservazione}{Osservazione}
\newtheorem{note}{Note}
\newtheorem{corollario}{Corollario}
\tableofcontents
\renewcommand{\chaptermark}[1]{
    \markboth{\chaptername
        \ \thechapter.\ #1}{}}
\renewcommand{\sectionmark}[1]{\markright{\thesection.\ #1}}

\chapter{Potential outcomes}
In the following course we will use the following notation:
\begin{itemize}
    \item $X$ to denote the random variable representing the \textbf{treatment}
          assignment.
    \item $Y$ to denote the random variable representing the \textbf{outcome} of
          interest.
    \item $Z$ to denote a set of random variables representing \textbf{covariates}.
\end{itemize}

Also, we will use uppercase letters to denote random variables and lowercase
letters to denote their realizations.

\begin{definition}[Potential outcomes]
    $Y(x)$ denotes what your outcome would be, if you were to take treatment $X = x$.
\end{definition}
A potential outcome $Y(x)$ is distinct from the observed outcome $Y$ in that not
all potential outcomes are observed. But, all potential outcomes can potentially
be observed.

The actually observed potential outcome depends on the given value $x$ of treatment $X$.

Up until now, we have been considering an individual. However, the population
consists of many individuals. Each individual is tipically associated with one or
more variables, referred to as covariates $Z$. We denote each individual using $i$
as as subscript.

\begin{definition}[Individual treatment effect]
    The individual treatment effect (ITE) for the $i^{th}$ individual
    is defined as the difference between the potential outcomes:
    \begin{equation}
        \tau_i \triangleq Y_i(1) - Y_i(0)
    \end{equation}

    The main different is that $Y(x)$ is a random variable because different individuals
    have different potential outcomes. Meanwhile, $Y_i(x)$ is a treated as a
    non-random variable because it is the potential outcome is deterministic.
\end{definition}

\begin{nota}
    If the individual treatment effect is different from zero, we can call it
    the \textbf{causal effect} of the treatment. Otherwise, we can call it the
    \textbf{no causal effect}.
\end{nota}

\section{Fundamental problem of causal inference}
As we said before, it's impossible to observe all potential outcomes for a given
individual. Therefore, we can't observe the causal effect:
\begin{equation*}
    \tau_i \triangleq Y_i(1) - Y_i(0)
\end{equation*}

The potential outcome that you don't observe are known as \textbf{counterfactuals}.
So, a potential outcome $Y(x)$ doesn't become counterfactual until another
potential outcome $Y(x')$ is observed.

\begin{nota}
    There are no counterfactuals or factuals until the outcome is observed.
\end{nota}
\begin{definition}
    The \textbf{average treatment effect} (ATE) is obtained by taking an average
    over the individual treatment effects:
    \begin{equation}
        \tau \triangleq \mathbb{E}[\tau_i] = \mathbb{E}[Y_i(1) - Y_i(0)] = \mathbb{E}[Y(1) - Y(0)]
    \end{equation}
    where we recall that the average is over the individuals $i$ if $Y_i(x)$ is
    deterministic.
\end{definition}

The fundamental problem can be see as a missing data problem. Therefore, we cannot
compute directly the average treatment effect. We could be tempted to use the
associational difference:
\begin{equation}
    \mathbb{E}[Y|X = 1] - \mathbb{E}[Y|X = 0]
\end{equation}

Unfortunately, this is not true in general to compute the average treatment effect.
Because $\mathbb{E}[Y|X = 1] - \mathbb{E}[Y|X = 0]$ is an associational quantity,
while $\mathbb{E}[Y(1)] - \mathbb{E}[Y(0)]$ is a causal quantity.

In general, they are not equal due to the \textbf{confounding} effect of the covariates
$Z$. For example, using the representation in figure \ref{fig:confounding} we can
say that the covariate $Z$ confounds the effect of $X$ on $Y$, because of the
following path:
\begin{equation*}
    X \rightarrow Y \leftarrow Z
\end{equation*}

\begin{figure}[!ht]
    \centering
    \includegraphics[width=0.35\textwidth]{img/confounding.png}
    \caption{Confounding effect of the covariate $Z$}
    \label{fig:confounding}
\end{figure}

We can consider average treatment effect equal to the associational difference
assuming the \textbf{ignorability} assumption.
\begin{definition}[Ignorability]
    The \textbf{ignorability} assumption states that the treatment assignment
    is random and independent of the potential outcomes:
    \begin{equation}
        Y(0), Y(1) \perp X
    \end{equation}
\end{definition}

Another name for this property is \textbf{exchangeability} because it can be
interpreted as we can exchange the groups can be exchanged without changing the
distribution of the potential outcomes.

We can distinguish between two types of ignorability: Mean ignorability and Full
ignorability. In general, mean ignorability is enough to compute the average
treatment effect.

This property is fundamental because it allows us to compute the average treatment
effect as the associational difference:
\begin{equation*}
    \tau \triangleq \mathbb{E}[Y(1) - Y(0)] = \mathbb{E}[Y|X = 1] - \mathbb{E}[Y|X = 0]
\end{equation*}

The assumption of ignorability allow us to identify causal effects. This can be
done reducing a causal expression to a purely statistical expression. We can
calculate the causal effect from just the observational distribution $P(X, Y, Z)$.

\begin{definition}[Identifiability]
    A causal quantity is identifiable if we can compute it from a purely statistical
    quantity.
\end{definition}

Unfortunately, the assumption of ignorability is completely unrealistic, confounding
is likely to happen in most data we observe. We can make the assumption of ignorability
more realistic by performing a \textbf{randomized experiment}.

In observational data, it's unrealistic to assume that the groups are exchangeable.
However, if we control for relevant variables by conditioning, then maybe the groups
will be exchangeable.

\begin{definition}[Conditional exchangeability]
    The \textbf{conditional exchangeability} or \textbf{unconfoundeness} assumption
    states that the treatment assignment is the potential outcomes conditional
    on the covariates:
    \begin{equation}
        Y(0), Y(1) \perp X | Z
    \end{equation}
\end{definition}

Indeed, when conditioning on $Z$, non-causal associations between $X$ and $Y$ no
longer exists. Non-causal association is blocked by conditioning on $Z$.

This is the main assumption necessary for causal inference. We can now identify
the causal effect within levels of $Z$, just like we did with ignorability:
\begin{equation}
    \mathbb{E}[Y(1) - Y(0) | Z] = \mathbb{E}[Y(1)| Z] - \mathbb{E}[Y(0)| Z] =
    \mathbb{E}[Y|X = 1, Z] - \mathbb{E}[Y|X = 0, Z]
\end{equation}

If we want the marginal effect that we had before when assuming ignorability, we
can get that by simply marginalizing over $Z$ as follows:
\begin{equation}
    \mathbb{E}[Y(1) - Y(0)] = \mathbb{E}_Z[\mathbb{E}[Y(1) - Y(0) | Z]]
\end{equation}

\begin{definition}[Adjustment formula]
    Given the assumption of unconfoundeness, positivity, consistency, and no interference,
    we can identify the average treatment effect as:
    \begin{equation}
        \tau = \mathbb{E}[Y(1) - Y(0)] = \mathbb{E}_Z[\mathbb{E}[Y(1) - Y(0) | Z]]
        = \mathbb{E}_Z[\mathbb{E}[Y|X = 1, Z] - \mathbb{E}[Y|X = 0, Z]]
    \end{equation}
\end{definition}

There may be some unobserved confounders that we can't control, this mean that
the assumption of unconfoundeness is violated. The best we can do is to observe
and fit as many covariates as possible to try to ensure unconfoundeness.

Indeed, it can be the case we obtain more biased estimates by including when
adjusting for the "wrong" covariates.

\begin{definition}[Positivity]
    Positivity is the condition that all the subgroups af the data with different
    value $z$ for covariates $Z$ have some probability of receiving treatment $X$.

    For all values $z$ of covariates $Z$ present in the population of interest,
    we have:
    \begin{equation}
        0 < P(X = 1 | Z = z) < 1
    \end{equation}
\end{definition}

If we have positivity violation, then we will be conditioning on a zero probability
event. This will lead to biased estimates.

Positivity is also referred to as the \textbf{overlap} assumption. This is because
we want the covariates distribution of the group to overlap.

\begin{figure}[!ht]
    \centering
    \includegraphics[width=\textwidth]{img/overlap.png}
    \caption{Example of overlap}
    \label{fig:positivity}
\end{figure}

Adjusting on more covariates can lead to curse of dimensionality. 

Another important assumption is the \textbf{no interference} assumption. This
\begin{definition}[No interference]
    The outcome $Y_i$ of the $i^{th}$ individual is not affected by anyone else's
    treatment $X_j$ for $j \neq i$.
    \begin{equation}
        Y_i(x_1, x_2, \ldots, x_i, \ldots, x_n) = Y_i(x_i)
    \end{equation}
\end{definition}

The last assumption is the \textbf{consistency} assumption.
\begin{definition}[Consistency]
    If the treatment is $X$, then the observed outcome $Y$ is the potential outcome
    under treatment $X$. Formally:
    \begin{equation}
        X = x \rightarrow Y = Y(X)
    \end{equation}
\end{definition}

\begin{definition}[Stable Unit Treatment Value Assumption]
    The \textbf{Stable Unit Treatment Value Assumption} (SUTVA) is satisfied if
    individual $i$'s outcome $Y_i$ is simply a function of the treatment $X_i$.
\end{definition}
\chapter{Flow of Association And Causation in Graphs}
\section{Bayesian Networks and Causal Graphs}
Assume we only care about modeling association, without any causal modeling. If
we want to model the data distribution $P(X_1, X_2, \ldots, X_n)$, we can use
the chain rule of probability to decompose it into a product of conditional
distributions:
\begin{equation}
    P(X_1, X_2, \ldots, X_n) = P(X_1) \cdot P(X_2|X_1) \cdot \ldots \cdot P(X_n|X_1,
    \ldots, X_{n-1}) = P(X_1) \cdot \prod_{i=2}^n P(X_i|X_1, \ldots, X_{i-1})
\end{equation}
However, if we were to model discrete random variables by using probability
tables, it would take an exponential number of parameters. To solve this we
can model local dependencies between variables to reduce the number of
parameters.

\begin{definition}[\textbf{Local Markov Assumption}]
    Given all the parents $pa(x)$ of a node $x$ in a DAG, the local Markov
    assumption states that $x$ is independent of all its non-descendants.
\end{definition}
Talking about Bayesian networks, the local Markov assumption is equivalent to
the \textbf{bayesian network factorization}:
\begin{equation}
    P(X_1, X_2, \ldots, X_n) = \prod_{i=1}^n P(X_i|pa(X_i))
\end{equation}

At this point we can define a \textbf{Markov Probability Distribution} as a
probability distribution $P$ that satisfies the local Markov assumption with
respect to a DAG $G$.

As important as the local Markov assumption is, it only gives us information
about the independencies in $P$ that a DAG $G$ implies.

To get this guaranteed dependence between adjacent nodes, we will generally
assume a slightly stronger assumption than the local Markov assumption, called
the \textbf{minimality assumption}.
\begin{definition}[\textbf{Minimality Assumption}]
    The minimality assumption means:
    \begin{enumerate}
        \item \textbf{local markov assumption}: Given all the parents $pa(x)$ of
              a node $x$ in a DAG, the local Markov assumption states that $x$
              is independent of all its non-descendants.
        \item Adjacent nodes in the DAG $G$ are dependent.
    \end{enumerate}
\end{definition}

Because removing edges in a Bayesian network is equivalent to adding independencies,
the minimality assumption is equivalent to saying that we can't remove any more
edges from the graph $G$.

\begin{definition}
    $P$ and $G$ are Markov Compatible if $P$ factorize according to $G$.
\end{definition}

Up to now all we presented was about statistical models and modeling association.
We now need to introduce some \textit{causal assumptions}, turn them into causal
models for allowing the study of causation. In order to introduce causal assumptions,
we must first understand what it means for $X$ to be a cause of $Y$. We can simply
define \textbf{cause} as follows:
\begin{definition}[\textbf{Cause}]
    $X$ is a cause of $Y$ if changing $X$ changes $Y$.
\end{definition}

Also, we can define \textbf{(strict) causal edges assumption} in a DAG $G$ as
every parent is a direct cause of all its children. Given this assumption, we can
define a \textbf{Causal graph} as a DAG $G$ that satisfies the strict causal edges
assumption.

Adding the causal edges assumption, implies that directed paths in the DAG take
on a very special meaning; they correspond to \textbf{causation}. This is in
contrast to other paths in the graph, which association may flow along, but
causation certainly may not.
\section{Chains, Forks, and Colliders}
To understand the difference between association flow and causal flow in DAGs,
we need the following minimal building blocks:
\begin{itemize}
    \item Two un-connected nodes;
    \item Two connected nodes;
    \item Chain;
    \item Fork;
    \item Collider.
\end{itemize}

By \textit{flow of association}, we mean whether any two nodes in a graph are
associated or not associated. In other terms, we want to know whether two nodes
are (statistically) dependent or (statistically) independent. However, we will
also study whether two nodes are conditionally independent or not.
\subsection{Two un-connected nodes}
Given a graph consisting of just two unconnected nodes, as reported in Figure~\ref{fig:two_unconnected_nodes},
these nodes are not associated, because there is no edge between them. To show
this, consider the factorization of the joint probability $P(X, Y)$ that the
Bayesian network factorization gives us: $P(X, Y) = P(X)P(Y)$. This factorization
immediately gives us a proof that the two nodes are unassociated (independent).

\begin{figure}[!ht]
    \centering
    \includegraphics[width=0.2\textwidth]{img/flow/two_unconnected_nodes.png}
    \caption{Two unconnected nodes}
    \label{fig:two_unconnected_nodes}
\end{figure}
\subsection{Two connected nodes}
On the contrary, if there is an edge between the two nodes, as reported in Figure~\ref{fig:two_connected_nodes},
then the two nodes are associated. We exploit the causal edges assumption which
means that $X$ is a direct cause of $Y$. They are dependent and conditional dependent.

\begin{figure}[!ht]
    \centering
    \includegraphics[width=0.2\textwidth]{img/flow/two_connected_nodes.png}
    \caption{Two connected nodes}
    \label{fig:two_connected_nodes}
\end{figure}

\subsection{Chain and Fork}
We can consider this two building blocks together, because they share the same
set of dependencies. In both cases, the association flows from $X$ to $Z$ through
$Y$.

They also share the same set of independencies. When we condition on $Y$ in both
graphs, it blocks the flow of association from $X$ to $Y$. Therefore, when we
condition on $Y$ ( $Z$'s parent in both graphs), $Z$ becomes independent of (and
viceversa). This independence is an instance of a \textbf{blocked path}.

\begin{figure}[!ht]
    \centering
    \begin{subfigure}[b]{0.45\textwidth}
        \includegraphics[width=\textwidth]{img/flow/chain.png}
        \caption{Chain}
        \label{fig:chain}
    \end{subfigure}
    \hfill
    \begin{subfigure}[b]{0.45\textwidth}
        \includegraphics[scale=0.5]{img/flow/fork.png}
        \caption{Fork}
        \label{fig:fork}
    \end{subfigure}
\end{figure}

\begin{note}
    The flow of association in general is symmetric, but the flow of causation
    is not.
\end{note}
\subsection{Collider}
Association flows along any path that does not contain a \textbf{collider}. A
collider is a node with two or more parents that are independent $X \perp Z$, as
shown in Figure~\ref{fig:collider}. We can think of $X$ and $Z$ simply as unrelated
events that can happen, and which both contribute to some common effect ($Y$).
To show that $X \perp Z$ we apply the Bayesian network factorization and then
marginalize out $Y$.

\begin{figure}[!ht]
    \centering
    \includegraphics[width=0.2\textwidth]{img/flow/collider.png}
    \caption{Collider}
    \label{fig:collider}
\end{figure}

The collider $Y$ blocks the path from node $X$ to node $Z$ and blocks the path
from node $Z$ to node $X$. This is another example of a blocked path, but this
time the path is not blocked by conditioning; the path is un-blocked by conditioning
on a collider.

Conditioning on descendants of a collider also induces association in between the
parents of the collider. The intuition is that if we learn something about a
collider's descendant, we usually also learn something about the collider itself
because there is a direct causal path from the collider to its descendants, and
we know that nodes in a chain are usually associated assuming minimality.
\section{d-separation}
\begin{definition}[\textbf{d-separation}]
    A path $p$ is blocked by a set of nodes $\mathbb{S}$ if and only if:
    \begin{itemize}
        \item $p$ contains a \textit{chain} of nodes or a \textit{fork} such that
              the middle node is in $\mathbb{S}$.
        \item $p$ contains a \textit{collider} such that the collision node is
              not in $\mathbb{S}$, and no descendant of the collision node is in
              $\mathbb{S}$.
    \end{itemize}
    If $\mathbb{S}$ blocks \textbf{every} path between two nodes $X$ and $Y$, then
    $X$ and $Y$ are \textbf{d-separated}, conditional on $\mathbb{S}$, and thus
    are independent conditional on $\mathbb{S}$.
\end{definition}
D-separation is an extremely important concept, because it implies conditional
independence. This is a very powerful tool for reasoning about causal models.
\begin{itemize}
    \item $X \perp_G Y | \mathbb{S}$ implies that $X$ and $Y$ are d-separated in the DAG
          $G$ when conditioning on $\mathbb{S}$.
    \item $X \perp_P Y | S$ implies that $X$ and $Y$ are independent in the
          distribution $P$ when conditioning on $\mathbb{S}$.
\end{itemize}
\begin{definition}[\textbf{Global Markov Assumption}]
    Given that $P$ is Markov with respect to a DAG $G$, if $X$ and $Y$ are d-separated
    in $G$ conditional on $\mathbb{S}$, then $X$ and $Y$ are independent conditional
    on $\mathbb{S}$.
\end{definition}

Not only is association not causation, but causation is a sub-category of association,
thus association and causation both flow along directed paths.

We can tell if two nodes are not associated by whether or not they are d-separated.

If we want to measure the causal effect of $X$ on $Y$, we need to ensure that $X$
and $Y$ are d-separated in the augmented graph where we remove outgoing edges
from $X$. This is the only way to ensure that the association we measure is causation.

\begin{note}
    Association is symmetric while causation not.
\end{note}

To summarize the assumptions we made:
\begin{itemize}
    \item local/global Markov Assumption: tells us which nodes are unassociated,
          tells along which paths the association doesn't flow.
    \item Minimality assumption: tells us which paths association does flow along
    \item Causal edges assumption: tells us causation flows along directed paths.
\end{itemize}
\begin{figure}[!h]
    \centering
    \includegraphics*[width=\textwidth]{img/flow/causal_networks_assumptions.png}
    \caption{Causal networks assumptions}
    \label{fig:causal_networks_assumptions}
\end{figure}


\chapter{Causal Models}
As you have undoubtedly heard many times in statistics classes, “correlation is
not causation.” A mere association between two variables does not necessarily
mean that one of those variables causes the other. For this reason, the
\textbf{randomized controlled experiment} is considered the standard of statistics.

We can define a randomized controlled experiment as an experiment in which all
the \textbf{factors} that influence the outcome of the experiment are either
static, or vary at random, except for one. This implies that any change in the
outcome variable must be due to that one input variable.

In cases where randomized controlled experiments are not practical, researchers
instead perform \textbf{observational studies}, in which they merely record data,
rather than controlling it. In these cases, the problem is that is difficult to
untangle the effects of different variables.

We can summarize the difference between these two types of studies as follows:
\begin{itemize}
      \item \textbf{Intervening}: we change the system assigning values to the a
            variable and observing the effect on the other variables. When we
            intervene to fix the value of a variable, we curtail the natural
            tendency of that variable to vary in response to other variables in
            nature. This amounts to performing a surgery on the graphical model,
            which we do by removing all edges directed into that variable.
            Intervening would be to take the whole population and give everyone
            treatment.
      \item \textbf{Conditioning}: we merely narrow our focus to the subset of
            cases in which the variable takes the value we are interested in.
            So, conditioning on $X = x$ just means that we are restricting our
            focus to the subset of the population to those who received treatment.
\end{itemize}
We denote intervention with the \textbf{do-operator} $do(X=x)$. So, if we have
an expression which contains the do-operator, we know that we are intervening
on that variable and we call this expression a \textbf{interventional expression}.

An interventional expression which can be reduced to an observational expression
is said to be \textbf{identifiable}. This means that we can estimate the effect
of the intervention from the observational data.

Whenever, $do(x)$ appears in expression after the conditioning bar, it means
that everything in that expression is in the \textbf{post-intervention world}
where intervention $do(x)$ occurs.
\section{Adjustment Formula}
We can define a \textbf{causal mechanism} as a mechanism that generates $X_i$ as
the conditional distribution of $X_i$ given its parents (causes) $pa(X_i)$.

Also, we want to show that intervention are local. This means that intervening on
a variable $X_i$ only changes the causal mechanism for $X_i$; it does not change
the causal mechanisms that generate any other variables $X_j$.
\begin{definition}[\textbf{Modularity of Causal Models}]
      If we intervene on a set of nodes $S$, setting them to constants, then for
      all $X_i \in \{X_1, \ldots, X_n\}$ we have the following:
      \begin{itemize}
            \item If $X_i \notin S$ then the causal mechanism that generates $X_i$
                  is unchanged by the intervention.
            \item If $X_i \in S$ then the causal mechanism that generates $X_i$ is
                  replaced by a constant. In other words, $P(X_i | pa(X_i)) = 1$
                  if $x$ is the value that $X_i$ is set to by the intervention
                  $do(X_i = x)$. Otherwise, we have $P(X_i | pa(X_i)) = 0$.
      \end{itemize}
\end{definition}
\begin{note}
      The causal graph for interventional (experimental) distributions is simply
      the same graph that was used for the observational joint distribution, but
      with all of the edges to the intervened node(s) removed.
\end{note}
Using do-expressions and graph surgery, we can begin to untangle the causal
relationships from the purely associative.
\begin{note}
      It is worth noting here that we are making a tacit assumption that
      the \textbf{intervention} has no side effects.
\end{note}

The intervention procedure, which led to the \textbf{Adjustment Formula},
dictates that $Z$ should coincide with the parents $pa(X)$ of $X$, because it is
the influence of these parents that we neutralize when we fix $X$ by external
manipulation $do(X)$.

We can therefore write a general Adjustment Formula and summarize it in a rule:
\begin{definition}[\textbf{Causal effect rule}]
      Given a graph $G$ in which a set of variables $pa(X)$ are designed as the
      parents of $X$, the \textbf{causal effect} of $X$ on $Y$ can be computed
      as follows:
      \begin{equation}
            P(Y = y| do(X = x)) = \sum_{z} P(Y | X = x, pa(X) = u)P(pa(X) = u)
      \end{equation}
      where $u$ ranges over all the combinations of values that the variables in
      $pa(X)$ can take.
\end{definition}
If we apply some manipulation on the formula, we can obtain a more convenient
form:
\begin{equation}
      P(Y = y| do(X = x)) = \frac{\sum_{z} P(Y = y, X = x, pa(X) = u)P(pa(X) = u)}{P(X = x | pa(X) = u)}
\end{equation}
In the equation above, the denominator represents the \textbf{propensity score}
which displays the role played by the parents $pa(X)$ of $X$ in determining the
result of the intervention $do(X = x)$.
\section{Truncated Factorization}
In some circumstances, we can involve multiple interventions in the same time.

The previous consideration also allows us to generalize the Adjustment Formula to
\textbf{multiple intervansions}, that is, interventions that fix the values of a
set of variables $S$ to constants $s$. We simply write down the Factorization of
the pre-intervention distribution and strike out all factors that correspond to
variables in the intervention set $S$.

\begin{definition}[\textbf{Truncated Factorization}]
      We assume that $P$ and $G$ satisfy the Markov assumption and modularity.
      Given, a set of intervention nodes $S$, if $x_i$ is consistent with the
      intervention $S = s$, then:
      \begin{equation}
            P(x_1, x_2, \dots, x_n| do(S = s)) = \prod_{i = 1}^{n} P(x_i | pa(x_i))
      \end{equation}
      otherwise $P(x_1, x_2, \dots, x_n| do(S = s)) = 0$.
\end{definition}
\begin{definition}[\textbf{Bayesian Network Factorization}]
      Given a probability distribution $P$ and a graph $G$, $P$ factorizes
      according to $G$ if:
      \begin{equation}
            P(x_1, x_2, \dots, x_n) = \prod_{i = 1}^{n} P(x_i | pa(x_i))
      \end{equation}
\end{definition}
Often, we know, that the variables have \textbf{unmeasured parents}, also known as
\textbf{latent}, that, though represented in the graph, may be inaccessible for
measurement. In those case, we need to find an alternative set of variables
to adjust for.

\begin{center}
      Under what conditions, is the structure of the causal graph sufficient for
      computing a causal effect from a given data set?
\end{center}

One of the most important tools we use to determine whether we can compute a
causal effect is a simple test called the \textbf{backdoor criterion}. Using it,
we can determine, for any two variables $X$ and $Y$ in a causal model represented
by a DAG $G$, which set of variables $S$ in that model should be conditioned on
when searching for the causal relationship between $X$ and $Y$.

\begin{definition}[\textbf{Backdoor criterion}]
      Given an ordered pair of variables $(X, Y)$ in a DAG $G$, a set of variables
      $S$ satisfies the backdoor criterion relative to $(X, Y)$ if no nodes
      in $S$ are descendants of $X$ and $S$ blocks all backdoor paths between $X$
      and $Y$ that contain an arrow into $X$.
\end{definition}
If a set of variables $S$ satisfies the backdoor criterion relative for $X$ and
$Y$, then the causal effect of $X$ on $Y$ is given by the following formula:
\begin{equation}
      P(Y = y| do(X = x)) = \sum_{s} P(Y = y| X = x, S = s)P(S = s)
\end{equation}
just as when we adjust for the parents of $X$ in the Adjustment Formula.

In general, we would like to condition on a set of nodes $S$ such that we:
\begin{itemize}
      \item Block all the spurious paths between $X$ and $Y$. We want the
            conditioning set $S$ to block any \textbf{backdoor path} in which
            one end has an arrow into $X$, because such paths may make $X$ and
            $Y$ dependent.
      \item Leave all directed paths from $X$ to $Y$ unperterbed. We don't want
            to condition on any nodes that are descendants of $X$.
      \item Create no spurious paths
\end{itemize}
\begin{figure}[!ht]
      \centering
      \includegraphics[width=\textwidth]{img/backdoor.png}
      \caption{Backdoor paths}
      \label{fig:backdoor}
\end{figure}
\begin{definition}[\textbf{Backdoor Adjustment Formula}]
      Give the modularity assumption, that, $S$ satisfies the backdoor criterion,
      and positivity, we can identify the causal effect of $X$ on $Y$ as follows:
      \begin{equation}
            P(Y = y| do(X = x)) = \sum_{s} P(Y = y| X = x, S = s)P(S = s)
      \end{equation}
\end{definition}
We can use the backdoor adjustment formula if, $S$ d-separates $X$ from $Y$ in
the augmented graph obtained by removing all outgoing edges from $X$. 

We would be able to isolate the causal association if $X$ is d-separated from 
$Y$ in the augmented graph.
\begin{center}
      \textbf{Isolation of the causal association is identification}
\end{center}

We can also isolate the causal association if $X$ is d-separated from $Y$ in
the augmented graph, conditional on $S$. This is what the first part of the 
backdoor criterion is about and what we've codified in the backdoor adjustment.
\chapter{Structural Causal Model}
\section{Structural Causal Models}
Graphical causal models such as causal Bayesian networks give us powerful ways to
encode statistical and causal assumptions, but we have yet to explain them exactly
what an \textit{intervention} is or exactly what a \textit{causal mechanism} is.

Moving from causal Bayesian networks to \textbf{full Structural causal models}
will give us this additional clarity along with the power to compute \textbf{counterfactuals}.

We need a way to specify the following concept:
\begin{center}
    $A$ causes $B$
\end{center}
this means that changing $A$ changing also $B$ but changing $B$ doesn't mean changing $A$.

We use the notation $B := f(A)$, which means that we assign to $B$ the value of
$f(A)$. This symbol ($:=$) denotes an \textbf{asymmetric assignment} and differs
from the symmetric equality symbol ($=$).

\begin{note}
    Using $B := f(A)$ we are using a \textbf{causal model}, while $B = f(A)$ we
    are using a \textbf{statistical model}
\end{note}

This operator ($:=$) is deterministic. Ideally, we'd like to allow it to be
probabilistic, which allows room for some unknown causes ($U$) of $B$ that factor
into this mapping. Then, we write the \textbf{structural equation} as:
\begin{equation*}
    B := f(A, U)
\end{equation*}
This means that the function $f$ is deterministic (is a stochastic mapping)
applied on $A$ that we know with some noise $U$ that models the unknown causes.
An example of this is reported in Figure~\ref{fig:unknown_graphs}.

\begin{figure}[!ht]
    \centering
    \includegraphics[width=0.5\textwidth]{img/structural_causal_model/unknown_graphs.png}
    \caption{Graphical representation of the structural equation}
    \label{fig:unknown_graphs}
\end{figure}

When $f$ is not specified we are in the \textbf{nonparametric regime} because we
aren't making any assumptions about parametric form. However, \textit{structural
    equations} can represent any stochastic mapping, so generalize the
\textit{probabilistic factor}:
\begin{equation*}
    P(X_i | pa(X_i))
\end{equation*}

Therefore, all the results that we've seen such as the truncated factorization
and the backdoor adjustment still holds when we introduce structural equations.

We have now come to the more precise definitions of what a cause is and what a
causal mechanism is.

\begin{definition}[\textbf{Causal Mechanism}]
    A causal mechanism that generates a variable is the structural equation
    that corresponds to that variable.
\end{definition}
\begin{note}
    The cause is more general than the directed cause.
\end{note}
From the initial model $B := f(A, U)$ the $A, U$ are directed causes for $B$.

In general, a model can consist of more than one structural equation, for example:
\begin{equation}
    M : \,\, \begin{array}{l}
        B := f_B(A, U_B) \\ C := f_C(A, B, U_c) \\ D := f_D(A, C, U_D)
    \end{array}
\end{equation}
Starting from this model we can build the associated causal graph, where we draw
an edge from every variable on the right-hand side to the variable on the left-hand side.
The associate causal graph is reported in Figure~\ref{fig:associate_graph}.
\begin{figure}[!ht]
    \centering
    \includegraphics[width=0.35\textwidth]{img/structural_causal_model/graph_associate_structural_equation.png}
    \caption{Associate causal graph}
    \label{fig:associate_graph}
\end{figure}

Based on the previous example, we can distinguish $A, U_B, U_C, U_D$ as \textbf{
    exogenous variables}, i.e., they can not be descendant of any other variables,
and in particular, they can not be descendant of an endogenous variable; they have
no ancestors and are represented as root nodes in graphs.

\begin{definition}[\textbf{Exogenous Variables}]
    The exogenous variables are the variables for which we don't have a structural
    equation that explains their behavior, so they are the root nodes in the graph.
\end{definition}

They are external to the model; we chose, for whatever reason, not to explain how
they are caused (we don't care about cause of cause).

While, $B, C, D$ are \textbf{endogenous variables}, i.e., the variables that we
write structural equations for, i.e., the variables whose causal mechanisms we
are modeling.

\begin{definition}[\textbf{Endogenous Variables}]
    The endogenous variables are the variables for which we have structural
    equations that define their behavior, so they have parents in the graph.
\end{definition}

If we know the value of every \textbf{exogenous variable}, then using the functions
of the structural equations, we can determine with perfect certainty the value
of every \textbf{endogenous variable}.

\begin{definition}[\textbf{Structural Causal Model}]
    A \textbf{structural causal model} (SCM) is a tuple $M = \langle U, V, F \rangle$
    of the following sets:
    \begin{itemize}
        \item $U$ is a set of exogenous variables.
        \item $V$ is a set of endogenous variables.
        \item $F$ is a set of functions, one to generate each endogenous variable
              as a function of other variables.
    \end{itemize}
\end{definition}

Every structural causal model implies an associated \textbf{causal graph}: for each
structural equation, draw an edge from every variable on the right-hand side to the
variable on the left-hand side.

If we have a causal graph contains no cycles (it is a DAG) and the \textbf{noise
    variable} $U$ are:
\begin{itemize}
    \item \textbf{Independent} so the causal model is \textbf{Markovian}
    \item \textbf{Dependent} so the causal model is \textbf{semi-Markovian}
\end{itemize}

\begin{note}
    If there is a \textbf{unobserved confounding} the model is \textbf{semi-Markovian}. And
    the graph of \textbf{non-Markovian} models contains cycles.
\end{note}

\textbf{Interventions} in SCMs consist of replacing the structural equation, i.e.,
if the intervention is $do(X = x)$ then we replace the structural equation associated
with $X$. For example:
\begin{equation}
    M : \begin{array}{l}
        Z := U_Z \\ X := f_X(Z, U_X) \\ Y:= f_Y(X, Z, U_Y)
    \end{array}
\end{equation}
The graph associated is Markovian. If we do the following intervention as $do(X = x)$
so the structural equation will be:
\begin{equation}
    M_X : \begin{array}{l}
        Z := U_Z \\ X := x \\ Y:= f_Y(X, Z, U_Y)
    \end{array}
\end{equation}
and the graph will be modified removing the edge between $U_X$ and $X$. The resulting
graph is reported in Figure~\ref{fig:intervention_graph}.
\begin{figure}[!ht]
    \centering
    \includegraphics[width=0.5\textwidth]{img/structural_causal_model/intervention_on_structural_eq.png}
    \caption{Intervention on structural equation}
    \label{fig:intervention_graph}
\end{figure}

Only changes the equation for $X$ and no other variables is a consequence of the
\textbf{modularity assumption}; these causal mechanisms are modular.

\begin{definition}[\textbf{Modularity assumption for SCM}]
    Consider an SCM $M = \langle U, V, F \rangle$ and an interventional SCM
    $M_x =\langle U, V, F_x \rangle$, that we get by performing the intervention
    $do(X = x)$.

    The modularity assumption states that $M$ and $M_x$ share all of their
    structural equations except the structural equation for $X$, which is
    $X := x$ in $M_x$.
\end{definition}
For the modularity, the intervention $do(X = x)$ is localized to $X$.

We can write unit-level potential outcomes as:
\begin{itemize}
    \item $Y_x (u)$ potential outcome that unit $u$ would observe if taking treatment
          $X = x$, given that the SCM is $M$.
    \item $Y_{M_x}(u)$ potential outcome that unit $u$ would observe if taking
          treatment $X = x$, given that the SCM is $M_x$ (in the manipulated model).
\end{itemize}

The \textbf{law of counterfactuals} gives us information about counterfactuals:
\begin{equation}
    Y_x(u) = Y_{M_x}(u)
\end{equation}

Given an SCM with enough details about it specified, we can compute
counterfactuals. This is extremely important because it is exactly what the
fundamental problem of causal inference tells us we cannot do.

Not only did we specify that the adjustment set blocks all backdoor paths from
$X$ to $Y$, but we also specified that does not contain any descendants of $X$,
otherwise, we have a selection bias effect.

There are two categories of things that could go wrong if we condition descendants
of $X$:
\begin{enumerate}
    \item We block the flow of causation from $X$ to $Y$;
    \item We induce non-causal association between $X$ and $Y$, we generate a
          the spurious path did not exist before.
\end{enumerate}

In the first case, if we have a chain between $X$ and $Y$ and we have at least a
node $L$ that is a \textbf{mediator} of the causal effect of $X$ on $Y$. So when
we observe $L$ we measure a $0$ causal effect between $X$ and $Y$ because it's
blocked by the observation on the mediator.

If we have a mediator with an observation and direct link from $X$ to $Y$ we
can measure the direct cause-effect of $X$ on $Y$ but not the un-direct.

In the second case, if we have a direct link between $X$ and $Y$ and a collider
on $L$ that is a descendant of $X$ and $Y$, if we condition on a descendant of $L$
that isn't a mediator, it could unblock a path from $X$ to $Y$ that was blocked
by a collider, so we are introducing a \textbf{collider bias}. The same holds
for any descendant of $L$. So we should only control the counterfactuals.

It is worthwhile to notice that we actually can condition some descendants of
$X$ without inducing non-causal associations between $X$ and $Y$. Conditioning
on descendants of $X$ that aren't on any causal paths from $X$ and $Y$ won't
induce bias.

Even outside of graphical causal models, this rule is often applied; it is usually described
as not conditioning on any \textbf{post treatment variables}.

Unfortunately, even if we only condition on \textbf{pre-treatment covariates}, we
can still induce \textbf{collider bias} if we condition on the collider. Doing
this opens up a backdoor path, along which non-causal association can flow. This
is known as $M$-Bias due to the $M$-Shape that this non-causal association flows
along when the graph is drawn with children below their parents.

\section{Application of Backdoor Adjustment}
In this section, we will show how to derive the \textbf{associational} quantity
$\mathbb{E}[Y|X = x]$ (it is the observational value obtained without intervention),
that will be compared to the \textbf{causal} quantity $\mathbb{E}[Y|do(x)]$. There
are situations where the causal quantity is the same as the associational.

If the treatment were binary, then we would just look at the difference between
the quantities with the respected value. However, if we consider \textbf{linear
    generative process}, thus quantities are:
\begin{equation}
    \frac{\partial\mathbb{E}[Y |x]}{\partial x } \, \land \, \frac{\partial\mathbb{E}[Y |do(x)]}{\partial x}
\end{equation}
this formulation gives us all the information about the treatment effect, regardless
of if treatment is continuous, binary, or multi-valued.

\begin{note}
    In this section,n we will assume \textbf{infinite} data so we can work with expectations.
    It's a strong assumption but helps explain the situation.
\end{note}

For the rest of this section, we will consider the example given in Figure~\ref{fig:example}.
This can also be represented as follows:
\begin{equation}
    \begin{array}{l}
        X := \alpha_1 Z \\ Y := \beta X + \alpha_2 Z
    \end{array}
\end{equation}
where $\beta$ is the causal effect of $X$ on $Y$ and $\alpha_1, \alpha_2, \beta \in \mathbb{R}$.
\begin{figure}[!ht]
    \centering
    \includegraphics[width=0.2\textwidth]{img/structural_causal_model/example.png}
    \caption{Basic example}
    \label{fig:example}
\end{figure}

Let's now prove that $ beta$ is the causal effect of $X$ on $Y$. Given the ordered
pair $(X, Y)$ and the set $S = \{Z\}$ which represents the \textbf{sufficient
    adjustment set} we have the following:
\begin{equation}
    \begin{array}{ll}
        \mathbb{E}[Y | do(x)] = \mathbb{E}_Z[\mathbb{E}[Y | X = x, Z]] & = \mathbb{E}_Z[\mathbb{E}[\beta X + \alpha_2 Z | X = x, Z]] \\
                                                                       & = \mathbb{E}_Z[\mathbb{E}[\beta x + \alpha_2 Z]]            \\
                                                                       & = \mathbb{E}_Z[\beta x + \alpha_2 Z]                        \\
                                                                       & = \beta x + \alpha_2\mathbb{E}[Z]
    \end{array}
\end{equation}
Then, we have the following:
\begin{equation}
    \frac{\partial \mathbb{E}[Y |do(x)]}{\partial x} = \frac{\partial [\beta x + \alpha_2\mathbb{E}[Z]]}{\partial x} = \beta
\end{equation}

While, when we are talking about \textbf{associational effect} from $X$ to $Y$ we have:
\begin{equation}
    \begin{array}{ll}
        \mathbb{E}[Y | x] = \mathbb{E}[Y | X = x] & = \mathbb{E}[\beta X + \alpha_2 Z | X = x] \\
                                                  & = \beta x + \mathbb{E}[\alpha_2 Z |X = x]  \\
                                                  & = \beta x + \alpha_2 \mathbb{E}[Z |X = x]  \\
                                                  & = \beta x + \frac{\alpha_2}{\alpha_1} x
    \end{array}
\end{equation}
Then, we have the following:
\begin{equation}
    \frac{\partial \mathbb{E}[Y |x]}{\partial x} = \frac{\partial[\beta x + \alpha_2 \mathbb{E}[Z |X = x]]}{\partial x} = \beta + \frac{\alpha_2}{\alpha_1}
\end{equation}
Therefore, it is clear that \textbf{associational effect} of  $X$ on $Y$ is
different from the \textbf{causal effect} of $X$ on $Y$.

In the potential outcomes framework, it is common to only condition on pre-treatment
covariates. This would prevent a practitioner who adheres to this rule from
conditioning on the collider. However, there is no reason that there can't be
pre-treatment colliders that induce $M$ bias.

Given, the \textbf{modularity assumption} and \textbf{markov assumption} and
\textbf{positivity}, if the backdoor criterion is satisfied in our assumed causal
graph, then we achieve \textbf{identification}.

\begin{note}
    Note that although the backdoor criterion is a sufficient condition for
    identification, it is not a necessary condition.
\end{note}
\begin{note}
    No interference assumption is commonly implicit in causal graphs because the $Y$
    usually only has a single node $X$ for treatment as a parent. In another way,
    consistency follows from the axioms of SCMs. On contrary the positivity is still
    an important assumption that we must make.
\end{note}


\chapter{Randomized Experiments}
\textbf{Randomized experiments} are noticeably different from observational studies.
In randomized experiments, the experimenter has complete control over the treatment
assignment mechanism.

This complete control over how treatment is chosen is what distinguishes randomized experiments from observational studies. In this simple experimental setup, the treatment
isn’t a function of covariates at all, it's assigned randomly. In observational studies, the treatment is almost always a function of some of the covariate(s).

This difference is key to whether or not confounding is present in our data.

In randomized experiments, \textit{association} = \textit{causation} because they guarantee that there is no confounding.
\begin{equation}
    \mathbb{E}[Y(1)] -  \mathbb{E}[Y(0)] = \mathbb{E}[Y| X = 1] - \mathbb{E}[Y | X = 0]
\end{equation}

Since the treatment and control groups would be the same, in all aspects, except for treatment $X$, any difference in the outcomes $Y$ of the treatment and control groups is due to the treatment $X$.
\begin{definition}[\textbf{Covariate balance}]
    We have \textbf{covariate balance} if the distribution of covaraiates $Z$ is the same across treatment groups.
    \begin{equation}
        P(Z|X = 1) = P(Z|X = 0)
    \end{equation}
\end{definition}

Randomization implies covariate balance, across all covariates $Z$, even unobserved ones. Intuitively, this is because the treatment is chosen at random, regardless of $Z$, so the treatment and control groups should look very similar.
This is confirmed by the fact that in randomaze experiment, we chose randomly how
assign treatment so we are removing dipendences between the confounding variables
and the treatment, so there aren't any backdoor path from $X$ to $Y$ and $S=\{\emptyset\}$
is a sufficient adjustment to block any association flowing from $X$ to $Y$ except for
causal one.

We will now prove that association is equal to causation in randomized experiments by proving that:
\begin{equation*}
    P(Y|do(x)) = P(Y|x)
\end{equation*}
For the proof, the main property we utilize is that covariate balance implies $Z$ and $X$ are independent.

First, let $S = \{\emptyset\}$ be a \textbf{sufficent adjustment set} that potentially contains unobserved variables. Such an \textbf{adjustment set} $S$ must exist because we allow it to contain any variables, observed or unobserved.

Then, from the backdoor adjustements we get:
\begin{equation}
    \begin{array}{ll}
        P(y|do(x)) & = \sum_s P(y|x, S = s)P(S = s)                \\
                   & = \sum_s \frac{P(y|x, s)P(x|s)P(s)}{P(x | s)} \\
                   & = \sum_s \frac{P(y, x, s)}{P(x | s)}          \\
                   & = \sum_s P(y, s| x)                           \\
                   & = P(y | x)
    \end{array}
\end{equation}

These means that all associations are causation.

\textbf{Exchangeability} gives us another perspective on why randomization makes causation equal to association. If the groups are exchangeable, we could exchange these groups, and the average outcomes would remain the same.

The final perspective that we'll look at to see why association is causation in randomized experiments is that of \textbf{graphical causal models}.

In regular observational data, there is almost always confounding. However, if we randomize  it doesn't depend on anything other than the output of a coin toss. Since the node has no incoming edges and thus no backdoor paths exist.
\chapter{Non-parametric Identification}
\section{Frontdoor Adjustment}
We have seen that the \textbf{backdoor criterion} is sufficient for identification
but not necessary. In other words, is it possible to get identifiability without
being able to block all backdoor paths? Also, if we have unobserved variables we
\textbf{cannot} block the backdoor path.

If we are only focusing on the data we can have multiple interpretations of them.
So we need a method that we can use for causal estimation.

This can be obtained by chaining together the partial effects to obtain the overall effect.
\begin{definition}[\textbf{Frontdoor Criterion}]
    A set of variables $\mathbf{S}$ is said to satisfy the front-door criterion
    relative to an ordered pair of variables $(X, Y)$ if:
    \begin{enumerate}
        \item $\mathbf{S}$ intercepts all directed paths from $X$ to $Y$;
        \item There is no unblocked backdoor path from $X$ to $\mathbf{S}$;
        \item All backdoor paths from $\mathbf{S}$ to $Y$ are blocked by $X$.
    \end{enumerate}
\end{definition}
The conditions are overly conservative; some of the backdoor paths excluded by
conditions (2) and (3) can be allowed provided they are blocked by some
variables.

There is a powerful symbolic machinery, called the \textbf{do-Calculus}, that allows
analysis of such intricate structures. The do-Calculus uncovers all causal
effects that can be identified from a given graph.

\begin{definition}[\textbf{Frontdoor Adjustment}]
    If $\mathbf{S}$ satisfies the \textbf{frontdoor criterion} relative to $(X, Y)$
    and if $P(x, s) > 0$, then the causal effect of $X$ on $Y$ is identifiable
    and is given by the formula:
    \begin{equation}
        P(y | do(x)) = \sum_{\mathbf{s}} P(\mathbf{s} | x) \sum_{x'} P(y | \mathbf{s}, x') \cdot P(x')
    \end{equation}
\end{definition}

Satisfying the backdoor criterion isn't necessary to identify causal effects. If
the front-door criterion is satisfied, that also gives us \textit{identifiability}.

The combination of the adjustment formula, the backdoor criterion, and the frontdoor
criterion covers numerous scenarios. It proves the enormous, even revelatory, power
that causal graphs have in not merely representing but discovering causal
information.
\section{Do-calculus}
We can \textbf{identify} a causal query even if the causal graph doesn't satisfy
backdoor and frontdoor criterion. We can use \textbf{do-calculus} to
identify any identifiable causal effect where there are multiple treatments and/or
multiple outcomes.

To introduce and discuss do-calculus we first need to give some more notation:
\begin{itemize}
    \item $\perp_\mathcal{G}$ is the d-separation in the causal graph $\mathcal{G}$;
    \item $\mathcal{G}_{\overline{X}}$ is the graph where all the incoming edges
          in $X$ have been removed.
    \item $\mathcal{G}_{\underline{X}}$ is the graph where all the outgoing edges
          from $X$ have been removed.
    \item $\mathcal{G}_{\overline{Y}\underline{X}}$ is the graph where all the
          incoming edges in $Y$ and all the outgoing edges from $X$ have been removed.
\end{itemize}
\begin{definition}[\textbf{Rules of do-calculus}]
    Given a causal graph $\mathcal{G}$, an associated distribution $P$, and
    disjoint set of variables $\mathbf{Y}, \mathbf{X}, \mathbf{Z}$ and $\mathbf{W}$,
    the following rules hold:
    \begin{itemize}
        \item \textbf{Rule 1}:
              \begin{equation}
                  P(\mathbf{y} | do(\mathbf{x}), \mathbf{z}, \mathbf{w}) =
                  P(\mathbf{y} | do(\mathbf{x}), \mathbf{w}) \, \text{ if } \, \mathbf{Y}
                  \perp_{\mathcal{G}_{\overline{\mathbf{X}}}} \mathbf{Z} | \mathbf{X}, \mathbf{W}
              \end{equation}
              This is true because if we remove $do(x)$ we obtain:
              \begin{equation*}
                  P(\mathbf{y} |\mathbf{z}, \mathbf{w}) = P(\mathbf{y} | \mathbf{w}) \, \text{ if } \, \mathbf{Y}
                  \perp_{\mathcal{G}_{\overline{\mathbf{X}}}} \mathbf{Z} | \mathbf{W}
              \end{equation*}
              We can see that is a d-separation under the global Markov assumption,
              because d-separation in the graph $\mathcal{G}$ implies conditional
              independence in $P$ between $\mathbf{Y}$ and $\mathbf{Z}$ giving
              $\mathbf{W}$, so the separating set is $\mathbf{W}$.

              This means that Rule 1 is simply a generalization of d-separation to
              interventional distributions;
        \item \textbf{Rule 2}:
              \begin{equation}
                  P(\mathbf{y} | do(\mathbf{x}), do(\mathbf{z}), \mathbf{w}) =
                  P(\mathbf{y} | do(\mathbf{x}), \mathbf{z}, \mathbf{w}) \,
                  \text{ if } \, \mathbf{Y} \perp_{\mathcal{G}_{\overline{\mathbf{X}}\underline{\mathbf{Z}}}} \mathbf{Z} | \mathbf{X}, \mathbf{W}
              \end{equation}
              Also in this case we can remove $do(x)$ and we obtain:
              \begin{equation*}
                  P(\mathbf{y} | do(\mathbf{z}), \mathbf{w}) = P(\mathbf{y} | \mathbf{z}, \mathbf{w}) \,
                  \text{ if } \, \mathbf{Y} \perp_{\mathcal{G}_{\overline{\mathbf{Z}}}} \mathbf{Z} | \mathbf{W}
              \end{equation*}
              This corresponds to apply the backdoor adjustment using the backdoor
              criterion. So the rule is the generalization of backdoor adjustment to
              ana interventional distribution;
        \item \textbf{Rule 3}:
              \begin{equation}
                  P(\mathbf{y} | do(\mathbf{x}), do(\mathbf{z}), \mathbf{w}) =
                  P(\mathbf{y} | do(\mathbf{x}), \mathbf{w}) \, \text{ if } \,
                  \mathbf{Y} \perp_{\mathcal{G}_{\overline{\mathbf{X}}\overline{\mathbf{Z}(\mathbf{W})}}} \mathbf{Z} | \mathbf{X}, \mathbf{W}
              \end{equation}
              where $\overline{\mathbf{Z}(\mathbf{W})}$ denotes the set of nodes
              of $\mathbf{Z}$ that aren't ancestor of any node of $\mathbf{W}$
              in $\mathcal{G}_{\overline{\mathbf{X}}}$.
    \end{itemize}
\end{definition}
We could ask whether there could exist causal estimands that are identifiable but
that can't be identified using only the rules of do-calculus. Fortunately, it
has been proved that do-calculus is \textbf{complete}, which means that these
three rules are sufficient to identify all identifiable causal estimands. Because
these proofs are constructive, they also admit algorithms that identify any causal
estimand in polynomial time.

Do-calculus tells us if we can identify a given causal estimand using only the
causal assumptions encoded in the causal graph. If we introduce more assumptions
about the distribution ex: linearity, we can identify more causal estimands, but
this is known as \textbf{parametric identification}.

\section{Determining identifiability from the graph}
We previously mentioned that do-calculus is complete, which means that three rules
are sufficient to identify all identifiable causal estimands, also that estimands
that aren't identifiable by just using the causal graph. However, it would be
much more satisfying to know whether a causal estimand is identifiable by simply
looking at the causal graph.

For example, the backdoor criterion and the frontdoor criterion gave us simple
ways to know for sure that a causal estimand is identifiable. However, plenty of
causal estimands are identifiable, even though the corresponding
causal graphs don't satisfy the backdoor or frontdoor criterion.

Tian and Pearl provide a relatively simple graphical criterion that is sufficient
for identifiability: the \textbf{Unconfounded Children Criterion}.

\begin{definition}[\textbf{Unconfounded Children Criterion}]
    This criterion is satisfied if it is possible to block all backdoor paths
    from the treatment variable $X$ to all of its children ($ch(X)$) that are
    ancestors ($an(Y)$) of $\mathbf{Y}$ with a \textbf{single} conditioning set $\mathbf{S}$.
\end{definition}

This criterion generalizes the backdoor criterion and the frontdoor criterion.
Like the backdoor criterion and the frontdoor criterion, \textbf{Unconfounded children
    criterion} is a sufficient condition for identifiability.

\begin{definition}[\textbf{Unconfounded Children Identifiability}]
    Let $\mathbf{Y}$ be the set of outcome variables and $X$ be a single variable.
    If the unconfounded children criterion and positivity are satisfied, then:
    \begin{equation}
        P(\mathbf{Y} = \mathbf{y} |do(x))
    \end{equation}
    is identifiable.
\end{definition}

If we can isolate all of the causal association flowing out of treatment $X$ along
directed paths to $\mathbf{Y}$, we have identifiability.
\begin{enumerate}
    \item First, consider that all of the causal associations from $X$ to $\mathbf{Y}$
          must flow through its children $ch(X)$.
    \item We can isolate this causal association if there is no confounding between
          $X$ and any of its children $ch(X)$.
    \item This isolation of all of the causal associations is what gives us
          identifiability of the causal effect of $X$ on any other node in the graph.
    \item This intuition might lead you to suspect that this criterion is necessary
          in the very specific case where the outcome set $\mathbf{Y}$ is all of
          the other variables in the graph other than $X$; it turns out that this
          is true. But this condition is not necessary if $\mathbf{Y}$ is a smaller
          set than that.
\end{enumerate}

In conclusion, the unconfounded children's identifiability is sufficient but not
necessary, and this related condition is necessary but not sufficient and works
on single variable $X, Y$.

Shpitser and Pearl provide a necessary and sufficient criterion for identifiability of:
\begin{equation}
    P(\mathbf{Y} = \mathbf{y}|do(\mathbf{X} = \mathbf{x}))
\end{equation}
when $\mathbf{Y}$ and $\mathbf{X}$ are arbitrary sets of variables: \textbf{the hedge criterion}.

Moving further along, Shpitser and Pearl provide a necessary and sufficient criterion
for the most general type of causal estimand: \textbf{conditional causal effects},
which takes the form:
\begin{equation}
    P(\mathbf{Y} = \mathbf{y}|do(\mathbf{X} = \mathbf{x}), \mathbf{Z} = \mathbf{z})
\end{equation}
where $\mathbf{Y}, \mathbf{X}$, and $\mathbf{Z}$ are all arbitrary sets of variables.
\chapter{Counter Factuals}
To make any sense on responsability, we must be able to compare what did happend
with what would have happened under some alternative hypothesis.
\section{Introduction to counterfactuals}


\textbf{Hypochetical condision} or \textbf{Antecendent} is the kind of decision
we could have taken in the counter factual worlds.


Counter facuals are used to compare two outcomes under the exact same conditions,
differing only in antecedent.

The fact that we know the outcome of our actual decision is important, because my estimand
after seeing the consequences of my actual decision may be totally different from
my etimate prior to seeing the consequence.

We cannot express this estimate using the do-expression because the do-operator
is too crude To distinguish: The \textbf{actual value} and the \textbf{hypothetical one}.
We are trying to compare between two choice after being choose one.

We need this distinction in order to let the actual value inform our assessment
of the hypothetical value. The do-operator is not capable to respond at this question.

We will use potential outcame frameworks:
$$Y_{X=x} = Y_x = Y(X=x)$$
Because we need to discover the values of $Y$ befour the choise of $X=x$.


We can formulare it as follow:

$$\mathbb{E}[Y_1|X=0,Y=Y_0]$$

Where $Y_1$ is the \textbf{hypothetical condition} and $X=0,Y=Y_0$ are the observed
variables.

\begin{note}
    Randomized control experiments cannot help us because they give us the information
    on hypothetical condition after the intervention:
    $$\mathbb{E}[Y_1|do(X=x)]$$
    but it doesn't add information of the fact that we had observed $Y_0$.
\end{note}

\section{Defining and computing counterfactual}

Consider a fully specified model SCM $M=<U, V,F>$ for which we know both the
functions $F$ and the values of all exogenous variables $U$.

In such a deterministic model every assignment $U=u$ to the exogenous variables
corresponds to a single member of, or a unit in a population, or to a situation in nature.
Each assignment $U=u$ uniquely determines the value of all variables in $V$. This
is given by the definintion of $M$.

The characteristic of each unit in a population have unique values, depending on
that individual's identity.

Consider now the \textbf{counterfactual sentence}
\begin{center}
    "$Y$ would be $y$ had $X$ been $x$, in situation $U=u$"
\end{center}
Which can be summarize in $Y_x(u) = y$


had $X$ been $x$ need to be interpreted as an istruction to more a minimal modification
in the \textbf{current model} so as to establish the antecedent condition $X = x$ which is
likely to conflict with the observed valve $X$, $X(u)$. Such a minimal modification
amounts to replacing the equation for $X$ with a constant $x$, which may be thought
as an external intervention $do(X=x)$ not necessary by a human experimenter.

This replacement permits the constant $x$ to differ from the actual value of $X$ ($X(u)$)
without rendering the system of equations inconsistent, and in this way, it allows
all variables to serve as antecedent to other variables.

Each SCM encodes within it many counterfactuals, corresponding to the various values
that its variable can take.

Counterfactuals are different than the ordinary inverventions, captured by the do-operacor.
We want to work at individual level: for each sizuation $U =u$, we obtained a definite
number, $Y_x (u)$, which stands for than hypothesical value of $Y$ in that situation.

The do-operator is only defined on probability distributions and always delivers
probabilistic results such as $\mathbb{E}[Y| do(x)]$. We are working on a population
level, while with counter factuals we are working on a unit $u$ of the population $U$
computing the outcome $Y_x(u)$, we are computing the hypothetical value of $Y$
in the situation of $X=x$ for the unit $u$.

We are now ready to generalize the concept of counterfactuals to any structural
model $M$. Consider any arbitrary two variables $X$ and $Y$, not necessarily connected
by a single equation.

We can pass from a model $M$ to $M_x$ by replacing the equation of $X$ with $X=x$,
so we can introduce the formal definition of counterfactuals $Y_x (u)$:
$$Y_x (u) = Y_{M_x}(u)$$
The counterfactual is the counterfactual obtained after the modification of the model
$M$ in $M_x$. So each model $M$ has a graph $G$ and $M_x$ has a graph $G_X$, where
$G_X$ is the graph of $M$ ($G$) where all incoming edges to $X$ have been removed.

The same definition is applicable when $X$ and $Y$ are sets of variables.

Counterfactuals allow us to take our scientific conception of reality, $M$, and
use it to generate answers to an enormous number of hypothetical question.
The values than these counterfacuals recieve are not totally arbitrary, but must
cohere with each other to be consistent with an underlying model.

\begin{note}
    If we observed $X(u)=1$ and $Y(u) = 0$ then the counterfactual:
    $$Y_{X=1}(u)=0$$
    is coeherent with the world, because you are whatching the actual world
\end{note}

\begin{definition}[\textbf{Counterfactual consistency rule}]
    If $X= x$ then $Y_x=Y$.

    If $X$ is binary then we have the convenient form
    $$Y = X_1 + (1-X)Y_0 $$
\end{definition}
This is coherent because:
\begin{itemize}
    \item $Y_1$ is equal to observed value of $Y$ whenever $X$ takes the value of $1$
    \item $Y_0$ is equal to observed value of $Y$ whenever $X$ takes the value of $0$
\end{itemize}

Any \textbf{deterministic counterfactuals} is computed by the following steps:
\begin{itemize}
    \item \textbf{Abduction}: use evidence $E = e$ to derermine the value of $U$
    \item \textbf{Action}: modify the model, $M$, by removing the structural equations for
          the variables in $X$ and replacing them with the appropriate functions $X = x$,
          to obrain the modified model $M_x$
    \item \textbf{Prediction}: use the modified model $M_x$ and the values of $U$ to compute
          the value of $Y$, the consequence of the counterfacuals.
\end{itemize}

So to compute counterfactual we need a unit and each observed variables of that unit.
Structural equation models are able to answer counterfactual queries of this nature because
each equation represents the mechanism by which the variable obtains its values.

If we know these mechanism, we should also be able to predict what values would be
obtained had some of these mechanisms been altered, given alterations.

\subsection{Probabilistic counterfactual}
Counterfactual can also be probabilistic, pertaining to a class of units within the population (ex: $Y<2$).
This probabilistic counterfactuals differ from do-operator intervencions because
they restrict the set of individual intervened upon, which do-expression cannot do.
We do not have information on all variables (we can't compute $u$) and probabilistic counterfactual allows
us to ask question about probabilisties and expectations of counterfactuals.

Nondeterminism enters causal models by assigning probabilizies $P(U=u)$ (\textbf{exogenous probability}) over the
exogenous variables
\begin{center}
    Uncertaty = Probability distribution
\end{center}

Exogenous probability induce a unique probability distriburion on the endogenous
variables.

$$P(U=u) \implies P(V=v)$$

This allow us to compute the joint distribuzion of all combinations of
observed and counterfactuals variables.

A typical query will be
\begin{center}
    Given that we observe feature $E=e$ for e given individual, what would we
    expect the value of $Y$ for that individual to be if $X$ had been $x$?
\end{center}

This expected is denoted $\mathbb{E}[Y_{X=x}|E=e]$
where we allow $E=e$ to conflict wich the antecedent $X=x$

The three-step process for probabilistic counterfactual:
\begin{itemize}
    \item \textbf{Abduction}: Update $P(U)$ by the envidence to obtain $P(U |E=e)$
    \item \textbf{Action}: modify the model, $M$, by removing the structural equations for
          the variables in $X$ and replacing them with the appropiate function $X=x$ to
          obtain the modified model $M_X$
    \item \textbf{Prediction}: use the modify model $M_X$ and the updated probabilities
          over $U$ variables, $P(U|E=e)$, to compure the expectation of $Y$, the consequence
          of the countertactuals
\end{itemize}


\end{document}
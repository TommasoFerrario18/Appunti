\chapter{Logiche temporali e model checking}
\section{Introduzione}
Le \textbf{logiche temporali} sono una famiglia di logiche matematiche che
permette di esprimere proprietà che cambiano nel tempo.

Quando si analizzano sistemi reattivi (quindi concorrenti) allora non vale più
il ragionamento di avere uno stato precedente e uno stato successivo che varia nel
tempo.

Sono stati introdotti diversi modelli per modellare i sistemi concorrenti, ora
si introdurranno nuove tecniche di analisi.

Esempio produttore consumatore in Java. Esempio concorrente e gli elementi sono
indipendenti quindi si devono sincronizzare quando si accede al buffer. Vogliamo
garantire la correttezza del programma:
\begin{itemize}
    \item Ogni oggetto deve essere prima prodotto
    \item Ogni oggetto non può essere consumato più di una volta
    \item Il sistema non raggiunge mai uno stato di \textbf{deadlock}
\end{itemize}
Le reti di Petri ci permettono di controllare se si mantiene la mutua esclusione,
ovvero che non esista una  marcatura in cui entrambi i processi non possono essere
negli stati $c_1$ e $c_2$. Noi vorremmo modellare il blocco di uno dei due
processi se uno si trova nella zona critica.

Vogliamo quindi un modello per verificare se sono valide delle proprietà di
interesse e se il modello rappresenta effettivamente il sistema.
\begin{definizione}
    Un sistema è \textbf{reattivo} se è un sistema concorrente, distribuito e
    asincrono.
\end{definizione}
Una sottoclasse dei sistemi reattivi sono quelli sincroni con un clock ma è una
semplificazione.

I sistemi reattivi non obbediscono più al paradigma input-computazione output,
rendendo quindi impossibile utilizzare le triple di Hoare per dimostrare la
correttezza di un programma. Possiamo sempre riutilizzare la logica di hoare per
dimostrare alcuni pezzi ma dovremmo usare un modo per unire i risultati.

Al posto delle triple di Hoare, utilizzeremo delle \textbf{asserzioni}, ovvero
delle frasi conteneti elementi temporali che descrivono il comportamento del sistema.
\begin{esempio}
    Se è stato spedito un messaggio allora questo prima o poi verrà ricevuto dal
    destinatario
\end{esempio}
\begin{esempio}
    Se si accennde una spia di allarme allora questa sarà accesa fino a quando non
    si spegne il dispositivo
\end{esempio}
Vediamo ora come possiamo procedere nell'analisi dei sistemi concorrenti. Il
problema che vogliamo risolvere è quello di stabilire se un sistema reattivo è
corretto. Per fare ciò, seguiremo il seguente schema:
\begin{enumerate}
    \item Si esprime il criterio di correttezza come una formula di un opportuno
          linguaggio logico.
    \item Si modella il sistema nella forma di un sistema di transizioni.
    \item Si valuta se la formula è vera nel sistema di transizioni. (algoritmi)
\end{enumerate}
\begin{definizione}[\textbf{Sistema di transizioni}]
    Un \textbf{sistema di transizioni} è definito da:
    \begin{equation}
        A=(Q,T)
    \end{equation}
    dove:
    \begin{itemize}
        \item $Q$ insieme degli stati (non per forza finito)
        \item $T\subseteq Q\times Q$ insiemi di transizioni di stato
    \end{itemize}
\end{definizione}
\begin{definizione}[\textbf{Cammino}]
    Definiriremo un \textbf{cammino} come una sequeza di stati $\pi$ dove
    \begin{equation}
        \pi = q_0q_1\dots q_n \ (q_i,q_{i+1})\in T \ \forall i
    \end{equation}
    (combacia con la definizione di cammino su grafi).
\end{definizione}
\begin{definizione}[\textbf{Suffisso}]
    Definiriremo un \textbf{suffisso} di ordine $i$ per un cammino $\pi$ è il
    cammino che iniziada uno stato di indice $i$.
    \begin{equation}
        \pi^{(i)} = q_iq_{i+1}\dots q_n
    \end{equation}
\end{definizione}
\begin{definizione}[\textbf{Cammino massimale}]
    Definiriremo un \textbf{cammino massimale} se il cammino è finito e non può
    essere esteso, cioè solo quando il cammino raggiunge uno stato che non ha
    transizioni uscenti. (cammino massimale finito)
\end{definizione}
Il cammino massimale è infinito quando non si ha uno stato pozzo.
\begin{nota}
    Ogni suffisso di un cammino massimale è a sua volta un cammino massimale.
\end{nota}
Posso rappresentare un cammino espandendo la definizione delle espressioni regolari,
aggiungendo $^\omega$ per rappresentare una sequenza infinita. Bisogna prestare
particolare attenzione al fatto che $\omega \neq \ast$, inaftti $\ast$ mi rappresenta
una sequenza arbitraria di valori.
\begin{nota}
    Posso esprimere un cammino come un suffisso di ordine 0.
\end{nota}
\section{Logica Temporale Lineare}
Definiamo la \textbf{sintassi} partendo dalle proposizioni atomiche:
\begin{equation}
    AP=\{ p_1,p_2,\dots,p_i,\dots \}
\end{equation}
composte dalle asserzioni considerate vere a prescindere.
\begin{esempio}
    Il messaggio è stato spedito, la spia di allarme è accesa
\end{esempio}
Definiamo anche le formule ben formate $FBF_{LTL}$:
\begin{itemize}
    \item Ogni proposizione atomica è una formula ben formata.
    \item $\top$ e $\bot$ sono formule ben formate.
    \item se $\alpha,\beta\in FBF$ allora $\alpha\lor \beta,\lnot \alpha \in FBF$
          (da queste si derivano tutti i connettivi logici).
    \item Operatori temporali siano $\alpha$ e $\beta$ formule ben formate, allora
          anche:
          \begin{itemize}
              \item $X\alpha$ "next $\alpha$" nel prossimo stato $\alpha$ è vera.
              \item $F\alpha$ "future $\alpha$" prima o poi $\alpha$ sarà vera.
              \item $G\alpha$ "globally $\alpha$" $\alpha$ è sempre vera.
              \item $\alpha\mathbb{U}\beta$ "until $\alpha$" $\alpha$ è vera fino
                    a quando $\beta$ diventa vera.
          \end{itemize}
          sono formule ben formate.
\end{itemize}
Per definire la sintassi utilizzeremo i \textbf{Modelli di Kripke} che prendono
i sistemi di transizione e li arricchiscono associando a ogni stato $q \in Q$
l'insieme delle proposizioni atomiche che sono vere in $q$. Quindi possiamo
definire:
\begin{equation}
    I \ : \ Q \rightarrow 2 ^{AP}
\end{equation}
dove $2^{AP}$ è l'insieme delle parti di $AP$. Un modello di Kripke sarà definito
come:
\begin{equation}
    A=(Q,T,I)
\end{equation}
Con un modello di Kripke possiamo identificare per uno stato quali sono le
proposizioni atomiche vere e per capire se le FBF sono vere allora dovrò analizzare
i cammini che partono dallo stato in esame.

Vediamo ora come associare una semantica alle formule ben formate. Per fare ciò
procediamo in due fase:
\begin{enumerate}
    \item Definiamo un criterio per stabilire se una formula $\alpha$ è vera in
          un cammino massimale $\pi$.
    \item Diciamo che la formula è vera rispetto a uno stato $q$ se è vera in
          tutti i cammini massimali che partono da $q$.
\end{enumerate}
Fissiamo quindi un cammino generico $\pi$ e $\alpha$ una formula ben formata allora
\begin{equation}
    \pi \vDash \alpha \iff \ \text{significa che} \ \alpha \ \text{è vera nel cammino}
    \ \pi
\end{equation}
Definiremo la relazione $\vDash$ per induzione. Supponiamo che $\alpha$ e $\beta$
siano due formule ben formate e $p$ una preposizione atomica:
\begin{itemize}
    \item $\pi \vDash \top$
    \item $\pi \not\vDash \bot$
    \item $\pi \vDash p \iff p\in I(q_0)$
    \item $\pi \vDash \lnot \alpha \iff \pi \not\vDash \alpha$
    \item $\pi \vDash  \alpha \lor \beta \iff (\pi \vDash \alpha)\lor (\pi \vDash
              \beta) $
    \item Operatori temporali: supponiamo $\beta, \gamma \in FBF$
          \begin{itemize}
              \item $\pi\vDash X\beta \iff \pi^{(1)}\vDash \beta$
              \item $\pi\vDash F\beta \iff \exists i \in \mathbb{N}:\pi^{(i)}
                        \vDash \beta$
              \item $\pi\vDash G\beta \iff \forall i \in \mathbb{N}:\pi^{(i)}
                        \vDash \beta$
              \item $\pi \vDash \beta \mathbb{U} \gamma \iff$:
                    \begin{itemize}
                        \item $\exists i \in \mathbb{N}$ tale che $\pi^{(i)}\vDash
                                  \gamma$ cioè $\pi\vDash F\gamma$
                        \item $\forall h, 0\le h < i, \pi^{(h)}\vDash \beta$
                    \end{itemize}
                    Se $\gamma$ è vera fin da subito, quindi $i=0$ allora $\beta$
                    è superfluo
          \end{itemize}
\end{itemize}
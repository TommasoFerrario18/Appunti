\chapter{Deeplearning}
Le principali tipologie di apprendimento sono:
\begin{itemize}
    \item \textbf{supervisionato}
    \item \textbf{non supervisionato}
\end{itemize}
Le architetture di deeplearning sono inizialmente state composte spesso da particolari strutture
simili agli enconder-decoder chiamate autoencoder. Queste strutture sono reti 
neurali che hanno il compito di imparare come ricostruire l'input. La metodologia
di apprendimento è il supervisionato ma senza label, più precisamente si da in 
input l'immagine e l'output della rete verrà confrontato con l'input. L'obiettivo
è ridurre l'errore sulla ricostruzione dell'input.
Il loro risultato è proprio dovuto al loro collo di bottiglia che separa la rete
in due:
\begin{itemize}
    \item \textbf{enconder}: impara una funzione di codifica
    \item \textbf{denconder}: impara una funzione di decodifica
\end{itemize}
L'algortimo di apprendimento di queste reti spesso è una delle tante implementazioni
di backpropagation, applicandolo su ciascun neurone di uscita.

Per gli autoencoder si deve obbligatoriamente avere sempre una struttura a clessidra
con la strozzatura centrale, la quale permette di generare la codifica e separare
le due reti. In aggiunta, l'architettura non prevede che ci siano collegamenti
tra encoder e decoder se no la cofica nella strozzatura non è più consistente.

Le architetture odierne di deeplearning si basano su strutture neurali e utilizzano i 
\textbf{tranformer}, componenti neurali che prendono in input sia il nuovo input, 
sia un vecchio output del trasformer. Sono componenti fondamentali per la generazione
di informazioni. L'apprendimento di queste reti si basa su strutture \textbf{feed-forward}
e su componenti \textbf{Attention}, quest'ultimi fondamentali perché permettono
l'effettivo cambiamento dei valori. 

Prima di arrivare ai transformer si è passati dagli autoencoder, a \textbf{Word2Vec},
un modello neurale utilizzato per problemi di \textbf{NPL}. Più precisamente si occupa
di associare per ogni parola del linguaggio un vettore di numeri reali, in questo
modo è possibile rappresentare le singole parole in uno spazio $n$-dimensionale.
Questa proprietà ha permesso di scoprire una proprietà: parole simili vengono 
rappresentate in una regione vicina dello spazio. Questo è dato dal fatto che 
le reti vengono allenate su tutti i testi raggiungibili e l'assegnamento del vettore
si effettua in base a quante volte due parole vengono veste vicine in una frase. 
Quindi la rete riesce a codificare come vettori simili parole simili, infatti c'è 
una corrispodenza geometrica col significato. 

Successivamente si è passati ai \textbf{denoising autoencoder}, ovvero rimozione del rumore
e ricostruzione dell'input originale. Queste reti vengono allenate dando in input 
il dato sporcato in precedenza, successivamente l'output viene confrontato con 
l'input originale. 

In seguito, sono stati introdotti i componenti \textbf{Attention}, composti da 3 
elementi in input e il suo compito è di assegnare in automatico diversi pesi 
agli elementi in input. Queste reti permettono quindi di cambiare i pesi degli input
portando a dei vantaggi notevo, per esempio, si possono interpretare parole ambigue 
in base al contesto.

Infine, si è passati ai tranformer, introcendo la ricorrenza dei dati in modo da 
considerare in input i vecchi dati della rete, sfasati, con un meccanismo di Attention.

I modelli di deeplearning vengono allenati in 2 fasi:
\begin{itemize}
    \item \textbf{pretraining generico}: allenamento non supervisionato, si allena
    la rete in modo generale senza specificare un particolare task. Questa fase 
    è la più pesante e permette di definire un primo stato di partenza dei parametri.
    \item \textbf{finetuning}: si prende la rete preallenata e si addestra per un task
    specifico utilizzando la metodologia supervisionata. 
\end{itemize}

In questo modo è possibile ridurre i tempi di addestramento e rendere riutilizzabile
la stessa rete per task simili. 

Alla fine si arriva alla creazione di reti più complesse come GPT, basata sempre
su trasformer e attention, ma addestrata in 2 fasi, con la possibilità di effettuare 
del reinforcement learning per migliorare ancora di più i risultati.
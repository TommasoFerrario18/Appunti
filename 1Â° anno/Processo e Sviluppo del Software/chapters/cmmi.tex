\chapter{Capability Maturity Model Integration}
Il \textbf{Capability Maturity Model Integration} (CMMI) che è un programma di
formazione e valutazione per il miglioramento a livello di processo gestito dal
CMMI Institute.

Bisogna prima introdurre il concetto di \textbf{maturità dei processi}. La
probabilità di portare a termine un progetto dipende dalla maturità del progetto
e la maturità dipende dal grado di controllo che si ha sulle azioni che si vanno
a svolgere per realizzare il progetto. Si ha quindi che:
\begin{itemize}
      \item Il progetto è \textbf{immaturo} quando le azioni legate allo sviluppo non sono
            ben definite o ben controllate e quindi gli sviluppatori hanno troppa
            libertà alzando la probabilità di fallimento.
      \item Il progetto è \textbf{maturo} quando le attività svolte sono ben definite,
            chiare a tutti i partecipanti e ben controllate. Si ha quindi un modo
            per osservare quanto si sta svolgendo e verificare che sia come
            pianificato, alzando le probabilità di successo e riducendo quelle di
            fallimento.
\end{itemize}
Risulta quindi essenziale ragionare sulla maturità del processo. Tale valore è
definito tramite un insiemi di livelli di maturità con associate metriche per
gestire i processi, questo è detto \textbf{Capability Maturity Model} (CMM). In
altri termini il modello CMM è una collezione dettagliata di best practices che
aiutano le organizzazioni a migliorare e governare tutti gli aspetti relativi al
processo di sviluppo.
\begin{center}
      \textit{Un processo migliore porta ad un prodotto migliore.}
\end{center}
Il modello CMMI è composto da:
\begin{itemize}
      \item \textbf{Process area}: che racchiude al suo interno una collezione di
            pratiche organizzate secondo obiettivi e riguarda una certa area del
            processo. Nel CMMI abbiamo 22 diverse process area. Ciascuna process
            area ha:
            \begin{itemize}
                  \item \textbf{Purpose statement}: descrive lo scopo finale
                        della process area stessa.
                  \item \textbf{Introductory notes}: descrivano i principali
                        concetti della process area.
                  \item \textbf{Related process area}: se utile, con la lista
                        delle altre process area correlate a quella corrente.
            \end{itemize}
      \item Le process area si dividono in due tipologie di obiettivi:
            \begin{enumerate}
                  \item \textbf{Specific goals}: ovvero gli obiettivi specifici
                        della singola process area in questione. All'interno di
                        ognuno abbiamo una serie di \textbf{specific practices},
                        ovvero quelle azioni che se svolte permettono di raggiungere
                        quell'obiettivo specifico e, a loro volta, tali pratiche
                        sono organizzate in:
                        \begin{itemize}
                              \item \textbf{Example work product}: elenchi di
                                    esempi di prodotti che possono essere generati
                                    attraverso l'adempimento delle pratiche.
                              \item \textbf{Subpractices}: descrizione dettagliata
                                    per l'interpretazione e l'implementazione.
                        \end{itemize}
                  \item \textbf{Generic goals}: gli obiettivi comuni a tutte
                        le process area. Questi obiettivi rappresentano quanto la
                        process area sia ben integrata e definita nel contesto
                        del processo ma questi criteri sono generali. All'interno
                        dei quali abbiamo una serie di \textbf{generic practices},
                        comuni a tutti, con le pratiche che devono essere svolte
                        per gestire positivamente una qualsiasi process area e,
                        a loro volta, tali pratiche sono organizzate in:
                        \begin{itemize}
                              \item \textbf{Generic practices elaborations}:
                                    ulteriori informazioni di dettaglio
                                    per la singola pratica.
                              \item \textbf{Subpractices}: descrizione
                                    dettagliata per l'interpretazione e
                                    l'implementazione.
                        \end{itemize}
                        Tra i principali generic goals (GG) abbiamo:
                        \begin{itemize}
                              \item \textbf{GG1}: raggiungere i specific goals,
                                    tramite l'esecuzione delle specific practices.
                              \item \textbf{GG2}: “ufficializzare” un managed
                                    process, tramite training del personale,
                                    pianificazione del processo, controllo dei
                                    work product etc$\dots$
                              \item \textbf{GG3}: “ufficializzare” un defined
                                    process, tramite la definizione rigorosa del
                                    progetto e la raccolta di esperienze legate
                                    al processo.
                        \end{itemize}
            \end{enumerate}
\end{itemize}
Per capire quanto un processo software è organizzato secondo questo standard
bisogna “mappare” quali goals e quali pratiche si stanno svolgendo e usare CMMI
non solo come \textit{ispirazione} ma come vero e proprio standard per definire
le azioni da svolgere nonché per confrontare il nostro operato e studiarlo
qualitativamente. Lo studio qualitativo mi permette di stabilire la maturità del
progetti, secondo un certo livello di compliance, detto CMMI level. Tale qualità
che può essere certificata da enti certificatori appositi.

Studiamo a fondo questi livelli di maturità e la loro codifica. Si hanno due
linee di sviluppo/miglioramento:
\begin{enumerate}
      \item \textbf{Capability levels} (CL): indica quanto bene si sta gestendo
            una particolare process area. Quindi per una singola process area mi
            dice quanto bene sto raggiungendo i generic goals.

            Il capability level ha valore compreso tra 0 e 3, estremi inclusi:
            \begin{itemize}
                  \item \textbf{Level 0} o \textbf{incomplete}: dove le pratiche
                        per una specifica process area sono state svolte
                        parzialmente o probabilmente non vengono svolte.
                  \item \textbf{Level 1} o \textbf{performed}: dove si eseguono
                        le pratiche e i vari specific goals sono soddisfatti. (GG1)
                  \item \textbf{Level 2} o \textbf{managed}: dove oltre alle
                        pratiche si ha anche una gestione delle attività stesse.
                        Si ha una policy per l'esecuzione delle pratiche. (GG2)
                  \item \textbf{level 3} o \textbf{defined}: dove l'intero processo
                        è ben definito secondo lo standard, descritto rigorosamente
                        e si ha un processo completamente su misura dell'organizzazione.
                        (GG3)
            \end{itemize}
            I capability levels di ciascuna process area possono essere rappresentati
            su un diagramma a barre, dove viene indicato il livello attuale e il
            profile target, ovvero il livello a cui quella process area deve arrivare.
      \item \textbf{Maturity levels} (ML): indica il livello di maturità raggiunto
            dall'intero processo di sviluppo, basandosi su tutte le process area
            attivate.

            Il Maturity level ha valore compreso tra 1 e 5:
            \begin{itemize}
                  \item \textbf{Level 1} o \textbf{initial}: dove si ha un processo
                        gestito in modo caotico.
                  \item \textbf{Level 2} o \textbf{managed}: dove si ha un processo
                        ben gestito secondo varie policy.
                  \item \textbf{Level 3} o \textbf{defined}: dove si ha un processo
                        ben definito secondo lo standard aziendale.
                  \item \textbf{Level 4} o \textbf{quantitatively managed}: dove
                        si stabiliscono obiettivi quantitativi per la qualità e
                        le performance del processo, in modo da poterli utilizzare
                        per la gestione.
                  \item \textbf{Level 5} o \textbf{optimizing}: dove grazie alle
                        informazioni raccolte ottimizzo il processo, in un'idea di
                        continuous improvement del progetto.
            \end{itemize}
\end{enumerate}
Il raggiungimento della \textbf{maturità del processo} quando si utilizza la rappresentazione a stadi
si ha quando il \textbf{maturity level} è 4 o 5. Il raggiungimento del \textbf{maturity level} 
4 implica l'implementazione dei \textbf{maturity level} 2, 3 e 4 in tutte le aree del proccesso. 
Allo stesso modo, il raggiungimento del \textbf{maturity level} 5 
implica l'implementazione di tutte le aree di processo per \textbf{maturity level} 
2, 3, 4 e 5.

Mentre, quando si utilizza la rappresentazione continua, si raggiunge un'elevata
maturità usando il concetto di stadiazione equivalente. La maturità che è
equivalente al \textbf{maturity level} 4 utilizzando lo staging equivalente
si raggiunge quando si raggiunge a \textbf{capability level} 3 per tutte le aree di
processo, ad eccezione dell'Organizational Performance Management (OPM) e Causal
Analysis and Resolution (CAR). L'elevata maturità, equivalente al 
\textbf{maturity level} 5 utilizzando una equivalente è raggiunta quando si raggiunge 
il \textbf{capability level} 3 per tutte le aree di processo.

Si possono confrontare CMMI e le pratiche agili. Ciò che viene svolto ai livelli
2 e 3 (con qualche piccolo adattamento) di maturity level si fa ciò che viene
fatto anche coi metodi agili. In merito ai livelli 4 e 5 di maturity level si hanno
pratiche che non rientrano nell'ottica dei metodi agili. Quindi un'organizzazione
può usare i metodi agili ed essere standardizzata rispetto CMMI raggiungendo
un maturity level 2 o 3. CMMI è quindi uno standard industriale con certificazioni
ufficiali.
\chapter{Introduzione}
I sistemi complessi sono costituiti da un gran numero di componenti interagenti 
tra loro, e sono caratterizzati da un comportamento globale emergente che non 
può essere spiegato dalla somma dei comportamenti locali. Questi sistemi sono 
presenti in natura, in ambito biologico, fisico e sociale, e sono oggetto di 
studio di molte discipline scientifiche. La teoria dei sistemi complessi è un 
campo di ricerca interdisciplinare che si occupa di studiare i meccanismi che 
regolano il comportamento di questi sistemi, e di sviluppare modelli matematici e 
computazionali per descriverne le proprietà e prevederne l'evoluzione.

Questi sistemi non sono facili da studiare e risulta difficile effettuare delle
previsioni. In compenso, si possono studiare le proprietà costruendo dei modelli
matematici e simulando il comportamento del sistema. 

\begin{definizione}[\textbf{Agente}]
    Un \textbf{agente} è un'entità autonoma che può percepire l'ambiente,
    elaborare informazioni e agire in base a queste informazioni.

    Un agente è caratterizzato da:
    \begin{itemize}
        \item \textbf{Architettura}: la struttura interna dell'agente.
        \item \textbf{Programma}: l'insieme di regole che l'agente segue.
    \end{itemize}
    \begin{center}
        Agente = Architettura + Programma
    \end{center}
\end{definizione}

Il concetto di agente può essere esteso a un sistema composto da più agenti,
chiamato \textbf{sistema multi-agente}. Questi sistemi sono costituiti da un
insieme di agenti che interagiscono tra loro e con l'ambiente, e sono in grado
di raggiungere obiettivi che sarebbero difficilmente raggiungibili da un singolo
agente.
\begin{definizione}[\textbf{Sistema}]
    Un \textbf{sistema} è un insieme di componenti interagenti tra loro, che
    cooperano per raggiungere un obiettivo comune.
\end{definizione}
\chapter{Modelli}
\section{Automi cellulari}
Possono riprodurre fenomeni di auto-riproduzione e auto-organizzazione, sono 
utili per modellare sistemi complessi dinamici e la simulazione. Utili per simulare 
e studiare sistemi come: traffico, afflusso di persone, percolazione (studio della 
fluido dinamica), sistema immunitario, sistemi sociali\dots.

L'idea è di non discrivere il sistema complesso dall'alto, ad alto livello ex: sistemi
di equazione differenziali. In realtà lo fa definendo delle regole locali e 
la combinazione delle regole apporta dei comportamenti complessi (interazioni 
semplici degli individui che generano comportamenti complessi).

Gli automi cellulari sono:
\begin{itemize}
    \item \textbf{sistemi}: insieme di entità che interagiscono
    \item \textbf{dinamici}: evolvono nel tempo in un insieme di passi
    \item \textbf{discreti}: spazio, tempo e proprietà devono essere solo finiti
    con un numero numerabile di stati. Spazio finito perché dobbiamo rappresentare
    la realtà.
\end{itemize}

Esistono automi cellulari a più dimensioni che specificano quante dimensioni è lo spazio.

Lo spazio è rappresentato da celle discrtete, si ha un tempo di evoluzione discreto.
Una cella può assumere un particolare stato e la sua evoluzione dipende dagli stati
delle celle adiacienti. Si guardano solo le celle del vicinato quindi si ha una relazione 
locale e uniforme con le celle adiacenti.

\begin{definizione}
    Un automa cellulare è
    $$\leftangle L, Q, q_0, u,f\rightangle$$
    dove:
    \begin{itemize}
        \item $L$ è l'array di automi a stati finiti uniformi
        \item $Q$ insieme di stati finiti
        \item $q_0$ è lo stato iniziale
        \item $u$ è la connessione della singola cella tra quelle adiacenti
        \item $f$ la regola di transizione locale
    \end{itemize}
\end{definizione}

Lavorando con automi cellulari finiti sarà importante definire una \textbf{condizione 
di bordo}, ovvero come deve essere la regola quando le celle sono vicino al bordo,
quando non hanno la specifica dei vicini richiesta. La scelta della condizione di
bordo può essere fatta in questo modo:
\begin{itemize}
    \item collegare i bordi
    \item guardare i bordi come uno specchio
\end{itemize}

Per gli automi cellulari 1-D possono codificare le regole come numeri decimali.
Per mostrare l'evoluzione dell'automa 1D si usa il diagramma spazio-tempo, dove 
permette di mostrare l'evoluzione dell'automa nel tempo.   

\begin{esempio}[Traffico]
    La regola del traffico 184 permette di modellare semplicemente il flusso del
    traffico su una strada a singola corsia.

    Possiamo complicare tutto il modello e al posto di avere la cella che specifica
    se è presente il veicolo, si specifica la velocità del veicolo in quella cella.
    Si aumenta il raggio delle celle che si considera vicine. Ogni veicolo accellera 
    di una velocità fino a quando non rischia di fare incidenti.
\end{esempio}
 
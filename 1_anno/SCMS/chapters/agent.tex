\chapter{Agent}
Nell'Intelligenza artificiale si è passati da Classical AI verso la Distribuited
AI.

\begin{definizione}[\textbf{AI}]
    Si definisce \textbf{AI} come lo studio degli agenti che ricevono delle
    percezioni dall'ambiente ed effettuano delle azioni.
\end{definizione}
Ciascun agente implementa una funzione che mappa la sequenza di percezioni in azioni,
queste funzioni possono essere rappresentate in diversi modi:
\begin{itemize}
    \item \textbf{production system}
    \item \textbf{reactive agents}
    \item \textbf{real-time conditional planners}
    \item \textbf{neural network}
    \item \textbf{decision-theorem system}
\end{itemize}

\begin{definizione}[\textbf{Agentification AI}, \textbf{Classical AI}]
    L'\textbf{Agentification AI} significa avere un singolo agente che risolve
    problemi adottando diverse tecniche e approcci.
\end{definizione}

\begin{definizione}[\textbf{Agent}]
    L'\textbf{Agent} è qualsiasi cosa che può essere visto come una percettore del
    suo ambiente attraverso sensori e effettua azioni sull'ambiente attraverso attuatori.
\end{definizione}

\begin{esempio}
    Un esempio sono gli agenti umani composti da sensori come occhi, orecchie e altri
    organi, mentre è composto da attuatori che sono mani, gambe, bocca e parti del corpo.

    Oltre agli esseri umani si possono avere agenti robotici composti dai rispettivi
    sensori e attuatori.
\end{esempio}

% TODO: schema ambiente e agente

Le azioni dell'agente non sono interne all'agente, ma bensì cambiano
il suo stato iterno. Per ciascun agente abbiamo \textbf{funzioni agente} che mappano
le percezioni in azioni.
$$f:\mathcal{P}(P)\rightarrow \mathcal{A}$$
Dove:
\begin{itemize}
    \item $P$: è l'insieme delle percezioni
    \item $\mathcal{A}$: è l'insieme delle azioni
\end{itemize}
Si considera l'insieme potenza delle percezioni perché ci si aspetta una sequenza.

L'agente sarà anche formato dal \textbf{programma di un agente} che è la specifica
configurazione di un'architettura che produce $f$.

L'agente sarà composto dall'architettura (ex: Prolog) più il programma (ex: programma in Prolog).

\begin{definizione} [\textbf{Distribuited AI}]
    La \textbf{Distribuited AI} consiste in un sistema di entità le quali possono
    risolvere un problema effettuando delle azioni e interagendo con l'ambiente,
    sia in collaborazione, sia in competizione.
\end{definizione}

A differenza della Classical AI in cui si ha un solo agente, nella Distribuited AI
si ha un sistema di agenti.

\begin{definizione} [\textbf{Sistema}]
    ?
\end{definizione}

La parte distribuita può essere a più livelli:
\begin{itemize}
    \item \textbf{Distribuited solving of problem}: distribuzione della risoluzione
          del problema tra diversi specialisti
    \item \textbf{Solving of distribuited problem}: distribuzione a livello di dominio
          del problema
    \item \textbf{Distribuited techniques for problem solving}: distribuzione a livello
          di tecniche da utilizzare per la risoluzione del problema.
\end{itemize}


% Definiremo un sistema di entità che risolve un problema che interagiscono tra di loro.
Approccio ai sistemi orientato agli agenti.

Soluzione distribuita per un problema, ex: riconoscimento del parlato
in modo distribuito a livelli gerarchici, ogni livello si occupa di un particolare
task (ing. del SW si suddivide la responsabilità per risolvere un problema)

Soluzione per un problema distribuito, ex: problema della logistica, siamo in un
dominio distribuito con repati che si occupano di effettuare dei particolari task.
Si hanno agenti che mettono in comunicazione diversi reparti oppure collaborano
nei reparti per contribuoire nel task. Il confine dell'ambiente è il magazzino.
L'approccio generale può essere quello centralizzato con un server centrale che
riceve le informazioni degli agenti e attiva gli attuatori presenti nel magazzino.
I vantaggi dell'approccio centralizzato sono la configurabilità e la predicibilità,
quindi posso dimostrare proprietà. Posso predire malfunzionamenti degli agenti,
quindi posso sovradimensionare la struttura per gestire i probabili malfunzionamenti.
Si ha quindi una scalabilità, flessibilità e adattabilità limitata.

Quindi è stato pensato di evolvere il sistema trasfmormando i muletti da
attuatori ad agenti che comunicano col sistema centrale per effettuare i task.
Quindi da un paradigma: muletto x porta al posto y un prodotto z, ad chi si offre
di portare al posto y il prodotto z? Il muletti risponderanno e il sistema selezionerà
il migliore. Aumenta la complessità degli agenti perché devono gestire le richieste
centrali e inoltre deve comunicare con gli altri agenti per non scontrarsi negli
incroci. Quindi gli agenti devono avere una rappresentazione dell'ambiente, cosa
che non avevano nella soluzione centralizzata, perché questo veniva cestito tutto
dal server centrale.

Distribuzione a livello di tecniche di risoluzione di un problema non perforza
distribuite. Ex: nemici che cooperano che sconfiggere il giocatore. Rappresento 
e gestisco i bot come agenti.

Nel passato per implementare l'distribuited AI in diversi modi:
\begin{itemize}
    \item memoria distribuita: struttura dati condivisa con problemi sull'accesso
    concorrente e con strategie di controllo degli agenti. Implementato con Linda 
    che suggerisce un nuovo modo per gestire la concorrenza
    \item processing autonomo concorrente e il coordinamento tra gli agenti 
    con scambio di messaggi. Implementato con modello ad Attori.
\end{itemize}
Ora si usano dei middleware per sistemi distribuiti come code, publih and subscribe 
ecc$\dots$ (ex: Redis, Kafka).

Successivamente si è cercato di rinnovare la definizione di agente perché non 
per forza ricevere stimoli ed eseguire azioni.
\begin{definizione} [\textbf{Agente}, Means]
    Agenti autonomi sono sistemi computazionali ...
\end{definizione}

\begin{definizione} [\textbf{Agente}, Wooldridge-Jennings]
    Un agente è un sistema computerizzato che ha delle proprietà:
    \begin{itemize}
        \item autonomo: 
        \item abilità sociali: comunica con altri agenti
        \item reativo: percepisce dei cambiamenti e risponde di conseguenza
        \item proattivo: effettua azioni che modificano l'ambiente
    \end{itemize}
\end{definizione}

\begin{definizione} [\textbf{Agente}, Jacques]
    SKIP
\end{definizione}

Per gli approcci basati su agenti si hanno diverse criticità:
\begin{itemize}
    \item si modella il sistema definendo l'ambiente, l'architettura e poi 
    si ragiona sul singolo agente
    \item non per tutti i sistemi si può ragionare sul singolo agente. Spesso
    esistono azioni esterne.
    \item la modifica sull'ambiente può limitare la scelta delle azioni degli agenti
    quindi l'autonomia spesso non è buona e complica la gestione degli agenti.
\end{itemize}

\section{Architetture degli Agenti}
Esistono 2 tassonomie per classificare gli agenti:
\begin{itemize}
    \item classificazione rispetto la conoscienza (Genesereth): si hanno $3$ categorie 
    di agenti:
    \begin{itemize}
        \item \textbf{Topistic}: agenti definiti dalla $6$-tupla:
        $$<E,P,A,see,do,action>$$
        dove:
        \begin{itemize}
            \item $E$: l'ambiente indirettamente visto attraverso il sistema precettivo
            \item $P$: partizione di $E$ che rappresenta gli insiemi di stati dell'ambiente
            che l'agente può percepire dall'ambiente 
            \item $A$: insieme di azioni disponibili che l'agente può usare in risposta 
            agli stati dell'ambiente
            \item $see$: funzione definita come $E\rightarrow P$. Mappa l'ambiente
            negli stati dell'ambiente (discretizza l'ambiente)
            \item $action$: funzione definita come $P\rightarrow A$. Funzione interna
                nell'agente che seleziona l'azione da fare data una percezione 
                (non l'insieme potenza che è specificata nella definizione di agente generico).
            \item $do$: funzione definita come $A\times E\rightarrow E$. Funzione
            eseguita dagli attuatori.
        \end{itemize} 
        Il ciclo è l'agente che osserva l'enviroment (see), trova la funzione appropriata
        (action) e esegue l'azione (do). Ex: robottino che pulisce i pavimenti e si 
        sposta in modo randomico. 
        \item \textbf{Hysteretic}: agenti definiti dalla $9$-tupla:
        $$<I,E,P,A,i_0,see,internal,do,action>$$
        dove:
        \begin{itemize}
            \item $I$: insieme di stati interni dell'Agente
            \item $i_0$: stato interno iniziale
            \item $action$: funzione definita come $I\times P\rightarrow A$. 
            Considera lo stato interno e le percezioni per scegliere l'azione.
            \item $internal$: funzione definita come $I\times P\rightarrow I$.
            Funzione del cambio di stato interno.
            \item gli altri elementi sono uguali alla classificazione precedente.
        \end{itemize}
        Ex: robottino lavapavimenti che ha la possibilità di sapere la sua carica
        che viene salvata nello stato interno, inoltre tiene traccia della sua 
        posizione per poter poi ritornornare alla base di ricarica. La base di
        ricarica sarà la posizione iniziale (Stato iniziale). Non può avere la 
        mappa dell'ambiente.
        \item \textbf{Knowledge-level}: agenti definiti dalla $9$-tupla:
        $$<D,E,P,A,d_0,see,database,do,action>$$
        Dove:
        \begin{itemize}
            \item $D$: insieme di predicati (base di conoscienza)
            \item $d_0$: inizializzazione del database
            \item $action$: funzione definita come $D\times P\rightarrow A$. 
            Considera lo stato interno e le percezioni per scegliere l'azione.
            \item $database$: funzione definita come $D\times P\rightarrow D$.
            Funzione per aggiornare il database
            \item gli altri elementi sono uguali alla classificazione iniziale.
        \end{itemize}
        Ex: robottini che riescono a mappare l'ambiente, aggiornano il proprio
        database a partire dai fatti noti.
    \end{itemize}
    \item classificazione rispetto alla decisionalità (Norwig): classificazione 
    rispetto a come scelgono l'azione rispetto alla base di conoscienza.
    \begin{itemize}
        \item basati su obiettivi
    \end{itemize}
\end{itemize}


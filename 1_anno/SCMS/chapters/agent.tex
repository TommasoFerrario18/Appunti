\chapter{Agent}
Nell'Intelligenza artificiale si è passati da Classical AI verso la Distribuited
AI.

\begin{definizione}[\textbf{AI}]
    Si definisce \textbf{AI} come lo studio degli agenti che ricevono delle 
    percezioni dall'ambiente ed effettuano delle azioni.
\end{definizione}
Ciascun agente implementa una funzione che mappa la sequenza di percezioni in azioni,
queste funzioni possono essere rappresentate in diversi modi:
\begin{itemize}
    \item \textbf{production system}
    \item \textbf{reactive agents}
    \item \textbf{real-time conditional planners}
    \item \textbf{neural network}
    \item \textbf{decision-theorem system}
\end{itemize}

\begin{definizione}[\textbf{Agentification AI}, \textbf{Classical AI}]
    L'\textbf{Agentification AI} significa avere un singolo agente che risolve
    problemi adottando diverse tecniche e approcci.
\end{definizione}

\begin{definizione}[\textbf{Agent}]
    L'\textbf{Agent} è qualsiasi cosa che può essere visto come una percettore del 
    suo ambiente attraverso sensori e effettua azioni sull'ambiente attraverso attuatori.
\end{definizione}

\begin{esempio}
    Un esempio sono gli agenti umani composti da sensori come occhi, orecchie e altri
    organi, mentre è composto da attuatori che sono mani, gambe, bocca e parti del corpo.

    Oltre agli esseri umani si possono avere agenti robotici composti dai rispettivi
    sensori e attuatori.
\end{esempio}

% TODO: schema ambiente e agente

Le azioni dell'agente non sono interne all'agente, ma bensì cambiano
il suo stato iterno. Per ciascun agente abbiamo \textbf{funzioni agente} che mappano 
le percezioni in azioni.
$$f:\mathcal{P}(P)\rightarrow \mathcal{A}$$
Dove:
\begin{itemize}
    \item $P$: è l'insieme delle percezioni
    \item $\mathcal{A}$: è l'insieme delle azioni
\end{itemize}
Si considera l'insieme potenza delle percezioni perché ci si aspetta una sequenza.

L'agente sarà ache formato dal \textbf{programma di un agente} che è la specifica 
configurazione di un'architettura che produce $f$.

L'agente sarà composto dall'architettura (ex: Prolog) più il programma (ex: programma in Prolog).

\begin{definizione} [\textbf{Distribuited AI}]
    La \textbf{Distribuited AI} consiste in un sistema di entità le quali possono
    risolvere un problema effettuando delle azioni e interagendo con l'ambiente,
    sia in collaborazione, sia in competizione.
\end{definizione}

A differenza della Classical AI in cui si ha un solo agente, nella Distribuited AI
si ha un sistema di agenti. 

\begin{definizione} [\textbf{Sistema}]
    ?
\end{definizione}

La parte distribuita può essere a più livelli:
\begin{itemize}
    \item \textbf{Distribuited solving of problem}: distribuzione della risoluzione
    del problema tra diversi specialisti
    \item \textbf{Solving of distribuited problem}: distribuzione a livello di dominio
    del problema
    \item \textbf{Distribuited techniques for problem solving}: distribuzione a livello
    di tecniche da utilizzare per la risoluzione del problema.
\end{itemize}


% Definiremo un sistema di entità che risolve un problema che interagiscono tra di loro.

% Soluzione distribuita di un problema, ex: approccio riconoscimento distribuito 
% del parlato che sfrutta diverse tecnologie. Un altro esempio è il sistema di diagnosi
% medica, il dottore ti manda dagli specialisti. Il problema non è distribuito,
% ma bensì è il metodo di risoluzione distribuito.

% soluzione per un problema distribuito, ex: problema della logistica, posso quindi
% trattarlo in modo centralizzato se faccio centralizzare tutte le informazioni in
% un server.
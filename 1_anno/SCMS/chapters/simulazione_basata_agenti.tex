\chapter{Simulazione di un sistema complesso basato su Agenti}
\section{Introduzione}
Iniziamo analizzando un esempio di sistema complesso ovvero il movimento delle
folle di persone.

I sistemi complessi sono composti da tanti elementi che si integrano e collaborano.
Oltre a questo si ha la non linearità, una struttura gerarchica o connessa, si
ha robustezza e plasticità del sistema.
\subsection{Feedback}
Un sistema complesso ha la possibilità di osservare dei \textbf{feedback} che
possono essere positivi o negativi dal sistema.
\begin{esempio}
      Ipotizziamo di dover scegliere tra tre ristoranti e stiamo osservando
      le seguenti situazioni:
      \begin{itemize}
            \item Il primo ristorante è completamente vuoto.
            \item Il secondo ristorante è abbastanza affollato ma non è pieno.
            \item Il terzo ristorante è pieno e c'è una fila fuori.
      \end{itemize}
      In questo caso il feedback positivo è rappresentato dal secondo ristorante
      che è abbastanza affollato, quindi è probabile che sia buono. Il feedback
      negativo è rappresentato dal terzo e dal primo ristorante i quali
      rappresentano situazioni opposte ma entrambe possono essere viste come
      negative.
\end{esempio}
I feedback si possono ottenere come output del sistema, e in determinate circostanze,
può essere utile fornire in input tale valore per modificare l'ambiente o le
scelte che l'agente effettua. Ad esempio, un feedback positivo può amplificare
l'effetto di una scelta, mentre un feedback negativo può ridurre l'effetto di
una scelta. Quindi si hanno meccanismi di inibizione e stimolazione nel sistema.
\subsection{Motivazioni}
La ricerca e lo studio dei sistemi complessi è utile per chi si occupa di progettare,
pianificare e sviluppare , come ad esempio gli ingegneri, prodotti in quanto
permette di prendere spunto dai sistemi naturali per quelli artificiali.

L'obiettivo di questo studio è quello di studiare attraverso la simulazione del
modello quello che potrebbe essere il comportamento del sistema reale. Questo
permette di evitare di dover effettuare test sul sistema reale, che potrebbero
essere pericolosi o costosi.
\begin{esempio}
      Lo studio delle simulazioni del movimento delle folle di persone permette
      ad esempio di studiare il comportamento in caso di evacuazione di un edificio
      in caso di emergenza senza dover mettere in pericolo le persone.
\end{esempio}
\subsection{Folla di persone come sistema complesso}
La folla di persone è un sistema complesso perché è composto da tanti agenti,
ovvero i pedoni, che possono prendere decisioni individualmente o in gruppo. Oltre
a ciò, il comportamento delle persone può essere influenzato dall'ambiente in
cui si trovano. In generale, il comportamento delle folle è difficile da prevedere
perché è influenzato da molti fattori.

Quando si studia la folla si devono anche considerare situazioni di competizione
per lo spazio condiviso, ma anche di cooperazione per evitare situazioni di
stallo.

Inoltre, i pedoni possono avere stimoli di imitazione verso altri agenti, ad
esempio, se un pedone vede un altro attraversare la strada, potrebbe decidere
di attraversare anche lui. È anche possibile osservare una tendenza a stare a
distanza dagli altri.

Possiamo avere anche dei \textbf{fenomeni emergenti}, ovvero comportamenti che
si osservano a livello globale ma che non sono direttamente riconducibili al
comportamento degli agenti singoli. Ad esempio la ola degli stadi. Questo fenomeno
è risultante da un comportamento aggregato di tanti agenti, non si riconosce la
derivazione dal singolo.
\section{Simulazione}
\begin{definizione}[\textbf{Modello Computazionale}]
      Un \textbf{modello computazionale} è un qualcosa di sufficientemente ben
      definito per essere implementato.
\end{definizione}
\begin{definizione}[\textbf{Simulazione}]
      La \textbf{simulazione} al computer è un modo per sfruttare un modello
      computazionale, con lo scopo di:
      \begin{itemize}
            \item Valutare piani e design prima di effettivamente metterli in
                  pratica nel mondo reale.
            \item Valutare teorie e modelli esistenti di un sistema complesso
                  attraverso l'analisi degli effetti delle scelte modellistiche.
                  Un esempio è la ricerca della cura di una cellula attraverso
                  dei processi chimici, si potrebbe simulare l'ambiente per
                  capire come funziona.
      \end{itemize}
\end{definizione}
L'uso dell'ambiente simulato risulta a volte necessario perché il sistema reale
non può essere osservato, ad esempio quando lo si sta progettando o per motivi
etici o pratici.

Vogliamo ora analizzare il ciclo vita del processo di simulazione riportato in
figura \ref{fig:ciclo_simulazione}. Tale processo è composto dalle seguenti fasi:
\begin{itemize}
      \item Si parte da un sistema reale o da una sua porzione anche detta
            \textbf{Target System}.
      \item Partendo dal target system si costruisce un modello che rappresenta
            il sistema reale. Questo modello può essere di diversi tipi (es.
            fisico, matematico, algoritmico, ad agenti).
      \item Si implementa il modello, creando un simulatore.
      \item Una volta implementato il simulatore lo si esegue la simulazione,
            usando diverse condizioni iniziali. Per ogni esecuzione vengono
            generati dei dati di simulazione.
      \item Ottenuti i dati si analizzano e si confrontano con il target system.
            In questo modo valuto il simulatore utilizzando delle informazioni
            reali. Se i risultati ottenuti da questa analisi sono soddisfacenti
            allora posso usarlo per fini predittivi e di spiegazione.
\end{itemize}
\begin{figure}[!ht]
      \centering
      \includegraphics[width=0.5\textwidth]{./img/sim/ciclobase.png}
      \caption{Ciclo di vita della simulazione}
      \label{fig:ciclo_simulazione}
\end{figure}

Il processo di simulazione che abbiamo analizzato, lo possiamo suddividere nelle
seguenti macro-categorie:
\begin{itemize}
      \item \textbf{Sintesi}: si azzarda una sintesi del sistema reale e si crea
            un simulatore, formalizzo i fenomeni del sistema, ciò mi permette di
            definire indicatori, metriche.
      \item \textbf{Analisi}: si analizzano i risultati ottenuti dal simulatore e
            li confronto con i dati reali per validare i simulatori.
\end{itemize}
\begin{figure}[!ht]
      \centering
      \includegraphics[width=0.7\textwidth]{./img/sim/lifecycle.png}
      \caption{Sintesi e Analisi}
      \label{fig:sintesi_analisi}
\end{figure}

Per quanto riguarda l'analisi delle folle possiamo definire i seguenti livelli:s
\begin{itemize}
      \item \textbf{Livello operazionale}: rappresenta l'insieme di azioni che
            sono definite nel sistema, come ad esempio camminare, aspettare,
            effettuare un'attività, scelta della traiettoria a livello geometrico
            e ad ostacoli. (agente semplice)
      \item \textbf{Livello tattico} (pianifico): si discretizza il livello
            precedente spesso in un grafo, aggiungendo uno scheduling delle
            attività, scelta della strada. È quindi richiesta una base di
            conoscenza.
      \item \textbf{Livello strategico}
\end{itemize}
\begin{figure}[!ht]
      \centering
      \includegraphics[width=0.5\textwidth]{./img/sim/levelsAnalysis.png}
      \caption{Livelli di analisi delle folle}
      \label{fig:livelli_folle}
\end{figure}

Le folle di persone possono essere modellate sotto diversi aspetti, ad esempio:
\begin{itemize}
      \item \textbf{Macroscopico}: modello solamente gli aspetti globali della
            folla, spesso si specificano dei vincoli a livello globale, attraverso
            un sistema di equazioni differenziali. Ha diversi problemi:
            \begin{itemize}
                  \item Gli agenti hanno lo stesso obiettivo e comportamento.
                  \item Risulta difficile considerare gli aspetti dinamici
                        dell'ambiente perché dovremo specificare un nuovo sistema
                        differenziale che modella il secondo stato dell'ambiente
                        e abilitare i singoli sistemi in base alla tempo.
                  \item Non si riescono a considerare tutti gli aspetti dinamici
                        della folla.
                  \item Utile per risolvere problemi di ottimizzazione nei contesti
                        specifici.
            \end{itemize}
            Spesso sono simulazioni approssimative, si fa variare il tempo e si
            prende uno screenshot del modello a due tempi differenti. Buone
            prestazioni computazionali perché sono indipendenti dal numero
            di pedoni, però sono più limitati.
      \item \textbf{Microscopico}: si specifica il modello dei singoli agenti
            secondo la loro architettura e devo tenere traccia degli agenti.
            Serve maggior attenzione sul sistema modellato per renderlo
            compatibile con la versione macroscopica.

            Con questo modello si possono sempre generare le stesse dinamiche
            aggregate della modellazione macroscopica. La modellazione
            microscopica può essere realizzata in diversi modi:
            \begin{itemize}
                  \item \textbf{Particelle}: gli agenti sono rappresentati da
                        particelle. Questa soluzione permette di mantenere la
                        componente fisica, ma si modellano i singoli e non le
                        componenti aggregate. Si specifica una velocità delle
                        particelle e si applicano delle forse su di esse anche in
                        base ai vicini. Le forze sono generate dagli obiettivi e dalle
                        altre particelle.
                  \item \textbf{Automi cellulari}
            \end{itemize}
      \item \textbf{Mesoscopiche}: rappresenta una via di mezzo tra le precedenti
            due. Si modellano i singoli agenti ma si considerano anche le componenti
            aggregate. Si considerano le interazioni tra gli agenti e le componenti
            globali. Si ha però un dettaglio inferiore sulla rappresentazione
            spaziale.
\end{itemize}
Per complicare i simulatori, perché ho parametri liberi, non posso modellare
l'eterogeneità, non posso specificare strutture particolari dell'ambiente \dots
\begin{nota}
      Non esiste un approccio modellistico migliore, dipende tutto da quanto
      conosciamo il fenomeno, gli obiettivi e i dati che abbiamo o misuriamo.
\end{nota}
Il processo di definizione dei simulatori coinvolge diverse fasi, regole e
tipologie di conoscenze. I passaggi tra i diversi livelli di astrazione possono
portare all'introduzione di errori e di incertezze.
\chapter{Simulazione di un sistema complesso basato su Agenti}
Si tratterà dell'esempio di sistema complesso di una folla di persone.

definizoni sui sistemi complessi

si ha tanti elementi interagenti che collaborano, si ha non linearità, una struttura 
gerarchica o connessa, si ha robustezza e plasticità del sistema. Si hanno feedback
positivi o negativi dal sistema, ex: affolamento nel ristorante. Questi feedback
possono essere degli output del sistema che vengono ridati in input per modificare 
l'ambiente, ex: feedback positivi e negativi. Quindi si hanno meccanismi di inibizione 
e stimulazione nel sistema, ex: sistemi sociali.

La ricerca nei sistemi complessi è utile anche per gli ingegneri che possono prendere 
spunto dai sistemi naturali per quelli artificiali.

L'obiettivo dei sistemi complessi è quello di studiare la simulazione del modello
per scoprire proprietà del sistema reale.

Modellare sistemi di folla perchette di effettuare simulazioni sulla sicurezza,
sul posizionamento dei cartelli per luoghi con un afflusso di persone.
L'obiettivo è studiare il comportamento dei pedoni all'interno di un edificio e dello spazio,
è utile studiare le folle perché permette di effettuare dei test sulla macchina.

Il sistema della folla è complesso perché si hanno dei pedoni (agenti), le loro 
decisioni sono individuali ma dipendenti dal contesto (ambiente). Il comportamento 
delle persone è difficile da prevedere perché è influenzato da moltissimi fattori.

Nella folla si ha una competizione per avere lo spazio condiviso, si ha anche una 
cooperazione per evitare situazioni di stallo (ex: norme sociali, lasciare uscire 
le persone dal vagone prima di entrare). I pedoni possono anche avere stimuli 
di imitazione (Ex: attraversare perché lo fa già qualcuno), inoltre si ha una 
tendesta a stare a distanza dagli altri (ex: quando le persone si siedono in un
aula tendono a lasciare dello spazio agli altri). Nel sistema si possono avere anche
dei \textbf{fenomeni emergenti} (ex: Hola degli stadi perché è risultante da un 
comportamento aggregato di tanti agenti, non si riconosce la derivazione dal singolo)
(ex: code a tratti di una strada).

Possiamo avere meccanismi di folla non solo fisici, ma anche i meme su internet.

\begin{definizione}[\textbf{modello computazionale}]
    Un modello computazionale è un qualcosa di sufficientemente ben definito per esere 
    implementato.
\end{definizione}

\begin{definizione}[\textbf{Simulazione}]
    La simulazione al pc è un modo per sfruttare un modello computazionale, con lo 
    scopo di:
    \begin{itemize}
        \item valutare piani di design prima di effettivamente metterli in pratica 
        nel mondo reale
        \item valutare teorie e modelli esistenti di un sistema complesso attraverso l'analisi
        degli effetti delle scelte modellistiche (ex: ricerca della cura di una cellula
        attraverso dei processi chimici, si potrebbe simulare l'ambiente per capire 
        come funziona)
    \end{itemize}
\end{definizione}
L'uso dell'ambiente simulativo è necessario perché spesso il sistema reale è inesistente
 (si sta progettando) o per motivi etici o pratici (ex: studio delle vie di fuga 
 di un'edificio, non posso appiccare un incendio).

Il ciclo vita della simulazione:
\begin{itemize}
    \item si parte da un sottosistema del sistema reale
    \item creo il modello (matematico, algoritmo, ad agenti) legato al sottosistema
    \item implemento il modello, creando un simulatore
    \item si crea una campagna di simulazione, eseguo il simulatore su diverse 
    condizioni iniziali. Per ogni esecuzione genero dei dati di simulazione.
    \item analizzo i risultati ottenuti dal simulatore e li confronto sul sottosistema
    reale. In questo modo valuto il simulatore e se i risultati sono realistici 
    allora posso usarlo per fini predittivi e di spiegazione.
\end{itemize}

Si suddivide quindi il processo di simulazione in:
\begin{itemize}
    \item sintesi: si azzarda una sintesi del sistema reale e si crea un simulatore,
    formalizzo i fenomeni del sistema, ciò mi permette di definire indicatori, metriche. 
    \item analisi: si analizzano i risultati ottenuti dal simulatore e li confronto 
    con i dati reali per validare i simulatori.
\end{itemize}

Analizzando le folle di persone possiamo definire i seguenti livelli:
\begin{itemize}
    \item livello operazionale: insieme di azioni che ci sono nel sitema: camminare
    aspettare, effettuare un'attività, scelta della traiettoria a livello geometrico e 
    ad ostacoli. (agente semplice)
    \item livello tattico (pianifico): si discretizza il livello precedente spesso in un grafo,
    aggiungendo uno scheduling delle attività, scelta della strada. (base di conoscenza)
    \item livello strategico
\end{itemize}

Possiamo modellare 2 aspetti delle folle:
\begin{itemize}
    \item macroscopico: modello gli aspetti globali, spesso si specificano dei 
    vincoli a livello globale (sistema di equazioni differenziali). Ha diversi
    problemi:
    \begin{itemize}
        \item gli agenti hanno lo stesso obiettivo e comportamento
        \item difficile da considerare gli aspetti dinamici dell'ambiente perché
        dovremo specificare un nuovo sistema differenziale che modella il secondo
        stato dell'ambiente e abilitare i singoli sistemi in base alla tempo.
        \item \dots
    \end{itemize}
    Spesso sono simulazioni approssimative, si fa variale il tempo e si prende 
    uno screenshot del modello a due tempi differenti. 
    Buone prestazioni computazionali perchè sono indipendenti dal numero di pedoni,
    però sono più limitati.
    \item microscopico: modello il singolo agente o pedone. Si specifica il modello 
    dei singoli agenti secondo la loro architettura e poi devo tener traccia degli 
    agenti. Serve maggior attenzione sul sistema modellato per renderlo compatibile 
    con la versione macroscopica. Con questo modello si possono sempre generare le 
    stesse dinamiche aggregate della modellazione macrospopica.
    La modellazione microscopica può essere realizzata in diversi modi:
    \begin{itemize}
        \item particelle (gli agenti sono particelle ma permette di mantenere la 
        componente fisica, ma si modellano i singoli e non le componenti aggragate): 
        si specifica una velocità delle particelle e si applicano 
        delle forse su di esse anche in base ai vicini. Le forze sono generate 
        dagli obietivi e dalle altre particelle (chiamate forze sociali).
        \item automi cellulari
    \end{itemize}
    \item mesoscopiche: si hanno gli individui con delle semplificiazioni sui loro 
    comportamenti.
\end{itemize}

Per complichiamo i simulatori, perché ho parametri liberi, non posso modellare 
l'eterogeneità, non posso specificare strutture particolari dell'ambiente\dots
Non esiste un approcco modellistico migliore, dipende tutto da quanto conosciamo 
il fenomeno, gli obientivi e i dati che abbiamo o misuriamo.

Simulare è difficile, passiamo dal reale al SW. Prendiamo un sottosistema, astraiamo il
modello rimuovendo i dettagli reali, creiamo il modello computazionale e poi lo 
implementiamo. Tutte queste fasi possono essere suggette ad errori e inserimenti 
di bayes.
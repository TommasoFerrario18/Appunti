\chapter{Introduzione}
Studieremo i sistemi complessi attraverso la loro rappresentazione sotto forma di
modelli, in questo modo si possono studiare i fenomeni mappandoli come proprietà
sui modelli.

\begin{definizione} [\textbf{Sistema Semplice}]
    Un \textbf{Sistema Semplice} è un sistema avente relazioni di causa-effetto
    note, stabili, ripetibili e predicibili.
\end{definizione}
\begin{esempio}
    Per una bicicletta è banale sapere che se si pedala si avanza.
\end{esempio}
\begin{definizione} [\textbf{Sistema Complicato}]
    Un \textbf{Sistema Complesso} è un sistema semplice ma su scala molto più
    ampia e conoscenze specialistiche diverse.
\end{definizione}
\begin{esempio}
    Una macchina, un autobus e un razzo.
\end{esempio}
\begin{definizione} [\textbf{Sistema Complesso}]
    Un \textbf{Sistema Complesso} è un sistema avente relazioni di causa-effetto
    spesso comprensibili in retrospettiva, non necessariamente facilmente
    riproducibili e predicibili. I sistemi complessi possono, fino ad un certo 
    punto, essere analizzati tramite simulazione.
\end{definizione}
\begin{esempio}
    Organismi viventi, organizzazioni umane.
\end{esempio}

I sistemi complessi hanno diverse proprietà:
\begin{itemize}
    \item \textbf{Non linearità}: date due variabili osservabili nel sistema
          si hanno relazioni di non linearità. (ex: traffico)
\end{itemize}

Problema N-Regine, 2 soluzioni:
\begin{itemize}
    \item centralizzato: genero da 0 una soluzione oppure possono generare una
          configurazione randomica e correggerla. Uso un array al posto di una matrice
          perché le regine non possono stare sulla stessa colonna quindi ogni cella
          rappresenta la riga di una regina.
    \item distribuito: ogni regina è un agente indipendente e la soluzione è l'interno
          sistema.
\end{itemize}

Per valutare l'approccio distribuito non è semplice perché non si ha più un
algoritmo effettivo.

\section{Agent}
\begin{definizione}
    Un agente è un qualsiasi cosa che può essere visto come una percezione nel
    suo ambiente attraverso sensori e azioni sull'ambiente attraverso attuatori.
\end{definizione}

\begin{esempio}
    Un esempio sono gli agenti umani composti da sensori come occhi, orecchie e altri
    organi, mentre mani, gambe, bocca e parti del corpo per attuatori.
\end{esempio}

Gli agenti vedono l'ambiente come una raffigurazione dell'ambiente, ha dei sensori,
percezioni e attuatori. Le azioni non sono interne all'agente, ma le azioni cambiano
lo stato dell'agente. Abbiamo \textbf{funzioni agente} che mappano le percezioni
in azioni.
$$f:P(P)\rightarrow A$$
Il \textbf{programma di un agente} è la specifica configurazione di un'architettura che
produce $f$.
L'agente è composto dall'architettura (ex: Prolog) più il programma (ex: programma in Prolog).
Definiremo un sistema (definizione)
Definiremo un sistema di entità che risolve un problema che interagiscono tra di loro.

Soluzione distribuita di un problema, ex: approccio riconoscimento distribuito
del parlato che sfrutta diverse tecnologie. Un altro esempio è il sistema di diagnosi
medica, il dottore ti manda dagli specialisti. Il problema non è distribuito,
ma bensì è il metodo di risoluzione distribuito.

soluzione per un problema distribuito, ex: problema della logistica, posso quindi
trattarlo in modo centralizzato se faccio centralizzare tutte le informazioni in
un server.
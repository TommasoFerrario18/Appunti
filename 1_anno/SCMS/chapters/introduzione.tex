\chapter{Introduzione}
Studieremo i sistemi complessi attraverso la loro rappresentazione sottoforma di
modelli, in questo modo si possono studiare i fenomeni mappandoli come proprietà
che devono essere dimostrate sui modelli.

\begin{definizione} [\textbf{Sistema Semplice}]
    Un \textbf{Sistema Semplice} è un sistema avente relazioni di causa-effetto
    note, stabili, ripetibili e predicibili.
\end{definizione}

\begin{esempio}
    Una bicicletta è banale sapere che se si pedala si avanza.
\end{esempio}

\begin{definizione} [\textbf{Sistema Complicato}]
    Un \textbf{Sistema Complicato} è un sistema semplice ma su scala molto più ampia
    e conoscienze specialistiche diverse.
\end{definizione}

\begin{esempio}
    Il funzionamento di una macchina, un autobus e un razzo. Sono tutti sistemi 
    facilmente prevedibili, ma ovviamente è difficile trovare una persona che conosce
    in dettaglio il funzionamento di ogni singolo componente, per questo si hanno
    dei specialisti. 
\end{esempio}

\begin{definizione} [\textbf{Sistema Complesso}]
    Un \textbf{Sistema Complesso} è un sistema avente relazioni di causa-effetto
    spesso comprensibili in retrospettiva, non necessariamente facilmente 
    riproducibili e predicibili. I sistemi complessi possono, fino ad un certo punto,
    essere analizzati tramite simulazione. 
\end{definizione}

Si parla di essere simulati fino ad un certo punto perché le simulazioni sono 
limitate, dal momento che non si possono prevere tutte le possibili interferenze che
si hanno all'interno del sistema.

\begin{esempio}
    Ex: organismi viventi, organizzazioni umane.    
\end{esempio}

I sistemi complessi hanno una proprietà fondamentale, ovvero la \textbf{non linearità}.
Più precisamente date due o più variabili osservabili nel sistema, esse hanno 
relazioni di non linearità. (ex: traffico)

In questo corso si passerà dal risolvere i problemi in modo centralizzato alla
risoluzione distribuita basata su agenti.
\begin{esempio}
    Riguarda l'esempio delle $N$-Queen
\end{esempio}

L'approccio distribuito può essere a livello di soluzione, problema o tecninche per
la risoluzione del problema.
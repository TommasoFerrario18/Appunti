\chapter{Risk management}
\section{Definizione del rischio}
\begin{definizione}[\textbf{Risk management}]
    Il \textbf{risk management} è la disciplina che si occupa di identificare,
    gestire e potenzialmente eliminare i rischi prima che questi diventino un
    problema per il successo del progetto.
\end{definizione}
\begin{definizione}[\textbf{Rischio}]
    Definiamo \textbf{rischio} come la possibilità che ci sia un danno.
\end{definizione}
\begin{definizione}[\textbf{Risk exposure}]
    Definiamo \textbf{risk exposure} come una grandezza, per calcolare quanto
    un progetto sia esposto ad un rischio. Viene calcolato come:
    \begin{equation}
        RE = P(UO) \cdot L(UO)
    \end{equation}
    dove:
    \begin{itemize}
        \item $P(UO)$: è la probabilità di un unsatisfactory outcome (UO), ovvero
              la probabilità che effettivamente un danno sia prodotto.
        \item $L(UO)$: è l'entità del danno stesso, ovvero è la perdita per le
              parti interessate se il risultato non è soddisfacente.
    \end{itemize}
    Tanto più un rischio è probabile e tanto più il rischio crea un danno, di
    conseguenza cresce il risk exposure.
\end{definizione}
\begin{definizione}[\textbf{Outcome unsatisfactory}]
    Definiamo \textbf{outcome unsatisfactory} come un risultato negativo che
    riguarda diverse aree:
    \begin{itemize}
        \item L'area relativa all'\textbf{esperienza degli utenti}, con un progetto che
              presenta le funzionalità sbagliate. In questo caso se i problemi sono
              gravi si hanno alti rischi che portano al fallimento del prodotto.
        \item L'area relativa agli \textbf{sviluppatori}, con rischi che possono riguardare
              superamento del budget oppure prolungamenti delle deadlines.
        \item L'area riguardante i \textbf{manutentori}, con rischi che impattano nella
              qualità bassa di software e hardware.
    \end{itemize}
\end{definizione}
Se abbiamo un rischio che produce un outcome unsatisfactory il primo elemento su
cui soffermarsi è lo studio degli eventi che abilitano il rischio, detti
\textbf{risk triggers}.

Possiamo distinguere due principali classi di rischio:
\begin{enumerate}
    \item \textbf{Process-related risks}: rischi con impatto negativo sul processo
          e sugli obiettivi di sviluppo.
    \item \textbf{Product-related risks}: rischi con impatto sul prodotto e su
          obiettivi del sistema funzionali o meno, come fallimenti riguardanti
          la qualità del prodotto o la distribuzione dello stesso.
\end{enumerate}
Entrambe le classi possono portare al fallimento del progetto e quindi vanno
gestite entrambe.
\section{Risk Management}
Bisogna quindi imparare a gestire i rischi. Per fare ciò si utilizza un processo
strutturato in due fasi:
\begin{enumerate}
    \item \textbf{Risk Assessment}: in questa fase si vanno a valutare i rischi
          attraverso:
          \begin{itemize}
              \item \textbf{Risk Identification}: fase di identificazione dei
                    rischi.
              \item \textbf{Risk Analysis}: l'analisi dei rischi identificati
                    nella fase precedente tramite calcolo del \textit{risk
                        exposure} e studio dei \textit{triggers}.
              \item \textbf{Risk Prioritization}: si vanno a definire delle
                    priorità nei rischi analizzati, in modo da concentrarsi sui
                    più pericolosi per poi passare a quelli meno pericolosi.
          \end{itemize}
          Alla fine di questa fase, si produce una lista ordinata sulla pericolosità
          dei rischi.
    \item \textbf{Risk Control}: consiste nel risk management planning, producendo
          piani di controllo di due tipi:
          \begin{enumerate}
              \item \textbf{Piani di management}: per la gestione del rischio
                    prima che si verifichi.
              \item \textbf{Piani di contingency}: per il contenimento di rischi
                    divenuti realtà qualora il piano di management fallisca,
                    sapendo cosa fare a priori in caso di emergenza.
          \end{enumerate}
          Si hanno quindi due sotto-fasi per i due tipi di piani:
          \begin{enumerate}
              \item \textbf{Risk monitoring}.
              \item \textbf{Risk resolution}.
          \end{enumerate}
\end{enumerate}
Queste due fasi vengono ciclicamente ripetute durante il ciclo di vita dello
sviluppo di un software.
\subsection{Risk identification}
In questa fase, si studia come identificare i rischi. Tale operazione è legata
alle competenze degli analisti. Un modo comune di realizzare questa operazione è
mediante l'uso delle \textbf{check-list}, liste che includono un insieme di rischi
plausibili comuni a molti progetti. L'analista scorre tale lista cercando rischi
che possono essere applicati al progetto in analisi.

Alcuni rischi possono verificarsi sempre ma va preso in considerazione solo per
motivi specifici identificabili nel mio progetto. Si hanno altri metodi per
identificare i rischi:
\begin{itemize}
    \item \textbf{Riunioni di confronto}, \textbf{brainstorming} e \textbf{workshop}.
    \item Confronto con altre \textbf{organizzazioni} e con altri \textbf{prodotti}.
\end{itemize}
\subsection{Risk analysis}
L'analisi dei rischi viene realizzata sfruttando l'esperienza e valutando le
reali probabilità che un rischio diventi reale. Come per la fase precedente, si
hanno degli schemi su cui basarsi:
\begin{itemize}
    \item \textbf{modelli di stima dei costi}
    \item \textbf{modelli delle prestazioni} basati su simulazioni, prototipi e analogie con altri progetti
    \item \textbf{check-list}
\end{itemize}

L'analisi dei rischi può anche comportare lo studio delle decisioni da prendere
al fine di minimizzare il risk exposure, scegliendo o meno tra varie opzioni,
scegliendo in modo guidato dai rischi. 

A tal fine si usano i \textbf{decision tree} nei quali la radice rappresenta il problema.
Si hanno di volta in volta i vari scenari, con le stime di probabilità di trovare
un errore critico, di fallimento, di non avere errori. Tali
probabilità verranno usate per il calcolo del risk exposure insieme ad un
quantificatore di $L(UO)$, spesso pari all'effettivo costo che conseguirebbe al
risultato ottenuto. Infine, i vari risk exposure di ogni caso vengono sommati per
ottenere il risk exposure finale. 

Si può fare un'\textbf{analisi di sensitività}
cambiando le percentuali o i costi al fine di capire come comportarsi, riducendo
al minimo il danno e il rischio.

\begin{esempio}[Decision tree]
    Vediamo ora un esempio di come utilizzare un albero di decisione per effettuare
    l'analisi di un rischio figura \ref{fig:tree}.
    \begin{figure}[!ht]
        \centering
        \includegraphics[width=0.5\textwidth]{img/risk/tree.png}
        \caption{Decision tree}
        \label{fig:tree}
    \end{figure}
\end{esempio}

Per ragionare sulle cause dei rischi usiamo il cosiddetto \textbf{risk tree}. Questo
albero ha come radice il rischio. Ogni nodo, detto \textbf{failure node}, è un
evento che si può scomporre in altri eventi, fino alle foglie. La scomposizione
è guidata da due tipi di nodi link:
\begin{enumerate}
    \item \textbf{and-node}: dove i figli di tali nodi sono eventi legati dal un
          and.
    \item \textbf{or-node}: dove i figli di tali nodi sono eventi legati dal un
          or.
\end{enumerate}
\begin{esempio}[Risk tree]
    Vediamo ora un esempio di risk tree figura \ref{fig:risk-tree}.
    \begin{figure}[!ht]
        \centering
        \includegraphics[width=0.5\textwidth]{img/risk/risktree.png}
        \caption{Risk tree}
        \label{fig:risk-tree}
    \end{figure}
\end{esempio}
Nodi $and/or$ vengono rappresentati tramite i simboli delle porte logiche. Dato un
risk tree cerco le combinazioni di eventi atomici che possono portare al rischio.
Per farlo si esegue la \textbf{cut-set tree derivation}, ovvero, partendo dalla radice, si
riporta in ogni nodo la combinazione di eventi che possono produrre il fallimento
e si vanno a calcolare le varie combinazioni degli eventi foglia. Praticamente si
deriva un insieme di eventi non scomponibili sulle combinazioni dell'and.

\begin{esempio}[Cut-set tree]
    Vediamo ora un esempio di cut-set tree figura \ref{fig:cut-set-tree}.
    \begin{figure}[!ht]
        \centering
        \includegraphics[width=0.5\textwidth]{img/risk/cut-set_tree.png}
        \caption{Cut-set tree}
        \label{fig:cut-set-tree}
    \end{figure}
\end{esempio}

\subsection{Risk prioritization}
Bisogna capire quali rischi sono più importanti di altri. Per farlo si pongono i
valori di $P(UO)$ e $L(UO)$ in un range, per esempio, da 1 a 10, ricalcolando il
risk exposure. Una volta fatto si lavora in base al risk-exposure.

\begin{esempio}[Risk-exposure analysis]
    Vediamo ora un esempio del piano cartesiano di analisi della risk-exposure in
    figura \ref{fig:risk-exposure-analysis}. In questo modo è possibile identificare
    delle regioni del grafico con valori particolari di risk-exposure.
    \begin{figure}[!ht]
        \centering
        \includegraphics[width=0.5\textwidth]{img/risk/risk-prioritization.png}
        \caption{Risk-exposure analysis}
        \label{fig:risk-exposure-analysis}
    \end{figure}
\end{esempio}

Si procede disegnando i dati su un piano con $P(UO)$ sull'asse delle y e $L(UO)$
su quello  delle x e posizionandogli eventi su tale piano. Qualora i valori di RE siano
in un range, allora si rappresentano le RE minima e massima. 
Una volta rappresentati gli eventi, posso utilizzare delle curve per
identificare zone di rischio diverse in base al risk exposure in modo da
classificare i vari eventi.

\subsection{Risk control}
Bisogna quindi capire come gestire i rischi. Per ogni rischio bisogna definire e
documentare un piano specifico indicante:
\begin{itemize}
    \item Cosa si sta gestendo.
    \item Come mitigare il rischio e quando farlo.
    \item Di chi è la responsabilità.
    \item Come approcciarsi al rischio.
    \item Il costo dell'approccio al rischio.
\end{itemize}
Anche in questo caso ci vengono incontro \textbf{liste} e \textbf{check-list} con le tecniche di risk
management più comuni in base al rischio specifico. Ci sono comunque strategie generali:
\begin{itemize}
    \item Abbassare la probabilità di realizzazione del rischio, lavorando sulla probabilità 
          dei triggers. ($P(UO) \to 0$)
    \item Lavorare sull'eliminazione del rischio. ($P(UO) = 0$)
    \item Lavorare sulla riduzione delle conseguenze del danno,
          non viene quindi ridotto il rischio. ($L(UO) \to 0$)
    \item Lavorare sull'eliminazione del danno conseguente al rischio. ($L(UO) = 0$)
    \item Lavorare sul mitigare le conseguenze di un rischio, diminuendo l'entità
          del danno. ($L(UO) \to 0$)
\end{itemize}
Bisogna anche studiare le \textbf{contromisure}, da scegliere e attivare in base
alla situazione. Si hanno due metodi quantitativi principali per ragionare
quantitativamente sulle contromisure:
\begin{enumerate}
    \item \textbf{Risk-reduction leverage}: dove si calcola quanto una certa
          contromisura può ridurre un certo rischio, utilizzando la seguente
          formula:
          \begin{equation}
              RRL(r, cm) = \frac{RE(r) - RE(\frac{r}{cm})}{cost(cm)}
          \end{equation}
          dove $r$ rappresenta il rischio, $cm$ la contromisura e $\frac{r}{cm}$
          la contromisura $cm$ applicata al rischio $r$.

          Calcolo quindi la differenza di risk exposure avendo e non avendo la
          contromisura e la divido per il costo della contromisura. La miglior
          contromisura è quella con il RRL maggiore, avendo minor costo e maggior
          efficacia dal punto di vista del risk exposure.
    \item \textbf{Defect detection prevention}: questo metodo confronta le varie
          contromisure, confrontando anche gli obiettivi del progetto, in modo
          quantitativo facendo un confronto indiretto, producendo matrici in cui
          si ragiona in modo indipendente sulle singole contromisure e sui singoli
          rischi ma confrontando anche in modo multiplo.

          Si ha un ciclo a tre step:
          \begin{enumerate}
              \item \textbf{Elaborare la matrice di impatto dei rischi} (\textit{risk
                        impact matrix}) \ref{fig:impact-matrix}: questa matrice
                    calcola l'impatto dei rischi sugli obiettivi del progetto.
                    I valori della matrice variano da 0, nessun impatto, a 1,
                    completa perdita di soddisfazione. Ogni rischio viene
                    accompagnato dalla probabilità $P$ che accada. Ogni obiettivo
                    è accompagnato dal peso $W$ che ha nel progetto.
                    \begin{figure}[!ht]
                        \centering
                        \includegraphics[width=0.5\textwidth]{img/risk/riskImpact.png}
                        \caption{Esempio di matrice di impatto dei rischi}
                        \label{fig:impact-matrix}
                    \end{figure}

                    La \textbf{criticità} di un rischio rispetto a tutti gli
                    obiettivi indicati:
                    \begin{equation}
                        \text{criticality}(r) = P(r) \cdot \sum_{obj}
                        (\text{impact\_matrix}[r, obj] \cdot W(obj))
                    \end{equation}
                    La criticità sale se sale l'impatto e se sale la probabilità
                    del rischio. Un altro dato è la \textbf{perdita} di raggiungimento
                    di un obiettivo qualora tutti i rischi si verificassero:
                    \begin{equation}
                        loss(obj) = W(obj) \cdot \sum_{r}
                        (\text{impact\_matrix}[r, obj] \cdot P(r))
                    \end{equation}
              \item \textbf{Elaborare contromisure efficaci per la matrice}:
                    \ref{fig:eff-matrix} si usa il fattore di criticità del
                    rischio. Viene prodotta una nuova matrice con colonne pari
                    ai rischi e righe pari alle contromisure. I valori saranno
                    le riduzioni di rischio di una contromisura $cm$ sul rischio
                    $r$. La riduzione va da 0, nessuna riduzione, a 1, rischio
                    eliminato.
                    \begin{figure}[!ht]
                        \centering
                        \includegraphics[width=0.5\textwidth]{img/risk/Effectiveness Matrix.png}
                        \caption{Esempio di matrice di efficacia delle contromisure}
                        \label{fig:eff-matrix}
                    \end{figure}

                    Possiamo calcolare la \textbf{combineReduction}, che ci dice
                    quanto un rischio viene ridotto se tutte le contromisure sono
                    attivate:
                    \begin{equation}
                        \text{combineReduction}(r) = 1 - \prod_{cm}(1 -
                        \text{reduction\_matrix}[cm, r])
                    \end{equation}
                    Un altro valore è l'\textbf{overallEffect}, ovvero l'effetto
                    di ogni contromisura sull'insieme dei rischi considerato:
                    \begin{equation}
                        \text{overallEffect}(cm) = \sum_{r} (\text{reduction\_matrix}[cm, r]
                        \cdot \text{criticality}(r))
                    \end{equation}
                    si avrà effetto maggior riducendo rischi molto critici.
              \item \textbf{Determinare il bilanciamento migliore tra riduzione
                        dei rischi e costo delle contromisure}.
                    Bisogna considerare anche il costo di ogni contromisura e
                    quindi si fa il rapporto tra effetto di ciascuna contromisura
                    e il suo costo e scegliendo il migliore.
          \end{enumerate}
\end{enumerate}
Il \textbf{contingency plan} viene attuato qualora il rischio si traduca in realtà.
Si passa quindi al risk monitoring/resolution. Queste due parti sono tra loro
integrate. I rischi vanno monitorati e all'occorrenza vanno risolti il prima
possibile. Tutte queste attività sono costose e si lavora su un insieme limitato
di rischi.
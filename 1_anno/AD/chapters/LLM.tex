\chapter{LLM}
I \textbf{Large Language Model} LLM sono modelli di machine learning che predicono
del testo a seguito di un prompt fornito dall'utente. Questa generazione viene
fatta utilizzando metodi probabilistici, sfruttando meccanismi di attention.

Sfrutta la probabilità condizionata delle parole.

Esistono 2 tipi di modelli:
\begin{itemize}
    \item medium size
    \item very large size
\end{itemize}

Si crede che più parametri permettano al modello di fare cose. In realtà si possono
avere risultati comparabili con meno parametri.

Inoltre spesso i modelli non riescono a rispondere direttamente alla domanda ma
bisogna fare prompt engineering.

Con questi modelli è possibile scrivere query ad un db in linguaggio naturale e
ottenere risposta in linguaggio naturale.

Inoltre i modelli hanno anche la possibilità di ricevere input multi-modale ovvero
diverse tipologie di media.

I limiti sono che sono costosi, sono pesanti, si inviano dati e molto spesso
sbaglia. Inoltre spesso non sono aggiornati e un altro problema è dove prende i
dati (bias, disinformazione, errori, proprietà intellettuale).

Ci sono stati attacchi ai modelli in base ai prompt.

Esistono anche Small model ovvero si utilizza la quantizzazione sui pesi per
perdere qualità ma riduco il peso degli iperparametri. Quindi spesso non serve
avere molti parametri ma bastano queste versioni quantizzate.

Posso prendere i modelli e specializzarli su un particolare dominio effettuando il
fine-tuning, questo può essere fatto:
\begin{itemize}
    \item supervisionato:
    \item non supervisionato
\end{itemize}

In generale l'apprendimento dei modelli può essere fatto in modo supervisionato
e non supervisionato, i dataset possono essere creati usando LLM oppure effettuare
dei labelling.
Si può effettuare il RL.

Utilizzando il fine-tuning significa prendere un modello e modificare un modello
già esistente e riapprendere dai parametri del modello vecchio. Bisogna
stare attenti perché se si modificano troppo si rischia di perdere la conoscenza
iniziale. Questo permette di ridurre i tempi di apprendimento e aumentare le
performance.

Il problema è che non abbiamo delle metriche standard per valutare i modelli,
spesso sono in inglese e spesso confrontano la sintassi. Si può usare una persona
oppure un terzo llm per la valutazione.

Un altro problema dei LLM è l'impossibilità di fornire il diritto all'oblio.

Si è notato che i modelli con finetuning difficilmente sbagliano.

La qualità del dataset impatta molto sul modello, posso avere un dataset ridotto
ma di alta qualità e ottenere risultati notevoli.

Nei LLM non posso utilizzare dati sensibili (context sensitive) si possono utilizzare
modelli RAG che combinano modelli basati su retrieval e modelli generativi
per poter generare dati di contesto. Sono basati da due componenti:
\begin{itemize}
    \item nella fase di retrieval si estraggono le informazioni rilevanti dalla
          conoscenza
    \item nella fase generativa generano nuovi testi basati su un prompt e un
          contesto
\end{itemize}
Esistono diverse tecniche di retrieval TF-IDF, BM25 oppure usare delle tecniche che
che trasformano il linguaggio naturale in linguaggio di query e utilizzare LLM
per trasformare il risultato in linguaggio naturale.

I modelli di retrieval si devono basare su una base di conoscenza spesso composta
da diversi documenti. Questi documenti devono essere partizionati in chunk di
lunghezza fissa, ogni chunk viene rappresentato come un vettore e questo viene
salvato in un db vettoriale.

La fase di retrieval si traduce il prompt in una ricerca dei simili vettori nel
db vettoriale per dare in input il contesto.

Nella base di conoscenza posso inserire qualsiasi cosa rappresentabile tramite
vettori. In aggiunta posso arricchire il chunk con un grafo di conoscenza.
I RAG permettono di utilizzare i dati sensibili per ottenere il contesto dalla base
di conoscenza quindi il modello non vedrà mai i dati sensibili, infatti i dati
sensibili vengono salvati in un db.  (i rag permettono anche di selezionare
le info sensibili da usare dal momento che possiamo implementare meccanismi di
controllo degli accessi)

Posso usare un LLM per costruire un KG e posso usare KG come supporto ai LLM.
LamaIndex è un framework per architetture RAG. Ex: gerarchie di tra vettori\dots
Langchain è un framework per lo sviluppo di applicazioni per LM attraverso una
pipeline di operazioni.
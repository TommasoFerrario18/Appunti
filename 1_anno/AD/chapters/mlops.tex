\chapter{MLOPS}
Prima in machine learning  prendavamo il training set, allenavamo l'argoritmo e 
ottenavamo il modello.

Possiamo pensare che nel mondo reale ci sia una pipeline di gestione e creazione 
dei modelli, un esempio è CRISP-DM che è una pipeline circolare composta 
dai seguenti passi:
\begin{itemize}
    \item comprensione del dominio
    \item data understanding che rappresenti il dominio, oltre a capirli bisogna 
    anche trovarli o produrli
    \item data preparation, ci si mobilita per integrare i dati provenienti da 
    diverse fonti, sistemazione del dataset per gestire dati mancanti, outliers ecc\dots
    \item creazione del modello
    \item pubblicazione del modello
\end{itemize}

% TODO: aggiungi la foto del ciclo

Avere i dati corretti è fondamentale per produrre la qualità del modello, se non avessimo
dati corretti (stessa distribuzione della production) allora si avranno problemi 
di bias. Infatti i dati di training e testing devono derivare dai dati reali che verranno 
dati in fase di production.

Una volta ottenuti i dati reali dovremo separarli in:
\begin{itemize}
    \item training data: dati da utilizzare per creare e valutare il modello
    \item serving data: dati da dare in fase di production
\end{itemize}

successivamente i dati, sia di training, sia serving, devono essere preparati,
successivamente una volta creato il modello dobbiamo valutarlo (ciclo che deve essere
ripetuto). 

Il problema è che nel tempo i dati cambiano o cambia il contesto quindi bisogna 
aggiungere una fase di validation dei dati quando si sta creando il modello.
L'obiettivo è capire le feature significative che fanno variare il modello,
questo serve per poi controllare in futuro se queste feature significative sono 
cambiate. Ovvero si confrontano le distribuzioni delle features ritenute significative 
durante il training con la distribuzione delle stesse feature prese dai dati odierni.
In questo modo possiamo segnalare quando le distribuzioni cambiano e nel caso devo
adottare delle soluzioni:
\begin{itemize}
    \item riaddestrare il modello
    \item avvisare l'utente che qualcosa sta cambiando
    \item non fare niente
\end{itemize}
Oppure spesso quello che succede è che cambiano nel tempo i tipi di dati, quindi, 
in questa fase di validation, bisogna correggere i dati per renderli compatibili col
modello.

Una delle fasi più importanti è Data Acquisition, infatti possiamo avere diversi 
problemi come bias nei dati che invalidano i risultati del modello.

In aggiunta dobbiamo fare Data Provenance ovvero documentare come ho costruito 
il dataset di training e validation, ovvero quali sono i dati utilizzati, da dove 
vengono, come ho costruito il dataset. In questo modo si risolvono eventuali problemi
dovuti a risultati sballati per dati che sono stati iniettati nel dataset.

Successivamente dobbiamo effettuare la Data Exploration che permette di studiare 
le distribuzioni dei valori, gli outliers ecc\dots. (Questa fase si fa in tutti i rettangoli 
dello schema) Questo viene continuamente fatto in CI.

Per effettuare tutti i passi della pipeline servono diverse figure lavorative:
\begin{itemize}
    \item ML expert 
    \item SW engineer
    \item Site Reability Expert engineer: esperto che si occupa di gestire la produzione
\end{itemize}



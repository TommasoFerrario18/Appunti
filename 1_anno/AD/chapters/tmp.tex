\chapter{NoSQL}
\begin{nota}
      In quest capitolo non si vuole sostenere che i database NoSQL siano
      migliori di quelli relazionali. I database NoSQL sono un'alternativa
      ai database relazionali e non sempre sono la scelta migliore. Per ogni
      applicazione è necessario valutare quale sia la scelta migliore.
\end{nota}
Fino a questo momento abbiamo utilizzato database relazionali per gestire le
nostre applicazioni. Tuttavia, i database relazionali non sono l'unica
opzione disponibile. In questo capitolo, esploreremo un'alternativa ai
database relazionali: i database NoSQL.

I database NoSQL sono nati con lo scopo di risolvere alcuni problemi di quelli
relazionali come la scalabilità e la flessibilità. Inoltre, le assunzioni che
si trovano dietro i database relazionali non sono sempre adatte per tutti i
casi d'uso.
\section{Introduzione}
Con il termine NoSQL si intende un insieme di tecnologie che si differenziano
dai database relazionali per il fatto che non utilizzano il modello relazionale
per la gestione dei dati.

Partiamo con il dire che il termine NoSQL significa \textit{Not Only SQL}.

Una caratteristica comune dei database NoSQL è che sono \textbf{schema-free} o
\textbf{schema-less}. Questo significa che non è necessario definire uno schema
prima di poter inserire i dati. L'inserimento dei dati implica per quel dato lo
schema associato, il successivo inserimento di un dato può avere uno schema diverso.

% Presa da internet la definizione
Oltre a ciò, per questi database vale il teorema CAP, che afferma che è
impossibile per un sistema distribuito garantire contemporaneamente le seguenti
tre proprietà:
\begin{itemize}
      \item \textbf{Consistency}: tutti i nodi vedono gli stessi dati allo stesso
            tempo.
      \item \textbf{Availability}: ogni richiesta riceve una risposta, anche in
            presenza di guasti.
      \item \textbf{Partition tolerance}: il sistema continua a funzionare anche
            se alcune parti del sistema non sono disponibili.
\end{itemize}

Mentre nei database relazionali avevamo le transazioni ACID, nei database NoSQL
abbiamo le transazioni BASE, che stanno per:
\begin{itemize}
      \item \textbf{Basically Available}: il sistema è sempre disponibile.
      \item \textbf{Soft state}: lo stato del sistema può cambiare anche senza
            input.
      \item \textbf{Eventual consistency}: il sistema diventerà consistente in un
            certo momento.
\end{itemize}
\section{Tipi di database NoSQL}
In questo corso vedremo i seguenti tipi di database NoSQL:
\begin{itemize}
      \item \textbf{Document-based}: i dati sono memorizzati in documenti, che
            possono essere in formato JSON, XML, BSON, YAML, etc.
      \item \textbf{Key-value}: i dati sono memorizzati in coppie chiave-valore.
      \item \textbf{Wide-column}: i dati sono memorizzati in colonne, simile ai
            database relazionali.
      \item \textbf{Graph-based}: i dati sono memorizzati in nodi e archi.
\end{itemize}
\subsection{Key value}
I database key-value sono i più semplici tra i database NoSQL. In questi
database, i dati sono memorizzati come tabelle di hash dove la chiave punta a un
particolare valore. Si utilizzano le tabelle di hash per massimizzare le prestazioni
di lettura e scrittura.
\subsection{Wide column}
I database wide-column sono simili ai database relazionali, ma invece di
memorizzare i dati in righe, i dati sono memorizzati in colonne. Questo permette
di avere una maggiore flessibilità rispetto ai database relazionali. La chiave 
punta a un insieme di colonne che può essere diverso per ogni riga.
\subsection{Document based}
I database document-based memorizzano i dati in documenti. I documenti sono 
indirizzati nel database tramite una chiave unica. La ricerca dei dati è
effettuata nei documenti stessi.
\subsection{Graph based}
I database graph-based memorizzano i dati in nodi e archi. Questi database sono
utili per memorizzare dati che hanno relazioni complesse. I nodi rappresentano
le entità e gli archi rappresentano le relazioni tra le entità.
\section{Confronto}
I database NoSQL sono una buona alternativa ai database relazionali, ma non
sono adatti per tutti i casi d'uso. I database relazionali sono adatti per
applicazioni che richiedono transazioni ACID e che hanno bisogno di una
struttura ben definita. I database NoSQL sono adatti per applicazioni che
richiedono scalabilità e flessibilità.

Una differenza fondamentale rispetto al modello relazionale che mette tutti i 
dati allo stesso livello di importanza, i modelli NoSQL mettono un concetto ad 
un livello di importanza maggiore.
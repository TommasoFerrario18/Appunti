\chapter{NoSQL}
\begin{nota}
      In quest capitolo non si vuole sostenere che i database NoSQL siano
      migliori di quelli relazionali. I database NoSQL sono un'alternativa
      ai database relazionali e non sempre sono la scelta migliore. Per ogni
      applicazione è necessario valutare quale sia la scelta migliore.
\end{nota}
Fino a questo momento abbiamo utilizzato database relazionali per gestire le
nostre applicazioni. Tuttavia, i database relazionali non sono l'unica
opzione disponibile. In questo capitolo, esploreremo un'alternativa ai
database relazionali: i database NoSQL.
\section*{Use Cases}
\subsection*{Crypto}
Iniziamo analizzando il seguente Use Case: un cliente vuole salvare dati legati 
alle crypto che vengono forniti da un API. Con le conoscenze conseguite fino a 
questo momento, salveremo tutto su una tabella di un DBMS relazionale dove si 
hanno 5 colonne:
\begin{itemize}
      \item id
      \item Simbolo
      \item Prezzo\_USD
      \item Prezzo\_EUR
      \item Data
\end{itemize}
Questa soluzione presenta dei problemi, ad esempio, se si volesse salvare
informazioni aggiuntive o se la risposta dell'API si dovesse modificare, si
dovrebbe modificare la tabella. Un altro problema sorge quando nella risposta 
sono contenuti elementi complessi, come ad esempio un array di oggetti. In questa 
situazione, usando un database relazionale, si dovrebbero creare tabelle diverse 
e ogni volta che si vuole fare una nuova interrogazione è necessario fare delle 
join tra le tabelle. Questo comporta un rallentamento delle prestazioni.

Una soluzione per ridurre il numero di join è quello di duplicare i dati, il
problema in questo caso non è più legato alla velocità delle interrogazioni, ma
al mantenere la consistenza dei dati. 

In aggiunta posso ridurre le tabelle unendole creando una grande tabella che può 
essere frammentata a livello fisico, il problema di questa soluzione è che si 
vanno a inserire un elevato numero di valori \texttt{null}.
\subsection*{Social network}
Un altro caso d'uso è il social network in cui si vuole salvare la relazione di
amicizia tra profili, il problema è che se si avessero 1M di utenti, ciascuno con
100 amici, allora la tabella amicizia diventa 100M di record. Troppa complessità
soprattutto per una tabella che verrà usata spesso con delle join.

Il DB relazionali non sono in grado di modellare efficientemente i contesti applicativi
con volumi molto significativi di dati.
\section{Introduzione}
Da questo use case possiamo osservare che il modello relazionale presenta dei
limiti nella gestione dei dati:
\begin{itemize}
      \item Si ha uno svantaggio nel salvare di dati poco omogenei, questo è dovuto
            principalmente alla rigidità dello schema dei modelli relazionali, in
            quanto prima di poter inserire i dati, è necessario definire la loro
            struttura.
      \item Si ha uno svantaggio in relazione al paradigma di programmazione che
            viene usato (Programmazione a oggetti). Questo obbliga a utilizzare
            un ORM per mappare i dati del database relazionale con gli oggetti
            del linguaggio di programmazione.
\end{itemize}
Inoltre, essendo le risposte delle API solitamente in formato JSON, si ha un
ulteriore svantaggio in quanto si deve fare un parsing della risposta per
poterla salvare nel database relazionale.

I database NoSQL sono nati con lo scopo di risolvere alcuni problemi di quelli
relazionali come la scalabilità e la flessibilità. Inoltre, le assunzioni che
si trovano dietro i database relazionali non sono sempre adatte per tutti i
casi d'uso.

Con questo non vogliamo dire che i database NoSQL siano migliori di quelli
relazionali, ma che sono un'alternativa ai database relazionali. Per ogni
applicazione è necessario valutare quale sia la scelta migliore. Di seguito
elenchiamo alcuni motivi per cui utilizzare i database relazionali:
\begin{itemize}
      \item SQL è semplice.
      \item Molto rigido
      \item Vincoli nel database allora le app possono non fare i controlli
      \item ipotesi di mondo chiuso: tutto quello che serve è nel database, se non
            lo so devo inserire il valore \texttt{null}.
      \item Tecnologia stabile e funzionante che alle spalle anni di ricerca e
            sviluppo.
      \item Sono valide le proprietà ACID, molto utili in contesti in cui si
            hanno transazioni.
\end{itemize}
È corretto presentare anche gli svantaggi dei database relazionali:
\begin{itemize}
      \item Modificare una tabella nella quale sono già presenti dei valori è molto
            complicato.
      \item Assunzione mondo chiuso: può risultare troppo pesante in determinate
            situazioni.
      \item Il concetto di minimizzare le ripetizioni dei valori implica la necessità
            di effettuare molte join per ottenere l'informazione desiderata.
      \item Si ha un mapping un attributo ha un valore.
      \item Non è compatibile con i linguaggi di programmazione ad oggetti.
      \item Presenta delle difficoltà nella gestione dei self-join.
      \item È difficilmente scalabile.
\end{itemize}
\begin{nota}
      Il termine NoSQL significa \textit{Not Only SQL}.
\end{nota}
Una caratteristica comune dei database NoSQL è che sono \textbf{schema-free} o
\textbf{schema-less}. Questo significa che non è necessario definire uno schema
prima di poter inserire i dati. L'inserimento dei dati implica per quel dato lo
schema associato, il successivo inserimento di un dato può avere uno schema diverso.

% Presa da internet la definizione
Oltre a ciò, per questi database vale il \textbf{teorema CAP}, il quale afferma
che è impossibile per un sistema distribuito garantire contemporaneamente le
seguenti tre proprietà:
\begin{itemize}
      \item \textbf{Consistency}: tutti i nodi vedono gli stessi dati allo stesso
            tempo.
      \item \textbf{Availability}: ogni richiesta riceve una risposta, anche in
            presenza di guasti.
      \item \textbf{Partition tolerance}: il sistema continua a funzionare anche
            se alcune parti del sistema non sono disponibili.
\end{itemize}
Mentre nei database relazionali avevamo le transazioni \textbf{ACID}, nei database
NoSQL abbiamo le transazioni \textbf{BASE}, che stanno per:
\begin{itemize}
      \item \textbf{Basically Available}: il sistema è sempre disponibile.
      \item \textbf{Soft state}: lo stato del sistema può cambiare anche senza
            input.
      \item \textbf{Eventual consistency}: il sistema diventerà consistente in un
            certo momento.
\end{itemize}

Inoltre, a differenza dei database relazionali, i database NoSQL si basano su
un'assunzione di mondo aperto, ovvero viene inserito solo quello che conosco se
mi mancano delle informazioni non le metto.

Si è passati dal progettare le architetture in modo da essere più specifiche
per l'applicazione, il contrario di come si faceva una volta: progettazione generalista.
\section{Tipi di database NoSQL}
Esistono diverse tipologie di database NoSQL, ognuna con le proprie caratteristiche
e i propri casi d'uso. Possiamo dire che più è semplice il modello, più è facile
scalarlo. Nella figura \ref{fig:tipi_nosql} possiamo osservare come si posizionano i
database NoSQL rispetto alla dimensione e complessità del modello.
\begin{figure}[!ht]
      \centering
      \includegraphics[scale=0.5]{./img/nosql/tipi_nosql.png}
      \caption{Tipi di database NoSQL}
      \label{fig:tipi_nosql}
\end{figure}

Rispetto questa rappresentazione possiamo posizionare il modello relazionale tra
il modello column-based e il modello document-based.

In questo corso vedremo i seguenti tipi di database NoSQL:
\begin{itemize}
      \item \textbf{Document-based}: i dati sono memorizzati in documenti, che
            possono essere in formato JSON, XML, BSON, YAML, etc.
      \item \textbf{Key-value}: i dati sono memorizzati in coppie chiave-valore.
      \item \textbf{Wide-column}: i dati sono memorizzati in colonne, simile ai
            database relazionali.
      \item \textbf{Graph-based}: i dati sono memorizzati in nodi e archi.
\end{itemize}
Una caratteristica che accomuna queste tipologie di database è legata al problema
di mettere insieme i dati. Ad esempio, i database relazionali usano le operazioni
di join, mentre i database document-based usano un unico documento.
\subsection{Key value}
I database \textbf{key-value} sono i più semplici tra i database NoSQL. In questi
database, i dati sono memorizzati come tabelle di hash dove la chiave punta a un
particolare valore. Si utilizzano le tabelle di hash per massimizzare le prestazioni
di lettura e scrittura.
\subsection{Wide column}
I database \textbf{wide-column} sono simili ai database relazionali, ma invece di
memorizzare i dati in righe, i dati sono memorizzati in colonne. Questo permette
di avere una maggiore flessibilità rispetto ai database relazionali. La chiave
punta a un insieme di colonne che può essere diverso per ogni riga.

\paragraph{HBASE}
HBase è un database wide-column che è stato ispirato da Google BigTable. In HBase
si ha una chiave che punta ad un insieme di valori che possono avere tipi differenti.

In questa struttura, possiamo definire delle \textbf{column family}, le quali
contengono più attributi che condividono delle informazioni. Il \textbf{qualifier} è
il nome dell'attributo all'interno di una column family. Ogni riga ha anche
un campo relativo al timestamp in cui è stata modificata.

L'utilizzo di timestamp fornisce un modo diverso per gestire le transazioni,
anche quelle distribuite, senza l'uso di lock. Questo metodo associa a ogni
transazione un timestamp, in modo che se due transazioni si sovrappongono,
possono essere risolte in base a tale valore.

In questa implementazione è possibile realizzare facilmente relazioni di tipo 1-n
nella stessa riga.
\paragraph{Cassandra}
Cassandra è un database wide-column che è stato realizzato da Facebook per gestire
la comunicazione con le mail.

Questo database si differenzia da HBase per il fatto si ha uno spazio delle chiavi,
il quale può essere visto come una tabella nel modello relazionale. Questo spazio
è suddiviso in column family, le quali sono differenti da quelle di HBase in quanto
la stessa chiave primaria può essere usata in più column family.

I valori che si trovano all'interno di una column family possono essere di tipo
diverso, inoltre, ogni riga ha un timestamp associato.

Quindi una column family equivale ad una \textit{tabella}, e si utilizza la chiave
primaria della riga per accedere ad un particolare dato di essa.

L'accesso viene effettuato specificando i seguenti campi:
\begin{itemize}
      \item \texttt{column family}
      \item \texttt{row key}
      \item \texttt{column name}
\end{itemize}
Per semplificare il passaggio da database relazionali a Cassandra, il query
language di Cassandra (CQL) è molto simile a SQL. Attraverso questo meccanismo
è possibile sviluppare un applicazione che è in grado di utilizzare sia database
relazionali che Cassandra senza apportare molte modifiche.

Un vantaggio dell'utilizzo di questo database è legato al fatto che ogni riga è
indipendente dalle altre. Quindi, in un contesto distribuito, è possibile frammentare
come si vuole il database senza dover fare join.
\subsection{Document based}
I database \textbf{document-based} permettono di memorizzare delle informazioni sotto
forma di documenti. Ognuno dei quali è identificato attraverso una chiave unica.
Tutte le operazioni di lettura e manipolazione dei dati avvengono all'interno
dei documenti stessi.
\paragraph{MongoDB}
MongoDB è un database document-based che memorizza i dati in documenti JSON, più
precisamente, utilizza un formato binario di JSON chiamato BSON. Ogni documento
è identificato da una chiave unica, chiamata \texttt{\_id}.

Possiamo immaginare la struttura di un file JSON come un albero, dove la radice
è la chiave \texttt{\_id} e i figli sono i campi del documento.

Questa tipologia permette di salvare i dati come si faceva negli schemi relazionali,
ovvero suddividendo le informazioni in diverse \textbf{collezioni}. Per fare ciò,
si utilizzano i \textbf{riferimenti} tra i documenti. Lo svantaggio di questa
soluzione è che si devono fare molte join per ottenere l'informazione desiderata.

Un'altra soluzione è quella di salvare i dati in un unico documento, sfruttando
l'embedding, in modo da evitare le join. Questo metodo è molto efficiente quando
si devono accedere alle informazioni più vicine alla radice, ma meno efficiente
quando si devono accedere alle foglie, si ha comunque un tempo di accesso minore
rispetto alle join.

Nei database document-based lo schema non è vincolante, in particolare, corrisponde
all'unione di tutti gli attributi e sotto-attributi del documento.

Vediamo ora un parallelismo tra i database relazionali e MongoDB:
\begin{itemize}
      \item \textbf{Database} $\rightarrow$ \textbf{Database}
      \item \textbf{Tabella} $\rightarrow$ \textbf{Collection}
      \item \textbf{Row} $\rightarrow$ \textbf{Document}
      \item \textbf{Column} $\rightarrow$ \textbf{Field}
      \item \textbf{Index} $\rightarrow$ \textbf{Index}
      \item \textbf{Join} $\rightarrow$ \textbf{Embedded document}
      \item \textbf{Foreign key} $\rightarrow$ \textbf{Reference}
      \item \textbf{Partition} $\rightarrow$ \textbf{Shard}
\end{itemize}
Le operazioni che si vogliono svolgere sulla base di dati vengono espresse
attraverso una notazione che ricorda quella della programmazione a oggetti, ovvero
attraverso la \textit{dot notation}. Inoltre, nelle operazioni tutti gli
operatori devono essere espressi in JSON. Questo permette di eliminare l'utilizzo
di framework come Hibernate per mappare gli oggetti sul database.
\begin{nota}
      Quando si devono caricare grandi quantità di dati è utile disabilitare
      l'ack che viene mandato per ogni operazione.
\end{nota}
Le query di modifica di un documento presentano degli oggetti JSON come
clausole, simile alla clausola \texttt{WHERE} delle query SQL. Oltre a questo
è possibile specificare di creare un documento nuovo nel caso non sia presente
quello che si vuole modificare.

Mongo permette di creare degli indici per velocizzare le operazioni.
\begin{esempio}
      Vediamo ora un esempio di parallelismo tra una query SQL e una per
      MongoDB:
      \begin{center}
            db.\textcolor{red}{users}.find(\textcolor{green}{\{age: \{ \$gt: 18 \}\}},
            \textcolor{cyan}{\{name: 1, age: 1\}}).\textcolor{magenta}{limits(10)}
      \end{center}
      \begin{center}
            \textcolor{cyan}{SELECT name, age} \textcolor{red}{FROM users}
            \textcolor{green}{WHERE age $>$ 18} \textcolor{magenta}{LIMIT 10}
      \end{center}
\end{esempio}
Nelle interrogazioni, tutte le clausole sono specificate in \texttt{AND}, se si
vuole specificare in \texttt{OR} è necessario utilizzare il comando $\$or$ seguito
da una lista di oggetti JSON.

Nelle versioni successive è stato definito il concetto di \textbf{aggregation} e
di \textbf{pipeline}. Una pipeline è un comando che rappresenta una sequenza di
operazioni che vengono eseguite in serie e possono essere usate per operazioni
di analytics.

\begin{nota}
      Dato un problema non sempre per avere efficienza dobbiamo scegliere la
      soluzione banale, spesso dobbiamo prendere soluzioni controintuitive. Ex:
      salvare i dati non in modo efficiente e sfruttare la potenza di esecuzione
      delle query dei DBMS.
\end{nota}
\begin{nota}
      Un documento è un albero.
\end{nota}
\subsection{Graph based}
I database graph-based memorizzano i dati salvando le informazioni nei nodi e
utilizzando gli archi per definire delle relazioni tra esse. Questi database sono
utili per memorizzare dati che hanno relazioni complesse, come ad esempio le
relazioni dei social network.

Spesso si usano per reti sociali, recommendations systems, logistica, financial
transaction, bioinformatica, autorizzation and access control.

I database basati sui grafi godono delle seguenti proprietà:
\begin{itemize}
      \item \textbf{Storage}:
            \begin{itemize}
                  \item \textbf{Nativo}: i dati vengono salvati sotto forma di
                        grafo, si ha la problematica legata alla scalabilità 
                        della soluzione.
                  \item \textbf{Non nativo}: i dati vengono salvati in una struttura 
                        differente da un grafo, ma che logicamente permette di essere
                        visto come un grafo.
            \end{itemize}
      \item \textbf{Processing engine}:
            \begin{itemize}
                  \item \textbf{Nativo}: permette di effettuare direttamente query
                        nel linguaggio dei grafi, sfruttando le proprietà dei 
                        grafi. Il problema di questa soluzione è dovuto al fatto 
                        che bisogna mappare le query reali utilizzando le proprietà 
                        del grafo.
                  \item \textbf{Non nativo}: permette di utilizzare un altro 
                        linguaggio (es. SQL) per effettuare query che non devono 
                        per forza essere proprietà del grafo.
            \end{itemize}
\end{itemize}
Lo storage nativo, spesso si basa sull'\textbf{index friend adjacency}, ovvero 
i nodi che sono vicini sono fisicamente salvati insieme. Questo permette di 
effettuare query molto velocemente, ma la costruzione del grafo è molto lenta.

Il collegamento tra concetti si effettua attraverso un cammino tra nodi, mentre 
nel relazionale si effettua tramite valori, invece nel documentale si usa l'embedding.

La struttura a grafo dei dati permette di rappresentare le informazioni 
attraverso un database documentale, ma essendo la base di tale modello una 
struttura ad albero risulta più complesso mappare le relazioni tra i dati. 
Per risolvere questo problema si utilizzando i riferimenti tra vari documenti.

Un modo efficiente per modellare un grafo in modo non nativo è quello di sfruttare 
i vicini. Ad esempio, in un modello documentale possiamo utilizzare l'embedding
per memorizzare i nodi vicini, mentre per i nodi lontani possiamo utilizzare i
riferimenti.

Una seconda soluzione è quella di usare relazioni differenti nel grafo per aumentare
l'espressività del grafo e semplifica i collegamenti. Questo può essere implementato
facilmente.

\begin{esempio}
      Metti l'esempio del confronto delle interrogazioni tra modelli.
\end{esempio}

Nel modello a grafo, ogni nodo rappresenta le informazioni al suo interno con 
delle proprietà, ovvero coppie chiave-valore dove la chiave è rappresentata 
dall'etichetta e il valore rappresenta l'informazione. Anche le relazioni possono
avere un etichetta a cui è associata una proprietà.

Una stessa informazione può essere modellata in modi diversi, la scelta dipende
dalle query che si vogliono effettuare. Le due soluzioni sono:
\begin{itemize}
      \item \textbf{Stessa etichetta}: si ha una sola etichetta per accedere
            a tutte le informazioni, mentre all'interno della relazione si specifica
            il tipo di informazione.
      \item \textbf{Etichette multiple}: si ha una relazione per ogni tipo di
            informazione.
\end{itemize}
Ad esempio, per modellare una situazione in cui a una persona sono associati più
indirizzi, si possono utilizzare le due soluzioni, la scelta dipende dalle 
tipologie di query che si vogliono effettuare. Se si vogliono trovare tutti 
gli indirizzi di una persona, allora è meglio utilizzare una soluzione in cui 
si ha una stessa etichetta per accedere a tutti gli indirizzi e poi all'interno 
della relazione si specifica il tipo di indirizzo. Mentre, se si vogliono 
effettuare query per sapere uno specifico tipo di indirizzo è più conveniente
utilizzare una relazione per ogni tipo di indirizzo. 

La modellazione tramite grafo risulta semplice a livello semantico da costruire,
il problema è che non si ha uno standard per il Query language, eccone alcuni:
\begin{itemize}
      \item cypher: linguaggio dichiarativo può effettuare aggregation, ordering
      e limit. L'obiettivo di cypher è fare pattern matching, ovvero cerca tutti 
      i sottografi che metchano con grafo di query. (esattamente come in prolog)

      Il risultato di una query grafo può essere una tabella o un sottografo.
      \item gremlin (graph traversal language): la query consiste nel specificare 
      un percorso, il percorso si inizializza in tutti i nodi e la query specifica
      dei constrain su archi e nodi per selezionare quali nodi attraversare. 

      Questo linguaggio è pensato per essere eseguito ad agenti, quindi è parallelizzabile
      ed eseguibile in ambienti distribuiti.
\end{itemize}

Un esempio è ArangoDB che supporta i seguenti modelli:
\begin{itemize}
      \item keyvalue
      \item graph
      \item document
      \item ricerca full text
\end{itemize}
Tutti i modelli sono gestiti come documenti JSON, il chiave valore è un documento 
chiave valore, document è banale, mentre i grafi si specificano in documenti JSON
mentre gli archi come reference a documenti quindi è pesante a causa dei Join.

\section{CAP theorem}
Nota che i sistemi RDBMS hanno un'upperbound di scalabilità fisico e teorico. I 
sistemi NoSQL sono stati implementati rispettando il CAP theorem e le proprietà
BASE.

\begin{teorema} [\textbf{CAP}]
      In un sistema distribuito non è possibile che contemporaneamente si hanno 
      le seguenti proprietà garantite:
      \begin{itemize}
            \item consistenza (Consistency)
            \item disponibilità (Availability)
            \item tolleranza alle partizioni (Partision tolerance)
      \end{itemize}      
      Il sistema può soddisfarne al più due contemporanemente.
\end{teorema}

Questo ha permesso di sviluppare DBMS che non rispettano le 3 proprietà.  Noi 
dovremmo scegliere con consapevolezza.

I principi BASE sono:
\begin{itemize}
      \item basic Availability: ricerve le richieste anche sotto partial consistency
      \item soft state:
\end{itemize}


Redis offre meccanismi di trasparenza automatici di replica per avere alta Availability.

HBase ha un modello Master Slave, il master conosce la locazione dei blocchi distribuiti
e poi si hanno gli slave che sono le macchine che contengono i dati frammentati.
Si implementa la frammentazione orizzontale e la replica per distribuire tra gli
slave. Si ha un orchestratore (zookeeper) che risolve problemi di fallimento.

Cassandra ha un'architettura peer to peer ad anello per ogni datacenter. Tutti i nodi sono collegati in modo da 
avere massima Availability e partition tolerance. In caso di guasto non ci sono 
problemi perché i dati sono replicati, ogni dato è replicato almeno 3 volte. I
vengono orchestrati con zookeeper. La frammentazione si decide sfruttando le funzioni
di hashing ovvero si mappa l'anello con valori da $0$ a $1$ e si usano le funzioni 
di hashing per salvare il dato in un nodo rispetto ad un altro. Più precisamente 
se l'hash della chiave mi restituisce un valore tra i nodi A-B allora inserisco il 
dato in B. Se aggiungo un nuovo nodo allora i nuovi dati che cadono tra il nodo 
precedente e il nuovo nodo verranno salvati su quello nuovo. Quelli che sono presenti 
nel nodo successivo che devono essere salvati in quello nuovo allora si replicano 
in quello nuovo. 
Dal momento che non si ha più un'architettura master-slave ma a cluster allora 
si usa il gossip protocol per far capire lo stato dei nodi nel cluster.
Per identificare il fallimento si utilizza una probabilità di fallimento che viene aggiornata 
in base alla latenza della rete. Questo perché il timeout è troppo stringente.
La scrittura avviene in questo modo:
\begin{itemize}
      \item scrivo nel LOG
      \item scrivo in memoria 
      \item scrivo su disco
\end{itemize}

Quando il client vuole scrivere nel cluster si deve specificare il grado di 
consistenza dei dati:
\begin{itemize}
      \item consistency one: si ritorna la risposta al client dopo una scrittura 
      su disco solo su un nodo del cluster
      \item consistency all: si ritorna la risposta al client dopo una scrittura 
      su disco solo su tutti i nodi del replication factor nodi del cluster
      \item consistency quorum: si ritorna la risposta al client dopo $n/2+1$ nodi che 
      hanno scritto su disco
\end{itemize}

La cancellazione avviene disabilitando \dots

L'allineamento

La lettura

MongoDB è CP, anche qui ha un'architettura master e slave. Ogni dato è frammentato
e replicato, si hanno nodi shard che sono slave, istanze diverse di mongo. Il master 
fa da orchestratore, se cade il master allora si promuove il primo slave. 

Voting

Frammentazione si 
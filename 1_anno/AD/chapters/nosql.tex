\chapter{NoSQL}
Parleremo dei modelli e successivamente delle tecnologie che li implementano.

Un coglione ci paga per salvare dati legati alle cripto su db nosql. In sostanza
si effettua una chiamata all'API che restituisce i valori in EUR e USD di bitcoin
ed etherium. Attualmente salveremo tutto su una tabella di un DBMS relazionale.
Per casi più complessi con risposte di query più complesse risulta essere molto
complesso gestire il salvataggio dei risultati. soprattutto quando si hanno risposte
molto complesse perché salvare dati in tante tabelle significa poi che l'accesso
porta all'utilizzo di join e quindi risulta essere molto complessa la sua gestione.

Posso ridurre le join duplicando i dati, il problema è che i dati che si duplicano
devono essere costanti. In aggiunta posso ridurre le tabelle unendole 
creando una grande tabella che può essere frammentata a livello fisico, il problema 
è che ho tanti null.

Nel relazionale ci sono tanti limiti:
\begin{itemize}
    \item salvataggio di dati poco omogenei
    \item il codice viene scritto OOP
    \item il db è in tabelle
    \item si dovrà utilizzare ORM per mappare OOP in db
\end{itemize}

Si aggiunge molta complessità per salvare una risposta JSON.

Un altro caso d'uso è il social network in cui si vuole salvare la relazione di 
amicizia tra profili, il problema è che se si avessero 1M di utenti, ciascuno con 
100 amici, allora la tabella amicizia diventa 100M di record. Troppa complessità
soprattutto per una tabella che verrà usata spesso con delle join.

Il DB relazionali non sono in grado di modellare efficientemente i contesti applicativi
con volumi molto significativi di dati.

I vantaggi dei relazionali:
\begin{itemize}
    \item SQL è semplice.
    \item molto rigido
    \item vincoli nel DB allora le app possono non fare i controlli
    \item ipotesi di mondo chiuso: tutto quello che serve è nel DB, se non lo so possono
    specificare null
    \item tecnologia stabile e funzionante
    \item proprietà ACID
\end{itemize}
Gli svantaggi dei relazionali:
\begin{itemize}
    \item cambiare un tipo di un attributo della tabella quando ci sono i dati 
    è molto complicato.
    \item assunzione di chiusura del mondo: troppo pesante
    \item minimizzazione dei valori questo implica molte join
    \item un attributo ha un solo valore
    \item non è compatibile con i linguaggi di programmazione
    \item non supporta i cicli nei dati
    \item difficile modificare le tabelle
    \item non è scalabile
\end{itemize}

I DB NoSQL non esiste lo schema, non si effettua più la progettazione dello schema.
I dati rappresentano lo schema. Questo porta flessibilità nei dati e nel salvataggio,
il problema è che non si conosce cosa c'è dentro.

Si ha un'assunzione di mondo aperto, inserisco solo dati che conosco.

Si hanno proprietà BASIC al contrario di ACID.

Più è semplice il modello allora più è facile scalarlo. Il modello relazione 
è a metà tra la complessità delle colonne e i documenti.

Questi modelli devono risolvere il problema di mettere insieme i dati. EX: relazionale 
usa i join, mentre per i documenti abbiamo un unico documento.

Si è passati dal progettare le architetture in modo da essere più specifiche 
per l'applicazione, il contrario di come si faceva una volta: progettazione generalista.



\section{Key value}
Si salva solo chiave-valore. Molto efficiente, la chiave viene utilizza con delle 
funzioni di hash.

Modello molto semplice, i valori non sono accessibili (blob). Ad ogni valore si 
può associare il tipo al valore. Usato per tenere traccia del carrello utilizzando 
la chiave della sessione, essendo semplice si può mantenere tutto in memory.

\section{Wide-column}
La chiave porta a colonne multiple senza schema fisso. Ogni riga può essere 
diversa dalle altre

\section{Document based}
Si indicizzano dei documenti nel db tramite una chiave unica, per ogni documento
si può avere una chiave che indicizza ad un altro documento.
\section{Graph}
Potente ma semplice.


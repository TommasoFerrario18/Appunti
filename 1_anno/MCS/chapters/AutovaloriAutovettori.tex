\chapter{Autovalori e autovalori}

\begin{definizione} [\textbf{Autovalore e autovettore}]
    Data una matrice $A\in \mathbb{R}^{n\times n}$ un autovalore di una matrice 
    e l'autovettore associato 
    sono una coppia $(\lambda, \underline{v}_\lambda)\in \mathbb{C}\times\mathbb{C}^{n}$
    tale che 
    $$A\underline{v}_\lambda = \lambda \underline{v}_\lambda $$
\end{definizione}

Possiamo sfruttare la norma di matrice $\|A\|=\max _{\underline{x}_\lambda\in \mathbb{R}^n}\frac{\|A\underline{x}_\lambda\|}{\|\underline{x}_\lambda\|} $

Quindi possiamo dire come sono fatti gli autovalori 
% TODO: scrivi procedimento
$|\lambda| \le \|A\|$. Il problema di questa caratterizzazione è che non è facilmente 
computabile per $A$ molto grandi.

\begin{teorema} [\textbf{Greshgorin}]
    Definniamo $\sigma(A):=\{\lambda | \lambda\text{ è autovalore di }A\}$ (è lo 
    spettro della matrice). Definiamo $R_j:=\{z\in \mathbb{C}| |z-A_{jj}| \le \sum_{l=1, l\ne j}^{n} |A_{jl}|\}$.
    Allora $\sigma(A)\subseteq \bigcup_{j=1}^n R_j$. 
    \begin{proof}
        Sia $\lambda = A_{jj}$ per qualche $j=1\dots n$ allora $\lambda$ è all'interno
        della palla.
        
        Quindi senza perdere generalità, sia $\lambda \ne A_{jj},\forall j=1\dots n$.
        Sia $\underline{v}_\lambda$ il vettore associato a $\lambda$, conosco che 
        $A \underline{v}_\lambda = \lambda \underline{v}_\lambda \implies (A-\lambda Id)\underline{v}_\lambda = \underline{0}$.
        Spezzo $A=D+E$ dove $D= diag(A)$ mentre $E= E-diag(A)$. Sappiamo che $D-\lambda Id$ è
        una matrice diagonale ed invertibile perché stiamo assumendo che $\lambda \ne A_{jj}, \forall j =1\dots n$.
        
        $(A-\lambda Id)\underline{v}_\lambda = \underline{0} \implies (D+E-\lambda Id)\underline{v}_\lambda = \underline{0}\implies (D-\lambda Id)\underline{v}_\lambda = -E\underline{v}_\lambda \implies
        \underline{v}_\lambda = -(D-\lambda Id)^{-1}E\underline{v}_\lambda$

        Quindi possiamo passare alle norme:
        
        $\|\underline{v}_\lambda\| = \|-(D-\lambda Id)^{-1}E\underline{v}_\lambda\|$

        Per ogni norma possiamo dire 
        $\|\underline{v}_\lambda\| = \|-(D-\lambda Id)^{-1}E\|\|\underline{v}_\lambda\|\implies 1\le \|(D-\lambda Id)^{-1}E\|$.
        
        Facciamo una digressione, scelgo la norma $\infty$, allora $\|M\underline{x}\|_\infty = \max_{j=1}^n|\sum_{l=1}^{n}M_{jl}x_j|\le \max_{j=1}^n\sum_{l=1}^{n}|M_{jl}||x_j|\le 
        \max_{j=1}^n|\sum_{l=1}^{n}M_{jl}|(\max_{j=1}^n|x_j|)\le\|\underline{x}\|_\infty  \max_{j=1}^n|\sum_{l=1}^{n}M_{jl}|$
        $\dots$

        Tornando alla dimostrazione, continuamo
        $$1\le \|(D-\lambda Id)^{-1}E\|_\infty \implies 1 \le \max_{j=1}^n\sum_{l=1,j\ne l}^{n} \frac{|A_{jl}|}{|A_{jj}-\lambda|}
        = \sum_{l=1,l\ne k}^{n} \frac{|A_{kl}|}{|A_{kk}-\lambda|}, \text{per qualche}k = 1\dots n\implies$$
        $$|A_{kk}-\lambda| \le \sum_{l=1,l\ne k}^{n} |A_{kl}| \implies \lambda \in R_k\implies \lambda \in \bigcup_{j=1}^n R_j, \forall \lambda \impl 
        \sigma(A)\subseteq \bigcup_{j=1}^n R_j$$

    \end{proof}
\end{teorema}

Questo mi serve perché posso trovare l'intervallo finale in cui sono inclusi tutti 
gli autovalori. Quindi se l'intervallo non contiene lo $0$ allora tutti gli autovalori 
sono $\ne 0$ e quindi il determinante $\ne 0$, quindi la matrice è invertibile.

\begin{nota}
    $R_j$ è l'insieme dei numeri complessi che distano $\le$ la somma delle entrate 
    alla riga $j$ esclusa l'entrata $A_{jj}$. Vengono chiamata \textbf{cerchi Greshgorin}.
\end{nota}

\section{Metodo delle potenze}
Algoritmo per calcolare l'autovalore della matrice $A$ di modulo massimo e l'autovettore 
associato.

Data $A\in \mathbb{R}^{n\times n}$, $\{\lambda_j\}^n_{j=1}$ autovalori tali che 
$$|\lambda_1|> |\lambda_2| \ge \dots \ge |\lambda_N|$$

Il problema di questo metodo è che si assume che $|\lambda_1|> |\lambda_2|$, 
se fosse $|\lambda_1| = |\lambda_2|$ allora non si avrebbe la convergenza. 

Dato $\underline{q}^{(0)}\in \mathbb{R}^n$ generico tale che $\|\underline{q}^{(0)}\|_2=1$.

% TODO: pseudocodice

\begin{nota}
    Se $A$ è diagonalizzabile con autovettori $\{\underline{v}_{\lambda_j}\}$,tale che 
    gli autovettori sono una base di $\mathbb{R}^n$
\end{nota}

Essendo che che gli autovettori sono dei vettori generatori allora 
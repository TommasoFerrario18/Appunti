\chapter{Formato jpeg}
Le immagini possono essere viste come una sequenza di $N \times M$ pixel e 
il formato jpeg combacia con l'applicazione della DCT2, si comprime l'immagine 
e per ritornale all'immagine iniziale, si applica IDCT2. Il problema è che si ottiene 
una matrice con la virgola mentre i pixel sono interi, perciò:
\begin{itemize}
    \item arrotonda la matrice
    \item si normalizzano le entry nell'intervallo $[0,255]$ 
\end{itemize}

\begin{nota}
    Il problema di queste procedure è che la complessità temporale è elevata se 
    applicata interamente su immagini molto grandi.
\end{nota}

\begin{nota}
    Si può identificare il fenomeno di Gibbs dal momento che si vedono degli artefatti.
\end{nota}

Nella compressione dell'immagine:
\begin{itemize}
    \item $M$ deve essere alto (la dimensione della sottomatrice delle frequenze 
    da preservare)
    \item diffusione del fenomeno di Gibbs
    \item costo computazionale elevato
\end{itemize}
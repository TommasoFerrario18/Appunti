\chapter{Formato jpeg}
Le immagini possono essere viste come una sequenza di $N \times M$ pixel e 
il formato jpeg combacia con l'applicazione della DCT2, si comprime l'immagine 
e per ritornale all'immagine iniziale, si applica IDCT2. Il problema è che si ottiene 
una matrice con la virgola mentre i pixel sono interi, perciò:
\begin{itemize}
    \item arrotonda la matrice
    \item si normalizzano le entry nell'intervallo $[0,255]$ 
\end{itemize}

\begin{nota}
    Il problema di queste procedure è che la complessità temporale è elevata se 
    applicata interamente su immagini molto grandi.
\end{nota}

\begin{nota}
    Si può identificare il fenomeno di Gibbs dal momento che si vedono degli artefatti.
\end{nota}

Nella compressione dell'immagine:
\begin{itemize}
    \item $M$ deve essere alto (la dimensione della sottomatrice delle frequenze 
    da preservare)
    \item diffusione del fenomeno di Gibbs
    \item costo computazionale elevato
\end{itemize}

Il formato jpeg risolve tutti questi problemi. 

Jpeg utilizza una funzione $qf:[0,100] \to \mathbb{R}$ tale che

$$qf(q)=\begin{cases}
    \frac{500}{q}\frac{1}{100}&1\le q \le 50\\
    \frac{200-2q}{100} & 50 \le q \le 100
\end{cases}$$

La quality function permette di specificare il bilanciamento tra compressione e 
visibilità 

In aggiunta è stata utilizzata la matrice di quantizzazione, definita in modo empirico, 
che insieme alla quality function permette di effettuare delle operazioni matriciali 
per azzerare la matrice delle frequenze della dct.




L'idea di creazione di un file jpeg è di:
\begin{itemize}
    \item suddividere l'immagine in blocchi da $8\times 8$ pixel. Questo permette 
    di risolvere il problema della complessità e della presenza del fenomeno di 
    Gibbs, il primo 
    possiamo velocizzarlo parallelizzandolo e il secondo permette di confinare 
    il fenomeno al blocco $8\times 8$ senza intaccare gli altri blocchi. 
    \item sottrarre i pixel di $128$, in questo modo si passa da un intervallo di 
    pixel di $[0, 255]$ a $[-127,128]$ per avere media dei pixel $0$.
    \item applicare dct 2 ad ogni blocco $8\times 8$ ottenendo le matrici $\alpha$ delle frequenze
    \item Si crea la matrice $\overline{Q} = \text{round} (qf(q)\cdot Q)\in \mathbb{Z}^{8\times 8}$
    \item compressione: costruiamo per ogni blocco $8\times 8$ delle frequenze 
    $\alpha$, calcoliamo la matrice $\overline{\alpha} = \text{round}(\alpha./Q)$,
    dove $./:= \alpha_{ij} / \overline{Q_{ij}}$ 
    \item alla fine il file contiene tutti i blocchi $\overline{\alpha}$, le dimensioni 
    dell'immagine, il quality factor $q$ e la matrice di quantizzazione $Q$.  
\end{itemize}

L'idea di lettuara di un file jpeg è di:
\begin{itemize}
    \item ottenere $\alpha = \overline{\alpha}.*\overline{Q}$
    \item applicare idct2 su $\alpha$
    \item sommiamo ai pixel $128$
    \item visualizza l'immagine
\end{itemize}
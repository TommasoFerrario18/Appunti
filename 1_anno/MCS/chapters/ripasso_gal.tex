\chapter{Ripasso Algebra Lineare}
\begin{definizione} [\textbf{Spazio vettoriale}]
    Un insieme $V$ si dice \textbf{spazio vettoriale} se sono definite su $V$ due
    operazioni che godono di particolari proprietà:
    \begin{itemize}
        \item \textbf{Somma}: $+: V \times V \rightarrow V$
        \item \textbf{Prodotto per uno scalare}: $\cdot : V \times \mathbb{R} \rightarrow V$
    \end{itemize}
\end{definizione}

Le operazioni dello spazio vettoriale godono di alcune proprietà:
\begin{itemize}
    \item \textbf{Somma}:
          \begin{itemize}
              \item \textbf{Commutativa}: $\forall u,v \in V, u+v = v+u$
                    \begin{proof}
                        $\forall u,v \in V:$
                        $$ u+v = \left[\begin{array}{c}
                                    u_1 \\u_2\\ \vdots \\u_n
                                \end{array}\right] + \left[\begin{array}{c}
                                    v_1 \\v_2\\\vdots\\v_n
                                \end{array}\right] = \left[\begin{array}{c}
                                    u_1 + v_1 \\u_2 + v_2\\\vdots\\u_n+v_n
                                \end{array}\right] = \left[\begin{array}{c}
                                    v_1 + u_1 \\v_2 + u_2\\\vdots\\v_n+u_n
                                \end{array}\right] = \left[\begin{array}{c}
                                    v_1 \\v_2\\\vdots\\v_n
                                \end{array}\right] + \left[\begin{array}{c}
                                    u_1 \\u_2 \\\vdots\\u_n
                                \end{array}\right] = v+ u$$
                    \end{proof}
              \item \textbf{Associativa}: $\forall u,v,z \in V, (u+v)+z = v+(u+z)$
              \item \textbf{Esistenza dell'elemento neutro}: $\exists 0 \in V: 0+v = v,
                        \forall v\in V$
              \item $\forall u \in V, \exists w\in V \text{ unico}: u+w=0$
          \end{itemize}
    \item \textbf{Prodotto per uno scalare}:
          \begin{itemize}
              \item \textbf{Distributivo rispetto alla somma}: $\forall u,v \in
                        V, \forall \lambda \in \mathbb{R}, \lambda(u+v) = \lambda
                        u+\lambda v$
              \item \textbf{Distributivo rispetto alla somma e prodotto tra scalari}:
                    $\forall u \in V, \forall \lambda,\mu\in \mathbb{R}: (\lambda +
                        \mu)u = \lambda \cdot u + \mu \cdot u \quad (\lambda \cdot
                        \mu)u = \lambda \cdot u \cdot \mu \cdot u$
              \item \textbf{Esistenza dell'elemento neutro}: $\exists 1 \in
                        \mathbb{R}:1\cdot v = v, \forall v\in V$
          \end{itemize}
\end{itemize}
\begin{esempio}
    Un esempio di spazio vettoriale è $V\in \mathbb{R}^n$
\end{esempio}
\begin{definizione} [\textbf{Prodotto scalare}]
    Sia $V$ uno spazio vettoriale, definiremo il \textbf{prodotto scalare} è un'operazione
    che a due elementi di $V$ associa un valore reale, $(\cdot, \cdot):V\times V
        \rightarrow \mathbb{R}$.Tale operazione soddisfa le seguenti proprietà:
    \begin{itemize}
        \item \textbf{Simmetrica}: $\forall u,v \in V, (u,v) = (v,u)$
        \item \textbf{Bi-lineare}: $\forall v_1,v_2,w \in V,\forall\alpha_1,
                  \alpha_2 \in \mathbb{R}:(\alpha_1v_1+\alpha_2v_2, w) =
                  \alpha_1(v_1, w) + \alpha_2(v_2, w)$
        \item $\forall v \in V, v\ne 0: (v,v) \ge 0$
    \end{itemize}
\end{definizione}
\begin{definizione} [\textbf{Norma di un vettore}]
    Sia $V$ uno spazio vettoriale, definiremo la \textbf{norma di un vettore} è
    un'operazione che a un elemento di $V$ associa un valore reale, $\|\cdot\|:
        V \rightarrow \mathbb{R}$. Tale operazione soddisfa le seguenti proprietà:
    \begin{itemize}
        \item \textbf{scalabilità}: $\forall v \in V,\forall \alpha \in \mathbb{R}:
                  \|\alpha v\| = |\alpha | \|v\|$
        \item \textbf{disuguaglianza triangolare}: $\forall v,w \in V: \|v+w\|
                  \le \|v\| + \|w\|$
        \item $\forall v \in V, v\ne 0: \|v\| > 0\implies \|0\| = 0$
    \end{itemize}
\end{definizione}
La definizione e le proprietà derivano dalla nota successiva
\begin{nota} [\textbf{Norma indotta dal prodotto scalare}]
    Sia $V$ uno spazio vettoriale in cui è definito un prodotto scalare allora è
    è possibile calcolare la norma da $(\cdot, \cdot)$
    \begin{equation}
        \|v\| = \sqrt{(v,v)}  \equiv \|v\|_2
    \end{equation}
    Tale norma viene chiamata \textbf{norma indotta dal prodotto scalare }
\end{nota}
\begin{nota}
    Esistono diverse norme oltre alla $2$:
    \begin{itemize}
        \item $\|v\|_2 = \sqrt{\sum_{i=1}^{N}v_i^2} = \|v-0\|$
        \item $\|v\|_1 = \sum_{i=1}^{N}|v_i|$
        \item $\|v\|_\infty = \max_{i= 1\dots n}|v_i|$
    \end{itemize}
\end{nota}
La norma euclidea ($\|\cdot\|_2$) è utile per calcolare la distanza di vettori
perché $\|v\|_2$ calcola la distanza di $v$ da $0$, mentre $\|v-w\|_2$ calcola
la distanza tra $v$ e $w$.

Quindi dato un prodotto scalare possiamo sempre definire la norma, ma non possiamo
dire il contrario.
\begin{nota}
    La norma ha una proprietà indotta dalla sua definizione e dalle sue proprietà,
    $\forall v,w \in V$
    \begin{equation*}
        \|\|v\| -\|w\|\| \le \|v-w\|
    \end{equation*}
\end{nota}
Possiamo definire anche le norme per le matrici $V\in \mathbb{R}^{r\times c}$:
\begin{itemize}
    \item $\|A\|_F = \sqrt{\sum_{i,j} |a_{i,j}|^2}$
    \item $\|A\|_1 = \max_{i=1\dots r}{\sum_{j=1}^{c} |a_{i,j}|}$
    \item $\|A\|_\infty = \max_{j=1\dots c}{\sum_{i=1}^{r} |a_{i,j}|}$
\end{itemize}

Altre operazioni utili per gli spazi vettoriali, generalmente vettori matrici, ad esempio
la \textbf{trasposizione}.
\begin{equation}
    A=\left[\begin{array}{ccc}
            a & b & c \\
            d & e & f
        \end{array}\right]\in \mathbb{R}^{2\times3},  A^t = \left[\begin{array}{cc}
            a & d \\
            b & e \\
            c & f
        \end{array}\right]\in \mathbb{R}^{3\times2}
\end{equation}
Il problema di questa operazione è che non è \textbf{chiusa} ovvero non rimane
nello stesso spazio.

In aggiunta si ha \textbf{prodotto tra matrici e vettori}. Dati $A\in \mathbb{R}^{r\times c}$
e $v\in \mathbb{R}^{c}$ si ha:
\begin{equation*}
    Av = \left[\begin{array}{cccc}
            a_{11} & a_{12} & \cdots & a_{1c} \\
            a_{21} & a_{22} & \cdots & a_{2c} \\
            \vdots & \vdots & \dots  & \vdots \\
            a_{r1} & a_{r2} & \cdots & a_{rc}
        \end{array}\right] \left[\begin{array}{c}
            v_1 \\v_2\\\vdots\\v_c
        \end{array}\right] = \left[\begin{array}{c}
            a_{11}v_1+a_{12}v_2+\cdots a_{1c}v_c \\
            a_{21}v_1+a_{22}v_2+\cdots a_{2c}v_c \\
            \vdots                               \\
            a_{r1}v_1+a_{r2}v_2+\cdots a_{rc}v_c \\
        \end{array}\right]
\end{equation*}
Questa operazione è vincolata dal fatto che la matrice e il vettore devono
essere compatibili, le colonne della matrice devono essere uguali alle righe del
vettore. Inoltre abbiamo anche l'operazione di \textbf{prodotto matrice-matrice}.
Sia $A\in \mathbb{R}^{3\times 2}, B\in \mathbb{R}^{2\times 4}$
\begin{equation*}
    AB = \left[\begin{array}{cc}
            a & b \\
            c & d \\
            e & f
        \end{array}\right]\left[\begin{array}{cccc}
            g & h & i & l \\
            m & n & o & p \\
        \end{array}\right] = \left[\begin{array}{cccc}
            ag+bm & ah+bn & ai+bo & al+bp \\
            cg+dm & ch+dn & ci+do & cl+bp \\
            eg+fm & eh+fn & ei+fo & el+fp \\
        \end{array}\right]
\end{equation*}
Anche questa operazione è vincolante perché può essere effettuata solo quando la
prima matrice ha lo stesso numero di colonne uguale al numero di righe della
seconda. Inoltre la commutazione non sempre è fattibile perché potrebbero non
essere compatibili, in aggiunta non è commutativa.
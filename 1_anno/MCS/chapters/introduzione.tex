\chapter{Introduzione}
\section{Rappresentazione di numeri reali rispetto ad una base}
Cosa vuol dire scrivere $12.211$ rispetto ad una certa base?

$$12.211_{10}\equiv 1\cdot 10^1+2\cdot 10^0 +2 \cdot 10^{-1} + 1 \cdot 10^{-2}+ 1 \cdot 10^{-3}$$
$$12.211_{3}\equiv 1\cdot 3^1+2\cdot 3^0 +2 \cdot 3^{-1} + 1 \cdot 3^{-2}+ 1 \cdot 3^{-3}$$

Più in generare, data una base $\beta$ possiamo rappresentare un qualsiasi
numero nel seguente modo:
$$\pm a_n a_{n-1}\dots a_{-n} = \pm \sum_{j = -n}^{n} a_j\cdot \beta^j $$
\begin{teorema}
    $1=0.\overline{9}$
    \begin{proof}
        possiamo dire che $$0.\overline{9} = \sum _{j=1}^{+\infty} 9\cdot 10^{-j}=9 \sum _{j=0}^{+\infty} \cdot 10^{-j-1}= $$
        $$=9 \sum _{j=0}^{+\infty} \cdot \frac{1}{10}^{j+1}=\frac{9}{10} \sum _{j=0}^{+\infty} \cdot \frac{1}{10}^{j}= $$
        $$=\frac{9}{10} \cdot \frac{1}{1-\frac{1}{10}} = 1$$
    \end{proof}
\end{teorema}
\begin{teorema}
    Sia dato $\frac{1}{q}$ un numero razionale ai minimi termini. Tale numero ha
    una rappresentazione finita se e solo se i divisori primi di $q$ dividono anche 
    la base $\beta$.
\end{teorema}
Con questo teorema abbiamo che sul pc non possiamo rappresentare i numeri reali 
e razionali perfettamente, dal momento che se non si ha il denominatore che non divide la base $2$
allora questo non ha una rappresentazione finita.

\section{Aritmetica floating point}
Sia $F$ un insieme di numeri rappresentabile su un computer. Un'aritmetica di macchina
per $F$ è una quadrupla:
\begin{equation}
    \left(\beta, t, L, U\right)
\end{equation}
dove:
\begin{itemize}
    \item $beta$: è la base
    \item $t$: è la \textbf{mantissa}, quanti numeri posso metere prima e dopo la virgola
    \item $L$: lower bound ($<0$)
    \item $U$: upper bound ($>0$)
\end{itemize}
$F$ consiste dei numeri rappresentabili come
\begin{equation}
    \pm 0.a_1a_2\dots a_t\cdot \beta ^e
\end{equation}
con $a_1\ne 0$, $0\le a_j\le \beta-1$, $L\le e\le U$.
\subsection{Standard Double Precision}
Avremo 
\begin{equation}
    \left(\beta = 2, t= 52, -1022, 1023\right)
\end{equation}
Si usano $52$ cifre per la mantissa ma si tiene sempre $1$ come cifra costante quindi
il numero finale avrà realmente $53$ cifre per la mantissa. In aggiunta per l'esponente si hanno 
$2^{11}-2$ valori.

Sul pc si usano $64$ bit di cui:
\begin{itemize}
    \item $1$ bit: per il segno
    \item $11$ bit: per l'esponente, le due configurazioni aggiuntive servono per 
    rappresentare le configurazioni \texttt{Inf} e \texttt{NaN}
    \item $52$ bit: per la mantissa
\end{itemize}

\subsection{Operazioni in Aritmetica floating point}
Se volessimo fare la somma floating point tra due numeri $x,y$ coincide a fare
\begin{equation}
    x \oplus y = fl(fl(x) + fl(y))
\end{equation}
Dove $fl(z)$ con $z\in \mathbb{R}$ è la rappresentazione approssimata in floating point di $z$.
\begin{nota}
    $sin(z)$ è sempre approssimato utilizzando le serie.
\end{nota}

\begin{definizione} [\textbf{Errore assoluto}]
    Se $x$ è una quantità \textbf{esatta}, sia $\stackrel{\sim}{x}$ la quantità approssimata
    relativa, sia $\|\cdot\|$ una misura di errore, allora l'\textbf{errore assoluto} è
    \begin{equation}
        \|x-\stackrel{\sim}{x}\|
    \end{equation} 
\end{definizione}

\begin{definizione} [\textbf{Errore relativo}]
    Se $x$ è una quantità \textbf{esatta}, sia $\stackrel{\sim}{x}$ la quantità approssimata
    relativa, sia $\|\cdot\|$ una misura di errore, allora l'\textbf{errore relativo} è
    \begin{equation}
        \frac{\|x-\stackrel{\sim}{x}\|}{\|x\|}
    \end{equation} 
\end{definizione}
Quindi possiamo calcolare 
\begin{equation}
    \frac{\|(x+y)-(x\oplus y)\|}{\|x+y\|}
\end{equation}
che corrisponde a 
%TODO: sistemare la semplificazione
\begin{equation}
    \frac{\|(x+y)-(x(1+\epsilon_x)+ y(1+\epsilon_y))\cdot (1+\epsilon_+)\|}{\|x+y\|}=
    =\frac{\|(x+y)-(x(1+\epsilon_x)+ y(1+\epsilon_y))\cdot (1+\epsilon_+)\|}{\|x+y\|}=
    \|\frac{x}{x+y}\cdot \epsilon_x + \frac{y}{x+y}\cdot \epsilon_y+\epsilon\|\le \|\frac{x}{x
    +y}\|\epsilon + \|\frac{y}{x+y}\|\epsilon + \epsilon
\end{equation}
Per $\epsilon$ piccolo l'errore cresce quando $y\sim -x$.

Quindi la somma non è \textbf{stabile} e l'errore aumenta quando $x+y\sim 0$.

Per quanto riguarda il prodotto effettuiamo lo stesso ragionamento.
calcoliamo
\begin{equation}
    \frac{\|(x\cdot y)-(x\odot y)\|}{\|x\cdot y\|} 
\end{equation}
stesse semplificazioni si ottiene che il prodotto è \textbf{stabile} quindi perché
rimangono solo gli $\epsilon$ che sono tendenti a $0$.
\chapter{Serie di Fourier}
\section{Serie di Fourier nel continuo}
Le serie di Fourier si basano sulle \textbf{funzioni perioche}, ovvero funzioni che
si ripetono ad intervalli regolari.
\begin{definizione}[\textbf{Funzione periodica}]
    Una funzione $f:\mathbb{R}\to \mathbb{R}$ si dice \textbf{periodica} se
    $\forall x\in \mathbb{R}, \exists T \in \mathbb{R}$ tale che:
    \begin{equation}
        f(x+n \cdot T) = f(x) \quad \forall n \in \mathbb{Z} \text{ e } \forall x \in \mathbb{R}
    \end{equation}
\end{definizione}
\begin{proposizione}
    Se una funzione periodica di periodo $p$ è notata in un generico intervallo
    di ampiezza $p$ allora è nota ovunque.
\end{proposizione}
Il risultato precedente sarà estremamente utile. Infatti, ci permette di ricondurre
lo studio/analisi di una funzione periodica di periodo $p$ solamente nell'intervallo
$[0, p]$.
\begin{proposizione}
    L'integrale su un intervallo di ampiezza $p$ di una funzione periodica $f$
    di periodo $p$ non dipende dagli estremi di integrazione.
    \begin{equation}
        \int_{a}^{a + p}f(t)dt = \int_{b}^{b + p}f(t)dt \quad \forall a,b
    \end{equation}
    \begin{proof}
        Significa dimostrare la seguente uguaglianza:
        \begin{equation*}
            \int_{a}^{a + p}f(t)dt = \int_{0}^{p}f(t)dt
        \end{equation*}
        Per le proprietà degli integrali sappiamo che:
        \begin{equation*}
            \int_{a}^{a+p}f(t)dt =  \int_{a}^{p}f(t)dt + \int_{p}^{a+p}f(t)dt =
            \int_{a}^{0}f(t)dt +\int_{0}^{p}f(t)dt+ \int_{p}^{a+p}f(t)dt
        \end{equation*}
        Utilizzando il metodo di sostituzione, introduciamo $s = t - p$, di conseguenza
        $t= p \iff s=0$ e $t = p + a \iff s = a$. Per concludere la sostituzione
        dobbiamo aggiornare i differenziali e gli estremi di integrazione.
        Per i differenziali avremo $ds = dt$ perché $\frac{\partial s}{\partial s} = 1$
        e $\frac{\partial t-p}{\partial t} = 1$.

        Quindi gli integrali diventano
        \begin{equation*}
            \int_{a}^{0}f(t)dt +\int_{0}^{p}f(t)dt+ \int_{p}^{a+p}f(t)dt =
            \int_{a}^{0}f(t)dt +\int_{0}^{p}f(t)dt+ \int_{0}^{a}f(s)ds =
        \end{equation*}

        Ma essendo $f$ periodica di periodo $p$, sapendo che $s = t + p$ e $p$ è
        il periodo allora $f(s) = f(t)$, quindi $\int_{0}^{a}f(s)ds = \int_{0}^{a}f(t)dt$,
        perciò:
        \begin{equation*}
            \begin{aligned}
                \int_{a}^{0}f(t)dt + \int_{0}^{p}f(t)dt + \int_{0}^{a}f(s)ds =
                \int_{a}^{0}f(t)dt + \int_{0}^{p}f(t)dt + \int_{0}^{a}f(t)dt = \\
                -\int_{0}^{a}f(t)dt + \int_{0}^{a}f(t)dt + \int_{0}^{p}f(t)dt =
                \int_{0}^{p}f(t)dt
            \end{aligned}
        \end{equation*}
    \end{proof}
\end{proposizione}
\begin{nota}
    Nei computer, la rappresentazione di un integrale viene approssimata nel
    seguente modo:
    \begin{equation*}
        \int_{0}^{p}f(t)dt \approx \sum_{i=1}^{x_n}\omega_i f(t_i)
    \end{equation*}
\end{nota}
Il nostro obiettivo è quello di modificare una funzione periodica in modo da
rendere più gestibile la periodicità. Consideriamo una funzione periodica di
di periodo $t$ e la vogliamo trasformare in una funzione periodica di un generico
periodo $p$.

Per ottenere questo risultato possiamo modificare l'argomento della funzione
periodica nel seguente modo:
\begin{equation}
    f(t) = f\left(\frac{t}{p}\right)
\end{equation}
Così facendo, se $p < t$ allora stiamo accorciando il periodo della funzione. Al
contrario, se $p > t$ stiamo allungando il periodo della funzione.
\begin{proof}
    Per dimostrarlo possiamo prendere una funzione periodica generica $f$ di
    periodo $t$ e posso testare che $f(x) = f\left( n \cdot \frac{t}{p} + x\right)$
    sia periodica di periodo $p$, ovvero che $f(x + np) = f(x) \iff
        f\left(\frac{t}{p}x + np\right) = f(\frac{t}{p}x)$.

    \begin{equation*}
        f\left(x + n \cdot p\right) = f\left(\frac{t}{p}\cdot (x + np)\right) =
        f\left(\frac{tx}{p} + \frac{tnp}{p}\right) = f\left(\frac{tx}{p} + nt\right)
        = f\left(\frac{tx}{p}\right) = f'(x)
    \end{equation*}
    perché $f$ è periodica di periodo $t$.
\end{proof}

Utilizzeremo per lo più le seguenti funzioni $\sin\left(\frac{2\pi k}{p}t\right)$
e $\cos \left(\frac{2\pi k}{p}t\right)$. Dove $k$ è un intero positivo che permette
di alterare la frequenza della funzione.
\begin{proposizione}\label{prop:integrali_sinusoidi}
    Per ogni $k \geq 1$ vale il seguente integrale:
    \begin{equation*}
        \int_{0}^{p}\sin \left(\frac{2\pi k}{p}t\right) dt =
        \int_{0}^{p}\cos \left(\frac{2\pi k}{p}t\right) dt = 0
    \end{equation*}
    \begin{proof}
        Possiamo risolvere i singoli integrali, per esempio con la sostituzione
        introduciamo la variabile $s= \frac{2\pi k}{p}t$ quindi $\frac{p}{2\pi k}ds =dt$
        allora:
        \begin{equation*}
            \int_{0}^{p}\sin \left(\frac{2\pi k}{p}t\right) dt =
            \int_{0}^{2k\pi}\sin \left(s\right)\frac{p}{2\pi k}ds =
            \frac{p}{2\pi k} \int_{0}^{2k\pi}\sin \left(s\right)ds =
            \frac{p}{2\pi k} \left[-cos(s)\right]_0^{2k\pi} = 0
        \end{equation*}
        Lo stesso ragionamento può essere applicato al coseno.
    \end{proof}
\end{proposizione}
\begin{proposizione}
    Presa una qualsiasi coppia di valori interi $l, k > 0$ vale il seguente integrale:
    \begin{equation}
        \int_{0}^{p}\sin\left(\frac{2\pi l}{p}t\right) \cdot \cos\left(\frac{2\pi k}{p}t\right) dt = 0
    \end{equation}
    \begin{proof}
        Risolviamolo con la sostituzione $s= \frac{2\pi t}{p}$ quindi $ds = \frac{p}{2\pi}dt$.
        \begin{equation*}
            \int_{0}^{p}\sin\left(\frac{2\pi l}{p}t\right)\cdot \cos\left(\frac{2\pi k}{p}t\right) dt =
            \int_{0}^{2\pi}\sin\left(ls\right)\cdot \cos\left(ks\right)\frac{p}{2\pi} ds =
            \frac{p}{2\pi}\int_{0}^{2\pi}\sin\left(ls\right)\cdot \cos\left(ks\right) ds
        \end{equation*}
        Utilizzando la relazione $\frac{1}{2}\left(\sin(\alpha +\beta) +
            \sin(\alpha-\beta)\right) = \sin\left(\alpha\right)\cos\left(\beta\right)$
        possiamo dire:
        \begin{equation*}
            \begin{aligned}
                \frac{p}{2\pi}\int_{0}^{2\pi}\sin\left(ls\right)\cdot \cos\left(ks\right) ds =
                \frac{p}{2\pi}\frac{1}{2}\int_{0}^{2\pi}\sin\left((l+k)s\right)+ \sin\left((l-k)s\right) ds= \\
                =\frac{p}{4\pi}\left(\int_{0}^{2\pi}\sin\left((l+k)s\right)ds + \int_{0}^{2\pi}\sin\left((l-k)s\right) ds\right)=0
            \end{aligned}
        \end{equation*}
        Si annullano perché coincide con l'integrale della nota precedente.
    \end{proof}
\end{proposizione}
\begin{proposizione}
    Presa una qualsiasi coppia di valori interi $l, k > 0$ vale il seguente integrale:
    \begin{equation*}
        \int_{0}^{p}\cos\left(\frac{2\pi l}{p}t\right) \cdot
        \cos\left(\frac{2\pi k}{p}t\right) dt = \begin{cases}
            \frac{p}{2} & k=l      \\
            0           & k \neq l \\
        \end{cases}
    \end{equation*}
    l'integrale tra il prodotto dei coseni con frequenza diversa allora il risultato
    è nullo.
    \begin{proof}
        Per dimostralo usiamo sempre la sostituzione $s=\frac{2\pi}{p}dt$ e $\frac{p}{2\pi}ds = dt$.
        Sostituendo otteniamo:
        \begin{equation*}
            \frac{p}{2\pi}\int_{0}^{2\pi} \cos(ks) \cdot \cos(ls) ds
        \end{equation*}
        Utilizzando un procedimento simile a quello di prima, possiamo scrivere:
        \begin{equation*}
            \frac{1}{2} (\cos(\alpha + \beta) + \cos(\alpha - \beta)) = \cos(\alpha)\cos(\beta)
        \end{equation*}
        Quindi:
        \begin{equation*}
            \frac{p}{2\pi}\int_{0}^{2\pi} \cos(ks) \cdot \cos(ls) ds =
            =\frac{p}{2\pi}\int_{0}^{2\pi} \frac{1}{2}\cos((k+l)s) +\cos((k-l)s) ds =
            \frac{p}{4\pi}\int_{0}^{2\pi}\cos((k+l)s) +\cos((k-l)s) ds
        \end{equation*}
        Per il caso $k \neq l$ allora otteniamo l'integrale della proposizione
        \ref{prop:integrali_sinusoidi} e quindi è uguale a $0$.
        Se $k = l$ allora $\int_{0}^{2\pi} \cos(2k s) ds = 0$ mentre
        $\int_{0}^{2\pi}cos(0s)ds = s|_{0}^{2\pi} = 2\pi$ il quale deve essere
        moltiplicato per $\frac{p}{4\pi}$ che era il coefficiente davanti all'integrale.

        Quindi se $k=l$ allora l'integrale vale $\frac{p}{2}$.
    \end{proof}
\end{proposizione}
Per il caso $\sin \cdot \sin$ si ottiene un risultato analogo al caso $\cos \cdot \cos$
modificando solamente la seguente relazione:
\begin{equation*}
    \frac{1}{2} (\cos(\alpha + \beta) + \cos(\alpha - \beta)) = \cos(\alpha)\cos(\beta) \to
    \frac{1}{2} (\cos(\alpha - \beta) - \cos(\alpha + \beta)) = \sin(\alpha)\sin(\beta)
\end{equation*}

La stessa dimostrazione si può effettuare per il caso $\sin \cdot \sin$
\begin{equation}
    \int_{0}^{p}\sin \left(\frac{2\pi}{p} kt\right) \sin \left(\frac{2\pi}{p} lt
    \right) dt = \begin{cases}
        \frac{p}{2} & l = k      \\
        0           & altrimenti
    \end{cases}
\end{equation}

L'idea alla basa della serie di Fourier è quella di scrivere una generica funzione
periodica $f:\mathbb{R}\to \mathbb{R}$ di periodo $p$ in una combinazione di $\sin$
e $\cos$.
\begin{definizione}[\textbf{Serie di Fourier}]
    Sia $f:\mathbb{R}\to \mathbb{R}$ una funzione periodica di periodo $p$.
    \begin{equation}
        f(t) = a_0 + \sum_{k=1}^{\infty} \left[a_k \cos (\frac{2\pi}{p}kt) +
            \beta_k \sin(\frac{2\pi}{p} kt)\right]
    \end{equation}
\end{definizione}

L'uguaglianza precedente è lecita perché specifica che una funzione periodica di
periodo $p$ può essere riscritta come combinazioni lineari di funzioni periodiche,
anche esse di periodo $p$. Importante è anche la limitazione della sommatoria a
$\infty$ perché significa una funzione è combinazione lineare di \textbf{tutte}
le funzioni periodiche.

All'atto pratico la sommatoria infinita è impossibile da calcolare, quindi si
approssima la funzione ad un $N$ molto grande.

In aggiunta si può notare che i coefficienti $a_0, a_1, \dots$ partono dall'indice
$0$ mentre $b_1,b_2,\dots$ partono dall'indice $1$. Questo è dato dal fatto che
quando $k = 0$ si ha:
\begin{equation*}
    a_0\cos (\frac{2\pi}{p}0t) +\beta_0\sin(\frac{2\pi}{p}0t) = a_0\cdot 1 + b_0
    \cdot 0 = a_0
\end{equation*}

Per utilizzare questa approssimazione mediante la serie, dobbiamo calcolare
facilmente e velocemente i coefficienti $a_i,b_i$.

Calcoliamo per prima cosa $a_0$, per ipotesi sappiamo che l'uguaglianza che esprime
la serie di Fourier è valida, allora anche gli integrali delle due parti saranno
uguali.
\begin{equation*}
    \int_{0}^{p}f(t) dt=\int_{0}^{p} a_0 + \sum_{k=1}^{\infty}\left[a_k\cos
        (\frac{2\pi}{p}kt) +\beta_k\sin(\frac{2\pi}{p}kt)\right]dt
\end{equation*}
Applicando le proprietà degli integrali si ottiene:
\begin{equation*}
    \int_{0}^{p}f(t) dt=a_0\int_{0}^{p} 1dt + \sum_{k=1}^{\infty}\left[a_k
        \int_{0}^{p}\cos (\frac{2\pi}{p}kt)dk +\beta_k\int_{0}^{p}\sin(
        \frac{2\pi}{p}kt)dt\right]
\end{equation*}
Grazie alle dimostrazioni effettuate precedentemente sappiamo che:
\begin{equation*}
    \int_{0}^{p}\sin \left(\frac{2\pi k}{p}t\right) dt =
    \int_{0}^{p}\cos \left(\frac{2\pi k}{p}t\right) dt = 0
\end{equation*}
di conseguenza abbiamo che gli integrali presenti nella sommatoria si annullano,
otteniamo quindi:
\begin{equation*}
    \int_{0}^{p}f(t) dt = a_0\int_{0}^{p} 1dt  = a_0p\implies a_0 = \frac{1}{p}
    \int_{0}^{p}f(t) dt
\end{equation*}
Questa espressione si può calcolare perché $f$ è sempre nota.
\begin{nota}
    La seguente espressione è chiamata anche media integrale nell'intervallo $[0,p]$.
    \begin{equation*}
        a_0 = \frac{1}{p}\int_{0}^{p}f(t) dt
    \end{equation*}
\end{nota}
Per calcolare i coefficienti $a_k$ possiamo moltiplicare entrambe le parti
dell'uguaglianza per $\cos\left(\frac{2\pi}{p}kt\right)$ e successivamente
integrare. In questo modo otteniamo:
\begin{equation*}
    \begin{aligned}
        \int_{0}^{p}f(t) \cdot \cos\left(\frac{2\pi}{p}kt\right) dt = \\ a_0\int_{0}^{p}
        \cos\left(\frac{2\pi}{p}kt\right) dt + \sum_{j=1}^{\infty}\left[a_j\int_{0}^{p}
            \cos \left(\frac{2\pi}{p}jt\right) \cos\left(\frac{2\pi}{p}kt\right)dk +
            \beta_j\int_{0}^{p}\sin\left(\frac{2\pi}{p}jt\right)\cos\left(\frac{2\pi}{p}kt\right)dt\right]
    \end{aligned}
\end{equation*}
Sfruttando le proposizioni definite in precedenza possiamo affermare che:
\begin{equation*}
    \begin{array}{c}
        a_0\int_{0}^{p}\cos \left(\frac{2\pi}{p}kt\right)dt = 0 \\
        \beta_j\int_{0}^{p}\sin\left(\frac{2\pi}{p}jt\right)\cos \left(\frac{2\pi}{p}kt\right)dt=0
    \end{array}
\end{equation*}
perciò, otteniamo:
\begin{equation*}
    \int_{0}^{p}f(t)\cos \left(\frac{2\pi}{p}kt\right) dt = \sum_{j=1}^{\infty}
    \left[a_j\int_{0}^{p}\cos \left(\frac{2\pi}{p}jt\right)\cos \left(
        \frac{2\pi}{p}kt\right)dk \right]
\end{equation*}
Per le proposizioni che abbiamo in precedenza, sappiamo che l'integrale del prodotto
di due coseni con frequenza diversa è nullo. Quindi rimane solamente il caso in
cui le frequenze sono uguali, ovvero $j=k$. Questo cui permette di ottenere:
\begin{equation*}
    \int_{0}^{p}f(t)\cos \left(\frac{2\pi}{p}kt\right) dt = a_k\int_{0}^{p}\cos
    \left(\frac{2\pi}{p}kt\right)\cos \left(\frac{2\pi}{p}kt\right)dk = a_k \frac{p}{2}
    \iff a_k =\frac{2}{p}  \int_{0}^{p}f(t)\cos \left(\frac{2\pi}{p}kt\right) dt
\end{equation*}
Il conto di $a_k$ è indipendente dagli altri valori di $a_j$ e quindi può essere
calcolato indipendentemente dagli altri coefficienti.

Successivamente dobbiamo calcolare i coefficienti $b_k$, la metodologia è la stessa
degli $a_k$ solo che in questo caso la funzione per cui moltiplichiamo entrambi
i membri è:
\begin{equation*}
    \sin\left(\frac{2\pi}{p}kt\right)
\end{equation*}
\begin{equation*}
    \begin{array}{l}
        \int_{0}^{p}f(t)\sin\left(\frac{2\pi}{p}kt\right) dt = \\
        = a_0 \int_{0}^{p} \sin\left(\frac{2\pi}{p}kt\right) dt + \sum_{j=1}^{\infty}
        \left[a_j\int_{0}^{p}\cos \left(\frac{2\pi}{p}jt\right)\sin\left(\frac{2\pi}{p}kt\right)dk
            + \beta_j \int_{0}^{p}\sin\left(\frac{2\pi}{p}jt\right)\sin\left(\frac{2\pi}{p}kt
            \right)dt\right]
    \end{array}
\end{equation*}
Grazie alle proposizioni definite in precedenza possiamo affermare che:
\begin{equation*}
    \begin{array}{c}
        a_0\int_{0}^{p}\sin \left(\frac{2\pi}{p}kt\right)dt = 0 \\
        a_j\int_{0}^{p}\cos\left(\frac{2\pi}{p}jt\right)\sin \left(\frac{2\pi}{p}kt\right)dt=0
    \end{array}
\end{equation*}
perciò, otteniamo:
\begin{equation*}
    \int_{0}^{p}f(t)\sin \left(\frac{2\pi}{p}kt\right) dt = \sum_{j=1}^{\infty}
    \left[\beta_j\int_{0}^{p}\sin \left(\frac{2\pi}{p}jt\right)\sin \left(
        \frac{2\pi}{p}kt\right)dk \right]
\end{equation*}
Per le proposizioni che abbiamo in precedenza, sappiamo che l'integrale del prodotto
di due seni con frequenza diversa è nullo. Quindi rimane solamente il caso in
cui le frequenze sono uguali, ovvero $j=k$. Questo cui permette di ottenere:
\begin{equation*}
    \int_{0}^{p}f(t)\sin \left(\frac{2\pi}{p}kt\right) dt = \beta_k\int_{0}^{p}\sin
    \left(\frac{2\pi}{p}kt\right)\sin \left(\frac{2\pi}{p}kt\right)dk = \beta_k \frac{p}{2}
    \iff \beta_k =\frac{2}{p}  \int_{0}^{p}f(t)\sin \left(\frac{2\pi}{p}kt\right) dt
\end{equation*}

\begin{definizione}[\textbf{Funzione pari}]
    Una funzione $f:\mathbb{R}\to \mathbb{R}$ si dice \textbf{pari} se:
    \begin{equation}
        f(t) = f(-t) \quad \forall t \in \mathbb{R}
    \end{equation}
\end{definizione}
\begin{definizione}[\textbf{Funzione dispari}]
    Una funzione $f:\mathbb{R}\to \mathbb{R}$ si dice \textbf{dispari} se:
    \begin{equation}
        f(t) = -f(-t) \quad \forall t \in \mathbb{R}
    \end{equation}
\end{definizione}
Sapendo che la funzione $\sin$ è dispari e la funzione $\cos$ è pari, possiamo
dire che se $f$ è pari allora i coefficienti $\beta_i = 0$ mentre se $f$ è dispari
allora i coefficienti $a_i = 0$.
\begin{proposizione}
    Prese due funzioni $f,g:\mathbb{R}\to \mathbb{R}$ tali per cui $f$ è pari e
    $g$ è dispari allora $h(x)=f(x)g(x)$ è dispari
    \begin{proof}
        Se $f(x)$ è pari se e solo se $f(x) = f(-x),\forall x\in \mathbb{R}$.
        Se $g(x)$ è dispari se e solo se $g(x) = -g(-x),\forall x\in \mathbb{R}$.
        Quindi per dimostrare $h(x)$ dispari allora possiamo partire dalla definizione.
        \begin{equation*}
            h(x) = f(x)g(x) = f(-x) (-g(-x)) = - f(-x)g(-x) = -h(-x)
        \end{equation*}
        Quindi abbiamo dimostrato che $h$ è dispari.
    \end{proof}
\end{proposizione}
Ricorda che la periodicità delle funzioni viene mantenuta.
\begin{proposizione}
    Presa una funzione $f:\mathbb{R}\to \mathbb{R}$ dispari e un intervallo simmetrico
    rispetto all'origine $[-a,a]$ con $a>0$ allora:
    \begin{equation}
        \int_{-a}^{a} f(x)dx = 0
    \end{equation}
    \begin{proof}
        Per dimostrare questa proposizione ci basiamo sul fatto che:
        \begin{equation*}
            \int_{-a}^{a} f(x)dx = \int_{-a}^{0} f(x)dx + \int_{0}^{a} f(x)dx =
            -\int_{0}^{-a} f(x) dx+ \int_{0}^{a} f(x)dx
        \end{equation*}
        Dato che $f$ è dispari allora $ \int_{0}^{-a} f(x) = \int_{0}^{a} f(x)$
        e quindi:
        \begin{equation*}
            -\int_{0}^{a} f(x) dx + \int_{0}^{a} f(x) dx = 0
        \end{equation*}
    \end{proof}
\end{proposizione}
Sfruttando i risultati delle due proposizioni appena presentate, possiamo dire che
se $f$ è una funzione dispari possiamo verificare che $a_k = 0, \, \forall k \ge 0$.
\begin{equation*}
    a_k = \int_{0}^{p} f(t) \cos\left(\frac{2 \pi}{p} kt\right) dt =
    \int_{-\frac{p}{2}}^{\frac{p}{2}} f(t) \cos\left(\frac{2 \pi}{p} kt\right) dt =
    \int_{-\frac{p}{2}}^{\frac{p}{2}} H(x) dx = 0
\end{equation*}
sapendo che $H(x)$ è dispari in quanto sto moltiplicando una funzione pari con una
funzione dispari. Nel caso specifico di $a_0$ ho che:
\begin{equation*}
    a_0 =\frac{1}{p} \int_{0}^{p}f(x)dx = \frac{1}{p} \int_{-\frac{p}{2}}^{\frac{p}{2}}f(x)dx = 0
\end{equation*}
In modo analogo si procede nel caso in cui $f$ è pari:
\begin{equation*}
    b_k = \int_{0}^{p} f(t) \sin\left(\frac{2 \pi}{p} kt\right) dt =
    \int_{-\frac{p}{2}}^{\frac{p}{2}} f(t) \sin\left(\frac{2 \pi}{p} kt\right) dt =
    \int_{-\frac{p}{2}}^{\frac{p}{2}} G(x) dx = 0
\end{equation*}
sapendo che $G(x)$ è dispari in quanto sto moltiplicando una funzione pari con una
funzione dispari.
\begin{nota}
    Posso cambiare l'intervallo di integrazione e passare da $[0,p]$ a $[-\frac{p}{2},\frac{p}{2}]$
    in quanto le funzioni sono periodiche di periodo $p$.
\end{nota}

Riassumendo, per calcolare la trasformata di Fourier $f(x)$ allora ci serve:
\begin{itemize}
    \item Essere certi che $f(x)$ sia periodica.
    \item Dobbiamo calcolare $\alpha_k$ e $\beta_k$, quindi devono esistere
          ed essere finiti i seguenti integrali.
          \begin{equation*}
              a_k = \frac{2}{p}\int_{0}^{p}f(x)\cos\left(\frac{2\pi k}{p}x\right)
              dx \quad \beta_k=\frac{2}{p}\int_{0}^{p}f(x)\sin\left(\frac{2\pi k}{p}x\right)dx
          \end{equation*}
          dal momento che $\cos$ e $\sin$ sono sempre integrabili allora i vincoli
          sono solo su $f$.
\end{itemize}
\begin{nota}
    Possiamo rappresentare la trasformata di Fourier per quelle funzioni che hanno
    un numero finito di discontinuità di tipo salto.
\end{nota}

La convergenza della serie di Fourier dipende dalla regolarità delle funzioni $f$,
infatti se ci sono delle discontinuità di tipo salto allora si hanno delle
convergenze meno precise. Si hanno i fenomeni di \textbf{Gibbs} ovvero si hanno andamenti
oscillatori vicino al punto di discontinuità, questi sono dovuti dal fatto che si
stanno approssimando dei punti di discontinuità di tipo salto con funzioni che non
li hanno.
Per risolvere questo problema possiamo introdurre nella sommatoria delle funzioni
che avranno gli stessi punti di discontinuità negli stessi valori.

\begin{teorema}[\textbf{Convergenza della serie di Fourier}]
    Sia $f:\mathbb{R}\to \mathbb{R}$ periodica, assumiamo che $f$ sia $C^1$ (continua
    fino alla prima derivata) a tratti allora se $f$ è continua in $t \in \mathbb{R}$
    la serie $f_N$ converge a $f(t)$, altrimenti se non è continua allora $f_N$
    converge a:
    \begin{equation*}
        \frac{\lim_{x\to t^-}f(x) +\lim_{x\to t^+}f(x)}{2}
    \end{equation*}
\end{teorema}
La serie converge alla media dei due limiti se $f$ non è continua, però prima e
dopo si ha fenomeni di Gibbs, però più incrementiamo il valore di $N$ maggiore
sarà la traslazione del movimento verso il punto di discontinuità.

Più precisamente la convergenza della serie dipende da come è fatta $f$:
\begin{itemize}
    \item Se $f$ è regolare la serie converge.
    \item Se $f$ presenta una discontinuità sulla derivata prima, allora nel
          punto in cui è presente tale discontinuità la convergenza sarà lenta.
    \item Se $f$ ha un punto di discontinuità di tipo salto, allora la serie
          converge alla media dei limiti nel punto con la conseguente presenza
          del fenomeno di GIBBS.
\end{itemize}
\subsection{Serie di Fourier per funzioni non periodiche}
Per ora abbiamo visto solo l'utilizzo della serie di Fourier per il caso di funzioni
$f$ periodiche di periodo $p$ e analizzandole nell'intervallo $[0,p]$. In realtà
possiamo generalizzare il ragionamento fatto finora per estendere la serie di Fourier
a una generica funzione $g(x)$ in un generico intervallo $[a,b]$.

Facendo opportuni cambi di variabili, anziché considerare un generico intervallo
$[a, b]$, si può sempre pensare che la funzione $f$ sia definita nell'intervallo
$[0, p]$. Questo può essere fatto utilizzando la seguente trasformazione:
\begin{equation}
    x = \frac{b - a}{p} x' + a \quad x' \in [0, p] \implies x \in [a, b]
\end{equation}

In questo modo si riesce a portare la funzione nell'intervallo canonico definendo:
\begin{equation*}
    \stackrel{\sim}{g}(x) = g\left(\frac{b-a}{p}x'+a\right), \quad \forall x' \in [a,b], x \in [0,p]
\end{equation*}
Si apre ora la questione di come creare una funzione periodica a partire da questa
funzione $\stackrel{\sim}{g}(x)$ definita nell'intervallo $[0, p]$. Un primo
tentativo potrebbe essere quello di periodizzarla per ripetizione.
\begin{figure}[!ht]
    \centering
    \includegraphics[width=0.5\textwidth]{img/Serie/ripetizione.png}
    \caption{Periodizzazione della funzione $\stackrel{\sim}{g}(x)$}
    \label{fig:ripetizione}
\end{figure}

Agendo in questo modo abbiamo creato delle discontinuità sulla funzione. Di conseguenza
l'approssimazione data dalle serie di Fourier sarà soggetta al fenomeno di Gibbs.

In realtà $\stackrel{\sim}{g}(x)$ in un modo più intelligente risolvendo eventuali
discontinuità di salto, ovvero presa $\stackrel{\sim}{g}(x)$, la si porta in $[0,p]$,
si specchia la funzione rispetto l'asse delle $y$ e poi replichiamo la funzione
infinite volte, ottenendo una funzione periodica di periodo $2p$.
\begin{figure}[!ht]
    \centering
    \includegraphics[width=0.5\textwidth]{img/Serie/specchio.png}
    \caption{Periodizzazione della funzione $\stackrel{\sim}{g}(x)$}
    \label{fig:specchio}
\end{figure}
Utilizzando quest'ultimo la trasformazione di Fourier si chiamerà \textbf{cosine trasformation},
perché $\stackrel{\sim}{g}(x)$ sarà pari e quindi la sua approssimazione sarà
composta solamente dai coefficienti del coseno.

Avendo definito la funzione $\stackrel{\sim}{g}(x)$ per renderla periodica, il
calcolo della trasformata di Fourier si complicherà, in quanto viene chiesto di
calcolare i coefficienti nell'intervallo di periodicità $[0, 2L]$.

Possiamo semplificarci la vita sfruttando la funzione iniziale nell'intervallo
$[a,b]$, sapendo che $\stackrel{\sim}{g}(x)$ coincide con quella di partenza
specchiata rispetto l'asse $y$ e poi resa periodica, abbiamo quindi costruito una
funzione pari.

Vediamo ora come possiamo calcolare i coefficienti della serie di Fourier per
questa funzione $\stackrel{\sim}{g}(x)$.

Iniziamo con il calcolo del coefficiente $a_0$:
\begin{equation}
    a_0 = \frac{1}{2L} \int_{0}^{2L} \stackrel{\sim}{g}(s)ds = \frac{1}{2L}
    \int_{-L}^{L} \stackrel{\sim}{g}(s)ds = \frac{1}{2L} 2 \int_{0}^{L} \stackrel{\sim}{g}(s)ds
    = \frac{1}{L} \int_{0}^{L} g(s)ds = \frac{1}{L} \int_{0}^{L} g\left(\frac{b - a}{L} s + a\right)ds
\end{equation}
Nello specifico, essendo la funzione periodica e pari, possiamo fare le seguenti
considerazioni:
\begin{itemize}
    \item Portare gli integrali dall'intervallo $[0,2L]$ all'intervallo $[-L,L]$
    \item Semplificare il calcolo di $2\int_{0}^{L}$ che coincide con la funzione iniziale
\end{itemize}
Per quanto riguarda il calcolo del generico coefficiente $a_k$:
\begin{equation}
    \begin{array}{l}
        a_k = \frac{2}{2L} \int_{0}^{2L} \stackrel{\sim}{g}(s) \cos\left(\frac{2\pi k}{2L}s\right)ds
        = \frac{1}{L} \int_{0}^{2L} \stackrel{\sim}{g}(s) \cos\left(\frac{\pi k}{L}s\right)ds \\
        = \frac{1}{L} \int_{-L}^{L} \stackrel{\sim}{g}(s) \cos\left(\frac{\pi k}{L}s\right)ds
        = \frac{2}{L} \int_{0}^{L} \stackrel{\sim}{g}(s) \cos\left(\frac{\pi k}{L}s\right)ds  \\
        = \frac{2}{L} \int_{0}^{L} g\left(\frac{b - a}{L} s  + a\right) \cos\left(\frac{\pi k}{L}s\right)ds
    \end{array}
\end{equation}
La serie di Fourier diventa quindi:
\begin{equation}
    f(x) = a_0 + \sum_{k=1}^{+\infty} a_k \cos\left(\frac{\pi k}{L}x\right)
\end{equation}
A prima vista si potrebbe pensare che esse permettano di rappresentare una
funzione $f$ con meno informazioni rispetto alle classiche serie di Fourier.
Infatti in questo caso non appaiono le funzioni seno. Tuttavia, in questo caso le
funzioni coseno che stiamo considerando sono associate a molte più frequenze
rispetto a quelle della classica serie di Fourier. In un certo senso con la serie
dei coseni abbiamo eliminato le funzioni seno a scapito di avere molte più funzioni
coseno rispetto alla serie di Fourier.
\section{Serie di Fourier nel discreto}
Fino a questo momento, abbiamo considerato solo funzioni continue definite come
$f:[a,b]\to \mathbb{R}$, il problema è che noi nella realtà non possiamo ragionare
nel continuo, ma nel discreto. Piuttosto che avere una funzione nota in tutti i
punti dell'intervallo di periodicità [0, p], possiamo pensare di conoscere solamente
i valori di questa funzione in alcuni punti e avere una sua approssimazione a “scalini”.

Supponiamo che $f:[0,p] \to \mathbb{R}$ continua nell'intervallo, vogliamo approssimare
la funzione con un totale $n = 4$ campionamenti, quindi vogliamo trovare $f_4 = (f_0, f_1,f_2,f_3)\in \mathbb{R}^n$.

Quindi $f_i = (\frac{i}{N} + \frac{i+1}{N}) \frac{1}{2} = \frac{2i+1}{2N}$ coicidono 
con il punto medio dell'$i$-esimo sottointervallo di $[0,p]$ suddiviso in $n$ intervalli.
Possiamo generalizzare questo ragionamento sfruttando la funzione reale
$$f_i = f(\frac{2i+1}{2n})$$

Possiamo vedere $f_N = \sum_{i=0}^{N-1} f_i \cdot e_i$ ovvero possiamo vedere il vettore 
$f_N$ come combinazione lineare dei valori $f_i$ per i vettori della base canonica 
$e_i$.

\begin{definizione}[\textbf{Base}]
    Sia $V$ uno spazio vettoriale, un insieme di vettori $\{v_i\}$ linearmente
    indipendenti è una base di $V$ se ogni vettore $v \in V$ può essere scritto
    come combinazione lineare dei vettori della base.
    \begin{equation}
        \vec{v} = \sum_{i=1}^{n} c_i \vec{v_i}, \quad \forall \vec{v} \in V, c_i \in \mathbb{R}
    \end{equation}
\end{definizione}
\begin{definizione}[\textbf{Base ortogonale}]
    Una base $\{v_i\}$ di uno spazio vettoriale $V$ è ortogonale se e solo se:
    \begin{equation}
        v_i\cdot v_j = \begin{cases}
            0     & i\ne j \\
            \ne 0 & i = j
        \end{cases}
    \end{equation}
\end{definizione}
Le basi ortogonali permettono di semplificare il calcolo del prodotto scalare
tra i vettori dello spazio. Nello specifico, permette di effettuare solo il prodotto 
scalare tra il vettore di coefficiente $j$ e il coefficiente $j$ della base.

\begin{nota}
    Per ogni spazio vettoriale possiamo definire la sua base canonica, ovvero
    la base formata dai vettori $\vec{e_i}$ tali che:
    \begin{equation}
        \vec{e_i} = \begin{cases}
            1 & i = j \\
            0 & i \ne j
        \end{cases}
    \end{equation}
\end{nota}

\begin{nota}
    Data la base canonica $\{e_i\}$ di $\mathbb{R}^n$ rispetta le seguenti proprietà:
    \begin{itemize}
        \item localizzata: guarda un'entrata alla volta
        \item generano $\mathbb{R}^n$
        \item ortogonale
    \end{itemize}
\end{nota}

A noi però interessa il fatto che una base non sia localizzata, perché se vogliamo 
rappresentare $f_N$ come combinazione della base canonica, ci servono un totale di 
$N$ vettori della base. Noi vogliamo trovare una base composta da un numero 
fisso di vettori in modo da risparmiare memoria. 

Quindi per il caso discreto, se vogliamo approssimare una funzione continua 
$f$ dobbiamo 
$$f= \sum_{k = 0}^{\infty} a_k\cos (\frac{k\pi}{L}x)$$

Nella possiamo rappresentare $f$ con un numero massimo di $N$ campioni quindi 
$$f \sim f_N = \sum_{k=0}^{N-1} b_k \cdot w_k$$

dove $f_i$ sono i famosi valori medi degli $N$ intervalli e $w_k$ è il vettore 
$k$-esimo della base di $\mathbb{R}^n$ tale che non sia localizzata. I vettori 
della base saranno:

$$w_k =\left[\begin{array}{c}
    w_k^0\\
    w_k^1\\
    \vdots\\
    w_k^N\\
\end{array}\right] = \left[\begin{array}{c}
    \cos (\frac{\pi k}{L}\cdot \frac{1}{2N})\\
    \cos (\frac{\pi k}{L}\cdot \frac{2}{2N})\\
    \vdots\\
    \cos (\frac{\pi k}{L}\cdot \frac{N-1}{2N})\\
\end{array}\right]$$

Quindi $w_k$ sono tutti i vettori che mi rappresentanto oscillazioni diverse del $\cos$
con periodi differenti. Quindi l'obiettivo è rappresentare 
$f_N$ con meno di $N$ vettori della base, cosa che non potevamo fare con la base
canonica.

Per calcolare i coefficienti $b_k$ dobbiamo innanzitutto far sì che la base sia 
ortogonale così risulta essere più semplice il conto.

\begin{teorema}
    Sia $\theta\in \mathbb{R}$ e $N\in \mathbb{N}$ allora 
    $$(\cos (\theta) + i\sin (\theta))^N = \cos(N \theta) + i \sin(N\theta)$$
    dove $i$ è la base immaginaria.
    \begin{proof}
        Dimostriamo per induzione:
        \begin{itemize}
            \item Se $N = 1$ allora è ovvio. 
            \item Supponiamo che valga $(\cos (\theta) + i\sin (\theta))^N= \cos(N \theta) + i \sin(N\theta)$
            dimostriamo che $$(\cos (\theta) + i\sin (\theta))^{N+1}= \cos((N+1) \theta) + i \sin((N+1)\theta)$$
            $$\cos((N+1) \theta) + i \sin((N+1)\theta) = \cos(N\theta+\theta) + i \sin(N\theta+\theta) =$$$$= \cos(N\theta)\cos(\theta) -\sin(N\theta)\sin(\theta) + i \left[\sin(N\theta)\cos(\theta)+\cos(N\theta)\sin(\theta)\right] =$$
            $$= \cos(N\theta)\cos(\theta) -\sin(N\theta)\sin(\theta) + i\sin(N\theta)\cos(\theta)+i\cos(N\theta)\sin(\theta)=$$
            $$= \cos(\theta) \left(\cos(N\theta) + i\sin (N\theta)\right)+\sin(\theta)(i\cos(N\theta)-\sin(N\theta))=$$
            $$= \cos(\theta) \left(\cos(N\theta) + i\sin (N\theta)\right)+\sin(\theta)(i\cos(N\theta)+i^2\sin(N\theta))=$$
            $$= \cos(\theta) \left(\cos(N\theta) + i\sin (N\theta)\right)+i\sin(\theta)(\cos(N\theta)+i\sin(N\theta))=$$
            $$= \left(\cos(\theta) +i\sin(\theta)\right) + \left(\cos(N\theta) + i\sin (N\theta)\right) = \left(\cos(\theta) +i\sin(\theta)\right) + \left(\cos(\theta) + i\sin (\theta)\right)^N =\left(\cos(\theta) + i\sin (\theta)\right)^{N+1} $$

        \end{itemize}
        
    \end{proof}
\end{teorema}

\begin{teorema}
    Sia $z\in \mathbb{C}$  $z\ne 1$ allora $\forall \in \mathbb{N}$ 
    $$\sum_{j=0}^{N-1}  z^j = \frac{1-z^N}{1-z}$$
    \begin{proof}
        Dimostrare $\sum_{j=0}^{N-1}  z^j = \frac{1-z^N}{1-z}$ allora è come dimostrare
        $$(1-z)\sum_{j=0}^{N-1}  z^j = 1-z^N$$
        Quindi 
        $$(1-z)\sum_{j=0}^{N-1} z^j = \sum_{j=0}^{N-1}z^j-z\sum_{j=0}^{N-1} z^j = 1+ z+z^1+\dots + z^{N-1} - z -z^1-z^2-\dots-z^{N} = 1- z^N$$
    \end{proof}
\end{teorema}
\begin{teorema}
    Per ogni $m\in \mathbb{Z}$ tale  che $m\in [-N+1,0)\cup (0, 2N-2]$
    $$\sum_{i=0}^{N-1} \cos(\pi m \frac{2i+1}{2N}) = 0$$
    \begin{proof}
        Possiamo dire $\cos (\frac{\pi m i}{N} + \frac{\pi m}{2N})= \cos(\frac{\pi m i}{N}) \cos(\frac{\pi m}{2N})-\sin(\frac{\pi m i}{N}) \sin(\frac{\pi m}{2N})$.
        Questo significa che la sommatoria si trasforma in 
        $$\sum_{i=0}^{N-1}  \cos(\frac{\pi m i}{N}) \cos(\frac{\pi m}{2N})-\sum_{i=0}^{N-1}\sin(\frac{\pi m i}{N}) \sin(\frac{\pi m}{2N}) = 
        \cos(\frac{\pi m}{2N})\sum_{i=0}^{N-1}  \cos(\frac{\pi m i}{N}) -\sin(\frac{\pi m}{2N})\sum_{i=0}^{N-1}\sin(\frac{\pi m i}{N})  $$

        Possiamo dire che $\cos(\frac{\pi m i}{N}) = \mathcal{R}\left[\cos(\frac{\pi m i}{N}) + \underline{i} \sin(\frac{\pi m i}{N})\right]$
        e di conseguenza $\sin(\frac{\pi m i}{N}) = \mathcal{I}\left[\cos(\frac{\pi m i}{N}) + \underline{i} \sin(\frac{\pi m i}{N})\right]$
        dove $\mathcal{R}$ $\mathcal{I}$ indicano rispettivamente la parte reale 
        e immaginaria del numero complesso e $\underline{i}$ è la costante immaginaria.

        Quindi possiamo utilizzare i teoremi dimostrati precedentemente e dire
        $$\cos(\frac{\pi m i}{N}) = \mathcal{R}\left[\cos(\frac{\pi m i}{N}) + \underline{i} \sin(\frac{\pi m i}{N})\right] =\mathcal{R}\left[\left(\cos(\frac{\pi m }{N}) + \underline{i} \sin(\frac{\pi m }{N})\right)^i\right]$$
        $$\sin(\frac{\pi m i}{N}) = \mathcal{I}\left[\cos(\frac{\pi m i}{N}) + \underline{i} \sin(\frac{\pi m i}{N})\right] =\mathcal{I}\left[\left(\cos(\frac{\pi m }{N}) + \underline{i} \sin(\frac{\pi m }{N})\right)^i\right]$$
    
        Quindi possiamo sostituire nella sommatoria per spostare l'indice all'esponente.

        $$\cos(\frac{\pi m}{2N})\sum_{i=0}^{N-1}  \cos(\frac{\pi m i}{N}) -\sin(\frac{\pi m}{2N})\sum_{i=0}^{N-1}\sin(\frac{\pi m i}{N}) = $$
        $$= \cos(\frac{\pi m}{2N})\sum_{i=0}^{N-1}  \mathcal{R}\left[\left(\cos(\frac{\pi m }{N}) + \underline{i} \sin(\frac{\pi m }{N})\right)^i\right] -\sin(\frac{\pi m}{2N})\sum_{i=0}^{N-1}\mathcal{I}\left[\left(\cos(\frac{\pi m }{N}) + \underline{i} \sin(\frac{\pi m }{N})\right)^i\right]$$
        Possiamo portare fuori l'operazione di estrazione della parte immaginaria
        e reale.
        $$\cos(\frac{\pi m}{2N})\sum_{i=0}^{N-1}  \mathcal{R}\left[\left(\cos(\frac{\pi m }{N}) + \underline{i} \sin(\frac{\pi m }{N})\right)^i\right] -\sin(\frac{\pi m}{2N})\sum_{i=0}^{N-1}\mathcal{I}\left[\left(\cos(\frac{\pi m }{N}) + \underline{i} \sin(\frac{\pi m }{N})\right)^i\right]=$$
        $$=\cos(\frac{\pi m}{2N})\mathcal{R}\left[\sum_{i=0}^{N-1}  \left(\cos(\frac{\pi m }{N}) + \underline{i} \sin(\frac{\pi m }{N})\right)^i\right] -\sin(\frac{\pi m}{2N})\mathcal{I}\left[\sum_{i=0}^{N-1}\left(\cos(\frac{\pi m }{N}) + \underline{i} \sin(\frac{\pi m }{N})\right)^i\right]$$
        
        Utilizziamo sempre i teoremi precedenti
    
        $$\cos(\frac{\pi m}{2N})\mathcal{R}\left[\sum_{i=0}^{N-1}  \left(\cos(\frac{\pi m }{N}) + \underline{i} \sin(\frac{\pi m }{N})\right)^i\right] -\sin(\frac{\pi m}{2N})\mathcal{I}\left[\sum_{i=0}^{N-1}\left(\cos(\frac{\pi m }{N}) + \underline{i} \sin(\frac{\pi m }{N})\right)^i\right]=$$
        $$=\cos(\frac{\pi m}{2N})\mathcal{R}\left[ \frac{1- \left(\cos(\frac{\pi m }{N}) + \underline{i} \sin(\frac{\pi m }{N})\right)^N}{1-\left(\cos(\frac{\pi m }{N}) + \underline{i} \sin(\frac{\pi m }{N})\right)}\right] -\sin(\frac{\pi m}{2N})\mathcal{I}\left[\frac{1- \left(\cos(\frac{\pi m }{N}) + \underline{i} \sin(\frac{\pi m }{N})\right)^N}{1-\left(\cos(\frac{\pi m }{N}) + \underline{i} \sin(\frac{\pi m }{N})\right)}\right]$$
        
        Utilizziamo sempre i teoremi precedenti
        $$\cos(\frac{\pi m}{2N})\mathcal{R}\left[ \frac{1- \left(\cos(\frac{\pi m }{N}) + \underline{i} \sin(\frac{\pi m }{N})\right)^N}{1-\left(\cos(\frac{\pi m }{N}) + \underline{i} \sin(\frac{\pi m }{N})\right)}\right] -\sin(\frac{\pi m}{2N})\mathcal{I}\left[\frac{1- \left(\cos(\frac{\pi m }{N}) + \underline{i} \sin(\frac{\pi m }{N})\right)^N}{1-\left(\cos(\frac{\pi m }{N}) + \underline{i} \sin(\frac{\pi m }{N})\right)}\right]=$$
        $$=\cos(\frac{\pi m}{2N})\mathcal{R}\left[ \frac{1- \left(\cos(\not N\frac{\pi m }{\not N}) + \underline{i} \sin(\not N\frac{\pi m }{\not N})\right)}{1-\left(\cos( \frac{\pi m }{ N}) + \underline{i} \sin( \frac{\pi m }{ N})\right)}\right] -\sin(\frac{\pi m}{2N})\mathcal{I}\left[\frac{1- \left(\cos(\not N\frac{\pi m }{\not N}) + \underline{i} \sin(\not N\frac{\pi m }{\not N})\right)}{1-\left(\cos(\frac{\pi m }{N}) + \underline{i} \sin(\frac{\pi m }{N})\right)}\right]$$
    
        Quindi possiamo dire che $\forall m \in \mathbb{Z}$ allora $\sin(\pi m) = 0$.
        Il problema è il $\cos$ che dipende da $m$ pari o $m$ dispari.
        Quindi:
        \begin{itemize}
            \item $m$ pari: allora $1- \left(\cos(\pi m ) + \underline{i} \sin(\pi m)\right) = 1 - 1 = 0$ 
            tutta l'equazione si annulla e quindi abbiamo dimostrato.
            \item $m$ dispari: allora $1- \left(\cos(\pi m ) + \underline{i} \sin(\pi m)\right) = 1 + 1 = 2$
        $$\cos(\frac{\pi m}{2N})\mathcal{R}\left[ \frac{2}{1-\left(\cos( \frac{\pi m }{ N}) + \underline{i} \sin( \frac{\pi m }{ N})\right)}\right] -\sin(\frac{\pi m}{2N})\mathcal{I}\left[\frac{2}{1-\left(\cos(\frac{\pi m }{N}) + \underline{i} \sin(\frac{\pi m }{N})\right)}\right]$$
        Consideriamo $\mathcal{I}\left[\frac{2}{1-\left(\cos(\frac{\pi m }{N}) + \underline{i} \sin(\frac{\pi m }{N})\right)}\right]$ allora possiamo
        manipolare il denominatore in questo modo:
        $$\mathcal{I}\left[\frac{2}{1-\left(\cos(\frac{\pi m }{N}) + \underline{i} \sin(\frac{\pi m }{N})\right)}\right]=$$
        $$=\mathcal{I}\left[\frac{2}{\left(1-\cos(\frac{\pi m }{N})\right) - \underline{i} \sin(\frac{\pi m }{N})}\right]= $$
        $$=\mathcal{I}\left[\frac{2}{\left(1-\cos(\frac{\pi m }{N})\right) - \underline{i} \sin(\frac{\pi m }{N})}\cdot \frac{\left(1-\cos(\frac{\pi m }{N})\right) + \underline{i} \sin(\frac{\pi m }{N})}{\left(1-\cos(\frac{\pi m }{N})\right) + \underline{i} \sin(\frac{\pi m }{N})}\right]= $$
        $$=\mathcal{I}\left[\frac{2\left[\left(1-\cos(\frac{\pi m }{N})\right) + \underline{i} \sin(\frac{\pi m }{N})\right]}{\left(1-\cos(\frac{\pi m }{N})\right)^2 - \left(\underline{i} \sin(\frac{\pi m }{N})\right)^2}\right]= $$
    \end{itemize}
    
    
    
    
    
    \end{proof}
\end{teorema}
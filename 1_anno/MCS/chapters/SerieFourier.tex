\chapter{Serie di Furier}

\section{Serie di Forier nel continuo}
Si basano sulle funzioni perioche. 

\begin{definizione} [\textbf{Funzione periodica}]
    Una \textbf{funzione periodica} $f:\mathbb{R}\to \mathbb{R}$ se 
    $\forall x\in \mathbb{R}, \exists t \in \mathbb{R}$ tale che $f(x+nt) = f(x),
    \forall n\in \mathbb{Z}$ 
\end{definizione}

\begin{nota}
    Se una funzione periodica di periodo $p$ è notata in un generico intervallo
    di ampiezza $p$ allora è nota ovunque.
\end{nota}

\begin{nota}
    L'integrale su un intervallo di ampiezza $p$ di una funzione periodica di periodo 
    $p$ non dipende da dagli estremi di integrazione.
    \begin{equation*}
        \int_{a}^{a+p}f(t)dt =  \int_{b}^{b+p}f(t)dt, \forall a,b
    \end{equation*}
    \begin{proof}
        Significa validare la seguente equazione
        \begin{equation*}            
            \int_{a}^{a+p}f(t)dt =  \int_{0}^{p}f(t)dt
        \end{equation*}

        Per le proprietà degli integrali sappiamo che 
        \begin{equation*}            
            \int_{a}^{a+p}f(t)dt =  \int_{a}^{p}f(t)dt + \int_{p}^{a+p}f(t)dt = 
            \int_{a}^{0}f(t)dt +\int_{0}^{p}f(t)dt+ \int_{p}^{a+p}f(t)dt
        \end{equation*}
        Dal metodo di sostituzione $s = t-p$ allora $t= p \iff s=0$ e $t=p+a 
        \iff s=a$ e i differenziali saranno $ ds = dt$ perché $\frac{\partial s}{\partial s} = 1$
        $\frac{\partial t-p}{\partial t} = 1$.

        Quindi gli integrali diventano 
        \begin{equation*}
            \int_{a}^{0}f(t)dt +\int_{0}^{p}f(t)dt+ \int_{p}^{a+p}f(t)dt = 
            \int_{a}^{0}f(t)dt +\int_{0}^{p}f(t)dt+ \int_{0}^{a}f(s)ds = 
        \end{equation*}

        Ma essendo $f$ periodica, sapendo che $s=t+p$ e $p$ è il periodo allora 
        $f(s) = f(t)$, quindi $\int_{0}^{a}f(s)ds = \int_{0}^{a}f(t)dt$, perciò
        
        \begin{equation*}
            \int_{a}^{0}f(t)dt +\int_{0}^{p}f(t)dt+ \int_{0}^{a}f(s)ds = 
            \int_{a}^{0}f(t)dt +\int_{0}^{p}f(t)dt+ \int_{0}^{a}f(t)dt = 
            -\int_{0}^{a}f(t)dt + \int_{0}^{a}f(t)dt +\int_{0}^{p}f(t)dt+  = \int_{0}^{p}f(t)dt 
        \end{equation*}
    \end{proof}
\end{nota}

\begin{nota}
    Un qualsiasi integrale può essere approssimato nel seguente modo

    \begin{equation*}
        \int_{0}^{p}f(t)dt \approx \sum_{i=1}^{x_n}\omega_1 f(t_1) 
    \end{equation*}
\end{nota}

Il nostro obiettivo è quello di riportare tutte le funzioni periodiche di periodo $t$
variabile ad un periodo $p$ costante. Possiamo moltiplicare i valori di input della 
funzione per $\frac{t}{p}$ (deriva dalla fisica). Se $p<t$ allora stiamo accorciando 
la funzione, se $p>t$ stiamo allungando la funzione.

\begin{proof}
    Per dimostrarlo possiamo prendere una funzione periodica generica $f$ di periodo $t$
    e posso testare che $f'(x) = f(\frac{t}{p}x)$ sia periodica di periodo $p$, 
    ovvero che $f'(x+np) = f'(x) \iff f(\frac{t}{p}x + np) = f(\frac{t}{p}x)$.

    $f'(x+np) = f(\frac{t}{p}(x + np))= f(\frac{tx}{p} + \frac{tnp}{p}) = f(\frac{tx}{p} + nt)
    =f(\frac{tx}{p}) =f'(x)$ perché $f$ è periodica di periodo $t$.
\end{proof}

Utilizzeremo per lo più le seguenti funzioni $\sin (\frac{2\pi k}{p}t)$ e $\sin (\frac{2\pi k}{p}t)$.
\begin{nota}
    Per ogni $k \ge 1$ vale
    \begin{equation*}
        \int_{0}^{p}\sin \left(\frac{2\pi k}{p}t\right) dt =  
        \int_{0}^{p}\cos \left(\frac{2\pi k}{p}t\right) dt = 0
    \end{equation*}
    \begin{proof}
        Possiamo risolvere i singoli integrali, per esempio con la sostituzione 
        introduciamo la variabile $s= \frac{2\pi k}{p}t$ quindi $\frac{p}{2\pi k}ds =dt$
        allora 
        $$\int_{0}^{p}\sin \left(\frac{2\pi k}{p}t\right) dt = 
        \int_{0}^{2k\pi}\sin \left(s\right)\frac{p}{2\pi k}ds = \frac{p}{2\pi k} \int_{0}^{2k\pi}\sin \left(s\right)ds =\frac{p}{2\pi k} \left[-cos(s)\right]_0^{2k\pi} = 0 $$
        La stessa roba si replica nel coseno. 
    \end{proof}
\end{nota}

\begin{nota}
    Presa una qualsiasi coppia $l,k>0$ vale 
    \begin{equation*}
        \int_{0}^{p}\sin\left(\frac{2\pi l}{p}t\right)\cdot \cos\left(\frac{2\pi k}{p}t\right) dt = 0
    \end{equation*}
    \begin{proof}
        Risolviamolo con la sostituzione $s= \frac{2\pi t}{p}$ quindi $ds = \frac{p}{2\pi}dt$.
        $$\int_{0}^{p}\sin\left(\frac{2\pi l}{p}t\right)\cdot \cos\left(\frac{2\pi k}{p}t\right) dt = 
        \int_{0}^{2\pi}\sin\left(ls\right)\cdot \cos\left(ks\right)\frac{p}{2\pi} ds =
        \frac{p}{2\pi}\int_{0}^{2\pi}\sin\left(ls\right)\cdot \cos\left(ks\right) ds$$
        Utilizzando la relazione $\frac{1}{2}\left(\sin(\alpha +\beta) \sin(\alpha-\beta)\right) = \sin\left(\alpha\right)\cos\left(\beta\right)$ 
        possiamo dire:
        $$\frac{p}{2\pi}\int_{0}^{2\pi}\sin\left(ls\right)\cdot \cos\left(ks\right) ds = 
        \frac{p}{2\pi}\frac{1}{2}\int_{0}^{2\pi}\sin\left((l+k)s\right)+ \sin\left((l-k)s\right) ds=$$
        $$
        =\frac{p}{4\pi}\left(\int_{0}^{2\pi}\sin\left((l+k)s\right)ds + \int_{0}^{2\pi}\sin\left((l-k)s\right) ds\right)=0
        $$
        Si annullano perché coincide con l'integrale della nota precedente.
    \end{proof}
\end{nota}
\begin{nota}
    PResa $l,k>0$ vale
    \begin{equation*}
        \int_{0}^{p}\cos\left(\frac{2\pi l}{p}t\right)\cdot \cos\left(\frac{2\pi k}{p}t\right) dt = \begin{cases}
            \frac{p}{2} & k=l\\
            0 & k\ne l\\
        \end{cases}
    \end{equation*}
    l'integrale tra il prodotto dei coseni con frequenza diversa allora il risultato 
    è nullo.
    \begin{proof}
        Per dimostralo usiamo sempre la sostituzione $s=\frac{2\pi}{p}dt$ e $\frac{p}{2\pi}ds = dt$.
        Come prima si ottiene la stessa roba di Prima
        $$\frac{p}{2\pi}\int_{0}^{2\pi} \cos(ks) \cdot \cos(ls) ds$$
        Utilizziamo $\frac{1}{2}\cos(\alpha + \beta) +\cos(\alpha - \beta) = \cos(\alpha) \cos(\beta)$ allora 
        otteniamo
        $$\frac{p}{2\pi}\int_{0}^{2\pi} \cos(ks) \cdot \cos(ls) ds = 
        =\frac{p}{2\pi}\int_{0}^{2\pi} \frac{1}{2}\cos((k+l)s) +\cos((k-l)s) ds = 
        =\frac{p}{4\pi}\int_{0}^{2\pi}\cos((k+l)s) +\cos((k-l)s) ds$$
        Per il caso $k\ne l$ allora otteniamo l'integrale della prima nota quindi $0$.
        Se $k=l$ allora $\int_{0}^{2\pi} \cos(2k s) ds = 0$ mentre $\int_{0}^{2\pi}cos(0s)ds = s|_{0}^{2\pi} = 2\pi$
    \end{proof}
\end{nota}
Per il caso $\sin \cdot \sin$ è uguale.
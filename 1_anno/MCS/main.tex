\documentclass[a4paper, 12pt, oneside]{book}
\usepackage[italian]{babel}
\usepackage[utf8]{inputenc}
\usepackage[a4paper,top=2.5cm,bottom=2.5cm,left=2cm,right=2cm]{geometry}
\usepackage{amssymb}
\usepackage{amsthm}
\usepackage{graphics}
\usepackage{amsfonts}
\usepackage{amsmath}
\usepackage{amstext}
\usepackage{engrec}
\usepackage{rotating}
\usepackage[safe,extra]{tipa}
\usepackage{multirow}
\usepackage{hyperref}
\usepackage{enumerate}
\usepackage{braket}
\usepackage{marginnote}
\usepackage{pgfplots}
\usepackage{cancel}
\usepackage{polynom}
\usepackage{booktabs}
\usepackage{enumitem}
\usepackage{algorithm}
\usepackage{algpseudocode}
\usepackage{framed}
\usepackage{pdfpages}
\usepackage{pgfplots}
\usepackage{fancyhdr}
\usepackage{caption}
\usepackage{subcaption}
\usepackage{setspace}
\usepackage{hyperref}
\pagestyle{fancy}
\fancyhead[L,RO]{\slshape \rightmark}
\fancyfoot[C]{\thepage}

\title{Metodi per il calcolo scientifico}
\author{Tommaso Ferrario (\href{https://github.com/TommasoFerrario18}{@TommasoFerrario18}) \\\\
Telemaco Terzi (\href{https://github.com/Tezze2001}{@Tezze2001}) \\\\
Simone Vendramini (\href{https://github.com/simone-vendramini}{@simone-vendramini})}
\date{Marzo 2024}

\pgfplotsset{compat=1.13}

\begin{document}

\maketitle
\newtheorem{teorema}{Teorema}
\newtheorem{dimostrazione}{Dimostrazione}
\newtheorem{definizione}{Definizione}
\newtheorem{esempio}{Esempio}
\newtheorem{osservazione}{Osservazione}
\newtheorem{nota}{Nota}
\newtheorem{corollario}{Corollario}
\tableofcontents
\renewcommand{\chaptermark}[1]{
    \markboth{\chaptername
        \ \thechapter.\ #1}{}}
\renewcommand{\sectionmark}[1]{\markright{\thesection.\ #1}}

\chapter*{Introduction}
\textbf{Deep Learning} is a subset of machine learning that is concerned with 
neural networks that are use to learn underlying features in data.

In general, we can define machine learning as a program that starting from the 
input and the output of a system, learns the rules that govern the system. In 
order to obtain high performance, machine learning algorithms depends heavily on 
the \textbf{representation} of the data. Representation is therefore the fundamental, 
and many artificial intelligence tasks can be solved by designing the right set of features.

The most difference between Deep ML and ML is that, the first one try to learn an
efficient representation of data and than use it to train a learn model. The latter 
one use a representation of data specified by an expert to train a learn model.

Deep Learning use neural networks with many layers of activity vectors as 
representations and learning the connection strengths between that give rise to 
these vectors by following the stochastic gradient of an objective function that
measures how well the network is performing.

So, the key ingredient of deep learning is \textbf{Depth}. There are two main 
ways to measure the depth of a model:
\begin{enumerate}
    \item in terms of depth of the graph describing how concepts are related to
        each other.
    \item in terms of number of sequential instructions that must be executed to 
        evaluate the architecture. This can be influenced by the choice of basic 
        functions used.
\end{enumerate}

One solution to the problem of feature representation is to use machine learning
not only to find the mapping between input and output, but also to find the
representation itself. This approach is called \textbf{Representation Learning}.
The goal of this task is to identify the \textit{factor of variations} that 
explain the observed data. The goal of this task is to identify the factor of 
variations that explain the observed data.

The most common example of representation learning is the use of autoencoders.

A key part of representation learning consists in the \textbf{distributed 
    representation}, which means a many to many relationship between two types
of representation:
\begin{itemize}
    \item Each concept is represented by many neurons.
    \item Each neuron participates in the representation of many concepts.
\end{itemize}

Deep learning solves this central problem in representation learning by introducing
representations that are expressed in therms of other, simpler representations.
An example is bunch of letters form words, sets of words form phrases.

\chapter{Ripasso Algebra Lineare}
\begin{definizione} [\textbf{Spazio vettoriale}]
    Un insieme $V$ si dice \textbf{spazio vettoriale} se sono definite su $V$ due
    operazioni che godono di particolari proprietà:
    \begin{itemize}
        \item \textbf{Somma}: $+: V \times V \rightarrow V$
        \item \textbf{Prodotto per uno scalare}: $\cdot : V \times \mathbb{R} \rightarrow V$
    \end{itemize}
\end{definizione}

Le operazioni dello spazio vettoriale godono di alcune proprietà:
\begin{itemize}
    \item \textbf{Somma}:
          \begin{itemize}
              \item \textbf{Commutativa}: $\forall u,v \in V, u+v = v+u$
                    \begin{proof}
                        $\forall u,v \in V:$
                        $$ u+v = \left[\begin{array}{c}
                                    u_1 \\u_2\\ \vdots \\u_n
                                \end{array}\right] + \left[\begin{array}{c}
                                    v_1 \\v_2\\\vdots\\v_n
                                \end{array}\right] = \left[\begin{array}{c}
                                    u_1 + v_1 \\u_2 + v_2\\\vdots\\u_n+v_n
                                \end{array}\right] = \left[\begin{array}{c}
                                    v_1 + u_1 \\v_2 + u_2\\\vdots\\v_n+u_n
                                \end{array}\right] = \left[\begin{array}{c}
                                    v_1 \\v_2\\\vdots\\v_n
                                \end{array}\right] + \left[\begin{array}{c}
                                    u_1 \\u_2 \\\vdots\\u_n
                                \end{array}\right] = v+ u$$
                    \end{proof}
              \item \textbf{Associativa}: $\forall u,v,z \in V, (u+v)+z = v+(u+z)$
                    La dimostrazione si può fare facilmente utilizzando sempre la proprietà
                    associativa della somma tra scalari.
              \item \textbf{Esistenza dell'elemento neutro}: $\exists 0 \in V: 0+v = v,
                        \forall v\in V$
              \item $\forall u \in V, \exists w\in V \text{ unico}: u+w=0$
          \end{itemize}
    \item \textbf{Prodotto per uno scalare}:
          \begin{itemize}
              \item \textbf{Distributivo rispetto alla somma}: $\forall u,v \in
                        V, \forall \lambda \in \mathbb{R}, \lambda(u+v) = \lambda
                        u+\lambda v$
              \item \textbf{Distributivo rispetto alla somma tra scalari}:
                    $\forall u \in V, \forall \lambda,\mu\in \mathbb{R}: (\lambda +
                        \mu)u = \lambda \cdot u + \mu \cdot u$
              \item \textbf{Associativo rispetto al prodotto tra scalari}:
                    $\forall u \in V, \forall \lambda,\mu\in \mathbb{R}: (\lambda \cdot
                        \mu)u = \lambda \cdot (\mu \cdot u)$
              \item \textbf{Esistenza dell'elemento neutro}: $\exists 1 \in
                        \mathbb{R}:1\cdot v = v, \forall v\in V$
          \end{itemize}
\end{itemize}
\begin{esempio}
    Un esempio di spazio vettoriale è $V\in \mathbb{R}^n$
\end{esempio}
\begin{definizione} [\textbf{Prodotto scalare}]
    Sia $V$ uno spazio vettoriale, definiremo il \textbf{prodotto scalare} è un'operazione
    che a due elementi di $V$ associa un valore reale, $(\cdot, \cdot):V\times V
        \rightarrow \mathbb{R}$.Tale operazione soddisfa le seguenti proprietà:
    \begin{itemize}
        \item \textbf{Simmetrica}: $\forall u,v \in V, (u,v) = (v,u)$
        \item \textbf{Bi-lineare}: $\forall v_1,v_2,w \in V,\forall\alpha_1,
                  \alpha_2 \in \mathbb{R}:(\alpha_1v_1+\alpha_2v_2, w) =
                  \alpha_1(v_1, w) + \alpha_2(v_2, w)$
        \item $\forall v \in V, v\ne 0: (v,v) \ge 0$
    \end{itemize}
\end{definizione}
\begin{definizione} [\textbf{Norma di un vettore}]
    Sia $V$ uno spazio vettoriale, definiremo la \textbf{norma di un vettore} è
    un'operazione che a un elemento di $V$ associa un valore reale, $\|\cdot\|:
        V \rightarrow \mathbb{R}$. Tale operazione soddisfa le seguenti proprietà:
    \begin{itemize}
        \item \textbf{scalabilità}: $\forall v \in V,\forall \alpha \in \mathbb{R}:
                  \|\alpha v\| = |\alpha | \|v\|$
        \item \textbf{disuguaglianza triangolare}: $\forall v,w \in V: \|v+w\|
                  \le \|v\| + \|w\|$
        \item $\forall v \in V, v\ne 0: \|v\| > 0\implies \|0\| = 0$
    \end{itemize}
\end{definizione}
La definizione e le proprietà derivano dalla nota successiva
\begin{nota} [\textbf{Norma indotta dal prodotto scalare}]
    Sia $V$ uno spazio vettoriale in cui è definito un prodotto scalare allora è
    possibile calcolare la norma da $(\cdot, \cdot)$
    \begin{equation}
        \|v\| = \sqrt{(v,v)}  \equiv \|v\|_2
    \end{equation}
    Tale norma viene chiamata \textbf{norma indotta dal prodotto scalare }
\end{nota}
\begin{nota}
    Esistono diverse norme oltre alla $2$:
    \begin{itemize}
        \item $\|v\|_2 = \sqrt{\sum_{i=1}^{N}v_i^2} = \|v-0\|$
        \item $\|v\|_1 = \sum_{i=1}^{N}|v_i|$
        \item $\|v\|_\infty = \max_{i= 1\dots n}|v_i|$
    \end{itemize}
\end{nota}
La norma euclidea ($\|\cdot\|_2$) è utile per calcolare la distanza di vettori
perché $\|v\|_2$ calcola la distanza di $v$ da $0$, mentre $\|v-w\|_2$ calcola
la distanza tra $v$ e $w$.

Quindi dato un prodotto scalare possiamo sempre definire la norma, ma non possiamo
dire il contrario.
\begin{nota}
    La norma ha una proprietà indotta dalla sua definizione e dalle sue proprietà,
    $\forall v,w \in V$
    \begin{equation*}
        \|v\| -\|w\| \le \|v-w\|
    \end{equation*}
\end{nota}
Possiamo definire anche le norme per le matrici $V\in \mathbb{R}^{r\times c}$:
\begin{itemize}
    \item $\|A\|_F = \sqrt{\sum_{i,j} |a_{i,j}|^2}$
    \item $\|A\|_1 = \max_{i=1\dots r}{\sum_{j=1}^{c} |a_{i,j}|}$
    \item $\|A\|_\infty = \max_{j=1\dots c}{\sum_{i=1}^{r} |a_{i,j}|}$
\end{itemize}

Altre operazioni utili per gli spazi vettoriali, generalmente per vettori e matrici, è
la \textbf{trasposizione}.
\begin{equation}
    A=\left[\begin{array}{ccc}
            a & b & c \\
            d & e & f
        \end{array}\right]\in \mathbb{R}^{2\times3},  A^t = \left[\begin{array}{cc}
            a & d \\
            b & e \\
            c & f
        \end{array}\right]\in \mathbb{R}^{3\times2}
\end{equation}
Il problema di questa operazione è che non è \textbf{chiusa} ovvero non rimane
nello stesso spazio.

In aggiunta si ha \textbf{prodotto tra matrici e vettori}. Dati $A\in \mathbb{R}^{r\times c}$
e $v\in \mathbb{R}^{c}$ si ha:
\begin{equation*}
    Av = \left[\begin{array}{cccc}
            a_{11} & a_{12} & \cdots & a_{1c} \\
            a_{21} & a_{22} & \cdots & a_{2c} \\
            \vdots & \vdots & \dots  & \vdots \\
            a_{r1} & a_{r2} & \cdots & a_{rc}
        \end{array}\right] \left[\begin{array}{c}
            v_1 \\v_2\\\vdots\\v_c
        \end{array}\right] = \left[\begin{array}{c}
            a_{11}v_1+a_{12}v_2+\cdots a_{1c}v_c \\
            a_{21}v_1+a_{22}v_2+\cdots a_{2c}v_c \\
            \vdots                               \\
            a_{r1}v_1+a_{r2}v_2+\cdots a_{rc}v_c \\
        \end{array}\right]
\end{equation*}
Questa operazione è vincolata dal fatto che la matrice e il vettore devono
essere compatibili, le colonne della matrice devono essere uguali alle righe del
vettore. Inoltre abbiamo anche l'operazione di \textbf{prodotto matrice-matrice}.
Sia $A\in \mathbb{R}^{3\times 2}, B\in \mathbb{R}^{2\times 4}$
\begin{equation*}
    AB = \left[\begin{array}{cc}
            a & b \\
            c & d \\
            e & f
        \end{array}\right]\left[\begin{array}{cccc}
            g & h & i & l \\
            m & n & o & p \\
        \end{array}\right] = \left[\begin{array}{cccc}
            ag+bm & ah+bn & ai+bo & al+bp \\
            cg+dm & ch+dn & ci+do & cl+bp \\
            eg+fm & eh+fn & ei+fo & el+fp \\
        \end{array}\right]
\end{equation*}
Anche questa operazione è vincolante perché può essere effettuata solo quando la
prima matrice ha il numero di colonne uguale al numero di righe della seconda.
Inoltre la commutazione non sempre è fattibile perché potrebbero non essere
compatibili e in generale \textbf{non} è commutativa. La soluzione del prodotto
può essere conosciuta in anticipato quando tra i fattori del prodotto si hanno
matrici particolari.
\begin{definizione}[\textbf{Matrice sparsa}]
    Se una matrice $A\in \mathbb{R}^{n \times m}$ ha poche entrate diverse da $0$
    allora si dice \textbf{sparsa}.
\end{definizione}

Supponiamo di avere $A\in \mathbb{R}^{3\times 3}$ sparsa e $B\in \mathbb{R}^{3\times3}$
\begin{equation*}
    \left[\begin{array}{ccc}
            1 & 0 & 0 \\
            0 & 0 & 1 \\
            0 & 1 & 0
        \end{array}\right]  \left[\begin{array}{ccc}
            a & b & c \\
            e & f & g \\
            h & i & l
        \end{array}\right] = \left[\begin{array}{ccc}
            a & b & c \\
            h & i & l \\
            e & f & g \\
        \end{array}\right]
\end{equation*}
In questo caso si ha uno scambio della seconda riga con la terza.
\begin{equation*}
    \left[\begin{array}{ccc}
            a & b & c \\
            e & f & g \\
            h & i & l
        \end{array}\right] \left[\begin{array}{ccc}
            1 & 0 & 0 \\
            0 & 0 & 1 \\
            0 & 1 & 0
        \end{array}\right] = \left[\begin{array}{ccc}
            a & c & b \\
            e & g & f \\
            h & l & i \\
        \end{array}\right]
\end{equation*}
In questo caso abbiamo fatto lo scambio della seconda colonna con la terza.
Quindi pre o post moltiplicare una matrice $B$ per una matrice $A$ composta da
un solo uno per ogni riga/colonna ha l'effetto di scambiare le corrispondenti
righe/colonne.
\begin{definizione}[\textbf{Matrice di permutazione}]
    Matrici $A \in \mathbb{R}^{n\times n}$ che permettono di effettuare scambi di
    righe o colonne sono dette \textbf{matrici di permutazione}.
\end{definizione}
Queste sono utili per risolvere i sistemi lineari e permettono di velocizzare le
operazioni di calcolo.

Supponiamo ora di considerare il seguente esempio:
\begin{equation*}
    \left[\begin{array}{ccc}
            1 & 0 & 0 \\
            0 & 1 & 0 \\
            0 & 0 & 1
        \end{array}\right]  \left[\begin{array}{ccc}
            a & b & c \\
            d & e & f \\
            g & h & i
        \end{array}\right] = \left[\begin{array}{ccc}
            a    & b    & c    \\
            2a+d & 2f+e & 2c+f \\
            g    & h    & i    \\
        \end{array}\right]
\end{equation*}
Da esso, possiamo affermare che meno coefficienti ha la matrice più è facile
capire come sarà la matrice risultante.

In aggiunta possiamo trasformare un sistema di equazioni lineari in un prodotto
tra matrice e vettore ($Ax+b=0\equiv a_1x_1+a_2x_2+\dots a_n x_n = b$). Dove $a_i$
è il vettore dei coefficienti della variabile $x_i$. Questa rappresentazione dei
sistemi permette di risolverli facilmente utilizzando dei metodi specifici per le
tipologie di matrici particolari, fondamentale sarà quindi riconoscere la tipologia
di matrici.

Altre matrici particolari sono quelle \textbf{triangolari inferiori/superiori}
che sono utili per risolvere i sistemi lineari usando il metodo di sostituzioni.
\begin{esempio}
    Esempio di matrice triangolare inferiore
    \begin{equation*}
        \left[\begin{array}{ccc}
                1 & 0 & 0 \\
                2 & 1 & 0 \\
                1 & 1 & 1
            \end{array}\right]
    \end{equation*}
\end{esempio}
\begin{esempio}
    Esempio di matrice triangolare superiore
    \begin{equation*}
        \left[\begin{array}{ccc}
                1 & 1 & 1 \\
                2 & 1 & 0 \\
                1 & 0 & 0 \\
            \end{array}\right]
    \end{equation*}
\end{esempio}

Un'altra matrice utile è quella \textbf{simmetrica}, ovvero una matrice quadrata
tale che $a_{ij} = a_{ji}$.

Un altro tipo di matrice particolare è quella dei sistemi con moltissime variabili,
moltissime equazioni ciascuna con poche variabili, quindi si parlerà di
\textbf{sistema lineare sparsi}, ovvero con un numero minore del $10\%$ di
entrate con valori diversi da $0$. Sarà importante trovare un modo per salvare i
valori delle matrici in modo efficiente.

Un operazione utile per la risoluzione dei sistemi lineari è il \textbf{determinante},
ovvero $det(A): \mathbb{R}^{n\times n} \rightarrow \mathbb{R}$. Il determinante
è utile per scoprire la presenza di soluzioni e se sono uniche.
\begin{equation}
    det(A) = \sum_{j = 1}^{n}(-1)^{i+j} \, a_{ij} \, det(A_{ij})
\end{equation}
\end{document}
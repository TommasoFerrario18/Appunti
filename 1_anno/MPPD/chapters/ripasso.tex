\chapter{Ripasso probabilità}
\begin{definizione}[\textbf{Spazio degli eventi}]
    Lo \textbf{spazio degli eventi} $\Omega$ è l'insieme di tutti gli esisti.
\end{definizione}
\begin{definizione}[\textbf{Evento}]
    Un \textbf{evento} $A$ è un qualunque sottoinsieme di $\Omega$ tale che:
    \begin{equation*}
        A \subseteq \Omega \quad \text{e} \quad P(A) = \sum_{\omega \in A} P(\omega)
    \end{equation*}
\end{definizione}
\begin{definizione}[\textbf{Variabile casuale}]
    Una \textbf{variabile casuale} può essere un'osservazione, un esito o un
    evento il cui valore incerto.
\end{definizione}
L'insieme dei possibili valori che può assumere una variabile casuale è chiamato
\textbf{dominio} o \textbf{spazio degli eventi}.
\begin{definizione}[\textbf{Spazio di probabilità}]
    Uno \textbf{spazio di probabilità} o \textbf{modello di probabilità} è uno
    spazio degli eventi corredato da un assegnamento $P(\omega)$ tale che:
    \begin{equation*}
        0 \leq P(\omega) \leq 1, \quad \sum_{\omega \in \Omega} P(\omega) = 1
    \end{equation*}
\end{definizione}
Si possono creare eventi più complessi combinando gli esiti di diverse variabili
casuali.
\begin{definizione}[\textbf{Evento atomico}]
    Un \textbf{evento atomico} o \textbf{campione} è una specificazione completa
    del valore delle variabili casuali di interesse.
\end{definizione}
Se nel contesto ci sono diverse variabili casuali, il numero di eventi atomici
è la combinazione dei valori tra le singole variabili.

L'insieme di tutti i possibili eventi atomici gode delle seguenti proprietà:
\begin{itemize}
    \item \textbf{Mutualmente esaustivo}: non ci sono altri eventi atomici.
    \item \textbf{Mutualmente esclusivo}: può verificarsi solo un evento atomico
          alla volta.
\end{itemize}
\begin{definizione}[\textbf{Variabile aleatoria}]
    Una \textbf{variabile aleatoria} è una variabile che può assumere valori
    diversi in corrispondenza di eventi che costituiscono una partizione dello
    spazio delle probabilità.
\end{definizione}
La teoria della probabilità può essere derivata dalle seguenti assunzioni:
\begin{itemize}
    \item Tutte le probabilità sono comprese tra 0 e 1.
          \begin{equation}
              0 \leq P(A) \leq 1
          \end{equation}
    \item Se qualcosa è necessariamente vero allora la probabilità è 1. Un
          ragionamento analogo vale per la probabilità nulla.
          \begin{equation}
              P(\top) = 1, \quad P(\bot) = 0
          \end{equation}
    \item La \textbf{probabilità disgiunta} che due variabili siano vere si
          ottiene come:
          \begin{equation}
              P(A \cup B) = P(A) + P(B) - P(A \cap B) \ \equiv \ P(A \lor B) =
              P(A) + P(B) - P(A \land B)
          \end{equation}
\end{itemize}
\begin{definizione}[\textbf{Probabilità congiunta}]
    La \textbf{probabilità congiunta}, ovvero il fatto che due eventi si
    verifichino contemporaneamente, si ottiene partendo dalla probabilità
    condizionata e dalle probabilità dei singoli eventi.
    \begin{equation}
        P(A \cap B) = P(A|B) \cdot P(B) = P(B|A) \cdot P(A) \equiv P(A, B) \equiv
        P(A \land B)
    \end{equation}
\end{definizione}
\begin{definizione}[\textbf{Probabilità condizionata}]
    La \textbf{probabilità condizionata} rappresenta la verosimiglianza che un
    evento $A$ si verifichi dato che un altro evento $B$ si è verificato.
    La probabilità di $A$ dato $B$ è definita come:
    \begin{equation}
        P(A|B) = \frac{P(A \cap B)}{P(B)} = \frac{P(A) + P(B) - P(A \cup B)}{P(B)}
    \end{equation}
\end{definizione}
Da questa definizione possiamo dire che le probabilità condizionate riflettono
il fatto che alcuni eventi rendono altri eventi più o meno verosimili. Sempre
in quest'ottica, se un evento non influisce sulla realizzazione di un altro
evento, allora i due sono detti \textbf{indipendenti}. Se due eventi sono
indipendenti allora:
\begin{equation}
    P(A | B) = P(A)
\end{equation}
\begin{definizione}[\textbf{Teorema di Bayes}]
    Il \textbf{teorema di Bayes} è un'importante relazione tra probabilità
    condizionata e probabilità congiunta. Esso afferma che la probabilità di
    un evento condizionato ad un altro evento può essere calcolata in termini
    della probabilità dell'evento condizionante e della probabilità
    condizionata dell'evento di interesse.
    \begin{equation}
        P(A|B) = \frac{P(B|A) \cdot P(A)}{P(B)}
    \end{equation}
    \begin{proof}
        La dimostrazione del teorema di Bayes si ottiene partendo dalla
        definizione di probabilità condizionata e dalla probabilità congiunta.
        \begin{equation*}
            P(A \cap B) = P(A|B) \cdot P(B) = P(B|A) \cdot P(A) \implies  P(A|B)
            \cdot P(B) = P(B|A) \cdot P(A)
        \end{equation*}
        Da cui si ottiene:
        \begin{equation*}
            P(A|B) = \frac{P(B|A) \cdot P(A)}{P(B)}
        \end{equation*}
    \end{proof}
\end{definizione}
Un altro modo di interpretare la regola di Bayes è quello di considerare
gli eventi come cause non necessariamente osservabili. In questo caso, la
verosimiglianza degli effetti osservabili date le cause non osservabili, in
formule:
\begin{equation*}
    P(\text{causa}|\text{effetto}) = \frac{P(\text{effetto}|\text{causa}) \cdot
        P(\text{causa})}{P(\text{effetto})}
\end{equation*}
La regola di Bayes ci permette di usare questo modello per inferire la
verosimiglianza della causa nascosta data l'osservazione dell'effetto.

Una semplificazione, che può tornare utile in determinate condizioni, della
regola di Bayes si può fare se si conosce la $P(E|C)$ per ogni causa $C$. In
queste condizioni possiamo evitare di conoscere $P(E)$.
\begin{equation}
    P(C|E) = \frac{P(E|C) \cdot P(C)}{\sum_{i} P(E|C_i) \cdot P(C_i)}
\end{equation}
Questa procedura consiste nel normalizzare le probabilità condizionate dividendo
per la sommatoria.
\begin{nota}
    È importante osservare che la regola di Bayes confida nel fatto che l'effetto
    deve essere scaturito da una delle cause ipotizzate. Se ci sono altre
    cause non considerate, la regola di Bayes non è applicabile.
\end{nota}
Il significato del teorema di Bayes può essere espresso come: apprendere
attraverso l'esperienza.

Posso riassumere nella seguente formula:
\begin{equation}
    P(A|B) = \alpha \cdot P(B | A) \cdot P(A)
\end{equation}
dove:
\begin{itemize}
    \item $P(A)$ è la probabilità a priori di $A$.
    \item $P(B|A)$ è la verosimiglianza di $B$ dato $A$.
    \item $P(A|B)$ è la probabilità a posteriori di $A$.
    \item $\alpha$ è un fattore di normalizzazione.
\end{itemize}

Supponiamo ora di voler calcolare la probabilità della causa date una congiunzione
di evidenze. Siccome abbiamo più effetti, il modello casuale si complica.
Ad esempio se abbiamo $N$ effetti booleani che portano a una determinata causa,
ci saranno $2^N$ diverse combinazioni di evidenze che dobbiamo modellare.
Questo porta a un notevole aumento della complessità del modello.
Per ridurre questa complessità, la probabilità congiunta di un insieme di eventi
può essere espressa come una \textbf{catena di probabilità condizionate}.
\begin{definizione}[\textbf{Catena di probabilità condizionate}]
    Una \textbf{catena di probabilità condizionate} è una sequenza di probabilità
    condizionate che rappresenta la probabilità congiunta di un insieme di eventi.
    \begin{equation}
        P(A_1 \cap A_2 \cap \ldots \cap A_n) = P(A_1) \cdot P(A_2|A_1) \cdot
        P(A_3|A_1 \cap A_2) \cdot \ldots \cdot P(A_n|A_1 \cap \ldots \cap A_{n-1})
    \end{equation}
\end{definizione}
Questo approccio risulta utile per fare \textbf{inferenza}. Consideriamo un
insieme di eventi $E_1, E_2, \dots, E_n$ e tutte le possibili combinazioni dei
loro valori, per semplicità usiamo $\top, \bot$. Inoltre, conosciamo tutti i
valori $P(E_1, E_2, \dots, E_n)$ e supponiamo che un sottoinsieme di questi
presenti un valore definito, come ad esempio $E_j = Vero = e$.
Chiamiamo \textbf{inferenza probabilistica} il processo di calcolo del valore:
\begin{equation}
    P(E_i = Vero|E_j=e)
\end{equation}
In generale l'inferenza probabilistica non è trattabile usando questo metodo
perché avendo $N$ eventi binari, dovremmo avere una lista di $2^N$ probabilità 
congiunte. Si riduce il numero utilizzando metodi approssimativi, qualitativi o 
sfruttando l'\textbf{indipendenza condizionata}.

Usando l'indipendenza condizionata allora per alcune variabili non c'è bisogno di
utilizzare il teorema di Bayes per calcolare la probabilità condizionata, perché
quella viene data dalla probabilità della causa.

Infatti, dati tre eventi $A, B, C$, sapendo che $A$ è indipendente da $C$ allora:
\begin{equation*}
    P(A | B, C) = P(A | B)
\end{equation*}
Sempre per l'indipendenza condizionata si ha anche:
\begin{equation*}
    P(A, B | C) = P(A | C) \cdot P(B | C)
\end{equation*}
\begin{nota}
    Quindi se $A$ e $B$ sono variabili indipendenti allora:
    \begin{equation*}
        P(A|B) = P(A) \iff P(B|A) = P(B) \iff P(A,B) = P(A)\cdot P(B)
    \end{equation*}
\end{nota}
\begin{nota}
    Quindi se $A$ e $B$ sono variabili condizionalmente indipendenti allora:
    \begin{equation*}
        P(A,B|C) = P(A|C) \cdot P(B|C), \quad C \neq \emptyset
    \end{equation*}
\end{nota}
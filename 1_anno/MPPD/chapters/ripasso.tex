\chapter{Ripasso probabilità}
\begin{definition}
    Una \textbf{variabile casuale} può essere un'osservazione, un esito o un evento
    il cui valore incerto.
\end{definition}

L'insieme dei possibili valori che può assumere una variabile casuale è chiamato
\textbf{dominio}. 

\begin{definition}
    Uno \textbf{spazio di probabilità} o \textbf{modello di probabilità} è uno spazio
    degli eventi corredato da un assegnamento $P(\omega)$ tale che:
    $$0 \le P(\omega)\le 1, \sum_{\omega\in \Omega} P(\omega) = 1$$
\end{definition}
Si possono creare eventi più complessi combinando gli esiti di diverse variabili 
casuali.
\begin{definition}
    Un \textbf{evento atomico} è una specificazione completa dei valori delle variabili
    di interesse
\end{definition}
Se nel constesto ci sono diverse variabili casuali, il numero di eventi atomici 
è la combinazione dei valori tra le singole variabili.

L'insieme di tutti i possibili eventi atomici ha le seguenti proprietà:
\begin{itemize}
    \item \textbf{completo}: non ci sono altri eventi atomici
    \item \textbf{mutualmente esclusivo}: può verificarsi solo un evento atomico alla volta
\end{itemize}

\begin{definition}
    Lo \textbf{spazio degli eventi} è l'insieme di tutti gli esisti
\end{definition}
\chapter*{Catene Markoviane}

Introdurremo dei modelli probabilistici che permettodo di descrivere un mondo 
mutevole, avremo una serie di variabili casuali descritte da uno stato al variare 
del tempo.  Le relazioni tra i cambiamenti di stato nel tempo descrivono l'evoluzione 
della variabile.

I modelli possono essere:
\begin{itemize}
    \item \textbf{statici}: il valore delle variabili non cambia nel tempo
    \item \textbf{dinamici}: il valore delle variabili cambia nel tempo, lo stato corrente 
    dipende dalla storia e il processo di cambiamento è descritto da una serie di fotografie al tempo $t$
    ogniuna delle quali contiene un insieme di variabili casuali.
\end{itemize}

\begin{definizione}
    Un \textbf{processo stocastico} $\{X(t), t\in T\}$ è un insieme di variabilli aleatorie 
    per cui ciascuna assume una distribuzione di valori al tempo $t$ dipendente dal 
    tempo precedente.    
\end{definizione}

Il tempo espresso dagli indici $t$ possono essere continui o discreti, inoltre possono 
essere finiti o infiniti. 

Avremo quindi diverse tipologie di processi stocastici:
\begin{itemize}
    \item processi stocastici a tempo continuo 
    $$\{X(t), t>0\}$$
    \item processi stocastici a tempo discreto 
    $$\{X(t), t=0,1,\dots\}$$
    \item processi stocastici a stati continui 
    $$\{X(t), t=0,1,\dots\}, X(t) \text{ distribuzione continua}$$
    \item processi stocastici a stati discreti
    $$\{X(t), t=0,1,\dots\}, X(t) \text{ distribuzione discreta}$$
\end{itemize}

La variabile $X(t)$ è un valore dato dal sistema al tempo $t$, cioè un valore della 
variabile che descrive uno stato del sistema al tempo $t$. Questo stato può essere 
puntuale o incrementale.

\begin{esempio}
    $X(t)$ può rappresentare il numero di visitatori fino al tempo $t$, oppure 
    può rappresentare il numero di visitatori al tempo $t$.
\end{esempio}

Un esempio di processo stocastico naive è il \textbf{Random Walk}, questo processo viene descrito 
dalla seguente equazione:
$$X(t) = X(t-1) + \epsilon_t$$
Dove $\epsilon_t \in \{-1, 1\}$,  $P(\epsilon_t = -1) = p$ e $P(\epsilon_t = 1) = 1- p$.
Se $p=\frac{1}{2}$ allora il processo è bilanciato, altrimenti il modello verrebbe 
chiamato random walk con \textbf{drift}.

In realtà si può esprimere la distribuzione $\epsilon_t$ come continua e questo 
crea delle randomicità maggiore. 

Esistono anche altri processi stocastici è il \textbf{processo autoregressivo del primo ordine}
dove 
$$X_t = \alpha X_{t-1}+b+\epsilon_t$$
con $a,b$ constanti, con $-1<a<1$ e $\epsilon_t \sim N(0,1)$.

I processi stocastici sono importanti perché posso effettuare delle predizioni.
Per esempio il tracking degli oggetti in movimento nei video, oltre al tracking 
possono essere utilizzati anche per generare dei processi (ex: generazioni di comportamenti 
casuali per le simulazioni).

\section{Processi Markoviani}
I processi stocastici possono rispettare una proprietà chiamata \textbf{proprietà Markoviana del 
primo ordine}, la proprietà assicura che la distribuzione di probabilità per tutti 
i possibili valori futuri del processo dipende solo dal valore corrente e non dai 
valori passati o da altre informazioni. 

La generalizzazione di questa proprietà specifica il fatto che la distribuzione di probabilità per tutti 
i possibili valori futuri del processo dipende da una storia finita.

Formalmente coincide con 
$$P(X_{t+1} = i_{t+1 } | X_t =i_t, \dots, X_{0} = i_{0})=P(X_{t+1} = i_{t+1 } | X_t =i_t, \dots, X_{t-k} = i_{t-k}), k < \infty$$

Per la proprietà Markoviana del primo ordine formalmente sarà
$$P(X_{t+1} = i_{t+1 } | X_t =i_t, X_{t-1} = i_{t-2}, \dots, X_{0} = i_{0}) = P(X_{t+1} = i_{t+1 } | X_t =i_t)$$

I processi stocastici che rispettano la \textbf{proprietà Markoviana} sono detti 
\textbf{processi Markoviani} e verranno rappresentati attraverso una \textbf{catena di Markov}.

Quindi un processo stocastico a tempi discreti che rispetta la proprietà Markoviana 
del primo ordine è una \textbf{catena di Markov del primo ordine} se per tutti gli stati si ha 
$$P(X_{t+1} = i_{t+1 } | X_t =i_t, \dots, X_{0} = i_{0}) = P(X_{t+1} = i_{t+1 } | X_t =i_t)$$
dove $P(X_0 = i_0) = q_i$, $q_i$ viene espressa come distribuzione di probabilità 
a priori che può essere data se si conosce la distribuzione iniziale, se non si 
ha inizialmente allora si può considerare la distribuzione uniforme. 

\begin{nota}
    Si può generalizzare a più ordini allora deovno rispettare 
    la proprietà Markoviana a più ordini.     
\end{nota}

Questi processi si riferiscono a stati direttamente osservabili, esistono anche 
le catene associate a stati non osservabili direttamente allora si parlerà di 
\textbf{Catena di Markov nascoste}.

Se la probabilità di un certo evento è indipendente dal tempo $t$ la catena di 
Markov si definisce \textbf{stazionaria} e si ha che 
$$P(X_{t+1}=j | X_t = i) = p_{ij}$$
$p_{ij}$  specifica la probabilità di passare dal valore $X_t = i$ a $X_{t+1} = j$,
quindi le catene di Markov possono essere rappresentate da una \textbf{Matrice di 
transizione} (ad un passo)

$$ P = \left[\begin{array}{cccc}
    p_{11} & p_{12} & \dots & p_{1n}\\
    p_{21} & p_{22} & \dots & p_{2n}\\
    \vdots & \vdots & \ddots & \vdots\\
    p_{n1} & p_{n2} & \dots & p_{nn}\\
\end{array}\right]$$
dove $1,\dots, n$ sono i possibili valori che può assumere la variabile aleatoria.
La matrice ha i seguenti vincoli $p_{ij}\ge 0$ e $\sum_{j=0}^{n}p_{ij}=1, \forall i$.

\begin{nota}
    Per i modelli stazionari la matrice di transizione rimane identica nel tempo, 
    altrimenti se varia nel tempo allora saranno modelli non stazionari.
\end{nota}

La rappresentazione grafica di una \textbf{Catena di Markov} è un grafo in cui 
gli stati sono i valori possibili che la variabile può assumere ($i,j,\dots$), gli archi rappresentano 
da che valore a quale valore si può passare nel tempo (ex: $(i,j), (j,k),\dots$) e gli archi vengono etichettati 
da $l(i,j) = p_{ij}$, quindi la matrice di transizione diventa la matrice di 
adiacienza.

A livello formale per definire la \textbf{Catena di Markov} devo avere l'insieme 
di stati $$\{S_1,S_2,\dots, S_n\}$$
Serve la probabilità di transizione tra stati
$$a_{ij} = P(X_{t+1} = S_i | X_t = S_j)$$
Successivamente serve la distribuzione iniziale degli stati
$$\pi_i = P[X_1 = S_i]$$
Se non si conosce la distribuzione iniziale allora si usa quella uniforme, se 
si conosce con certezza il valore che assume la variabile all'inizio allora se $i$
è il valore che assume, $\pi_i = 1, \forall j \ne i, \pi_j=0$.

Per calcolare la \textbf{probabilità di una serie storica} mi basta effettuare 
il prodotto della probabilità di ciascuno stato dato quello precedente.
\begin{esempio}
    Se vogliamo calcolare la probabilità della serie $Sn, R,R,R,Sw,Sw$, la probabilità 
    coincide con 
    $$P(Sn) \cdot P(R|Sn) \cdot P(R|R)\cdot P(R|R)\cdot P(Sw|R) \cdot P(Sw|Sw)$$
    La probabilità iniziale $P(Sn) = \pi_{Sn}$, con $\pi$ distribuzione iniziale 
    della variabile aleatoria.
\end{esempio}

Quando la serie storica è molto lunga il prodotto tra probabilità tende a $0$, quindi 
conviene passare al logaritmo.

\subsection{Inferenza}

Possiamo effettuare inferenza di valori per istanti di tempo più lontani.

Data la matrice di transizione, possiamo calcolare la probabilità che dopo $n$ 
passi si trovi in uno stato specifico $j$. Dovremo calcolare $P(X_{m+n} = j | X_{m}=i) = P(X_n=j | X_0=i)=P_{ij}(n)$.
Dove $P_{ij}(n) = P^n_{ij}$ 
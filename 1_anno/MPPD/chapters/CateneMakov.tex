\chapter{Catene Markoviane}
Introdurremo dei modelli probabilistici che permettono di descrivere un mondo
che cambia nel corso del tempo. Per fare ciò, avremo bisogno di una serie di
variabili casuali descritte da uno stato che varia nel tempo. Le relazioni tra i
cambiamenti di stato nel tempo descrivono l'evoluzione della variabile.

Possiamo quindi distinguere due tipologie di modelli:
\begin{itemize}
    \item \textbf{Statici}: il valore delle variabili non cambia nel tempo.
    \item \textbf{Dinamici}: il valore delle variabili cambia nel tempo. In questa
          tipologia, lo stato corrente dipende dalla storia e il processo di
          cambiamento è descritto da una serie di fotografie, o anche dette
          \textit{time slice}, ognuna delle quali contiene un insieme di variabili
          casuali.
\end{itemize}
\begin{definizione}[\textbf{Processo stocastico}]
    Un \textbf{processo stocastico} $\{X(t), t\in T\}$ è un insieme di variabili
    aleatorie che evolve nel tempo. Nello specifico, $X(t)$ rappresenta il valore
    della variabile aleatoria al tempo $t$.
\end{definizione}
Il tempo espresso dagli indici $t$ possono essere continui o discreti, inoltre
possono essere finiti o infiniti.

Avremo quindi diverse tipologie di processi stocastici:
\begin{itemize}
    \item Processi stocastici a tempo continuo:
          \begin{equation*}
              \{X(t), t>0\}
          \end{equation*}
    \item Processi stocastici a tempo discreto:
          \begin{equation*}
              \{X(t), t=0,1,\dots\}
          \end{equation*}
    \item Processi stocastici a stati continui:
          \begin{equation*}
              \{X(t), t=0,1,\dots\}, X(t) \text{ distribuzione continua}
          \end{equation*}
    \item Processi stocastici a stati discreti:
          \begin{equation*}
              \{X(t), t=0,1,\dots\}, X(t) \text{ distribuzione discreta}
          \end{equation*}
\end{itemize}
La variabile $X(t)$ è un valore dato dal sistema al tempo $t$, cioè un valore
della variabile che descrive uno stato del sistema al tempo $t$. Questo stato
può essere puntuale o cumulativo.
\begin{esempio}
    $X(t)$ può rappresentare il numero di visitatori fino al tempo $t$ (cumulativo),
    oppure può rappresentare il numero di visitatori al tempo $t$ (puntuale).
\end{esempio}
Un esempio di processo stocastico naive è il \textbf{Random Walk}, questo processo
viene descritto dalla seguente equazione:
\begin{equation*}
    X(t) = X(t-1) + \varepsilon_t
\end{equation*}
dove $\varepsilon_t \in \{-1, 1\}$,  $P(\varepsilon_t = -1) = p$ e $P(\varepsilon_t = 1) = 1- p$.
Se $p=\frac{1}{2}$ allora il processo è bilanciato, altrimenti il modello verrebbe
chiamato random walk con \textbf{drift}.

Un modo per generalizzare questo processo è quello di esprimere la distribuzione
$\varepsilon_t$ come continua, ad esempio $\varepsilon_t \approx N(0, 1)$. Questo
crea una randomicità maggiore.

Un'altro esempio di processo stocastico è il \textbf{processo autoregressivo del
    primo ordine} (\textit{first order autoregressive process}), il quale è descritto
dalla seguente equazione:
\begin{equation*}
    X_t = \alpha X_{t-1} + b + \varepsilon_t
\end{equation*}
con $\alpha,b$ constanti, con $-1<\alpha<1$ e $\varepsilon_t \approx N(0,1)$.

I processi stocastici sono importanti perché posso effettuare delle predizioni.
Per esempio il tracking degli oggetti in movimento nei video, oltre al tracking
possono essere utilizzati anche per generare dei processi (ex: generazioni di
comportamenti casuali per le simulazioni).
\section{Processi Markoviani}
I processi stocastici possono rispettare una proprietà chiamata \textbf{proprietà
    Markoviana del primo ordine}, la proprietà assicura che la distribuzione di
probabilità per tutti i possibili valori futuri del processo dipende solo dal
valore corrente e non dai valori passati o da altre informazioni.

La generalizzazione di questa proprietà specifica il fatto che la distribuzione
di probabilità per tutti i possibili valori futuri del processo dipende da una
storia finita.

Tale proprietà può essere espressa in formule come:
\begin{equation}
    P(X_{t+1} = i_{t+1 } | X_t =i_t, \dots, X_{0} = i_{0})=P(X_{t+1} = i_{t+1 } | X_t =i_t)
\end{equation}
I processi stocastici che rispettano la \textbf{proprietà Markoviana} sono detti
\textbf{processi Markoviani} e verranno rappresentati attraverso una \textbf{catena di Markov}.

Quindi un processo stocastico a tempi discreti che rispetta la proprietà Markoviana
del primo ordine è una \textbf{catena di Markov del primo ordine} se per tutti
gli stati si ha:
\begin{equation*}
    P(X_{t+1} = i_{t+1 } | X_t =i_t, \dots, X_{0} = i_{0}) = P(X_{t+1} = i_{t+1 } | X_t =i_t)
\end{equation*}
dove $P(X_0 = i_0) = q_i$, $q_i$ viene espressa come distribuzione di probabilità
a priori che può essere data se si conosce la distribuzione iniziale, se non si
ha inizialmente allora si può considerare la distribuzione uniforme.
\begin{nota}
    Si può generalizzare a più ordini allora devono rispettare la proprietà
    Markoviana a più ordini.
\end{nota}
Questi processi si riferiscono a stati direttamente osservabili, esistono anche
le catene associate a stati non osservabili direttamente allora si parlerà di
\textbf{Catena di Markov nascoste}.

Se la probabilità di un certo evento è indipendente dal tempo $t$ la catena di
Markov si definisce \textbf{stazionaria} e si ha che:
\begin{equation*}
    P(X_{t+1} = i_{t+1 } | X_t =i_t) = p_{ij}
\end{equation*}
dove $p_{ij}$ specifica la probabilità di passare dal valore $X_t = i$ a $X_{t + 1} = j$,
quindi le catene di Markov possono essere rappresentate da una \textbf{Matrice di
    transizione} (ad un passo), ovvero matrici quadrate di dimensione $n$ con $n$
numero di stati possibili. Ogni elemento della matrice rappresenta la probabilità
di passare da uno stato all'altro.
\begin{equation*}
    P = \begin{bmatrix}
        p_{11} & p_{12} & \dots  & p_{1n} \\
        p_{21} & p_{22} & \dots  & p_{2n} \\
        \vdots & \vdots & \ddots & \vdots \\
        p_{n1} & p_{n2} & \dots  & p_{nn} \\
    \end{bmatrix}
\end{equation*}
La matrice ha i seguenti vincoli $p_{ij} \geq 0$ e $\sum_{j=0}^{n}p_{ij}=1, \forall i$.
\begin{nota}
    Per i modelli stazionari la matrice di transizione rimane identica nel tempo,
    altrimenti se varia nel tempo allora saranno modelli non stazionari.
\end{nota}

La rappresentazione grafica di una \textbf{Catena di Markov} è un grafo in cui i
nodi rappresentano gli stati che la variabile può assumere ($i,j,\dots$), mentre
gli archi rappresentano la probabilità di passare da uno stato all'altro. Quindi
la matrice di transizione diventa la matrice di adiacenza del grafo.

A livello formale per definire la \textbf{Catena di Markov} devo avere l'insieme
di stati:
\begin{equation*}
    S = \{S_1, S_2, \dots, S_n\}
\end{equation*}
la probabilità di transizione tra stati:
\begin{equation*}
    p_{ij} = P(X_{t+1} = S_i | X_t = S_j)
\end{equation*}
e la distribuzione iniziale degli stati:
\begin{equation*}
    \pi_i = P[X_0 = S_i]
\end{equation*}
Se non si conosce la distribuzione iniziale allora si usa quella uniforme, se
si conosce con certezza il valore che assume la variabile all'inizio allora se $i$
è il valore che assume, $\pi_i = 1, \forall j \ne i, \pi_j=0$.

Per calcolare la \textbf{probabilità di una serie storica} mi basta effettuare
il prodotto della probabilità di ciascuno stato dato quello precedente.
\begin{esempio}
    Se vogliamo calcolare la probabilità della serie $Sn, R,R,R,Sw,Sw$, la
    probabilità coincide con:
    \begin{equation*}
        P(Sn, R,R,R,Sw,Sw) = P(Sn) \cdot P(R|Sn) \cdot P(R|R)\cdot P(R|R)\cdot P(Sw|R) \cdot P(Sw|Sw)
    \end{equation*}
    La probabilità iniziale $P(Sn) = \pi_{Sn}$, con $\pi$ distribuzione iniziale
    della variabile aleatoria.
\end{esempio}
Quando la serie storica è molto lunga il prodotto tra probabilità tende a $0$, 
quindi conviene passare al logaritmo.
\subsection{Inferenza}
Possiamo effettuare inferenza di valori per istanti di tempo più lontani.

Data la matrice di transizione, possiamo calcolare la probabilità che dopo $n$
passi si trovi in uno stato specifico $j$. Per fare ciò, dovremo calcolare:
\begin{equation}
    P(X_{m + n} = j | X_{m} = i) = P(X_n = j | X_0 = i)=P_{ij}(n)
\end{equation} 
dove $P_{ij}(n)$ rappresenta la probabilità di passare dallo stato $i$ allo stato
$j$ in $n$ passi. Questa probabilità può essere calcolata attraverso la seguente
formula:
\begin{equation}
    P_{ij}(n) = \prod_{i = 1}^{n} P_{ij}
\end{equation}
ovvero, si calcola il prodotto riga per colonna della matrice di transizione
per $n$ volte. In questo modo, si considerano tutti i possibili percorsi che
portano dallo stato $i$ allo stato $j$ in $n$ passi.
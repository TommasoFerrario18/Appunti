\chapter{Reti Bayesiane} \label{cap:RetiBayesiane}
\section{Introduzione}
Le reti bayesiane sono un \textbf{modello grafico probabilistico} che rappresenta le relazioni
probabilistiche tra un insieme di variabili. Queste reti sono utili per rappresentare
le relazioni di dipendenza tra le variabili e per effettuare inferenze su di esse.

Usando l'indipendenza e l'indipendeza condizionata il modello causale è molto più compatto, il
numero di parametri passa da $\mathcal{O}(2^n)$ a $\mathcal{O}(n)$, dove $n$ 
è il numero degli effetti.

\begin{definizione}
    Una rete bayesiana è un grafo orientato aciclico in cui ogni nodo (variabili aleatorie)
    ha associato una tabella di probabilità condizionata e gli archi definiscono 
    la dipendenza e indipendenza condizionale tra le variabili.
\end{definizione}

se $(x,y)\in E$ dice che $x$ causa $y$ quindi $x$ è genitore di $y$.

I nodi contengono le distribuzioni delle variabili e quindi la CPT (tabella delle probabilità condizionate).
Le CPT dicono quant'è la probabilità di assumere un valore per una variabile di un nodo, condizionato al valore delle variabili
dei genitori. L'assunzione delle reti è che ogni nodo è condizionalmente indipendente
dai suoi non discendenti dati i suoi genitori.

Non si vogliono cicli perché non si permette che un noti influenzi se stesso.
Se specifico solo il vero nella CPT posso ricavare il falso faccio 1-vero e quindi
il falso è un parametro dipendente. Parametri totali $8 = 2*2^2$: $2*$ per i valori che può
assumere il figlio $2^2$ sono i valori che possono assumere i genitori.

Per le reti bayesiane si può fare sia inferenza diagnostica (dai figli ai padri) o prognostica (dai genitori ai figli).
Posso anche calcolare la probabilità di variabili che non sono in relazione padre figlio ma che sono
connesse.

La semantica delle reti è di due tipi:
%tipi
importanti per:
\begin{itemize}
    \item primo tipo: ricopre importanza centrale per comprendere come progettare ed implementare procedure di inferenza
    \item secondo tipo: la importante per comprendere come sia possibile  contruire una rete
\end{itemize}

La costruzione delle reti si può effettuare o a mano se esperti del dominio oppure
con algoritmi di apprendimento.

Per fare inferenza utilizzando la probabilità congiunta dobbiamo 
$$P(v_1,...,v_n) = \prod_{i=1}^{n} P(v_i|parents(v_i))$$
Non si vedono più le dipendenze di una variabile da tutte le altre ma bensì probabilità della variabile 
condizionata al suo genitore.

La chain rule coincide con la formula di fattorizzazione se i parent sono 

costruzione incrementale:
\begin{itemize}
    \item scelgo l'ordinamento topologico
    \item prendo la prima variabile e la inserisco nella rete
    \item prendo i parents tra i nodi rimanenti, trovo i genitori
    \item calcolo la CPT, conteggio vero e proprio
    \item 
\end{itemize}

osservazioni sulle reti:
\begin{itemize}
    \item aderenza sul dominio
    \item definire le relazioni di dipendenza impatta notevolmente sulla quantità
    di parametri
\end{itemize}

LA costruzione deve essere fatta in modo attento perché bisogna definire le variabili,
l'ordinamento topologico (prima le cause e poi le conseguenze) e poi le dipendenze.

Le CPT rappresentano le relazioni di dipendenza e indipendenza condizionale
grazie alla formula dei parent.
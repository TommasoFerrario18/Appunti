\chapter{Introduzione}
In questo corso si analizzeranno modelli probabilistici. Questi modelli permettono
di quantificare l'incertezza delle informazioni e successivamente prendere delle
decisioni. Questi passi sono fondamentali perché siamo costantemente circondati
da tantissimi dati (evidenze) che per loro natura sono incerti, quindi dobbiamo
avere un modo per gestire l'incertezza con l'obiettivo di prendere delle
decisioni sensate.

Quando devo prendere delle decisioni sui dati che analizzo, posso avere diversi
problemi:
\begin{itemize}
    \item Dati mancanti/inesatti
    \item Evidenze inconsistenti
    \item Diverse fonti di incertezza
\end{itemize}
I \textbf{modelli probabilistici} rappresentano l'incertezza tramite le dipendenze
tra le variabili (struttura del modello) e le probabilità (parametri del modello).
Rappresentando l'incertezza, si riesce ad automatizzare la parte di previsione,
di conseguenza risultano:
\begin{itemize}
    \item Scalabili
    \item Robusti
    \item Adattivi
\end{itemize}

Per ora sono stati visti solo modelli \textbf{discriminativi}, al contrario i
modelli probabilistici sono \textbf{generativi} perché sono caratterizzati da
variabili aleatorie, le quali possono essere campionate e permettono di effettuare
delle simulazioni. Essi si basano sulla teoria della probabilità e quindi ci
permetteranno di inferire quantità sconosciute e allo stesso tempo apprendere.

A livello di matematica si partirà da Teorema di Bayes per adattare i modelli
nel tempo.

La fase di inferenza si classifica in due tipi:
\begin{itemize}
    \item \textbf{Diagnostica}: dalla classe si ricava l'evidenza.
    \item \textbf{Prognostica}: dall'evidenza si ricava la classe.
\end{itemize}
\begin{esempio}
    Definizione di un sistema di Ranking per gli scacchi basato sulla statistica.
    L'idea si basa sul fatto che non è possibile la misura diretta, ogni risultato
    della partita dipende dalle persone. Quindi al termine di ogni partita bisogna
    aggiornare la posizione in base al risultato della partita e chi ha giocato.

    Si esprime l'assunzione che la variabile che modella la mossa ha una distribuzione
    normale.

    Abbiamo due variabili gaussiane che rappresentano il valore del giocatore, Quando
    si conosce l'esito della partita, si cambia la distribuzione delle due variabili.

    $y$ è il risultato della partita
    $\pi$ sono le caratteristiche del giocatore
    $s$ è il livello del problema.

    L'inferenza sarà diagnostica (dall'esito all'inizio).
\end{esempio}
\begin{esempio}
    Supponiamo di chiedere un finanziamento di $10000$ per un auto con scadenza
    a $1$ anno. Il finanziamento può essere o al tasso fisso o al tasso variabile.
    I modelli probabilistici permettono di identificare il miglior tasso pur non
    sapendo come cambieranno.
\end{esempio}
In generale i sistemi che modellano anche l'incertezza dovrebbero funzionare meglio
rispetto a quelli che non la gestiscono. Per modellare l'incertezza, si utilizzerà
la teoria della probabilità.

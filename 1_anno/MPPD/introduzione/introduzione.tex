\chapter{Introduzione}
Reti bayesiane sono strutture più complesse rispetto a Naive Bayes.

Affrontiamo la parte di incertezza perché siamo circondati da tantissimi dati 
che per loro natura sono incerti, quindi dobbiamo saperla gestire e decidere su 
questi dati, in questo modo peso capire cosa mi permette di decidere. (incertezza e decisioni)

Quando devo prendere delle decisoni posso avere:
\begin{itemize}
    \item dati mancanti
    \item evidenze inconsistenti
    \item errori
\end{itemize}

I modelli probabilistici modellano l'incertezza e la decisione, rappresentandola 
come dipendenze tra variabili (struttura) e le probabilità (parametri). Questo 
mi permette di automatizzare la parte di previsione, questi modelli sono scalabili
e si evolvono.

I modelli che descrivono i dati che possono essere generati da un sistema. SVM 
è un modello discriminativo, mentre NB è un modello generativo perché è caratterizzato
da variabili aleatorie, le quali possono essere campionate e si usano per fare delle
simulazioni. I modelli si basano sulla teoria della probabilità e quindi ci permetterò
di inferire quantità sconosciute e allo stesso tempo apprendere.

A livello di matematica si partirà da Teorema di Bayes per adattare i modelli 
nel tempo.

I modelli sono fondamentali perché permette di fare diagnostica e prognostica.

\begin{esempio}
    Definizione di un sistema di Ranking per gli scacchi basato sulla statistica.
    L'idea si basa sul fatto che non è possibile la misura diretta ogni risultato 
    della partita dipende dalle persone. In base al risultato dipende dai punteggi
    dei partecipanti. Quindi al termine di ogni partita bisogna aggiornare la posizione
    in base al risultato della partita e contro chi a giocato.

    Un'assunzione è che la variabile che modella la mossa è normale.

    Abbiamo due variabili gaussiane che rappresentano il valore del giocatore, Quando
    si conosce l'esito della partita, si cambia la distribuzione delle due variabili.
    
    $y$ è il risultato della partita
    $\pi$ sono le caratteristiche del giocatore
    $s$ è il livello del problema.

    L'inferenza sarà diagnostica (dall'esito all'inizio).
\end{esempio}

\begin{esempio}
    supposiano di chiedere un finanziamento di $10000$ per un auto con scadenza 
    a $1$ anno. Il finanziamento può essere o al tasso fisso o al tasso variabile.
    I modelli probabilistici permettono di identificare il miglior tasso pur non 
    sapendo come cambieranno.
\end{esempio}


\chapter{Ripasso di algebra lineare}
Per praticità ripasseremo i concetti fondamentali facendo riferimento a $\mathbb{R}^2$,
formato quindi da elementi, dette coordinate, che sono coppie ordinate $(x_1,x_2)$
(rappresentabili con un punto nel piano o con un segmento orientato con partenza
nell'origine e destinazione nelle coordinate del punto nel piano).

Ricordiamo le operazioni fondamentali, dati $R$ pari a $(x_1,x_2)$ e $Q$ pari a $(x_3,x_4)$
\begin{itemize}
    \item addizione: $P+Q=(x_1+x_3, x_2+x_4)$
    \item prodotto per uno scalare $\lambda\in\mathbb{R}$: $\lambda\cdot R=(\lambda\cdot x_1,\lambda\cdot x_2)$
    \item prodotto scalare tra vettori: $\langle P,Q\rangle\equiv P\cdot Q^T = \sum_{i=1}^n r_i\cdot q_i$
          (dove $r_i$ e $q_i$ sono rispettivamente gli elementi di $R$ e $Q$ all'indice $i$)
\end{itemize}

Ricordiamo la \textit{norma} di un vettore $X$:
\begin{equation}
    \lVert X \rVert=\equiv\sqrt{X\cdot X^T}=\sqrt{\sum_{i=1}^n x_i\cdot x_i}=\sqrt{\langle X,X\rangle}
\end{equation}

Con $X=0$ indichiamo il \textit{vettore nullo} (che ha anche norma nulla).

Definiamo il \textit{versore} (\textit{vettore unitario}) come:
\begin{equation}
    \frac{X}{\lVert X \rVert},\ \ \ X \neq 0
\end{equation}
In $\mathbb{R}^2$ l'angolo $\theta$ sotteso tra due vettori $X$ e $Y$ è:
\begin{equation}
    \cos\theta=\frac{\langle X,Y\rangle}{\lVert X \rVert\cdot \lVert Y \rVert}
\end{equation}

La proiezione di un vettore $X$ sul vettore $Y$ è:
\begin{equation}
    X_Y = \lVert X \rVert \cdot \cos\theta
\end{equation}
Si hanno quindi tre casi:
\begin{enumerate}
    \item $\theta < 90 \iff \langle X,Y\rangle >0$
    \item $\theta > 90 \iff \langle X,Y\rangle <0$
    \item $\theta = 90 \iff \langle X,Y\rangle =0$
\end{enumerate}
(quindi disegnando una retta sul piano tutti i punti sopra di essa apparterranno
ad una certa classe e quelli sotto ad un'altra).

Posso definire una retta $r$ che passa per l'origine in  $\mathbb{R}^2$ assegnando
un vettore $W=(w_1,w_2)$ ad essa ortogonale, infatti tutti i punti, ovvero
vettori, $X=(x_1,x_2)$ sulla retta sono ortogonali a $W$:
\begin{equation}
    \langle W,X\rangle=w_1\cdot x_1+w_2\cdot x_2=0
\end{equation}


Quindi la retta (ovvero l'iperpiano) mi separa due semispazi, a seconda che
$\langle X,W\rangle$ sia strettamente positivo o strettamente negativo.

Generalizzando ora a $n$ dimensioni ho che, dato l'iperpiano $h$ (di dimensione $n-1$):
\begin{itemize}
    \item se $h$ passa dall'origine allora si ha l'equazione $\langle X,Y\rangle=0$
    \item se non passa per l'origine $\langle X,Y\rangle +b=0$
\end{itemize}

I vettori in un iperpiano si proiettano tutti nello stesso modo e i punti ad un
lato e all'altro dell'iperpiano sono distinti dal fatto che $\langle X,Y\rangle +b$
sia strettamente positiva o strettamente negativa.
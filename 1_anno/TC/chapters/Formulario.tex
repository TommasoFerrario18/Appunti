\chapter{Appendice}
\section{Formulario}
\subsection{Pattern Matching algoritmo di Knuth-Morris-Pratt}
Notazione utilizzata:
\begin{itemize}
    \item $P$ = pattern
    \item $T$ = testo
    \item $m$ = lunghezza del pattern
    \item $n$ = lunghezza del testo
    \item $B(P[1, j])$ = bordo di $P[1, j]$
    \item $p$ = posizione della nuova finestra
    \item $i$ = posizione della finestra
    \item $j$ = posizione della scansione nel pattern
\end{itemize}
Formule utili:
\begin{itemize}
    \item Posizione della nuova finestra:
          \begin{equation}
              p = i + j - \phi(j - 1) - 1
          \end{equation}
    \item Posizione da cui riparte la scansione del testo:
          \begin{equation}
              i = i + j - 1
          \end{equation}
    \item Posizione da cui riparte la scansione del pattern:
          \begin{equation}
              j = \phi(j - 1) + 1
          \end{equation}
    \item Calcolo della prefix function:
          \begin{equation}
              \phi(j) = \begin{cases}
                  |B(P[1, j])| & \text{se } 1 \leq j \leq m \\
                  -1           & \text{se } j = 0
              \end{cases}
          \end{equation}
\end{itemize}
\subsection{Pattern Matching algoritmo di Baeza-Yates e Gonnet}
Notazione utilizzata:
\begin{itemize}
    \item $P$ = pattern
    \item $T$ = testo
    \item $m$ = lunghezza del pattern
    \item $n$ = lunghezza del testo
    \item $D_i$ = word riferita al carattere $T[i]$, il bit $j$-esimo è 1 se e
          solo se $P[1, j] = suff(T[1, i])$.
\end{itemize}
Formule utili:
\begin{itemize}
    \item Posizione dell'occorrenza esatta:
          \begin{itemize}
              \item $i = i - m + 1$
          \end{itemize}
    \item Calcolo della $i$-esima word:
          \begin{equation}
              D_i = \textbf{RSHIFT1}(D_{j - 1}) \ \textbf{AND} \ B_{T[i]}
          \end{equation}
\end{itemize}
\subsection{Pattern Matching algoritmo di Wu-Manber}
Notazione utilizzata:
\begin{itemize}
    \item $P$ = pattern
    \item $T$ = testo
    \item $m$ = lunghezza del pattern
    \item $n$ = lunghezza del testo
    \item $k$ = numero di errori
    \item $suff_h$ = suffisso con al più $h$ errori
    \item $D_i^h$ = word di bit, riferita al carattere $T[i]$ dove il bit
          $j$-esimo è 1 se e solo se $P[1, j] = suff_h(T[1, i])$.
\end{itemize}
Formule utili:
\begin{itemize}
    \item Posizione dell'occorrenza approssimata:
          \begin{itemize}
              \item $i - m + 1$ se $h = 0$ (Posizione di inizio, in quanto si
                    tratta di un match esatto)
              \item $i$ se $h = k$  (Posizione di fine, in quanto si tratta di
                    un match approssimato)
          \end{itemize}
    \item Calcolo della $i$-esima word:
          \begin{equation}
              D_i^h = \begin{array}{l}
                  \textbf{RSHIFT1}(D_{i - 1}^h) \ \textbf{AND} \ B_{T[i]} \\
                  \textbf{OR}                                             \\
                  \textbf{RSHIFT1}(D_{i - 1}^{h - 1})                     \\
                  \textbf{OR}                                             \\
                  D_{i - 1}^{h - 1}                                       \\
                  \textbf{OR}                                             \\
                  \textbf{RSHIFT1}(D_{i}^{h - 1})
              \end{array}
          \end{equation}
\end{itemize}
\subsection{Strutture di indicizzazione}
Notazione utilizzata:
\begin{itemize}
    \item $P$ = pattern
    \item $T$ = testo
    \item $m$ = lunghezza del pattern
    \item $n$ = lunghezza del testo
    \item $B$ = BWT del testo
    \item $SA$ = suffix array del testo
    \item $C$ = array dei caratteri FM-index
    \item $Occ$ = array delle occorrenze FM-index
    \item $Q-suffisso$ = suffisso che inizia con $Q$
    \item $LF(i)$ = posizione del suffisso precedente a $i$ nel suffix array
    \item $B[i]$ = carattere in posizione $i$ della BWT
    \item $F$ = array dei first character, si ottiene ordinado i valori della BWT.
\end{itemize}
Formule utili:
\begin{itemize}
    \item Relazione tra BWT e SA:
          \begin{equation}
              B[i] = T[SA[i] - 1] \ \ T[SA[i]] = F[i]
          \end{equation}
    \item Per ogni posizione $i$, il simbolo $B[i]$ precede nel testo il simbolo $F[i]$.
    \item Posizione $j$ nel suffix array del suffisso che inizia con $B[i]$ si ottiene come:
          \begin{equation}
              j = LF(i) = C(B[i]) + Occ(i, B[i]) + 1
          \end{equation}
    \item Calcolo del $Q$-intervallo ($[b, e)$) dopo la backward extension con
          $\sigma$:
          \begin{equation}
              \begin{array}{l}
                  b' = C(\sigma) + Occ(b, \sigma) + 1 \\
                  e' = C(\sigma) + Occ(e, \sigma) + 1
              \end{array}
          \end{equation}
\end{itemize}